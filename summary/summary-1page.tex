\documentclass[11pt,a4paper]{article}
\usepackage[utf8]{inputenc}
\usepackage[T1]{fontenc}
\usepackage{geometry}
\geometry{hmargin=0.7in, vmargin=0.55in}

\title{\textbf{Harnessing Individual Participant Data from Clinical Trials to Predict Relapse in Visceral Leishmaniasis}}
\author{Thesis Summary -- James Wilson}
\date{\today}


\begin{document}
\maketitle
\thispagestyle{empty}

\subsection*{Background}
Visceral leishmaniasis (VL) is a neglected tropical disease caused by Leishmania parasites that disproportionately affects the poorest populations. The disease is typically fatal without timely treatment. The Indian subcontinent (ISC) and East Africa have historically borne the greatest burden, with a WHO-supported elimination campaign in the ISC achieving remarkable success over the last 20 years, and a similar framework recently launched in East Africa. Sustained progress, however, hinges on addressing treatment failure—particularly relapse—which fuels ongoing transmission, raises concerns about drug resistance, and demands early identification and effective management. To help meet these challenges, the Infectious Diseases Data Observatory (IDDO) has established an individual participant data (IPD) platform, curating over 14,000 records from 50 clinical studies to allow insights beyond the scope of individual trials.

\subsection*{Aim}

The aim of this thesis is to explore how IPD from the IDDO VL data platform can be harnessed to better understand both the causes and predictors of VL treatment failure. The ultimate goal is to inform clinical decision-making and public health policy to improve patient outcomes and support the success of the ongoing elimination campaigns in the ISC and East Africa.

\subsection*{Summary of methods and results}

First, a systematic review of prognostic models was conducted following a published protocol and using established reporting (TRIPOD-SRMA), extraction (CHARMS) and risk-of-bias (PROBAST) tools. Eight studies describing 12 models were identified, all predicting mortality rather than relapse, with uniformly high risk of bias. The absence of any models predicting relapse highlights an important evidence gap and justifies the subsequent models presented in this thesis.

Second, using IPD from over 6,000 VL patients across 27 studies, prognostic models predicting 6-month relapse were developed and internally validated separately for the ISC and East Africa. Eligible studies were prospective with at least 6 months' follow-up. Multivariable generalised linear mixed-effects models were applied with multiple imputation for missing data and bootstrap validation. Relapse risk was associated with extremes of age, shorter fever duration, anaemia severity, malnutrition severity, pre-treatment parasite density, smaller baseline spleen size, and treatment regimen. Model performance showed moderate discrimination (c-statistics 0.64--0.71) and good calibration. Manuscripts in preparation.

\subsection*{Discussion and conclusions}

Model findings are discussed in the context of regional VL epidemiology and elimination campaigns. Several novel predictors are identified alongside established risk factors. Plausible biological mechanisms for the identified predictors are proposed, and the implications for clinical practice and public health policy are explored.

This thesis presents the first prognostic models for VL relapse, demonstrating the value of collaborative data sharing and IPD meta-analysis to overcome the limitations of otherwise sparse data from individual trials. The models offer new insights into disease dynamics and have real potential to inform clinical decision-making and public health policy, supporting elimination efforts in both the ISC and East Africa.

\end{document}