\chapter{\label{ch:7-discussion}Discussion}
\minitoc{}

\section{Overview}

This thesis demonstrates both the feasibility and the challenges of predicting VL relapse using routinely collected clinical data. In response to a clear public health need and the lack of existing prognostic tools, the prognostic models presented here represent the first systematic attempt to predict relapse risk across the ISC and East Africa.

Despite substantial heterogeneity between contributing studies, several important relapse-predictor relationships were identified. Fever duration and baseline parasite grade showed effects shared across regions, supporting their role as core prognostic features. In contrast, other predictors demonstrated region-specific effects: the association between anaemia and relapse differed in direction between regions, while WBC count and malnutrition were predictive only in East Africa, and age only in the ISC. Models incorporating parasite grade achieved superior discrimination where available. Excluding parasite grade, however, still allowed for meaningful discriminatory performance.

This final chapter provides a structured reflection on the models presented in the Results chapters. Insights from the Background and Systematic Review chapters are drawn upon to interpret, contextualise, and assess the implications of these findings.

First, the models are explored and critically appraised from two complementary angles; considering \textbf{(i)} the mechanistic plausibility of the identified predictor-relapse relationships, \textemdash\ `\textit{what do the models teach us about the nature and determinants of VL relapse?}', and \textbf{(ii)} the clinical implications, impact, challenges, and practicalities related to model deployment in routine settings \textemdash\ `\textit{how can the models be applied to improve outcomes at both the patient and policy levels?}'.

Second, key limitations are explored relating to the underlying data, study design, and assumptions made during the modelling process. The potential impact of such limitations on model performance, interpretability, and generalisability are discussed, alongside consideration of how they may influence the translation of findings into real-world clinical or programmatic settings.

Third, directions for future research are considered. These consider how data-driven approaches can be leveraged to shine new light onto both the nature and determinants of VL outcomes, and how such efforts can be translated into meaningful policy that benefits patients.

Finally, the thesis concludes with a high-level summary of how the models collectively advance understanding of VL relapse risk, highlighting their key strengths, novel insights, and potential contribution to both clinical decision-making and public health policy.

\section{Plausibility of Relapse Predictors}

It deserves recognition that the relapse-predictor relationships identified in the relapse models, and illustrated in Figures~\ref{fig:isc_var_forest_combined} and \ref{fig:ea_var_forest_combined}, reflect statistical associations arising under the stated modelling assumptions alone. In the absence of additional context, these associations do not elucidate the biological mechanisms underlying relapse and should not be interpreted as causal. In this section, we explore ...

\section{Model Implementation}

\section{Limitations}

\section{Future Research}

\section{Conclusion}

\begin{itemize}
    \item High level summary of prediction models
          \begin{itemize}
              \item Impactful statement - emphasising large datasets, novelty, and rigorous statistical approach
              \item Review of model predictors and performance
              \item Link to structure of discussion
          \end{itemize}
    \item Causal plausibility of identified relationships
          \begin{itemize}
              \item What is relapse?
              \item Is relapse different to treatment failure?
              \item East Africa models
          \end{itemize}
    \item Implementation considerations
          \begin{itemize}
              \item Model form (considering clinical prediction rule, vs.\ application)
              \item Choice of intercept
              \item Patient counselling
              \item Policy implementation
              \item Stakeholder engagement
          \end{itemize}
    \item Strengths
          \begin{itemize}
              \item Novelty
              \item Comprehensive statistical approach
          \end{itemize}
    \item Limitations
          \begin{itemize}
              \item Underlying data: Applicability of prediction models to non-trial patients: need for external validation. Need to be cautious when using models to risk stratify patients with severe disease, immunocompromised.
              \item Underlying data: 'Noisy' data. For example, affected by patient recall.
              \item Modelling assumptions: Treatment
              \item Modelling assumptions: Age
              \item Modelling assumptions: East Africa
          \end{itemize}
    \item Directions for future research
    \item
    \item Conclusion
\end{itemize}

•	Biological and clinical plausibility:
•	Host factors
•	Treatment-related factors
•	Immunological hypotheses
•	Discussion of whether predictors are:
•	Causal
•	Proxies
•	Markers of unmeasured processes

\section{Implementation Considerations}

\subsection{Indian subcontinent}
•	Model performance and strengths
•	Key predictors and their regional relevance
•	Implications for:
•	Follow-up strategies
•	Targeted surveillance
•	Resource allocation

\subsection{East Africa}
•	Differences in predictors and performance
•	Role of:
•	Comorbidities
•	Treatment heterogeneity
•	Programmatic constraints
•	Implications for clinical and public health practice

5. Clinical and programmatic implementation considerations

(Strongly recommend keeping this separate)
•	Who would use these models?
•	At what point in the care pathway?
•	Data availability in routine care
•	Risks of misclassification
•	Ethical considerations:
•	Risk stratification
•	Stigma or differential care
•	Alignment with VL elimination goals

This section is especially valuable for examiners and policy-minded readers.

\section{Strengths}
\section{Limitations}

Your list is good; consider adding framing:
•	Data limitations
•	Trial populations
•	Missingness and measurement error
•	Methodological limitations
•	Overfitting
•	Outcome definition
•	Lack of time-to-event modelling
•	Conceptual limitations
•	Prediction vs causation
•	Immune mechanisms inferred but not measured

Explicitly note which limitations are unlikely vs likely to materially affect conclusions.

\section{Future Research}

Strong already. You could sharpen by grouping:
•	Methodological
•	Dynamic prediction models
•	Time-to-relapse modelling
•	Incorporation of biomarkers
•	Biological
•	Immune correlates of relapse
•	Implementation
•	Prospective validation
•	Impact studies
•	Stakeholder-driven refinement

\section{Conclusion}

•	One-two paragraph synthesis of:
•	What is now known that wasn't before
•	Why this matters for VL control
•	Explicit statement of contribution:
•	    Scientific
•	    Clinical
•	    Methodological

% summarise main story

% importance of relapse -> widespread recognition of urgent need to determine cure and monitor treatment response -> 
% prediction model for relapse -> currently no prediction models for relapse -> development of new relapse models!

