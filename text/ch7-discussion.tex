% \begin{savequote}[8cm]
%     Quote goes here.
%     \qauthor{--- James Wilson}
% \end{savequote}

\chapter{\label{ch:7-discussion}Discussion}
\minitoc{}

\section{Structure outline}

\begin{itemize}
    \item High level summary of prediction models
          \begin{itemize}
              \item Impactful statement - emphasising large datasets, novelty, and rigorous statistical approach
              \item Review of model predictors and performance
              \item Link to structure of discussion
          \end{itemize}
    \item Causal plausibility of identified relationships
          \begin{itemize}
              \item What is relapse?
              \item Is relapse different to treatment failure?
              \item East Africa models
          \end{itemize}
    \item Implementation considerations
          \begin{itemize}
              \item Model form (considering clinical prediction rule, vs.\ application)
              \item Choice of intercept
              \item Patient counselling
              \item Policy implementation
              \item Stakeholder engagement
          \end{itemize}
    \item Strengths
          \begin{itemize}
              \item Novelty
              \item Comprehensive statistical approach
          \end{itemize}
    \item Limitations
          \begin{itemize}
              \item Underlying data: Applicability of prediction models to non-trial patients: need for external validation. Need to be cautious when using models to risk stratify patients with severe disease, immunocompromised.
              \item Underlying data: 'Noisy' data. For example, affected by patient recall.
              \item Modelling assumptions: Treatment
              \item Modelling assumptions: Age
              \item Modelling assumptions: East Africa
          \end{itemize}
    \item Directions for future research
    \item Conclusion
\end{itemize}

•	Biological and clinical plausibility:
•	Host factors
•	Treatment-related factors
•	Immunological hypotheses
•	Discussion of whether predictors are:
•	Causal
•	Proxies
•	Markers of unmeasured processes

\section{Implementation Considerations}

\subsection{Indian subcontinent}
•	Model performance and strengths
•	Key predictors and their regional relevance
•	Implications for:
•	Follow-up strategies
•	Targeted surveillance
•	Resource allocation

\subsection{East Africa}
•	Differences in predictors and performance
•	Role of:
•	Comorbidities
•	Treatment heterogeneity
•	Programmatic constraints
•	Implications for clinical and public health practice

5. Clinical and programmatic implementation considerations

(Strongly recommend keeping this separate)
•	Who would use these models?
•	At what point in the care pathway?
•	Data availability in routine care
•	Risks of misclassification
•	Ethical considerations:
•	Risk stratification
•	Stigma or differential care
•	Alignment with VL elimination goals

This section is especially valuable for examiners and policy-minded readers.

\section{Strengths}
\section{Limitations}

Your list is good; consider adding framing:
•	Data limitations
•	Trial populations
•	Missingness and measurement error
•	Methodological limitations
•	Overfitting
•	Outcome definition
•	Lack of time-to-event modelling
•	Conceptual limitations
•	Prediction vs causation
•	Immune mechanisms inferred but not measured

Explicitly note which limitations are unlikely vs likely to materially affect conclusions.

\section{Future Research}

Strong already. You could sharpen by grouping:
•	Methodological
•	Dynamic prediction models
•	Time-to-relapse modelling
•	Incorporation of biomarkers
•	Biological
•	Immune correlates of relapse
•	Implementation
•	Prospective validation
•	Impact studies
•	Stakeholder-driven refinement

\section{Conclusion}

•	One-two paragraph synthesis of:
•	What is now known that wasn't before
•	Why this matters for VL control
•	Explicit statement of contribution:
•	    Scientific
•	    Clinical
•	    Methodological

% summarise main story

% importance of relapse -> widespread recognition of urgent need to determine cure and monitor treatment response -> 
% prediction model for relapse -> currently no prediction models for relapse -> development of new relapse models!

