\begin{savequote}[8cm]
    Quote goes here.
    \qauthor{--- James Wilson}
\end{savequote}

\chapter{\label{ch:4-isc-model}Development and internal validation of a prognostic model predicting visceral leishmaniasis relapse in the Indian subcontinent}
\minitoc

\section{Introduction}

This chapter describes the development and internal validation of prognostic models predicting visceral leishmaniasis (VL) relapse in immunocompetent patients from the Indian subcontinent (ISC).

As outlined in Chapter~\ref{ch:2-background}, relapse is not only consequential for individual patients but also represents an infection reservoir that threatens the success of ongoing VL elimination efforts. Development of a non-invasive tool predicting relapse, as a test of cure following initial treatment, has been identified by the WHO as a current research priority\cite{WHO2024_Leishmaniasis,WHO_2024_VL_easternAfrica,who_sea_elim}. A prediction model using routinely collected patient information could fulfil this role by identifying treatment-responsive patients at increased risk of relapse, enabling targeted counselling and follow-up to support early detection and management.

The Infectious Diseases Data Observatory (IDDO) VL data platform hosts over 14,000 individual participant data (IPD) from almost 50 clinical studies across all endemic regions\cite{IDDO_VisceralLeishmaniasis2025}. Since no prognostic models for VL relapse have previously been published (Chapter~\ref{ch:3-sys-review}), model updating is not possible. The IDDO VL data platform therefore provides a unique opportunity to develop and validate the first relapse prediction models tailored to patients in the ISC.

The current chapter is divided into a methodology section, explaining in detail the methods used to standardise, describe, and analyse the IPD, and a results section, where the principal findings are presented. A number of results, including figures, tables, model equations, and performance measures, are included in the Appendices and Supplemental files. The chapter concludes with a summary of the main findings. The mechanisms underlying the models' findings, and the practical ramifications of deploying such models in the field, are considered in the final discussion chapter (Chapter \ref{ch:6-discussion}).

Two complementary models are presented. Model~1 incorporates parasite grade from a pre-treatment tissue aspirate, whereas Model~2 excludes this variable, allowing for use in settings where diagnosis relies solely on serological or clinico-epidemiological criteria.

\section{Methodology}

\subsection{Study inclusion}
\subsection{Participants and outcomes}
\subsection{Sample size}
\subsection{Candidate predictors}
\subsection{Descriptive analysis}
\subsection{Missing data}
\subsection{Model specification and variable selection}
\subsection{Model performance}
\subsection{Internal validation}

%\subsection{Risk of bias and applicability assessment}

\section{Results}

\subsection{Studies}
\subsection{Participants}

%\subsection{Risk of bias and applicability assessment}
\subsection{Model development and internal validation}
\subsection{Model performance}

\section{Discussion}

\section{Conclusion}