% \begin{savequote}[8cm]
%   Quote goes here.
%   \qauthor{--- James Wilson}
% \end{savequote}

\chapter{\label{ch:6-ea-model-results}Results: East Africa Models}
\minitoc

\section{Descriptive Analysis}

A total of 2,095 patients from nine East African studies were identified after applying the eligibility criteria to the IDDO VL data platform\cite{hailu2010,khalil2014,mbui2019,musa2010,musa2012,ritmeijer2001,ritmeijer2006,veeken2000,wasunna2016}. At the participant selection stage, 370 participants (13.5\%, 370/2749) were excluded due to either having a positive HIV test, or lacking a HIV test and recruiting from a site outside of Sudan. 323 (11.7\%, 323/2749) participants were excluded due to not achieving initial cure. A flow-diagram is presented in Figure~\ref{fig:ea_flow_diagram}.

% flow diagram
\begin{figure}[tb]
  \centering
  \includegraphics[width = 0.8\textwidth, trim={4cm 0cm 2cm 2cm}, clip]{figures/ch6/ea_flow_chart.pdf}
  \caption{Flow diagram showing the studies and participants excluded from East Africa model development, following application of the eligibility criteria. HIV: human immunodeficiency virus; IDDO: Infectious Diseases Data Observatory; EA: East Africa; PKDL: post kala-azar dermal leishmaniasis; VL: visceral leishmaniasis.}
  \label{fig:ea_flow_diagram}
\end{figure}

\subsection{Study Characteristics}

Important study characteristics are presented in Table~\ref{tab:ea_study}.

A median of 138 patients per contributed study were included in model development, ranging from 29 to 638 patients (IQR: 92 to 289 patients). Five studies recruited from sites across multiple countries (1,186 [57.8\%] patients)\cite{hailu2010,khalil2014,mbui2019,musa2012,wasunna2016}, and the remaining four studies recruited from site(s) in single countries, including Sudan (two studies, 505 [24.6\%] patients)\cite{musa2010,veeken2000} and Ethiopia (two studies 360 [17.6\%] patients)\cite{ritmeijer2006,ritmeijer2001}. A full breakdown of patients by country is included in Section~\ref{sec:pt-char} below.

The majority of studies were sponsored by DND\textit{i} and conducted across the Leishmaniasis East Africa Platform (LEAP) (six studies, 1,226 [59.8\%] patients, conducted 2004 to 2016). These included LEAP~0104 \textemdash\ a large pivotal study comparing short course SSG \& paromomycin to SSG or paromomycin monotherapy (three contributed studies, 967 [47.1\%] patients)\cite{hailu2010,musa2010,musa2012}. LEAP~0106 was a smaller study with the aim of identifying the minimum safe and effective dose of liposomal amphotericin B (LAMB) (one study, 92 [4.5\%] patients)\cite{khalil2014}. LEAP~0208 compared miltefosine with two different combination therapies (LAMB + miltefosine, LAMB + SSG) (138 [6.7\%] patients)\cite{wasunna2016}, and LEAP~0714 was a smaller allometric dosing study of miltefosine in paediatric patients (29 [1.4\%] patients)\cite{mbui2019}.

The remaining studies were led by MSF and conducted between 1998 and 2005, including two studies comparing generic and branded SSG (577 [28.1\%] patients)\cite{ritmeijer2001,veeken2000}, and a further study comparing SSG and miltefosine (248 [12.1\%] patients)\cite{ritmeijer2006}.

{
\newgeometry{left=1.5cm, right=1.5cm, top=2cm, bottom=1.8cm}

\begin{landscape}
  \pagestyle{empty}

  % latex table generated in R 4.4.1 by xtable 1.8-4 package
% Wed Dec 10 17:36:05 2025

{
\newgeometry{left=1.5cm, right=1.5cm, top=2cm, bottom=1.5cm}
\captionsetup{width=1\textheight, font=scriptsize, skip=7pt}

\begin{landscape}
  % \singlespacing
  \pagestyle{empty}
  \scriptsize
  \begin{ThreePartTable}
    \begin{TableNotes}
      \item[1] Study name is composed of the lead author and year of publication.
      \item[2] Refer to the supplementary material of the publication for allometric dosing table\cite{mbui2019}.
      \item[3] SSG tested and dispensed by International Dispensary Association, The Netherlands; manufactored by Albert David Ltd, India.
      \item[4] For both arms in Veeken 2010, Ritmeijer 2001, and Ritmeijer 2006, if positive initial test--of--cure, treatment would continue with SSG, including if previously taking MF, until two subsequent consecutive tests of cure, performed weekly, were negative.
      \item[5] Actual dosing ranged from 2.0--3.33 mg/kg/day after rounding to nearest 10mg tablets; full regimen described in publication\cite{wasunna2016}
    \end{TableNotes}

    \begin{longtable}[c]{L{1cm} L{4cm} L{1cm} L{1.5cm} L{3.5cm} L{2.5cm} L{3.7cm} L{1cm} L{1cm} L{1cm} L{1.3cm}}
      \caption{Key characteristics of included studies from East Africa, ordered by lead author and year of publication/protocol. -: not reported; BD: twice daily (bis die); D: Day(s); IM: intramuscular; IV: intravenous; LAMB: Liposomal amphotericin B (Gilead formulation); MF: Miltefosine; mg/kg: milligrams per kilogram; OD: once daily (omni die); PO: per os (oral); PM: Paromomycin; SSG: Sodium stibogluconate; \label{tab:ea_study}}                                                                                                                                                                                                                                                                                                                                                                                                               \\
      \toprule
      Study\tnote{1}                     & Title                                                                                                                                                                                       & Journal                  & Sponsor/ funding            & Location(s)                                                                                                                                                                        & Study design                                                                        & Study arm(s)\tnote{4}                                                                                                                                                                                                       & Age (years) & Study period & n (model) & Relapses (\%) \\ \midrule
      \endfirsthead

      \caption[]{continued}                                                                                                                                                                                                                                                                                                                                                                                                                                                                                                                                                                                                                                                                                                                                                                                                                                       \\
      \toprule
      Study\tnote{1}                     & Title                                                                                                                                                                                       & Journal                  & Sponsor/ funding            & Location(s)                                                                                                                                                                        & Study design                                                                        & Study arm(s)\tnote{4}                                                                                                                                                                                                       & Age (years) & Study period & n (model) & Relapses (\%) \\ \midrule
      \endhead

      \multicolumn{11}{r}{\textit{continued on next page}}                                                                                                                                                                                                                                                                                                                                                                                                                                                                                                                                                                                                                                                                                                                                                                                                        \\
      \endfoot

      \insertTableNotes
      \endlastfoot

      Hailu 2010\cite{hailu2010}         & Geographical variation in the response of visceral leishmaniasis to paromomycin in East Africa: a multicentre, open-label, randomized trial                                                 & PLoS Negl Trop Dis       & DNDi/MSF                    & 5 centres: Ethiopia (Gondar University Hospital, Arba Minch Hospital); Sudan (MSF Treatment Centre, Um el Khjer; Ministry of Health Hospital, Kassab); Kenya (CCR, KEMRI, Nairobi) & Multicentre, open-label, randomised trial                                           & (1) SSG, 20 mg/kg, IV or IM, OD, 30 D; (2) PM, 15 mg/kg, IM, OD, 21 D; (3) Combination of PM and SSG, same dose, frequency and route, 17 D                                                                                  & 4--60       & 2004--2008   & 289       & 21 (7.3)      \\ \midrule
      Khalil 2014\cite{khalil2014}       & Safety and efficacy of single dose versus multiple doses of AmBisome for treatment of visceral leishmaniasis in eastern Africa: a randomised trial                                          & PLoS Negl Trop Dis       & DNDi (multiple funders)     & 3 centres: Ethiopia (Gondar University Hospital, Arba Minch Hospital); Sudan (Ministry of Health Hospital, Kassab)                                                                 & Multicentre, open-label, non-inferiority, randomised trial with adaptive design     & LAMB, IV, either single dose at (1) 7.5mg/kg; (2) 10 mg/kg; (3) 12.5 mg/kg; (4) 15 mg/kg, or multiple dose at (5) 3 mg/kg, OD, D1-5, 14, 21 (total 21 mg/kg)                                                                & $\geq$ 4    & 2009--2011   & 92        & 11 (12.0)     \\ \midrule
      Mbui 2019\cite{mbui2019}           & Pharmacokinetics, Safety, and Efficacy of an Allometric Miltefosine Regimen for the Treatment of Visceral Leishmaniasis in Eastern African Children: An Open-label, Phase II Clinical Trial & Clin Infect Dis          & DNDi (multiple funders)     & 2 clinical sites: Kachelibra, West Pokot County, Kenya; Amudat, Karamoja sub-region, Uganda                                                                                        & Open-label, phase II, clinical trial                                                & (1) MF, allometric dosing according to sex, height, and weight, BD, PO, 28 D\tnote{2}                                                                                                                                       & 4--12       & 2015--2016   & 29        & 2 (6.9)       \\ \midrule
      Musa 2010\cite{musa2010}           & Paromomycin for the treatment of visceral leishmaniasis in Sudan: a randomized, open-label, dose-finding study                                                                              & PLoS Negl Trop Dis       & DNDi (multiple funders)     & Ministry of Health Hospital, Kassab, Sudan                                                                                                                                         & Open-label, open-label, phase II, randomised, dose-finding study                    & (1) PM, 15 mg/kg, IM, OD, 28 D; (2) PM, 20 mg/kg, IM, OD, 21 D                                                                                                                                                              & 4--60       & 2005--2006   & 40        & 8 (20.0)      \\ \midrule
      Musa 2012\cite{musa2012}           & Sodium stibogluconate (SSG) \& paromomycin combination compared to SSG for visceral leishmaniasis in East Africa: a randomised controlled trial                                             & PLoS Negl Trop Dis       & DNDi/MSF (multiple funders) & 6 centres: 5 centres described in Khalil 2014 (above) and Amudat Hospital, Uganda.                                                                                                 & Multicentre, open-label, parallel-arm, randomised trial                             & (1) SSG, 20 mg/kg, IV or IM, OD, 30 D; (2) PM, 20 mg/kg, IM, OD, 21 D; (3) combination of SSG, 20 mg/kg, IV or IM, OD, 17 D and PM, 15 mg/kg, IM, OD, 17 D                                                                  & 4--60       & 2004--2010   & 638       & 30 (4.7\%)    \\ \midrule
      Ritmeijer 2001\cite{ritmeijer2001} & Ethiopian visceral leishmaniasis: generic and proprietary sodium stibogluconate are equivalent; HIV co-infected patients have a poor outcome                                                & Trans R Soc Trop Med Hyg & MSF                         & Temporary MSF treatment centre, Densha, Ethiopia                                                                                                                                   & Open-label, pseudo-randomised controlled trial                                      & (1) SSG (generic)\tnote{3}, 20 mg/kg, IM, 30 D; (2) SSG (Pentostam, GlaxoWellcome), 20 mg/kg, IM, 30 D                                                                                                                      & all         & 1998--1999   & 112       & 1 (0.9)       \\ \midrule
      Ritmeijer 2006\cite{ritmeijer2006} & A comparison of miltefosine and sodium stibogluconate for treatment of visceral leishmaniasis in an Ethiopian population with high prevalence of HIV infection                              & Clin Infect Dis          & MSF                         & 2 centres in Ethiopia: Humera Hospital, Mycadra Health Center                                                                                                                      & Open-label, randomised controlled trial                                             & (1) SSG, 20 mg/kg/day, IM, OD, 30 D (extended if HIV positive); (2) MF, 100 mg/day, PO, OD, 28 D                                                                                                                            & $\geq$ 15   & 2003--2005   & 248       & 6 (2.4)       \\ \midrule
      Veeken 2000\cite{veeken2000}       & A randomized comparison of branded sodium stibogluconate and generic sodium stibogluconate for the treatment of visceral leishmaniasis under field conditions in Sudan                      & Trop Med Int Health      & MSF                         & 2 MSF treatment centres in Gedaref State, Sudan: Um Kuraa, Kassab                                                                                                                  & Open-label, pseudo-randomised controlled trial                                      & (1) SSG (generic)\tnote{3}, 20 mg/kg, IM, 30 D; (2) SSG (Pentostam, GlaxoWellcome), 20 mg/kg, IM, 30 D                                                                                                                      & all         & 1998--1999   & 465       & 4 (0.9)       \\ \midrule
      Wasunna 2016\cite{wasunna2016}     & Efficacy and Safety of AmBisome in Combination with Sodium Stibogluconate or Miltefosine and Miltefosine Monotherapy for African Visceral Leishmaniasis: Phase II Randomized Trial.         & PLoS Negl Trop Dis       & DNDi (multiple funders)     & 3 centres: Kenya (Kimalel Health Centre); Sudan (Dooka Hospital and Ministry of Health Hospital, Kassab)                                                                           & Phase II, open-label, non-comparative randomised trial (adaptive-sequential design) & (1) Combination of LAMB, 10 mg/kg, IV, single dose, D1 and SSG, 20 mg/kg, IM, OD, D2-11; (2) Combination of LAMB, 10 mg/kg, IV, single dose, D1, and MF, 2.5 mg/kg, PO, OD, D2-11; (3) MF\tnote{5}, 2.5 mg/kg, PO, OD, 28 D & 7--60       & 2010--2012   & 138       & 16 (11.6)     \\ \bottomrule
    \end{longtable}
  \end{ThreePartTable}
\end{landscape}
}
\pagestyle{fancy}
\restoregeometry

  \begin{figure}[tb]
    \centering
    \includegraphics[width=1.35\textwidth]{figures/ch6/treat.pdf}
    \caption{Bar chart showing the distribution of treatment regimens across contributing studies from East Africa. Drugs are colour-coded (see legend). Important distinguishing dosing information provided in the overlaying labels. Full treatment details presented in Table \ref{tab:ea_study}. D: days; LAMB: liposomal amphotericin B (Gilead); PM: paromomycin; MF: miltefosine; mg: milligrams/kilogram; SSG: sodium stibogluconate.}
    \label{fig:ea_treat}
  \end{figure}

\end{landscape}
}
\pagestyle{fancy}
\restoregeometry

\subsubsection{Study Design and Treatment Arms}

All studies bar one allocated patients to more than one treatment arm (eight studies, 2,022 [98.6\%] patients)\cite{hailu2010,musa2010,musa2012,khalil2014,ritmeijer2006,ritmeijer2001,veeken2000,wasunna2016}. Six studies randomly allocated patients to treatment arms (1,445 [70.5\%] patients)\cite{hailu2010,khalil2014,musa2010,musa2012,ritmeijer2006,wasunna2016} and two studies were pseudo-randomised\footnote{Based on whether the direct agglutination test (DAT) batch number was odd or even.} (577 [28.1\%] patients)\cite{ritmeijer2001,veeken2000}. Three studies were described by the authors as Phase 2 trials (207 [10.1\%] patients)\cite{mbui2019,musa2010,wasunna2016}.

Treatment arms are summarised in Figure~\ref{fig:ea_treat}. Treatment allocations at the patient level are described further in Section~\ref{sec:ea_cat_var}.

\subsubsection{Study Eligibility Criteria}

All but two studies described age limits in their inclusion criteria\cite{hailu2010,khalil2014,mbui2019,musa2010,musa2012,ritmeijer2006,wasunna2016}. Two studies specified lower age limits only ($\geq$~15 years in one study; 248 [12.1\%] patients, and $\geq$~4 years in a further study; 92 [4.5\%] patients). Three studies limited inclusion to patients between 4 and 60 years (967 [47.1\%] patients)\cite{musa2010,musa2012,hailu2010}, and one study limited inclusion to patients between 7 and 60 years (138 [6.7\%] patients)\cite{wasunna2016}. One study recruited children between 4 and 12 years (29 [1.4\%] patients)\cite{mbui2019}.

Six studies (all DND\textit{i}, 1,226 [59.8\%] patients) reported exclusion criteria based on platelet count ($\geq$~40~$\times$10$^9$/L) and haemoglobin ($\geq$~40~g/L in one study, 92 [4.4\%] patients\cite{khalil2014}, and $\geq$~50~g/L in five studies, 1,134 [55.2\%] patients\cite{hailu2010,mbui2019,musa2010,musa2012,wasunna2016}). Four studies described excluding patients with `severe' VL (1,048 [51.1\%] patients)\cite{hailu2010,khalil2014,mbui2019,musa2012}, and four studies excluded patients based on malnutrition severity (three studies excluded patients with severe protein and/or caloric malnutrition; 467 [22.8\%] patients\cite{hailu2010,musa2010,wasunna2016}, and one study excluded children with weight-for-height or BMI-for-age z-scores < -3; 29 [1.4\%] patients\cite{mbui2019}). Seven studies (1,474 [71.9\%] patients) excluded patients with severe concomitant illness and/or co-infection\cite{hailu2010,khalil2014,mbui2019,musa2010,musa2012,ritmeijer2006,wasunna2016}.

Four studies (299 [14.6\%] patients) required a negative HIV test as a pre-requisite for inclusion\cite{wasunna2016,musa2010,mbui2019,khalil2014}. The remaining studies either did not report HIV testing or HIV-related exclusion criteria (one study, 465 [22.7\%] patients)\cite{veeken2000}, or permitted the inclusion of patients with VL/HIV co-infection (with variable testing strategies, 4 studies, 1,019 [49.7\%] patients)\cite{hailu2010,musa2012,ritmeijer2001,ritmeijer2006}.

Of the eight studies that included adults (2,022 [98.6\%] patients), four studies (1059/2022 [52.4\%] patients) excluded women who were pregnant or lactating\cite{musa2010,musa2012,hailu2010,khalil2014}, one study excluded women of child-bearing age (138/2022 [6.8\%] patients)\cite{wasunna2016}, one study excluded all female patients regardless of age (248/2022 [12.3\%] patients)\cite{ritmeijer2006},\footnote{Due to concern re: teratogenicity of miltefosine.} and two studies did not report excluding women based on pregnancy or lactating status(577/2051 [28.1\%] patients)\cite{ritmeijer2001,ritmeijer2006}.

Patients with prior VL treatment were excluded in seven studies; either within 6 months in five studies (1,088 [53.0\%] patients)\cite{hailu2010,khalil2014,mbui2019,musa2010,musa2012}, or at any prior time point in two studies (577 [28.1\%] patients)\cite{veeken2000,ritmeijer2001}.

Full study-specific eligibility criteria can be found in the \href{https://github.com/jpwil/dphil}{Supplementary Material}.

\subsubsection{Diagnostic Criteria}

A positive tissue aspirate was required for VL confirmation in all six DND\textit{i} studies (1,226 [59.8\%] patients), with tissue type including bone, spleen, or lymph node, depending on practice at the recruiting site. All studies required clinical evidence of VL, although the exact working varied across studies (presented in \href{https://github.com/jpwil/dphil}{Supplementary Material}). The three earlier MSF studies (825 [40.2\%] patients)\cite{veeken2000,ritmeijer2006,ritmeijer2001} confirmed VL with a DAT (titre $\geq$~1:6400), with borderline titres (1:8000--1:3200) requiring a confirmatory tissue aspirate. One study (112 [5.5\%] patients)reported some patients receiving a VL diagnosis on clinical grounds due to occasional disruption to the supply of DAT kits\cite{ritmeijer2001}.

\subsubsection{Relapse}

Relapse was directly defined in two studies (288 [14.0\%] patients)\cite{ritmeijer2006,musa2010}, and could be inferred in the remaining studies through definitions of definite cure, initial cure, and treatment failure. All relapse cases were confirmed with a tissue aspirate, with aspirate results available in the contributed IPD. All studies adopted an active follow-up strategy following test-of-cure. However, loss-to-follow-up rates varied across studies, with particularly high rates seen in the earlier pragmatic MSF studies\cite{veeken2000,ritmeijer2006,ritmeijer2001}. Four studies (1,059 [51.6\%] patients)\cite{hailu2010,khalil2014,musa2010,musa2012} performed routine 6 month aspirates.

\subsubsection{Initial Cure}

Definitions of initial cure varied considerably across studies, and are presented in the \href{https://github.com/jpwil/dphil}{Supplementary Material}. In the six DND\textit{i} studies, test-of-cure occurred at variable time points within 31 days of treatment initiation, depending on the treatment arm (range 18--31 days)\cite{hailu2010,mbui2019,musa2010,musa2012,wasunna2016,khalil2014}. In the three MSF studies, treatment was extended with SSG until two consecutive tests-of-cure were negative\cite{veeken2000,ritmeijer2006,ritmeijer2001}. Routine aspirates were performed in all studies, although two studies explicitly stated that aspirates were only performed in the presence of palpable splenomegaly or lymphadenopathy (360 [17.6\%] patients)\cite{ritmeijer2006,ritmeijer2001}.

In five studies (1,197 [58.4\%] patients), those showing clinical improvement despite a \textit{positive} initial test-of-cure aspirate, regardless of parasite grade, were considered to have failed the outcome of initial cure\cite{wasunna2016,musa2012,musa2010,khalil2014,hailu2010}. However, these patients remained in the study and could subsequently achieve definite cure at 6 months if clinical symptoms did not recur and a subsequent (e.g. 3 or 6 month) aspirate were negative. These patients, sometimes referred to as `slow-responders', were included in model development despite not meeting the study definition of initial cure.

\subsection{Patient Characteristics\label{sec:pt-char}}

Overall (marginal) distributions of categorical and continuous variables are tabulated in Tables~\ref{tab:ea_categorical} and~\ref{tab:ea_continuous}, respectively. These distributions are also displayed graphically in Figure~\ref{fig:ea_cat_comb} for bar charts of categorical variables, and Figures~\ref{fig:ea_pooled_dist_cont1} and~\ref{fig:ea_pooled_dist_cont1} for histograms of non-laboratory and laboratory variables, respectively. Study-specific distributions are presented for age, sex, and relapse in Figure~\ref{fig:ea_main_dist}. In the Appendix, study-specific distributions of all categorical and continuous variables are presented in Figures~\ref{fig:ea_comb_dist_cat} and~\ref{fig:ea_age_comb}--\ref{fig:ea_cr_log_comb}, respectively.

Patient numbers and proportions presented in this section exclude missing data. See Section \ref{sec:ea_missing_data} for further information on missing data.

% these are the study specific distributions of outcome, sex and age
% cp /Users/jameswilson/proj/vl_model_ea/figures/dist/main_dist.pdf figures/ch6/ea_main_dist.pdf
\newgeometry{left=1cm, bottom=2.5cm, right=2cm, top=3cm}
\begin{landscape}
  \begin{figure}[tb]
    \centering
    \includegraphics[width=1.35\textwidth]{figures/ch6/ea_main_dist.pdf}
    \caption{Graphical summary of East Africa study-specific sample sizes and distributions of relapse status, sex, and age.}
    \label{fig:ea_main_dist}
  \end{figure}
\end{landscape}
\restoregeometry

\subsubsection{Categorical Variables\label{sec:ea_cat_var}}

Almost three quarters of patients (1,519 [74.1\%]) were male, ranging from 59.1\%\cite{veeken2000} to 100\%\cite{ritmeijer2006} at the study level.

Relapse within 6 months was reported in 99 (4.8\%) patients overall, and varying considerably at the study level from 0.9\%\cite{ritmeijer2001,veeken2000} to 20\%\cite{musa2010}.

As defined in Section~\ref{sec:malnutrition}, almost a quarter of all patients were severely malnourished (509 [24.9\% patients]). Moderate malnutrition affected 800 (39.1\%) patients, and 735 (36.0\%) had mild/normal malnutrition. Higher rates of severe malnutrition were seen in the two early MSF studies conducted in the late 1990s (33.1\%)\cite{veeken2000,ritmeijer2001} (presented in Appendix~Figure~\ref{fig:ea_comb_dist_cat}).

Severe anaemia affected 999 (48.8\%) of patients overall.

A variety of treatment regimens were investigated across the studies. At the patient level, the single most common drug used was SSG (either generic or branded) \textemdash\ used as monotherapy in 1,048 (51.1\%) patients, or as part of combination therapy with paromomycin or LAMB in 326 (15.9\%) and 47 (2.2\%) patients, respectively. The second most frequently used drug was paromomycin (received by 623 [30.3\%] of patients overall) followed by LAMB (185 [9.1\%] patients) and miltefosine (195 [9.5\%] patients). The full range of treatment regimens, including dosing, are presented in Figure~\ref{fig:ea_treat}.

Parasite grade ranged from 1$+$ to 6$+$, with 1$+$ being the most common (511 [36.8\%] patients who underwent tissue aspirate), followed by 2$+$ (237 [17.1\%] patients), and decreasing to 77 (5.5\%) patients with 6$+$. Tissue aspirate source was not reported in 702 (50.5\%) of patients overall. Where reported, 393 (29.1\%) were from the spleen, 163 (12.1\%) were from lymph nodes, and 131 (9.7\%) from bone marrow.

Approximately half of all patients were recruited from sites in Sudan (1,104 [53.8\%] patients), followed by Ethiopia (695 [27.8\%] patients), Kenya (227 [11.1\%]), and Uganda (25 [1.2\%] patients).

% categorical variables table
\begin{table}[htbp]
    \centering
    \small
    \begin{threeparttable}
        \begin{tabular}{@{} l R{1.3cm} @{\hspace{4pt}} L{1.3cm} R{1.3cm} @{\hspace{4pt}} L{1.3cm} R{0.89cm} @{\hspace{4pt}} L{1.3cm} @{}}
            \toprule
            \textbf{Variable}                 & \multicolumn{2}{c}{\textbf{Overall (\%)}} & \multicolumn{2}{c}{\textbf{Final cure (\%)}} & \multicolumn{2}{c}{\textbf{Relapse (\%)}}                        \\
                                              & \multicolumn{2}{c}{n~=~2,051}             & \multicolumn{2}{c}{n~=~1,952}                & \multicolumn{2}{c}{n~=~99}                                       \\
            \midrule
            \textbf{Sex}                      &                                           &                                              &                                           &        &    &        \\
            \hspace{1em} Female               & 532                                       & (25.9)                                       & 503                                       & (25.8) & 29 & (29.3) \\
            \hspace{1em} Male                 & 1,519                                     & (74.1)                                       & 1,449                                     & (74.2) & 70 & (70.7) \\
            \textbf{Malnutrition}             &                                           &                                              &                                           &        &    &        \\
            \hspace{1em}Normal/mild           & 735                                       & (35.8)                                       & 710                                       & (36.4) & 25 & (25.3) \\
            \hspace{1em}Moderate              & 800                                       & (39.0)                                       & 760                                       & (38.9) & 40 & (40.4) \\
            \hspace{1em}Severe                & 509                                       & (24.8)                                       & 476                                       & (24.4) & 33 & (33.3) \\
            \hspace{1em}(Missing)             & 7                                         & (0.3)                                        & 6                                         & (0.3)  & 1  & (1.0)  \\
            \textbf{Anaemia}                  &                                           &                                              &                                           &        &    &        \\
            \hspace{1em}Non-severe            & 1,049                                     & (51.1)                                       & 1,015                                     & (52.0) & 34 & (34.3) \\
            \hspace{1em}Severe                & 999                                       & (48.7)                                       & 934                                       & (47.8) & 65 & (65.7) \\
            \hspace{1em}(Missing)             & 3                                         & (0.1)                                        & 3                                         & (0.2)  & 0  & (0.0)  \\
            \textbf{Parasite grade}           &                                           &                                              &                                           &        &    &        \\
            \hspace{1em}1+                    & 511                                       & (24.9)                                       & 488                                       & (25.0) & 23 & (23.2) \\
            \hspace{1em}2+                    & 237                                       & (11.6)                                       & 219                                       & (11.2) & 18 & (18.2) \\
            \hspace{1em}3+                    & 192                                       & (9.4)                                        & 181                                       & (9.3)  & 11 & (11.1) \\
            \hspace{1em}4+                    & 179                                       & (8.7)                                        & 169                                       & (8.7)  & 10 & (10.1) \\
            \hspace{1em}5+                    & 193                                       & (9.4)                                        & 180                                       & (9.2)  & 13 & (13.1) \\
            \hspace{1em}6+                    & 77                                        & (3.8)                                        & 60                                        & (3.1)  & 17 & (17.2) \\
            \hspace{1em}(Missing)             & 662                                       & (32.3)                                       & 655                                       & (33.6) & 7  & (7.1)  \\
            \textbf{Aspirate source}\tnote{1} &                                           &                                              &                                           &        &    &        \\
            \hspace{1em}Bone                  & 131                                       & (9.4)                                        & 110                                       & (8.6)  & 21 & (22.8) \\
            \hspace{1em}Spleen                & 393                                       & (28.3)                                       & 369                                       & (28.9) & 24 & (26.1) \\
            \hspace{1em}Lymph node            & 163                                       & (11.7)                                       & 151                                       & (11.8) & 12 & (13.0) \\
            \hspace{1em}(Missing)             & 702                                       & (50.5)                                       & 649                                       & (50.7) & 35 & (38.0) \\
            \bottomrule
        \end{tabular}
        \begin{tablenotes}
            \footnotesize
            \item[1] Denominator for \% in aspirate source: number of patients with documented parasite grade (overall: 1,389; final cure: 1,279; relapse: 92).

        \end{tablenotes}
    \end{threeparttable}
    \caption[East Africa: categorical variables]{Summary of categorical candidate predictors and parasite source across contributed studies from East Africa. Missing data are described where present.}
    \label{tab:ea_categorical}
\end{table}

\subsubsection{Continuous Variables}

The median age was 14 years, ranging from 1 to 60 years (IQR: 9 to 22 years). Age distributions by study are presented in Figure~\ref{fig:ea_main_dist}, showing right-skewed distributions reflecting the study specific age inclusion criteria. 129 (6.3\%) of patients were under 5 years at the time of recruitment.

In adults the median BMI was 17.6~m/kg$^2$ (IQR: 16.3 to 18.7~m/kg$^2$). In children ($\geq$~5 and $<$~19 years), the median BMI-for-age z-score was -2.32 (IQR: -3.10 to -1.50), and in younger children ($<$~5 years) the median weight-for-height z-score was -2.36 (IQR: -3.12 to -1.58) (data available for 129 patients).

The median spleen size was 8~cm (IQR: 5 to 11~cm), and ranged from 0 to 30~cm. Where measured, 138 (6.9\%) patients had non-palpable spleens.

The median duration of fever prior to recruitment was 40 days (IQR: 25 to 91 days), and ranging from 3 to 761~days.

Distributions of laboratory results are presented in Table~\ref{tab:ea_continuous} and Figure~\ref{fig:ea_pooled_dist_cont1}.

A correlation matrix showing associations between continuous variables is presented in Appendix Figure~\ref{fig:ea_cont_cont}, and between continuous and categorical variables in Appendix Figure~\ref{fig:ea_cont_cat}. These correlations are considered further in the Discussion section.

% continuous variables table
% \begin{table}[ht]
%     \centering
%     \begin{tabular}{rllllllll}
%         \hline
%            & VARIABLE        & median\_overall                                         & range\_overall                       & missing\_overall & median\_nr                                              & missing\_nr  & median\_r                                               & missing\_r \\
%         \hline
%         1  & AGE             & 14 (9 -- 22)                                            & 0.6 -- 60                            & 1 (0.0)          & 15 (9 -- 22)                                            & 1 (0.1)      & 12 (9 -- 20)                                            & 0 (0.0)    \\
%         2  & HEIGHT          & 153 (126 -- 168)                                        & 68.8 -- 205                          & 9 (0.4)          & 154 (126 -- 168)                                        & 8 (0.4)      & 143 (127 -- 164)                                        & 1 (1.0)    \\
%         3  & WEIGHT          & 35 (21 -- 49)                                           & 5.6 -- 70.5                          & 1 (0.0)          & 35 (21 -- 49)                                           & 1 (0.1)      & 28 (20.6 -- 44.5)                                       & 0 (0.0)    \\
%         4  & BMI             & 17.5626352555497 (16.2946453729421 -- 18.6851211072664) & 12.1107266435986 -- 32.6388888888889 & 1,299 (63.3)     & 17.5382653061225 (16.2911048048411 -- 18.7109492465502) & 1,229 (63.0) & 17.6308539944904 (16.6493236212279 -- 18.2183224271267) & 70 (70.7)  \\
%         5  & BMIZ            & -2.32 (-3.1 -- -1.495)                                  & -8.84 -- 3.13                        & 888 (43.3)       & -2.3 (-3.09 -- -1.4725)                                 & 854 (43.8)   & -2.58 (-3.43 -- -1.89)                                  & 34 (34.3)  \\
%         6  & WFHZ            & -2.36 (-3.12 -- -1.58)                                  & -6.81 -- 0.49                        & 1,922 (93.7)     & -2.32 (-3.01 -- -1.55)                                  & 1,827 (93.6) & -3.415 (-3.8925 -- -3.24)                               & 95 (96.0)  \\
%         7  & SPLEEN\_LENGTH  & 8 (5 -- 11)                                             & 0 -- 30                              & 69 (3.4)         & 8 (5 -- 11)                                             & 69 (3.5)     & 7 (4 -- 10)                                             & 0 (0.0)    \\
%         8  & FEVER\_DURATION & 40.44 (25 -- 91.32)                                     & 3 -- 761                             & 349 (17.0)       & 40.44 (25 -- 91.32)                                     & 320 (16.4)   & 30.44 (20.25 -- 60.88)                                  & 29 (29.3)  \\
%         9  & PARASITE        & 2 (1 -- 4)                                              & 1 -- 6                               & 662 (32.3)       & 2 (1 -- 4)                                              & 655 (33.6)   & 3 (1.75 -- 5)                                           & 7 (7.1)    \\
%         10 & LAB\_WBC        & 2.5 (1.8 -- 3.5)                                        & 0.6 -- 16.8                          & 887 (43.2)       & 2.5 (1.8 -- 3.5)                                        & 871 (44.6)   & 2.7 (1.84 -- 3.55)                                      & 16 (16.2)  \\
%         11 & LAB\_PLT        & 106 (73 -- 157)                                         & 5 -- 881                             & 890 (43.4)       & 105 (73 -- 155)                                         & 874 (44.8)   & 110 (74.5 -- 171)                                       & 16 (16.2)  \\
%         12 & LAB\_HGB        & 79 (67 -- 91)                                           & 22 -- 167                            & 2 (0.1)          & 80 (67 -- 92)                                           & 2 (0.1)      & 72 (62 -- 83.5)                                         & 0 (0.0)    \\
%         13 & LAB\_ALT        & 21 (14 -- 31)                                           & 2 -- 134                             & 920 (44.9)       & 21 (14 -- 31)                                           & 901 (46.2)   & 19 (14 -- 30.25)                                        & 19 (19.2)  \\
%         14 & LAB\_CREAT      & 44.2 (0.6 -- 79)                                        & 0.1 -- 134                           & 826 (40.3)       & 44.2 (0.6 -- 78.94)                                     & 815 (41.8)   & 44.2 (0.675 -- 79.5603825)                              & 11 (11.1)  \\
%         \hline
%     \end{tabular}
% \end{table}

\begin{landscape}
  \begin{table}[htbp]
    \centering
    \small
    \begin{threeparttable}
      \begin{tabular}{@{} l @{} r @{\hspace{4pt}} l @{} r @{\hspace{4pt}} l @{} r @{\hspace{4pt}} l @{} r @{\hspace{4pt}} l @{} r @{\hspace{4pt}} l @{} r @{\hspace{4pt}} l @{}}
        \toprule
        \textbf{Variable}           & \multicolumn{4}{@{}c@{}}{\textbf{Overall} (n = 2,051)} & \multicolumn{4}{@{}c@{}}{\textbf{Final cure } (n = 1,952)} & \multicolumn{4}{@{}c@{}}{\textbf{Relapse} (n = 99)}                                                                                                                        \\
        \cmidrule(r){2-5}\cmidrule(lr){6-9}\cmidrule(l){10-13}
                                    & Median                                                 & (IQR)                                                      & Missing\tnote{1}                                    & (\%)   & Median & (IQR)            & Missing & (\%)   & Median & (IQR)            & Missing\footnotemark[1] & (\%)   \\
        \midrule
        Age (years)                 & 14                                                     & (9 -- 22)                                                  & 1                                                   & (0.0)  & 15     & (9 -- 22)        & 1       & (0.1)  & 12     & (9 -- 20)        & 0                       & (0.0)  \\
        Height (cm)                 & 153                                                    & (126 -- 168)                                               & 9                                                   & (0.4)  & 154    & (126 -- 168)     & 8       & (0.4)  & 143    & (127 -- 164)     & 1                       & (1.0)  \\
        Weight (kg)                 & 35                                                     & (21 -- 49)                                                 & 1                                                   & (0.0)  & 35     & (21 -- 49)       & 1       & (0.1)  & 28     & (20.6 -- 44.5)   & 0                       & (0.0)  \\
        BMI (kg/m$^2$)\tnote{2}     & 17.56                                                  & (16.29 -- 18.69)                                           & 4                                                   & (0.5)  & 17.54  & (16.29 -- 18.71) & 3       & (0.4)  & 17.63  & (16.65 -- 18.22) & 1                       & (3.2)  \\
        BMI--FA z--score\tnote{3}   & -2.32                                                  & (-3.10 -- -1.50)                                           & 2                                                   & (0.2)  & -2.30  & (-3.09 -- -1.47) & 2       & (0.2)  & -2.58  & (-3.43 -- -1.89) & 0                       & (0.0)  \\
        WFH z--score\tnote{4}       & -2.36                                                  & (-3.12 -- -1.58)                                           & 0                                                   & (0.0)  & -2.32  & (-3.01 -- -1.55) & 0       & (0.0)  & -3.42  & (-3.89 -- -3.24) & 0                       & (0.0)  \\
        Spleen size (cm)            & 8                                                      & (5 -- 11)                                                  & 69                                                  & (3.4)  & 8      & (5 -- 11)        & 69      & (3.5)  & 7      & (4 -- 10)        & 0                       & (0.0)  \\
        Fever duration (days)       & 40.4                                                   & (25.0 -- 91.3)                                             & 349                                                 & (17.0) & 40.4   & (25.0 -- 91.3)   & 320     & (16.4) & 30.4   & (20.3 -- 60.9)   & 29                      & (29.3) \\
        Parasite grade              & 2                                                      & (1 -- 4)                                                   & 662                                                 & (32.3) & 2      & (1 -- 4)         & 655     & (33.6) & 3      & (2 -- 5)         & 7                       & (7.1)  \\
        WBC ($\times 10^9$/L)       & 2.5                                                    & (1.8 -- 3.5)                                               & 887                                                 & (43.2) & 2.5    & (1.8 -- 3.5)     & 871     & (44.6) & 2.7    & (1.8 -- 3.6)     & 16                      & (16.2) \\
        Platelets ($\times 10^9$/L) & 106                                                    & (73 -- 157)                                                & 890                                                 & (43.4) & 105    & (73 -- 155)      & 874     & (44.8) & 110    & (74.5 -- 171)    & 16                      & (16.2) \\
        Haemoglobin (g/L)           & 79                                                     & (67 -- 91)                                                 & 2                                                   & (0.1)  & 80     & (67 -- 92)       & 2       & (0.1)  & 72     & (62 -- 83.5)     & 0                       & (0.0)  \\
        ALT (IU/L)                  & 21                                                     & (14 -- 31)                                                 & 920                                                 & (44.9) & 21     & (14 -- 31)       & 901     & (46.2) & 19     & (14 -- 30)       & 19                      & (19.2) \\
        Creatinine ($\mu$mol/L)     & 44.2                                                   & (0.6 -- 79.0)                                              & 826                                                 & (40.3) & 44.2   & (0.6 -- 78.9)    & 815     & (41.8) & 44.2   & (0.7 -- 79.6)    & 11                      & (11.1) \\
        \bottomrule
      \end{tabular}
      \begin{tablenotes}
        \footnotesize
        \item[1] Denominator for missing \%: total number of patients in respective group (overall, relapse or final cure). For measures of malnutrition (BMI, BMI-for-age z--score, and weight-for-height z--score), see below.
        \item[2] Denominator for missing \%: number of patients aged $\geq$ 19 years, n = 756 (relapse: 30, final cure: 726).
        \item[3] Denominator for missing \%: number of patients aged 5--18 year inclusive, n = 1,165 (relapse: 65, final cure: 1,100).
        \item[4] Denominator for missing \%: number of patients aged $<$ 5 years, n = 129 (relapse: 4, final cure: 125).
      \end{tablenotes}
    \end{threeparttable}
    \caption{Summary of continuous candidate predictors across contributed studies from East Africa. Including additional variables used for the derivation of malnutrition status (height, weight, BMI, BMI-for-age z--score, weight-for-height z--score). ALT: alanine aminotransferase; BMI(-FA): body mass index(-for age); cm: centimetres; IQR: inter-quartile range, IU: international units; kg: kilograms; L: litres; m: metres; WBC: white blood cells; WFH: weight-for-height; g: grams; $\mu$mol: micromoles.}
    \label{tab:ea_continuous}
  \end{table}
\end{landscape}

% pooled distributions of categorical variables
% cp /Users/jameswilson/proj/vl_model_ea/figures/dist/catOut/comb_cat.pdf figures/ch6/ea_cat_comb.pdf
\newgeometry{left=1cm, bottom=2.5cm, right=2cm, top=3cm}
\begin{landscape}
  \begin{figure}[tb]
    \centering
    \includegraphics[width=1.35\textwidth]{figures/ch6/ea_cat_comb.pdf}
    \caption{Marginal distributions and predictor--outcomes relationships for categorical candidate predictors. Excluding missing data. 95\% binomial confidence intervals calculated using the Wilson method. Note: for parasite grade relapse \%, $y$--axis is rescaled to accommodate increased risk in 6+ group. Norm: normal.}
    \label{fig:ea_cat_comb}
  \end{figure}
\end{landscape}
\restoregeometry

\subsection{Univariable Associations}

Unadjusted relationships between variables (including all candidate predictors, excluding missing data) and relapse risk are presented both in tabular form (Tables~\ref{tab:ea_categorical}, \ref{tab:ea_continuous}) and visually alongside their distributions (Figure~\ref{fig:ea_cat_comb} for categorical variables, and Figures~\ref{fig:ea_pooled_dist_cont1} and \ref{fig:ea_pooled_dist_cont2} for non-laboratory and laboratory variables, respectively). Predictor-outcome relationships are presented on the log-odds (logit) scale for continuous candidate predictors in Appendix Figure~\ref{fig:ea_logodds}.

On review of the associations between the categorical predictors and relapse risk (upper facet row, Figure~\ref{fig:ea_cat_comb}), clear trends were seen for malnutrition, anaemia, and parasite grade. Increased relapse risk was seen with higher malnutrition severity, ranging from approximately 3.5\% for mild/normal to 6.5\% for severe categories. Perhaps more marked, with non-overlying confidence intervals, was the increased relapse risk seen in the severe anaemia group compared to the non-severe anaemia group, from approximately 3\% to 6.5\%. For relapse cases, 65/99 (65.7\%) had severe anaemia, compared to 476/1946 (24.5\%) of final cure cases (excluding missing data).

With parasite grade, a substantial increase in relapse risk was seen for patients with a 6$+$ grade compared to <~6$+$ grades, increasing from approximately 5--7.5\% to over 20\%. Among relapse cases, 17/92 (18.5\%) of patients had a 6$+$ grade aspirate, compared to 60/1297 (4.6\%) of the final cure cases (after excluding missing data).

Clear trends were observed between all non-laboratory continuous variables and relapse risk (Figure~\ref{fig:ea_pooled_dist_cont1}). For age, relapse risk followed a non-linear trend, peaking at approximately 10~years and decreasing to a nadir in the early 20s. Fluctuations in risk with height and weight were also noted, mirroring the relationship between age and relapse. Less marked downward trends between relapse risk and spleen size and fever duration were observed.

Among the laboratory continuous variables (Figure~\ref{fig:ea_pooled_dist_cont2}), a pronounced inverse relationship between relapse risk and haemoglobin is seen, concordant with the aforementioned increased risk seen in patients with severe vs. non-severe anaemia. Appreciable upwards trends between relapse risk and both white blood cell count and platelets are also seen, although the overall significance is unclear.

A correlation matrix showing associations between continuous variables is presented in Appendix Figure~\ref{fig:ea_cont_cont}, and between continuous and categorical variables in Appendix Figure~\ref{fig:ea_cont_cat}. These correlations are considered further in the Discussion section.

% these are the pooled continuous distributions and relationships with relapse - PART 1
% cp /Users/jameswilson/proj/vl_model_ea/figures/dist/contOut/age_comb.pdf figures/ch6/ea_pool_age_comb.pdf
% cp /Users/jameswilson/proj/vl_model_ea/figures/dist/contOut/ss_comb.pdf figures/ch6/ea_pool_ss_comb.pdf
% cp /Users/jameswilson/proj/vl_model_ea/figures/dist/contOut/fd_comb.pdf figures/ch6/ea_pool_fd_comb.pdf
% cp /Users/jameswilson/proj/vl_model_ea/figures/dist/contOut/height_comb.pdf figures/ch6/ea_pool_height_comb.pdf
% cp /Users/jameswilson/proj/vl_model_ea/figures/dist/contOut/weight_comb.pdf figures/ch6/ea_pool_weight_comb.pdf

\clearpage
\begin{figure}[H]
  \centering
  \begin{subfigure}{\textwidth}
    \centering
    \begin{overpic}[width=\textwidth]{figures/ch6/ea_pool_age_comb.pdf}
      \put(2,19){\small Age}
    \end{overpic}
  \end{subfigure}
  \begin{subfigure}{\textwidth}
    \centering
    \begin{overpic}[width=\textwidth]{figures/ch6/ea_pool_weight_comb.pdf}
      \put(2,19){\small Weight}
    \end{overpic}
  \end{subfigure}
  \begin{subfigure}{\textwidth}
    \centering
    \begin{overpic}[width=\textwidth]{figures/ch6/ea_pool_height_comb.pdf}
      \put(2,19){\small Height}
    \end{overpic}
  \end{subfigure}
  \begin{subfigure}{\textwidth}
    \centering
    \begin{overpic}[width=\textwidth]{figures/ch6/ea_pool_fd_comb.pdf}
      \put(2,19){\small FevDur}
    \end{overpic}
  \end{subfigure}
  \begin{subfigure}{\textwidth}
    \centering
    \begin{overpic}[width=\textwidth]{figures/ch6/ea_pool_ss_comb.pdf}
      \put(2,19){\small SpnSize}
    \end{overpic}
  \end{subfigure}
  \caption{Marginal distributions and predictor--outcome relationships for continuous non-laboratory candidate predictors. FevDur:~duration of fever; SpnSize:~spleen size. For each candidate predictor, left upper panel shows the overall density across studies and the left lower panel shows overlapping densities normalised by relapse status. The right panel shows a univariable generalised additive model spline fit, with 95\% confidence interval, of relapse.}
  \label{fig:ea_pooled_dist_cont1}
\end{figure}

% these are the pooled continuous distributions and relationships with relapse - PART 2
% cp /Users/jameswilson/proj/vl_model_ea/figures/dist/contOut/wbc_comb.pdf figures/ch6/ea_pool_wbc_comb.pdf
% cp /Users/jameswilson/proj/vl_model_ea/figures/dist/contOut/plt_comb.pdf figures/ch6/ea_pool_plt_comb.pdf
% cp /Users/jameswilson/proj/vl_model_ea/figures/dist/contOut/hb_comb.pdf figures/ch6/ea_pool_hb_comb.pdf
% cp /Users/jameswilson/proj/vl_model_ea/figures/dist/contOut/alt_comb.pdf figures/ch6/ea_pool_alt_comb.pdf
% cp /Users/jameswilson/proj/vl_model_ea/figures/dist/contOut/cr_comb.pdf figures/ch6/ea_pool_cr_comb.pdf
\begin{figure}[H]
  \centering
  \begin{subfigure}{\textwidth}
    \centering
    \begin{overpic}[width=\textwidth]{figures/ch6/ea_pool_hb_comb.pdf}
      \put(2,19){\small Hb}
    \end{overpic}
  \end{subfigure}
  \begin{subfigure}{\textwidth}
    \centering
    \begin{overpic}[width=\textwidth]{figures/ch6/ea_pool_plt_comb.pdf}
      \put(2,19){\small Plt}
    \end{overpic}
  \end{subfigure}
  \begin{subfigure}{\textwidth}
    \centering
    \begin{overpic}[width=\textwidth]{figures/ch6/ea_pool_wbc_comb.pdf}
      \put(2,19){\small WBC}
    \end{overpic}
  \end{subfigure}
  \begin{subfigure}{\textwidth}
    \centering
    \begin{overpic}[width=\textwidth]{figures/ch6/ea_pool_alt_comb.pdf}
      \put(2,19){\small ALT}
    \end{overpic}
  \end{subfigure}
  \begin{subfigure}{\textwidth}
    \centering
    \begin{overpic}[width=\textwidth]{figures/ch6/ea_pool_cr_comb.pdf}
      \put(2,19){\small Crt}
    \end{overpic}
  \end{subfigure}
  \caption{Marginal distributions and predictor--outcome relationships for continuous laboratory candidate predictors. All predictors presented on log scale. Hb:~haemoglobin; Plt:~platelet; WBC:~white blood cells; ALT:~alanine aminotransferase; Crt:~creatinine. For each candidate predictor, left upper panel shows the overall density across studies and the left lower panel shows overlapping densities normalised by relapse status. The right panel shows a univariable generalised additive model spline fit, with 95\% confidence interval, of relapse.}
  \label{fig:ea_pooled_dist_cont2}
\end{figure}

\subsection{\label{sec:ea_missing_data}Missing Data}

Missingness was common, and affecting one or more candidate predictors in 1,168 (56.9\%) patients. 215 (10.5\%) patients were missing 1 candidate predictor exactly. As appreciated from Figure~\ref{fig:ea_missing_summary}, the majority of missingness originated from the earlier MSF studies, where ALT, platelets, WBC, creatinine, and parasite grade were not routinely collected, resulting in almost a third of all patients (634 [30.9\%] patients) missing 5 candidate predictors.

Across all studies, ALT, platelets, WBC, and creatinine were missing in between 40 and 45\% of all patients. Parasite grade was missing for approximately one third of the total cohort (662 [32.3\%] of patients).

\subsubsection{Multiple Imputation}

Diagnostic plots for the 30 imputed datasets are available in the \href{https://github.com/jpwil/dphil}{Supplementary Material}. On review of the trace plots, there was good evidence for convergence of the imputation model. Density and scatter plots revealed overall good correlation between the imputed and original variables.

% missing data figure
% cp /Users/jameswilson/proj/vl_model_ea/figures/missing/summary.pdf figures/ch6/ea_missing_summary.pdf
\begin{figure}[tb]
  \centering
  \includegraphics[scale = 0.8, trim={0cm 3.8cm 0cm 3.3cm}, clip]{figures/ch6/ea_missing_summary.pdf}
  \caption{Density plot illustrating the amount of missing data overall and across contributing studies from East Africa. Study ordered by lead author and publication year. Variables ordered by amount of missingness. ALT: alanine aminotransferase; WBC: white blood cells.}
  \label{fig:ea_missing_summary}
\end{figure}

\section{Model Results}

East African IPD were used to develop two separate prediction models; one including parasite grade as a candidate predictor, and one exclude parasite grade.

\subsection{Model Specification and Coefficient Estimates}

Final (retained) predictors across both models are presented in Figure~\ref{fig:ea_var_forest_combined}. Full specification of the final models, including intercept terms, p-values, and predictor transformations, are presented in Appendix Tables~\ref{tab:ea_model_coeff_with_pg} and \ref{tab:ea_model_coeff_without_pg}.


% cp /Users/jameswilson/proj/vl_model_ea/results/var_forest_combined.pdf figures/ch6/var_forest_combined.pdf
\begin{figure}[tb]
  \centering
  \includegraphics[width=\textwidth]{figures/ch6/var_forest_combined.pdf}
  \caption{Forest plot of adjusted odds ratios with 95\% confidence intervals for final model predictors. Odds ratios are displayed on a logarithmic scale.}
  \label{fig:ea_var_forest_combined}
\end{figure}


% cp /Users/jameswilson/proj/vl_model_ea/figures/multiAssocM1.pdf figures/ch6/isc_multiassoc_with_pg.pdf
% cp /Users/jameswilson/proj/vl_model_ea/figures/multiAssocM2.pdf figures/ch6/isc_multiassoc_without_pg.pdf
\newgeometry{left=2.5cm, bottom=2.3cm, right=2cm, top=2.8cm}
\begin{landscape}
  \begin{figure}[tb]
    \centering
    \begin{subfigure}{1.5\textwidth}
      \centering
      \begin{overpic}[width=\textwidth]{figures/ch6/isc_multiassoc_without_pg.pdf}
        \put(2,25.5){\small\textbf{East Africa model: without parasite grade}}
      \end{overpic}
    \end{subfigure}
    \begin{subfigure}{1.5\textwidth}
      \centering
      \begin{overpic}[width=\textwidth]{figures/ch6/isc_multiassoc_with_pg.pdf}
        \put(2,22.4){\small\textbf{East Africa model: with parasite grade}}
      \end{overpic}
    \end{subfigure}
    \caption{Adjusted associations between final predictors and predicted relapse probability, as estimated from the final East Africa prognostic models. Probabilities were calculated from optimism--adjusted models and following logistic recalibration (intercept--term only) to data contributed from Musa 2012\cite{musa2012}. Where not varying in the plot, predictions are standardised to a representative reference participant: median age (13 years), median fever duration (45 days), median white cell count (2.5 x10$^9$/L), with severe anaemia, moderate malnutrition, and \textemdash\ for the model including parasite grade \textemdash\ a median parasite count of 2+.}
    \label{fig:ea_adjusted_assoc}
  \end{figure}
\end{landscape}
\restoregeometry

\subsection{Model Performance and Internal Validation}


\begin{table}[htbp]
  \begin{tabular}{@{}llll@{}}
    \toprule
                                   & Performance        & Average  & Optimism--adjusted \\
                                   & estimate (95\% CI) & optimism & performance        \\        \midrule
    \textbf{Model: with PG}        &                    &          &                    \\
    \hspace{1em} C-statistic       & 0.72 (0.63--0.82)  & 0.046    & 0.68               \\
    \hspace{1em} Calibration slope & 1.03 (0.64--1.42)  & 0.224    & 0.81               \\        \midrule
    \textbf{Model: without PG}     &                    &          &                    \\
    \hspace{1em} C-statistic       & 0.66 (0.56--0.76)  & 0.050    & 0.61               \\
    \hspace{1em} Calibration slope & 1.08 (0.47--1.68)  & 0.329    & 0.75               \\        \bottomrule
  \end{tabular}
  \caption{Apparent and optimism--adjusted performance measures. Abbreviations: CI: confidence interval; PG: parasite grade}
  \label{tab:ea-performance}
\end{table}


% cp /Users/jameswilson/proj/vl_model_ea/graphs/forestCIM1.pdf figures/ch6/forestCIM1.pdf
% forest plots for C-statistic
\begin{figure}[tb]
  \centering
  \begin{overpic}[width=\textwidth, trim={3.5cm 0cm 1cm 0.3cm}, clip]{figures/ch6/forestCIM1.pdf}
    \put(0,5.5){\fcolorbox{black}{white}{\scriptsize $\tau^2$ = 0.0040; $I^2$ = 46.6\%; p = 0.11}}
  \end{overpic}
  \caption{Forest plot showing individual and pooled study c--statistics, for the model \textbf{including} parasite grade. For the model excluding parasite grade refer to Appendix \ref{fig:ea_forestCIM2}. Pooled c--statistics are presented from both fixed--effects and random--effects meta-analysis models, after excluding studies with $\leq$ 5 relapse events. Study--specific confidence intervals with few events should be interpreted with caution (as discussed in the Methodology \ref{meth:discrimination}). Blue diamonds: pooled summary estimates with 95\% confidence intervals. For Mbui 2019, all bootstrapped c--statistic estimates were 1 exactly and therefore no variance is presented.}
  \label{fig:ea_forestCIM1}
\end{figure}

% cp /Users/jameswilson/proj/vl_model_ea/graphs/forestCalM1.pdf figures/ch6/forestCalM1.pdf
% forest plots for calibration
\begin{figure}[tb]
  \centering
  \begin{overpic}[width=\textwidth, trim={2.4cm 0cm 0.3cm 0.3cm}, clip]{figures/ch6/forestCalM1.pdf}
    \put(0,12){\fcolorbox{black}{white}{\tiny CITL: $\tau^2$ = 0.960; $I^2$ = 89.0\%; p < .0001}}
    \put(0,8.85){\fcolorbox{black}{white}{\tiny CS: $\tau^2$ = 0; $I^2$ = 0\%; p = 0.90}}
  \end{overpic}
  \caption{Forest plots showing individual and pooled study calibration measures for the model \textbf{including} parasite grade. Left: calibration intercept (calibration--in--the--large, CITL); Right: calibration slope (CS). Blue diamonds: summary estimates with 95\% confidence intervals. Calibration slope for Mbui 2019 not shown due to small sample size causing failure of model convergence.}
  \label{fig:ea_forestCalM1}
\end{figure}

% calibration plots (overall)
% cp /Users/jameswilson/proj/vl_model_ea/graphs/calPlot.pdf figures/ch6/ea_calPlot.pdf
\begin{figure}[tb]
  \centering
  \begin{overpic}[width=\textwidth, trim = 0 0 0 -25]{figures/ch6/ea_calPlot.pdf}
    \put(-2,52){\small\textbf{East Africa model: with parasite grade}}
    \put(54,52){\small\textbf{East Africa model: without parasite grade}}
  \end{overpic}
  \caption{Calibration plots showing observed versus predicted probabilities for deciles of predicted probability. Red dashed line represents perfect calibration.  Observed probabilities are presented with 95\% confidence intervals (black error bars). A generalised additive model is fitted to show the smoothed mean observed probability (blue dotted line) with 95\% confidence intervals (blue ribbon). Histograms, normalised by outcome, are overlaid to illustrate the distribution of relapses and cures across the expected probabilities.}
  \label{fig:ea_calPlot}
\end{figure}

% calibration plots (malnutrition)
% cp /Users/jameswilson/proj/vl_model_ea/graphs/calPlotMal1.pdf figures/ch6/ea_calPlotMal1.pdf
\begin{figure}[tb]
  \centering
  \includegraphics[width=\textwidth]{figures/ch6/ea_calPlotMal1.pdf}
  \caption{Calibration plots for malnutrition severity (model including parasite grade).}
  \label{fig:ea_calPlotMal1}
\end{figure}

\section{Summary}