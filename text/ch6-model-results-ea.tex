\begin{savequote}[8cm]
  Quote goes here.
  \qauthor{--- James Wilson}
\end{savequote}

\chapter{\label{ch:6-ea-model-results}East Africa model results}
\minitoc

\section{Descriptive analysis}

% flow diagram
\begin{figure}[tb]
  \centering
  \includegraphics[width = 0.8\textwidth, trim={4cm 0cm 2cm 2cm}, clip]{figures/ch6/ea_flow_chart.pdf}
  \caption{Flow diagram showing the studies and participants excluded from East Africa model development, following application of the eligibility criteria. HIV: human immunodeficiency virus; IDDO: Infectious Diseases Data Observatory; EA: East Africa; PKDL: post kala-azar dermal leishmaniasis; VL: visceral leishmaniasis.}
  \label{fig:ea_flow_diagram}
\end{figure}

% studies tables
% latex table generated in R 4.4.1 by xtable 1.8-4 package
% Wed Dec 10 17:36:05 2025

{
\newgeometry{left=1.5cm, right=1.5cm, top=2cm, bottom=1.5cm}
\captionsetup{width=1\textheight, font=scriptsize, skip=7pt}

\begin{landscape}
  % \singlespacing
  \pagestyle{empty}
  \scriptsize
  \begin{ThreePartTable}
    \begin{TableNotes}
      \item[1] Study name is composed of the lead author and year of publication.
      \item[2] Refer to the supplementary material of the publication for allometric dosing table\cite{mbui2019}.
      \item[3] SSG tested and dispensed by International Dispensary Association, The Netherlands; manufactored by Albert David Ltd, India.
      \item[4] For both arms in Veeken 2010, Ritmeijer 2001, and Ritmeijer 2006, if positive initial test--of--cure, treatment would continue with SSG, including if previously taking MF, until two subsequent consecutive tests of cure, performed weekly, were negative.
      \item[5] Actual dosing ranged from 2.0--3.33 mg/kg/day after rounding to nearest 10mg tablets; full regimen described in publication\cite{wasunna2016}
    \end{TableNotes}

    \begin{longtable}[c]{L{1cm} L{4cm} L{1cm} L{1.5cm} L{3.5cm} L{2.5cm} L{3.7cm} L{1cm} L{1cm} L{1cm} L{1.3cm}}
      \caption{Key characteristics of included studies from East Africa, ordered by lead author and year of publication/protocol. -: not reported; BD: twice daily (bis die); D: Day(s); IM: intramuscular; IV: intravenous; LAMB: Liposomal amphotericin B (Gilead formulation); MF: Miltefosine; mg/kg: milligrams per kilogram; OD: once daily (omni die); PO: per os (oral); PM: Paromomycin; SSG: Sodium stibogluconate; \label{tab:ea_study}}                                                                                                                                                                                                                                                                                                                                                                                                               \\
      \toprule
      Study\tnote{1}                     & Title                                                                                                                                                                                       & Journal                  & Sponsor/ funding            & Location(s)                                                                                                                                                                        & Study design                                                                        & Study arm(s)\tnote{4}                                                                                                                                                                                                       & Age (years) & Study period & n (model) & Relapses (\%) \\ \midrule
      \endfirsthead

      \caption[]{continued}                                                                                                                                                                                                                                                                                                                                                                                                                                                                                                                                                                                                                                                                                                                                                                                                                                       \\
      \toprule
      Study\tnote{1}                     & Title                                                                                                                                                                                       & Journal                  & Sponsor/ funding            & Location(s)                                                                                                                                                                        & Study design                                                                        & Study arm(s)\tnote{4}                                                                                                                                                                                                       & Age (years) & Study period & n (model) & Relapses (\%) \\ \midrule
      \endhead

      \multicolumn{11}{r}{\textit{continued on next page}}                                                                                                                                                                                                                                                                                                                                                                                                                                                                                                                                                                                                                                                                                                                                                                                                        \\
      \endfoot

      \insertTableNotes
      \endlastfoot

      Hailu 2010\cite{hailu2010}         & Geographical variation in the response of visceral leishmaniasis to paromomycin in East Africa: a multicentre, open-label, randomized trial                                                 & PLoS Negl Trop Dis       & DNDi/MSF                    & 5 centres: Ethiopia (Gondar University Hospital, Arba Minch Hospital); Sudan (MSF Treatment Centre, Um el Khjer; Ministry of Health Hospital, Kassab); Kenya (CCR, KEMRI, Nairobi) & Multicentre, open-label, randomised trial                                           & (1) SSG, 20 mg/kg, IV or IM, OD, 30 D; (2) PM, 15 mg/kg, IM, OD, 21 D; (3) Combination of PM and SSG, same dose, frequency and route, 17 D                                                                                  & 4--60       & 2004--2008   & 289       & 21 (7.3)      \\ \midrule
      Khalil 2014\cite{khalil2014}       & Safety and efficacy of single dose versus multiple doses of AmBisome for treatment of visceral leishmaniasis in eastern Africa: a randomised trial                                          & PLoS Negl Trop Dis       & DNDi (multiple funders)     & 3 centres: Ethiopia (Gondar University Hospital, Arba Minch Hospital); Sudan (Ministry of Health Hospital, Kassab)                                                                 & Multicentre, open-label, non-inferiority, randomised trial with adaptive design     & LAMB, IV, either single dose at (1) 7.5mg/kg; (2) 10 mg/kg; (3) 12.5 mg/kg; (4) 15 mg/kg, or multiple dose at (5) 3 mg/kg, OD, D1-5, 14, 21 (total 21 mg/kg)                                                                & $\geq$ 4    & 2009--2011   & 92        & 11 (12.0)     \\ \midrule
      Mbui 2019\cite{mbui2019}           & Pharmacokinetics, Safety, and Efficacy of an Allometric Miltefosine Regimen for the Treatment of Visceral Leishmaniasis in Eastern African Children: An Open-label, Phase II Clinical Trial & Clin Infect Dis          & DNDi (multiple funders)     & 2 clinical sites: Kachelibra, West Pokot County, Kenya; Amudat, Karamoja sub-region, Uganda                                                                                        & Open-label, phase II, clinical trial                                                & (1) MF, allometric dosing according to sex, height, and weight, BD, PO, 28 D\tnote{2}                                                                                                                                       & 4--12       & 2015--2016   & 29        & 2 (6.9)       \\ \midrule
      Musa 2010\cite{musa2010}           & Paromomycin for the treatment of visceral leishmaniasis in Sudan: a randomized, open-label, dose-finding study                                                                              & PLoS Negl Trop Dis       & DNDi (multiple funders)     & Ministry of Health Hospital, Kassab, Sudan                                                                                                                                         & Open-label, open-label, phase II, randomised, dose-finding study                    & (1) PM, 15 mg/kg, IM, OD, 28 D; (2) PM, 20 mg/kg, IM, OD, 21 D                                                                                                                                                              & 4--60       & 2005--2006   & 40        & 8 (20.0)      \\ \midrule
      Musa 2012\cite{musa2012}           & Sodium stibogluconate (SSG) \& paromomycin combination compared to SSG for visceral leishmaniasis in East Africa: a randomised controlled trial                                             & PLoS Negl Trop Dis       & DNDi/MSF (multiple funders) & 6 centres: 5 centres described in Khalil 2014 (above) and Amudat Hospital, Uganda.                                                                                                 & Multicentre, open-label, parallel-arm, randomised trial                             & (1) SSG, 20 mg/kg, IV or IM, OD, 30 D; (2) PM, 20 mg/kg, IM, OD, 21 D; (3) combination of SSG, 20 mg/kg, IV or IM, OD, 17 D and PM, 15 mg/kg, IM, OD, 17 D                                                                  & 4--60       & 2004--2010   & 638       & 30 (4.7\%)    \\ \midrule
      Ritmeijer 2001\cite{ritmeijer2001} & Ethiopian visceral leishmaniasis: generic and proprietary sodium stibogluconate are equivalent; HIV co-infected patients have a poor outcome                                                & Trans R Soc Trop Med Hyg & MSF                         & Temporary MSF treatment centre, Densha, Ethiopia                                                                                                                                   & Open-label, pseudo-randomised controlled trial                                      & (1) SSG (generic)\tnote{3}, 20 mg/kg, IM, 30 D; (2) SSG (Pentostam, GlaxoWellcome), 20 mg/kg, IM, 30 D                                                                                                                      & all         & 1998--1999   & 112       & 1 (0.9)       \\ \midrule
      Ritmeijer 2006\cite{ritmeijer2006} & A comparison of miltefosine and sodium stibogluconate for treatment of visceral leishmaniasis in an Ethiopian population with high prevalence of HIV infection                              & Clin Infect Dis          & MSF                         & 2 centres in Ethiopia: Humera Hospital, Mycadra Health Center                                                                                                                      & Open-label, randomised controlled trial                                             & (1) SSG, 20 mg/kg/day, IM, OD, 30 D (extended if HIV positive); (2) MF, 100 mg/day, PO, OD, 28 D                                                                                                                            & $\geq$ 15   & 2003--2005   & 248       & 6 (2.4)       \\ \midrule
      Veeken 2000\cite{veeken2000}       & A randomized comparison of branded sodium stibogluconate and generic sodium stibogluconate for the treatment of visceral leishmaniasis under field conditions in Sudan                      & Trop Med Int Health      & MSF                         & 2 MSF treatment centres in Gedaref State, Sudan: Um Kuraa, Kassab                                                                                                                  & Open-label, pseudo-randomised controlled trial                                      & (1) SSG (generic)\tnote{3}, 20 mg/kg, IM, 30 D; (2) SSG (Pentostam, GlaxoWellcome), 20 mg/kg, IM, 30 D                                                                                                                      & all         & 1998--1999   & 465       & 4 (0.9)       \\ \midrule
      Wasunna 2016\cite{wasunna2016}     & Efficacy and Safety of AmBisome in Combination with Sodium Stibogluconate or Miltefosine and Miltefosine Monotherapy for African Visceral Leishmaniasis: Phase II Randomized Trial.         & PLoS Negl Trop Dis       & DNDi (multiple funders)     & 3 centres: Kenya (Kimalel Health Centre); Sudan (Dooka Hospital and Ministry of Health Hospital, Kassab)                                                                           & Phase II, open-label, non-comparative randomised trial (adaptive-sequential design) & (1) Combination of LAMB, 10 mg/kg, IV, single dose, D1 and SSG, 20 mg/kg, IM, OD, D2-11; (2) Combination of LAMB, 10 mg/kg, IV, single dose, D1, and MF, 2.5 mg/kg, PO, OD, D2-11; (3) MF\tnote{5}, 2.5 mg/kg, PO, OD, 28 D & 7--60       & 2010--2012   & 138       & 16 (11.6)     \\ \bottomrule
    \end{longtable}
  \end{ThreePartTable}
\end{landscape}
}
\pagestyle{fancy}
\restoregeometry

% categorical variables table
\begin{table}[htbp]
    \centering
    \small
    \begin{threeparttable}
        \begin{tabular}{@{} l R{1.3cm} @{\hspace{4pt}} L{1.3cm} R{1.3cm} @{\hspace{4pt}} L{1.3cm} R{0.89cm} @{\hspace{4pt}} L{1.3cm} @{}}
            \toprule
            \textbf{Variable}                 & \multicolumn{2}{c}{\textbf{Overall (\%)}} & \multicolumn{2}{c}{\textbf{Final cure (\%)}} & \multicolumn{2}{c}{\textbf{Relapse (\%)}}                        \\
                                              & \multicolumn{2}{c}{n~=~2,051}             & \multicolumn{2}{c}{n~=~1,952}                & \multicolumn{2}{c}{n~=~99}                                       \\
            \midrule
            \textbf{Sex}                      &                                           &                                              &                                           &        &    &        \\
            \hspace{1em} Female               & 532                                       & (25.9)                                       & 503                                       & (25.8) & 29 & (29.3) \\
            \hspace{1em} Male                 & 1,519                                     & (74.1)                                       & 1,449                                     & (74.2) & 70 & (70.7) \\
            \textbf{Malnutrition}             &                                           &                                              &                                           &        &    &        \\
            \hspace{1em}Normal/mild           & 735                                       & (35.8)                                       & 710                                       & (36.4) & 25 & (25.3) \\
            \hspace{1em}Moderate              & 800                                       & (39.0)                                       & 760                                       & (38.9) & 40 & (40.4) \\
            \hspace{1em}Severe                & 509                                       & (24.8)                                       & 476                                       & (24.4) & 33 & (33.3) \\
            \hspace{1em}(Missing)             & 7                                         & (0.3)                                        & 6                                         & (0.3)  & 1  & (1.0)  \\
            \textbf{Anaemia}                  &                                           &                                              &                                           &        &    &        \\
            \hspace{1em}Non-severe            & 1,049                                     & (51.1)                                       & 1,015                                     & (52.0) & 34 & (34.3) \\
            \hspace{1em}Severe                & 999                                       & (48.7)                                       & 934                                       & (47.8) & 65 & (65.7) \\
            \hspace{1em}(Missing)             & 3                                         & (0.1)                                        & 3                                         & (0.2)  & 0  & (0.0)  \\
            \textbf{Parasite grade}           &                                           &                                              &                                           &        &    &        \\
            \hspace{1em}1+                    & 511                                       & (24.9)                                       & 488                                       & (25.0) & 23 & (23.2) \\
            \hspace{1em}2+                    & 237                                       & (11.6)                                       & 219                                       & (11.2) & 18 & (18.2) \\
            \hspace{1em}3+                    & 192                                       & (9.4)                                        & 181                                       & (9.3)  & 11 & (11.1) \\
            \hspace{1em}4+                    & 179                                       & (8.7)                                        & 169                                       & (8.7)  & 10 & (10.1) \\
            \hspace{1em}5+                    & 193                                       & (9.4)                                        & 180                                       & (9.2)  & 13 & (13.1) \\
            \hspace{1em}6+                    & 77                                        & (3.8)                                        & 60                                        & (3.1)  & 17 & (17.2) \\
            \hspace{1em}(Missing)             & 662                                       & (32.3)                                       & 655                                       & (33.6) & 7  & (7.1)  \\
            \textbf{Aspirate source}\tnote{1} &                                           &                                              &                                           &        &    &        \\
            \hspace{1em}Bone                  & 131                                       & (9.4)                                        & 110                                       & (8.6)  & 21 & (22.8) \\
            \hspace{1em}Spleen                & 393                                       & (28.3)                                       & 369                                       & (28.9) & 24 & (26.1) \\
            \hspace{1em}Lymph node            & 163                                       & (11.7)                                       & 151                                       & (11.8) & 12 & (13.0) \\
            \hspace{1em}(Missing)             & 702                                       & (50.5)                                       & 649                                       & (50.7) & 35 & (38.0) \\
            \bottomrule
        \end{tabular}
        \begin{tablenotes}
            \footnotesize
            \item[1] Denominator for \% in aspirate source: number of patients with documented parasite grade (overall: 1,389; final cure: 1,279; relapse: 92).

        \end{tablenotes}
    \end{threeparttable}
    \caption[East Africa: categorical variables]{Summary of categorical candidate predictors and parasite source across contributed studies from East Africa. Missing data are described where present.}
    \label{tab:ea_categorical}
\end{table}

% continuous variables table
% \begin{table}[ht]
%     \centering
%     \begin{tabular}{rllllllll}
%         \hline
%            & VARIABLE        & median\_overall                                         & range\_overall                       & missing\_overall & median\_nr                                              & missing\_nr  & median\_r                                               & missing\_r \\
%         \hline
%         1  & AGE             & 14 (9 -- 22)                                            & 0.6 -- 60                            & 1 (0.0)          & 15 (9 -- 22)                                            & 1 (0.1)      & 12 (9 -- 20)                                            & 0 (0.0)    \\
%         2  & HEIGHT          & 153 (126 -- 168)                                        & 68.8 -- 205                          & 9 (0.4)          & 154 (126 -- 168)                                        & 8 (0.4)      & 143 (127 -- 164)                                        & 1 (1.0)    \\
%         3  & WEIGHT          & 35 (21 -- 49)                                           & 5.6 -- 70.5                          & 1 (0.0)          & 35 (21 -- 49)                                           & 1 (0.1)      & 28 (20.6 -- 44.5)                                       & 0 (0.0)    \\
%         4  & BMI             & 17.5626352555497 (16.2946453729421 -- 18.6851211072664) & 12.1107266435986 -- 32.6388888888889 & 1,299 (63.3)     & 17.5382653061225 (16.2911048048411 -- 18.7109492465502) & 1,229 (63.0) & 17.6308539944904 (16.6493236212279 -- 18.2183224271267) & 70 (70.7)  \\
%         5  & BMIZ            & -2.32 (-3.1 -- -1.495)                                  & -8.84 -- 3.13                        & 888 (43.3)       & -2.3 (-3.09 -- -1.4725)                                 & 854 (43.8)   & -2.58 (-3.43 -- -1.89)                                  & 34 (34.3)  \\
%         6  & WFHZ            & -2.36 (-3.12 -- -1.58)                                  & -6.81 -- 0.49                        & 1,922 (93.7)     & -2.32 (-3.01 -- -1.55)                                  & 1,827 (93.6) & -3.415 (-3.8925 -- -3.24)                               & 95 (96.0)  \\
%         7  & SPLEEN\_LENGTH  & 8 (5 -- 11)                                             & 0 -- 30                              & 69 (3.4)         & 8 (5 -- 11)                                             & 69 (3.5)     & 7 (4 -- 10)                                             & 0 (0.0)    \\
%         8  & FEVER\_DURATION & 40.44 (25 -- 91.32)                                     & 3 -- 761                             & 349 (17.0)       & 40.44 (25 -- 91.32)                                     & 320 (16.4)   & 30.44 (20.25 -- 60.88)                                  & 29 (29.3)  \\
%         9  & PARASITE        & 2 (1 -- 4)                                              & 1 -- 6                               & 662 (32.3)       & 2 (1 -- 4)                                              & 655 (33.6)   & 3 (1.75 -- 5)                                           & 7 (7.1)    \\
%         10 & LAB\_WBC        & 2.5 (1.8 -- 3.5)                                        & 0.6 -- 16.8                          & 887 (43.2)       & 2.5 (1.8 -- 3.5)                                        & 871 (44.6)   & 2.7 (1.84 -- 3.55)                                      & 16 (16.2)  \\
%         11 & LAB\_PLT        & 106 (73 -- 157)                                         & 5 -- 881                             & 890 (43.4)       & 105 (73 -- 155)                                         & 874 (44.8)   & 110 (74.5 -- 171)                                       & 16 (16.2)  \\
%         12 & LAB\_HGB        & 79 (67 -- 91)                                           & 22 -- 167                            & 2 (0.1)          & 80 (67 -- 92)                                           & 2 (0.1)      & 72 (62 -- 83.5)                                         & 0 (0.0)    \\
%         13 & LAB\_ALT        & 21 (14 -- 31)                                           & 2 -- 134                             & 920 (44.9)       & 21 (14 -- 31)                                           & 901 (46.2)   & 19 (14 -- 30.25)                                        & 19 (19.2)  \\
%         14 & LAB\_CREAT      & 44.2 (0.6 -- 79)                                        & 0.1 -- 134                           & 826 (40.3)       & 44.2 (0.6 -- 78.94)                                     & 815 (41.8)   & 44.2 (0.675 -- 79.5603825)                              & 11 (11.1)  \\
%         \hline
%     \end{tabular}
% \end{table}

\begin{landscape}
  \begin{table}[htbp]
    \centering
    \small
    \begin{threeparttable}
      \begin{tabular}{@{} l @{} r @{\hspace{4pt}} l @{} r @{\hspace{4pt}} l @{} r @{\hspace{4pt}} l @{} r @{\hspace{4pt}} l @{} r @{\hspace{4pt}} l @{} r @{\hspace{4pt}} l @{}}
        \toprule
        \textbf{Variable}           & \multicolumn{4}{@{}c@{}}{\textbf{Overall} (n = 2,051)} & \multicolumn{4}{@{}c@{}}{\textbf{Final cure } (n = 1,952)} & \multicolumn{4}{@{}c@{}}{\textbf{Relapse} (n = 99)}                                                                                                                        \\
        \cmidrule(r){2-5}\cmidrule(lr){6-9}\cmidrule(l){10-13}
                                    & Median                                                 & (IQR)                                                      & Missing\tnote{1}                                    & (\%)   & Median & (IQR)            & Missing & (\%)   & Median & (IQR)            & Missing\footnotemark[1] & (\%)   \\
        \midrule
        Age (years)                 & 14                                                     & (9 -- 22)                                                  & 1                                                   & (0.0)  & 15     & (9 -- 22)        & 1       & (0.1)  & 12     & (9 -- 20)        & 0                       & (0.0)  \\
        Height (cm)                 & 153                                                    & (126 -- 168)                                               & 9                                                   & (0.4)  & 154    & (126 -- 168)     & 8       & (0.4)  & 143    & (127 -- 164)     & 1                       & (1.0)  \\
        Weight (kg)                 & 35                                                     & (21 -- 49)                                                 & 1                                                   & (0.0)  & 35     & (21 -- 49)       & 1       & (0.1)  & 28     & (20.6 -- 44.5)   & 0                       & (0.0)  \\
        BMI (kg/m$^2$)\tnote{2}     & 17.56                                                  & (16.29 -- 18.69)                                           & 4                                                   & (0.5)  & 17.54  & (16.29 -- 18.71) & 3       & (0.4)  & 17.63  & (16.65 -- 18.22) & 1                       & (3.2)  \\
        BMI--FA z--score\tnote{3}   & -2.32                                                  & (-3.10 -- -1.50)                                           & 2                                                   & (0.2)  & -2.30  & (-3.09 -- -1.47) & 2       & (0.2)  & -2.58  & (-3.43 -- -1.89) & 0                       & (0.0)  \\
        WFH z--score\tnote{4}       & -2.36                                                  & (-3.12 -- -1.58)                                           & 0                                                   & (0.0)  & -2.32  & (-3.01 -- -1.55) & 0       & (0.0)  & -3.42  & (-3.89 -- -3.24) & 0                       & (0.0)  \\
        Spleen size (cm)            & 8                                                      & (5 -- 11)                                                  & 69                                                  & (3.4)  & 8      & (5 -- 11)        & 69      & (3.5)  & 7      & (4 -- 10)        & 0                       & (0.0)  \\
        Fever duration (days)       & 40.4                                                   & (25.0 -- 91.3)                                             & 349                                                 & (17.0) & 40.4   & (25.0 -- 91.3)   & 320     & (16.4) & 30.4   & (20.3 -- 60.9)   & 29                      & (29.3) \\
        Parasite grade              & 2                                                      & (1 -- 4)                                                   & 662                                                 & (32.3) & 2      & (1 -- 4)         & 655     & (33.6) & 3      & (2 -- 5)         & 7                       & (7.1)  \\
        WBC ($\times 10^9$/L)       & 2.5                                                    & (1.8 -- 3.5)                                               & 887                                                 & (43.2) & 2.5    & (1.8 -- 3.5)     & 871     & (44.6) & 2.7    & (1.8 -- 3.6)     & 16                      & (16.2) \\
        Platelets ($\times 10^9$/L) & 106                                                    & (73 -- 157)                                                & 890                                                 & (43.4) & 105    & (73 -- 155)      & 874     & (44.8) & 110    & (74.5 -- 171)    & 16                      & (16.2) \\
        Haemoglobin (g/L)           & 79                                                     & (67 -- 91)                                                 & 2                                                   & (0.1)  & 80     & (67 -- 92)       & 2       & (0.1)  & 72     & (62 -- 83.5)     & 0                       & (0.0)  \\
        ALT (IU/L)                  & 21                                                     & (14 -- 31)                                                 & 920                                                 & (44.9) & 21     & (14 -- 31)       & 901     & (46.2) & 19     & (14 -- 30)       & 19                      & (19.2) \\
        Creatinine ($\mu$mol/L)     & 44.2                                                   & (0.6 -- 79.0)                                              & 826                                                 & (40.3) & 44.2   & (0.6 -- 78.9)    & 815     & (41.8) & 44.2   & (0.7 -- 79.6)    & 11                      & (11.1) \\
        \bottomrule
      \end{tabular}
      \begin{tablenotes}
        \footnotesize
        \item[1] Denominator for missing \%: total number of patients in respective group (overall, relapse or final cure). For measures of malnutrition (BMI, BMI-for-age z--score, and weight-for-height z--score), see below.
        \item[2] Denominator for missing \%: number of patients aged $\geq$ 19 years, n = 756 (relapse: 30, final cure: 726).
        \item[3] Denominator for missing \%: number of patients aged 5--18 year inclusive, n = 1,165 (relapse: 65, final cure: 1,100).
        \item[4] Denominator for missing \%: number of patients aged $<$ 5 years, n = 129 (relapse: 4, final cure: 125).
      \end{tablenotes}
    \end{threeparttable}
    \caption{Summary of continuous candidate predictors across contributed studies from East Africa. Including additional variables used for the derivation of malnutrition status (height, weight, BMI, BMI-for-age z--score, weight-for-height z--score). ALT: alanine aminotransferase; BMI(-FA): body mass index(-for age); cm: centimetres; IQR: inter-quartile range, IU: international units; kg: kilograms; L: litres; m: metres; WBC: white blood cells; WFH: weight-for-height; g: grams; $\mu$mol: micromoles.}
    \label{tab:ea_continuous}
  \end{table}
\end{landscape}

% pooled distributions of categorical variables
% cp /Users/jameswilson/proj/vl_model_ea/figures/dist/catOut/comb_cat.pdf figures/ch6/ea_cat_comb.pdf
\newgeometry{left=1cm, bottom=2.5cm, right=2cm, top=3cm}
\begin{landscape}
  \begin{figure}[tb]
    \centering
    \includegraphics[width=1.35\textwidth]{figures/ch6/ea_cat_comb.pdf}
    \caption{Pooled distributions and predictor--outcomes relationships for categorical candidate predictors. Excluding missing data. 95\% binomial confidence intervals calculated using the Wilson method. Note: for parasite grade relapse \%, y-axis is rescaled to accommodate increased risk in 6+ group. Norm: normal.}
    \label{fig:ea_cat_comb}
  \end{figure}
\end{landscape}
\restoregeometry

% these are the pooled continuous distributions and relationships with relapse - PART 1
% cp /Users/jameswilson/proj/vl_model_ea/figures/dist/contOut/age_comb.pdf figures/ch6/ea_pool_age_comb.pdf
% cp /Users/jameswilson/proj/vl_model_ea/figures/dist/contOut/ss_comb.pdf figures/ch6/ea_pool_ss_comb.pdf
% cp /Users/jameswilson/proj/vl_model_ea/figures/dist/contOut/fd_comb.pdf figures/ch6/ea_pool_fd_comb.pdf
% cp /Users/jameswilson/proj/vl_model_ea/figures/dist/contOut/height_comb.pdf figures/ch6/ea_pool_height_comb.pdf
% cp /Users/jameswilson/proj/vl_model_ea/figures/dist/contOut/weight_comb.pdf figures/ch6/ea_pool_weight_comb.pdf

\clearpage
\begin{figure}[H]
  \centering
  \begin{subfigure}{\textwidth}
    \centering
    \begin{overpic}[width=\textwidth]{figures/ch6/ea_pool_age_comb.pdf}
      \put(2,19){\small Age}
    \end{overpic}
  \end{subfigure}
  \begin{subfigure}{\textwidth}
    \centering
    \begin{overpic}[width=\textwidth]{figures/ch6/ea_pool_weight_comb.pdf}
      \put(2,19){\small Weight}
    \end{overpic}
  \end{subfigure}
  \begin{subfigure}{\textwidth}
    \centering
    \begin{overpic}[width=\textwidth]{figures/ch6/ea_pool_height_comb.pdf}
      \put(2,19){\small Height}
    \end{overpic}
  \end{subfigure}
  \begin{subfigure}{\textwidth}
    \centering
    \begin{overpic}[width=\textwidth]{figures/ch6/ea_pool_fd_comb.pdf}
      \put(2,19){\small FevDur}
    \end{overpic}
  \end{subfigure}
  \begin{subfigure}{\textwidth}
    \centering
    \begin{overpic}[width=\textwidth]{figures/ch6/ea_pool_ss_comb.pdf}
      \put(2,19){\small SpnSize}
    \end{overpic}
  \end{subfigure}
  \caption{Distributions and predictor--outcome relationships for continuous non-laboratory candidate predictors. FevDur:~duration of fever; SpnSize:~spleen size. For each candidate predictor, left upper panel shows the overall density pooled across studies and the left lower panel shows overlapping densities normalised by relapse status. The right panel shows a univariable generalised additive model spline fit, with 95\% confidence interval, of relapse.}
  \label{fig:ea_pooled_dist_cont1}
\end{figure}

% these are the pooled continuous distributions and relationships with relapse - PART 2
% cp /Users/jameswilson/proj/vl_model_ea/figures/dist/contOut/wbc_comb.pdf figures/ch6/ea_pool_wbc_comb.pdf
% cp /Users/jameswilson/proj/vl_model_ea/figures/dist/contOut/plt_comb.pdf figures/ch6/ea_pool_plt_comb.pdf
% cp /Users/jameswilson/proj/vl_model_ea/figures/dist/contOut/hb_comb.pdf figures/ch6/ea_pool_hb_comb.pdf
% cp /Users/jameswilson/proj/vl_model_ea/figures/dist/contOut/alt_comb.pdf figures/ch6/ea_pool_alt_comb.pdf
% cp /Users/jameswilson/proj/vl_model_ea/figures/dist/contOut/cr_comb.pdf figures/ch6/ea_pool_cr_comb.pdf
\begin{figure}[H]
  \centering
  \begin{subfigure}{\textwidth}
    \centering
    \begin{overpic}[width=\textwidth]{figures/ch6/ea_pool_hb_comb.pdf}
      \put(2,19){\small Hb}
    \end{overpic}
  \end{subfigure}
  \begin{subfigure}{\textwidth}
    \centering
    \begin{overpic}[width=\textwidth]{figures/ch6/ea_pool_plt_comb.pdf}
      \put(2,19){\small Plt}
    \end{overpic}
  \end{subfigure}
  \begin{subfigure}{\textwidth}
    \centering
    \begin{overpic}[width=\textwidth]{figures/ch6/ea_pool_wbc_comb.pdf}
      \put(2,19){\small WBC}
    \end{overpic}
  \end{subfigure}
  \begin{subfigure}{\textwidth}
    \centering
    \begin{overpic}[width=\textwidth]{figures/ch6/ea_pool_alt_comb.pdf}
      \put(2,19){\small ALT}
    \end{overpic}
  \end{subfigure}
  \begin{subfigure}{\textwidth}
    \centering
    \begin{overpic}[width=\textwidth]{figures/ch6/ea_pool_cr_comb.pdf}
      \put(2,19){\small Crt}
    \end{overpic}
  \end{subfigure}
  \caption{Pooled distributions and predictor--outcome relationships for continuous laboratory candidate predictors. All predictors presented on log scale. Hb:~haemoglobin; Plt:~platelet; WBC:~white blood cells; ALT:~alanine aminotransferase; Crt:~creatinine. For each candidate predictor, left upper panel shows the overall density pooled across studies and the left lower panel shows overlapping densities normalised by relapse status. The right panel shows a univariable generalised additive model spline fit, with 95\% confidence interval, of relapse.}
  \label{fig:ea_pooled_dist_cont2}
\end{figure}

% these are the study specific distributions of outcome, sex and age
% cp /Users/jameswilson/proj/vl_model_ea/figures/dist/main_dist.pdf figures/ch6/ea_main_dist.pdf
\newgeometry{left=1cm, bottom=2.5cm, right=2cm, top=3cm}
\begin{landscape}
  \begin{figure}[tb]
    \centering
    \includegraphics[width=1.35\textwidth]{figures/ch6/ea_main_dist.pdf}
    \caption{Graphical summary of East Africa study-specific sample sizes and distributions of relapse status, sex, and age.}
    \label{fig:ea_main_dist}
  \end{figure}

  % study specific treatments
  % cp /Users/jameswilson/proj/vl_model_ea/figures/treatment/treat_dist1a.pdf figures/ch6/treat.pdf

  \begin{figure}[tb]
    \centering
    \includegraphics[width=1.35\textwidth]{figures/ch6/treat.pdf}
    \caption{Bar chart showing the distribution of treatment regimens across contributing studies from East Africa. Drugs are colour-coded (see legend). Important distinguishing dosing information provided in the overlaying labels. Full treatment details presented in Table \ref{tab:ea_study}. D: days; LAMB: liposomal amphotericin B (Gilead); PM: paromomycin; MF: miltefosine; mg: miligrams/kilogram; SSG: sodium stibogluconate.}
    \label{fig:ea_treat}
  \end{figure}

\end{landscape}
\restoregeometry


% missing data figure
% cp /Users/jameswilson/proj/vl_model_ea/figures/missing/summary.pdf figures/ch6/ea_missing_summary.pdf
\begin{figure}[tb]
  \centering
  \includegraphics[scale = 0.8, trim={0cm 3.8cm 0cm 3.3cm}, clip]{figures/ch6/ea_missing_summary.pdf}
  \caption{Density plot illustrating the amount of missing data overall and across contributing studies from East Africa. Study ordered by lead author and publication year. Variable ordered by amount of missingness. ALT: alanine aminotransferase; WBC: white blood cells.}
  \label{fig:ea_missing_summary}
\end{figure}