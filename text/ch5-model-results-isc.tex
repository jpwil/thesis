% \begin{savequote}[8cm]
%   Quote goes here.
%   \qauthor{--- James Wilson}
% \end{savequote}

\chapter{\label{ch:5-isc-model-results}Results: Indian Subcontinent Models}
\minitoc{}

\section{Descriptive Analysis}

On application of the eligibility criteria to the IDDO VL data platform, a total of 19 studies and 4,599 patients were selected for inclusion\cite{bhattacharya2007,chakraborty2008,das2009,koirala2003,pandey2016,pandey2017,rijal2003,rijal2010a,rijal2010b,sundar2007a,sundar2008,sundar2008B,sundar2009,sundar2010,sundar2011,sundar2012,sundar2014,sundar2015,sundar2019}. At the participant selection stage, 27 participants (0.6\%, 27/4860) were excluded due to having a positive HIV test, and 201 (4.1\%, 201/4860) were excluded due to not achieving initial cure. A flow diagram is presented in Figure~\ref{fig:isc_flow_diagram}.

% flow diagram
\begin{figure}[tb]
  \centering
  \includegraphics[width = 0.9\textwidth, trim={4cm 0cm 2cm 2cm}, clip]{figures/ch5/isc_flow_chart.pdf}
  \caption[Indian subcontinent model: flow diagram]{Flow diagram showing the studies and patients excluded from Indian subcontinent model development. HIV\@: human immunodeficiency virus; IDDO\@: Infectious Diseases Data Observatory; ISC\@: Indian subcontinent; PKDL\@: post kala-azar dermal leishmaniasis; VL\@: visceral leishmaniasis.}\label{fig:isc_flow_diagram}
\end{figure}

\subsection{Study Characteristics}

Key characteristics of the 19 included studies are presented in Table~\ref{tab:isc_study}.

A median of 144 patients were recruited in each study, ranging from 6 to 928 patients (IQR: 87 to 330 patients). Patients were recruited between 2000 and 2017 from sites in India (15 studies, 4,171 [90.7\%] patients)\cite{bhattacharya2007,chakraborty2008, das2009,pandey2016,pandey2017,sundar2007a,sundar2008,sundar2008B,sundar2009,sundar2010,sundar2011,sundar2012,sundar2014,sundar2015,sundar2019} and Nepal (4 studies, 428 [9.3\%] patients)\cite{rijal2003, rijal2010a, rijal2010b,koirala2003}. All Indian studies recruited patients from the state of Bihar in northeastern India, with the majority of studies led by investigators from either a government referral hospital in Patna (Rajendra Memorial Research Institute of Medical Sciences: RMRIMS) (4 studies, 1,125 [24.5\%] patients)\cite{bhattacharya2007, das2009,pandey2016,pandey2017}, or a private research facility in Muzaffarpur (Kala-azar Medical Research Centre: KAMRC) (9 studies, 2,652 [57.7\%] patients)\cite{chakraborty2008,sundar2008,sundar2008B,sundar2009,sundar2010,sundar2011,sundar2012,sundar2015,sundar2019}. All Nepalese studies were led by investigators based at the B.P. Koirala Institute of Health Sciences, a health sciences university in the city of Dharan, Koshi Province, eastern Nepal.

  % studies tables

  {
    \newgeometry{left=1.5cm, right=1.5cm, top=2cm, bottom=1.8cm}

    \begin{landscape}
      \pagestyle{empty}

      % study characteristics
% need to have `\\` after caption entries, otherwise compiler hangs
% reduced font size, expanded margins, remove page numbers, footer & headers - this reduces to 3 pages instead of 4-5
% sum of n (model) = 4599; sum of relapses = 228. 

{
\newgeometry{left=1.5cm, right=1.5cm, top=2cm, bottom=1.5cm}
\captionsetup{width=1\textheight, font=scriptsize, skip=7pt}

\begin{landscape}
  % \singlespacing
  \pagestyle{empty}
  \scriptsize
  \begin{ThreePartTable}
    \begin{TableNotes}
      \item[1] Study name is composed of the lead author and year of publication or most recent protocol.
      \item[2] Contributed number of participants from KAMRC site only.
      \item[3] Amphotericin B deoxycholate arm not reported in publication.
      \item[4] Protocol not in public domain.
    \end{TableNotes}

    \begin{longtable}[c]{L{1.5cm} L{3.5cm} L{1.5cm} L{2.5cm} L{2.5cm} L{2.5cm} L{2.5cm} L{1.5cm} L{1.2cm} L{1.3cm} L{1.3cm}}
      \caption{Key characteristics of included studies, ordered by lead author and year of publication/protocol. Where information not presented in the publication, information is extracted from the study protocol. -: not reported; ABLE: Amphotericin B lipid emulsion (Bharat Serum and Vaccines Ltd.); alt: alternative; ABD: Amphotericin B deoxycholate; BMGF: Bill and Melinda Gates Foundation; BPKIHS: B.P. Koirala Institute of Health Sciences; D: Day(s); EC: European Commission; Govt.: Government; ICMR: Indian Council of Medical Research; IM: Intramuscular; KAMRC: Kala-azar Medical Research Center; LAMB: Liposomal amphotericin B (Gilead formulation unless otherwise specified); MF: Miltefosine; mg/kg: milligrams per kilogram; NCVBDC: National Center for Vector Borne Diseases Control; OD: Once daily; P: Publication; PATH: Program for Appropriate Technology in Health; PM: Paromomycin; Pr: Protocol; Ref: Reference; RMRIMS: Rajendra Memorial Research Institute of Medical Sciences; SDA: Single-dose administration; SSG: Sodium stibogluconate; TDR: UNICEF/UNDP/World Bank/WHO Special Programme for Research and Training in Tropical Diseases; UN: United Nations; WHO: World Health Organization\label{tab:isc_study}} \\
      \toprule
      Study\tnote{1}                                  & Title                                                                                                                                                                                                                       & Journal                  & Sponsor/funding                                                                      & Location(s)                                       & Study design                                                                       & Study arm(s)                                                                                                                   & Age (years) & Study period & n (model)    & Relapses (\%)                                                                                                                                                                                                                                                                                                                                                                                                                                                                                                                          \\ \midrule
      \endfirsthead

      \caption[]{continued}                                                                                                                                                                                                                                                                                                                                                                                                                                                                                                                                                                                                                                                                                                                                                                                                                                                                                                                                                                                                                                                                                                                                                                                                                                          \\
      \toprule
      Study\tnote{1}                                  & Title                                                                                                                                                                                                                       & Journal                  & Sponsor/funding                                                                      & Location(s)                                       & Study design                                                                       & Study arm(s)                                                                                                                   & Age (years) & Study period & n (model)    & Relapses (\%)                                                                                                                                                                                                                                                                                                                                                                                                                                                                                                                          \\ \midrule
      \endhead

      \multicolumn{11}{r}{\textit{continued on next page}}                                                                                                                                                                                                                                                                                                                                                                                                                                                                                                                                                                                                                                                                                                                                                                                                                                                                                                                                                                                                                                                                                                                                                                                                           \\
      \endfoot

      \insertTableNotes
      \endlastfoot

      Bhattacharya 2007\cite{bhattacharya2007}        & Phase 4 trial of miltefosine for the treatment of Indian visceral leishmaniasis                                                                                                                                             & J Infect Dis             & ICMR                                                                                 & 13 locations in Bihar, India (outpatient setting) & Open label; phase 4; safety/efficacy                                               & 1: MF 28D                                                                                                                      & 2--65       & 2002--2004   & 352\tnote{2} & 22 (6.3)                                                                                                                                                                                                                                                                                                                                                                                                                                                                                                                               \\ \midrule
      Chakraborty 2008\cite{chakraborty2008}          & Human placental extract offers protection against experimental visceral leishmaniasis: a pilot study for a phase-I clinical trial                                                                                           & Am J Trop Med Hyg        & Indian Council of Scientific and Industrial Research; Albert David Ltd.              & KAMRC, Muzaffarpur, India                         & ``Pre-phase 1''; pilot/preliminary                                                 & 2\tnote{3}: 2.06mg human placental extract IM, single dose (or ABD 1mg/kg alt. days for 30D)                                   & $\geq$5     & 2003--2005   & 6            & 0 (0)                                                                                                                                                                                                                                                                                                                                                                                                                                                                                                                                  \\ \midrule
      Das 2009\cite{das2009}                          & A controlled, randomized nonblinded clinical trial to assess the efficacy of amphotericin B deoxycholate as compared to pentamidine for the treatment of antimony unresponsive visceral leishmaniasis cases in Bihar, India & Ther Clin Risk Manag     & -                                                                                    & RMRIMS, Patna, India                              & Randomised; open label; efficacy                                                   & 2: ABD 1mg/kg alt. days for 30D vs pentamidine IM 4mg/kg alt days, 30D                                                         & 6--60       & 2002         & 73           & 5 (6.8)                                                                                                                                                                                                                                                                                                                                                                                                                                                                                                                                \\ \midrule
      Koirala 2003                                    & Phase IV trial of miltefosine in the treatment of visceral leishmaniasis                                                                                                                                                    & Protocol only\tnote{4}   & -                                                                                    & BPKIHS, Dharan, Nepal                             & Open label; phase 4; safety/efficacy                                               & 1: MF 28D                                                                                                                      & 2--65       & 2003--2004   & 116          & 12 (10.3)                                                                                                                                                                                                                                                                                                                                                                                                                                                                                                                              \\ \midrule
      Pandey 2016\cite{pandey2016}                    & Pharmacovigilance of miltefosine in treatment of visceral leishmaniasis in endemic areas of Bihar                                                                                                                           & Am J Trop Med Hyg        & NCVBDC, Govt. of India; World Bank                                                   & 4 locations in Bihar, India                       & Open label; safety/efficacy                                                        & 1: MF 28D                                                                                                                      & 6--70       & 2012--2015   & 600          & 45 (7.5)                                                                                                                                                                                                                                                                                                                                                                                                                                                                                                                               \\ \midrule
      Pandey 2017\cite{pandey2017}                    & Efficacy and safety of liposomal amphotericin B for visceral leishmaniasis in children and adolescents at a tertiary care center in Bihar, India                                                                            & Am J Trop  Med Hyg       & -                                                                                    & RMRIMS, Patna, India                              & Open label; safety/efficacy                                                        & 1: LAMB 10mg/kg SDA                                                                                                            & $<$15       & 2014--2016   & 100          & 2 (2.0)                                                                                                                                                                                                                                                                                                                                                                                                                                                                                                                                \\ \midrule
      Rijal 2003\cite{rijal2003}                      & Treatment of visceral leishmaniasis in south-eastern Nepal: decreasing efficacy of sodium stibogluconate and need for a policy to limit further decline                                                                     & Trans R Soc Trop Med Hyg & WHO, Geneva University Hospital; Novartis Foundation                                 & BPKIHS, Dharan, Nepal                             & Non-randomised; efficacy                                                           & 2: SSG 20mg/kg OD, 30D either (i) in hospital or (ii) first 5--7 days in hospital; extended to 40D if positive aspirate at 30D & All         & 2000--2001   & 102          & 1 (1.0)                                                                                                                                                                                                                                                                                                                                                                                                                                                                                                                                \\ \midrule
      Rijal 2010(A)\cite{rijal2010a}                  & Efficacy and safety of liposomal amphotericin B in Nepalese patients with visceral leishmaniasis                                                                                                                            & Protocol only            & TDR                                                                                  & BPKIHS, Dharan, Nepal                             & Phase 2/3; safety/efficacy                                                         & 1: LAMB, 3mg/kg OD, 5D                                                                                                         & 12--65      & 2010--2011   & 32           & 1 (3.1)                                                                                                                                                                                                                                                                                                                                                                                                                                                                                                                                \\ \midrule
      Rijal 2010(B)\cite{rijal2010b}                  & Clinical risk factors for therapeutic failure in kala-azar patients treated with pentavalent antimonials in Nepal                                                                                                           & Trans R Soc Trop Med Hyg & EC (5th Framework Programme)                                                         & BPKIHS, Dharan, Nepal                             & Prospective cohort                                                                 & 1: SSG 20mg/kg OD, 30D                                                                                                         & -           & 2001--2003   & 178          & 1 (0.6)                                                                                                                                                                                                                                                                                                                                                                                                                                                                                                                                \\ \midrule
      Sundar 2007\cite{sundar2007a,sundar2005prot}    & Injectable paromomycin for visceral leishmaniasis in India                                                                                                                                                                  & N Engl J Med             & PATH (including UN, BMGF, TDR)                                                       & 4 locations in Bihar, India                       & Randomised; open label; phase 3; non-inferiority; safety/efficacy                  & 2: PM 15mg/kg IM, 21D vs. ABD 1mg/kg alt. days, 30D                                                                            & 5--55       & 2003--2004   & 250\tnote{2} & 4 (1.6)                                                                                                                                                                                                                                                                                                                                                                                                                                                                                                                                \\ \midrule
      Sundar 2008(A)\cite{sundar2008protC,sundar2008} & New treatment approach in Indian visceral leishmaniasis: single-dose liposomal amphotericin B followed by short-course oral miltefosine                                                                                     & Clin Infect Dis          & Banaras Hindu University                                                             & KAMRC, Muzaffarpur, India                         & Partially randomised; open label; phase 2; non-comparative; sequential; triangular & 5: LAMB 5mg/kg SD alone, or comb. LAMB 3.75--5mg/kg SD + MF 7--14D                                                             & $\geq$12    & 2006--2007   & 225          & 7 (3.1)                                                                                                                                                                                                                                                                                                                                                                                                                                                                                                                                \\ \midrule
      Sundar 2008(B)\cite{sundar2008B}                & Safety of a pre-formulated amphotericin B lipid emulsion for the treatment of Indian kala-azar                                                                                                                              & Trop Med Int Health      & Bharat Serum and Vaccines Ltd.                                                       & KAMRC, Muzaffarpur, India                         & Non-randomised; non-comparative; open label; phase 2; safety/efficacy              & 3: ABLE OD for 3D at (i) 5mg/kg (ii) 4mg/kg (iii) 3mg/kg                                                                       & 12--65      & 2004--2005   & 45           & 4 (8.9)                                                                                                                                                                                                                                                                                                                                                                                                                                                                                                                                \\ \midrule
      Sundar 2009\cite{sundar2008protB, sundar2009}   & Short-course paromomycin treatment of visceral leishmaniasis in India: 14-day vs 21-day treatment                                                                                                                           & Clin Infect Dis          & Banaras Hindu University                                                             & KAMRC, Muzaffarpur, India                         & Randomised; open label; phase 3; safety/efficacy                                   & 2: PM OD 15mg/kg/day for either (i) 21D or (ii) 14D                                                                            & 5--55       & 2007--2008   & 307          & 26 (8.5)                                                                                                                                                                                                                                                                                                                                                                                                                                                                                                                               \\ \midrule
      Sundar 2010\cite{sundar2008protA, sundar2010}   & Single-dose liposomal amphotericin B for visceral leishmaniasis in India                                                                                                                                                    & N Engl J Med             & Banaras Hindu University                                                             & KAMRC, Muzaffarpur, India                         & Randomised; open label; phase 3; safety/efficacy                                   & 2: ABD 1mg/kg alt. days for 30D, vs. LAMB 10mg/kg SDA                                                                          & 2--65       & 2008--2009   & 412          & 14 (3.4)                                                                                                                                                                                                                                                                                                                                                                                                                                                                                                                               \\ \midrule
      Sundar 2011\cite{sundar2006prot,sundar2011}     & Ambisome plus miltefosine for Indian patients with kala-azar                                                                                                                                                                & Trans R Soc Trop Med Hyg & Banaras Hindu University                                                             & KAMRC, Muzaffarpur; RMRIMS, Patna, India          & Open label; phase 2; safety/efficacy                                               & 1: LAMB 5mg/kg SDA followed by MF D2-D15                                                                                       & 2--65       & 2007--2009   & 128          & 5 (3.9)                                                                                                                                                                                                                                                                                                                                                                                                                                                                                                                                \\ \midrule
      Sundar 2012\cite{sundar2012}                    & Efficacy of miltefosine in the treatment of visceral leishmaniasis in India after a decade of use                                                                                                                           & Clin Infect Dis          & EC (Kaladrug-R); Sitaram Memorial Trust                                              & KAMRC, Muzaffarpur, India                         & Open label; safety/efficacy                                                        & 1: MF 28D                                                                                                                      & 6--70       & 2009--2010   & 571          & 34 (6.0)                                                                                                                                                                                                                                                                                                                                                                                                                                                                                                                               \\ \midrule
      Sundar 2014\cite{sundar2009prot, sundar2014}    & Efficacy and safety of amphotericin B emulsion versus liposomal formulation in Indian patients with visceral leishmaniasis: a randomised, open-label study                                                                  & PLoS Negl Trop Dis       & Bharat Serums and Vaccines Ltd; Department of Science and Technology, Govt. of India & 4 locations in Bihar, India.                      & Randomised; open label; phase 3; safety/efficacy                                   & 2: LAMB 15mg/kg SDA vs. ABLE 15mg/kg SDA                                                                                       & 5--65       & 2009--2011   & 144\tnote{2} & 9 (6.3)                                                                                                                                                                                                                                                                                                                                                                                                                                                                                                                                \\ \midrule
      Sundar 2015\cite{sundar2011prot,sundar2015}     & Single-dose indigenous liposomal amphotericin B in the treatment of Indian visceral leishmaniasis: A phase 2 study                                                                                                          & Am J Trop Med Hyg        & Lifecare Innovations; Department of Science and Technology, Govt. of India           & KAMRC, Muzaffarpur, India                         & Non-randomised; non-comparative; open label; phase 2; safety/efficacy              & 2: LAMB (Lifecare Innovations) 10mg/kg SDA vs. 15mg/kg SDA                                                                     & 12--60      & 2012--2013   & 30           & 3 (10.0)                                                                                                                                                                                                                                                                                                                                                                                                                                                                                                                               \\ \midrule
      Sundar 2019\cite{sundar2012prot,sundar2019}     & Effectiveness of single-dose liposomal amphotericin B in visceral leishmaniasis in Bihar                                                                                                                                    & Am J Trop Med Hyg        & Banaras Hindu University                                                             & KAMRC, Muzaffarpur, India                         & Observational; efficacy                                                            & 1: LAMB 10mg/kg SDA                                                                                                            & All         & 2013--2017   & 928          & 33 (3.6)                                                                                                                                                                                                                                                                                                                                                                                                                                                                                                                               \\ \bottomrule
    \end{longtable}
  \end{ThreePartTable}
\end{landscape}
}
\pagestyle{fancy}
\restoregeometry

      \begin{figure}[tb]
        \centering
        \includegraphics[width=1.35\textwidth]{figures/ch5/treat.pdf}
        \caption{Bar chart showing the distribution of treatment regimens across contributing studies from the Indian subcontinent. Important distinguishing dosing information are provided in the overlaying labels, as space allows. Full treatment details are presented in Table~\ref{tab:isc_study}. 6 patients are included from Chakraborty 2008: 5 receiving alternate day amphotericin B deoxycholate and 1 receiving human placenta extract. ABD: amphotericin B deoxycholate; ALT: alternate days; CONS: consecutive days; D: days; HPE: human placenta extract; MF: miltefosine; PENT: pentamidine;~mg: milligrams/kilogram; SSG: sodium stibogluconate.}\label{fig:isc_treat}
      \end{figure}

    \end{landscape}
  }
\pagestyle{fancy}
\restoregeometry

\subsubsection{Study Design and Treatment Arms}

% for R: s <- c("das2009" = 73, "bhattacharya2007" = 352, "chakraborty2008" = 6, "koirala2003"=116,"pandey2016"=600,"pandey2017"=100, "rijal2003"=102, "rijal2010a"=32, "rijal2010b"=178,"sundar2007a"=250,"sundar2008"=225, "sundar2008b"=45, "sundar2009"=307, "sundar2010"=412, "sundar2011"=128,"sundar2012"=571,"sundar2014"=144,"sundar2015"=30,"sundar2019"=928)

Study designs are summarised in Table~\ref{tab:isc_study} as described in the publication or protocol. Four studies (428 [9.3\%] patients) were described as Phase 2 trials\cite{sundar2008,sundar2008B,sundar2011,sundar2015}, four studies (1,113 [24.2\%] patients) as Phase 3 trials\cite{sundar2007a,sundar2009,sundar2010,sundar2014}, two studies (468 [10.2\%] patients) as Phase 4 trials\cite{koirala2003,bhattacharya2007}, and one study (32 [0.7\%] patients) as a Phase 2/3 trial\cite{rijal2010a}.

Ten studies (1,625 [35.3\%] patients) allocated patients to more than one treatment arm\cite{das2009, sundar2007a,sundar2008, sundar2009, sundar2010, sundar2014, sundar2015, sundar2008B, chakraborty2008, rijal2010b}, of which six studies (1,411 [30.7\%] patients) were randomised\cite{das2009,sundar2007a,sundar2009,sundar2010,sundar2014} or partially randomised\cite{sundar2008}.

Treatment arms are summarised by study in Figure~\ref{fig:isc_treat}. The three largest studies were all monotherapy trials that treated with either 10~mg/kg single dose liposomal amphotericin B (1 study, 928 [20.2\%] patients)\cite{sundar2019}, or 28-day miltefosine (2 studies, 1,171 [25.5\%] patients).

Figure~\ref{fig:isc_treat} presents the different treatment regimens across all studies.

\subsubsection{Study Eligibility Criteria}

All studies recruited both male and female patients, and all but three studies restricted inclusion by age. Four studies only recruited patients aged 12 years and above (314 [6.8\%] patients)\cite{sundar2015,sundar2008B,sundar2008,rijal2010a}. Where specified, the \textit{lower} age limit was otherwise between 2 and 6 years. One study (100 [2.2\%] patients) recruited children only (< 15 years)\cite{pandey2017}. Otherwise, the \textit{upper} age limit ranged between 60 and 70 years.

In the 16 studies that included baseline haemoglobin level in their eligibility criteria, the median lower threshold was 50 g/L (range 35 to 60 g/L).\footnote{with one study reporting boosting haemoglobin with blood transfusions prior to recruitment\cite{das2009}.} Almost all studies excluded patients with serious illness or the presence of significant co-existing diseases. No study explicitly excluded patients based on malnutrition severity.

Twelve studies (2,021 [43.9\%] patients) explicitly reported performing HIV testing on all recruited patients\cite{koirala2003,bhattacharya2007,chakraborty2008,rijal2013,rijal2010a,sundar2007a,sundar2008,sundar2008B,sundar2009,sundar2010,sundar2014,sundar2015}, and patients with confirmed VL/HIV co-infection were excluded from all but two studies (280 [6.1\%] patients)\cite{rijal2003,rijal2010b}. Patients with HIV/VL co-infection from these two studies were excluded at the IPD level, where reported.

Inclusion criteria based on prior VL treatment were reported in 12 studies (2,629 [57.2\%] patients)\cite{sundar2019,sundar2015,sundar2014,sundar2011,sundar2010,sundar2009,sundar2008B,sundar2007a,rijal2010b,rijal2010a,rijal2003,das2009}, with timing since last VL treatment ranging from 10 days\cite{sundar2008B} to lifelong\cite{sundar2019}. One study (73 [1.6\%] patients) only recruited patients who had failed treatment with SSG\cite{das2009}.

Specific wording of the inclusion and exclusion criteria, and exclusion thresholds for platelets, white blood cells, clotting, renal, and liver function, are provided in the \href{https://github.com/jpwil/dphil}{Supplementary Material}.

\subsubsection{Diagnostic Criteria}

The majority of studies specified clinical criteria for inclusion, with six studies (1,984 [43.1\%] patients) requiring $\geq$ 2 weeks' fever and splenomegaly\cite{sundar2019,sundar2014,rijal2010b,rijal2010a,rijal2003,pandey2016}. The remaining studies either did not specify the clinical criteria (two studies, 36 [0.8\%] patients)\cite{chakraborty2008,sundar2015}, or referred more generally to the presence of typical signs and symptoms (11 studies, 2,579 [56.1\%] patients).

VL was confirmed by tissue aspirate in all but two studies (1,528 [33.2\%] patients)\cite{pandey2016,sundar2019}, where the rK39 RDT was instead used as the primary diagnostic method. Three studies (276 [6.0\%] patients)\cite{pandey2017,rijal2010a,sundar2014}, screened patients with the rK39 RDT prior to confirmation with a tissue aspirate.

\subsubsection{Relapse}

Nine studies directly defined relapse in their publications or protocols\cite{sundar2019,koirala2003,sundar2007a,rijal2010a, rijal2010b,rijal2003,pandey2017,das2009,bhattacharya2007}. In the remaining studies, relapse events could be inferred indirectly from definitions of initial cure and treatment failure/success.  Confirmation with tissue aspirates was performed in 13 studies, and reported (at least partially) in the IPD in 12 studies (see \href{https://github.com/jpwil/dphil}{Supplementary Material} for study-specific details).

In the three studies that reported relapse beyond 6 months\cite{sundar2019,sundar2008,rijal2010b}, IPD interrogation allowed identification of the subset of patients where relapse occurred within 6 months of initial cure assessment. Active follow-up strategies were used to identify relapse events in all but one study (928 [20.2\%] patients)\cite{sundar2019}.

\subsubsection{Initial Cure}

Initial cure was assessed 28--30 days after treatment initiation in 14 studies (3,740 [81.3\%] patients)\cite{bhattacharya2007,chakraborty2008,das2009,pandey2016,pandey2017,rijal2003,rijal2010b,sundar2010,sundar2011,sundar2012,sundar2014,sundar2015,sundar2019,koirala2003}. In the remaining five studies, initial cure was assessed either at two different time points, depending on the treatment arm (two studies, 532 [11.6\%] patients)\cite{sundar2007a,sundar2009}, or between days 16 and 19 (three studies, 302 [6.6\%] patients)\cite{rijal2010a,sundar2008,sundar2008B}. Clinical criteria for initial cure were described in all but two studies\cite{bhattacharya2007,chakraborty2008}, although often loosely defined (see \href{https://github.com/jpwil/dphil}{Supplementary Material} for study-specific details).

Tissue aspirate formed part of the initial cure assessment in all but one study (928 [20.2\%] of patients without tissue aspirate)\cite{sundar2019}. In nine studies (1,378 [30.0\%] patients)\cite{bhattacharya2007,koirala2003,rijal2003,rijal2010a,sundar2007a,sundar2008B,sundar2009,sundar2014,sundar2015}, patients with an initial $1+$ parasite grade underwent a repeat aspirate 10 to 60 days later. If the repeat aspirate were negative, the patient would often be considered a `slow-responder', allowing progression to either definite cure or relapse.

\subsection{Patient Characteristics}

Overall (marginal) distributions of categorical and continuous variables are tabulated in Tables~\ref{tab:isc_categorical} and~\ref{tab:isc_continuous}, respectively. These distributions are also displayed graphically in Figure~\ref{fig:isc_cat_comb} for bar charts of categorical variables, and Figures~\ref{fig:isc_pooled_dist_cont_nolab} and~\ref{fig:isc_pooled_dist_cont_lab} for histograms of non-laboratory and laboratory variables, respectively. Study-specific distributions are presented for age, sex, and relapse in Figure~\ref{fig:isc_main_dist}. In the Appendix, study-specific distributions of all categorical and continuous variables are presented in Figures~\ref{fig:isc_comb_dist_cat} and~\ref{fig:isc_age_comb}--\ref{fig:isc_cr_log_comb}, respectively.

Patient numbers and proportions presented in this section exclude missing data. See Section \ref{sec:isc_missing_data} for further information on missing data.

% these are the study specific distributions of outcome, sex and age
% cp /Users/jameswilson/proj/vl_model_isc/figures/dist/main_dist.pdf figures/ch5/isc_main_dist.pdf
\newgeometry{left=1cm, bottom=2.5cm, right=2cm, top=3cm}
\begin{landscape}
  \begin{figure}[tb]
    \centering
    \includegraphics[width=1.35\textwidth]{figures/ch5/isc_main_dist.pdf}
    \caption{Graphical summary of the Indian subcontinent study-specific sample sizes and distributions of relapse status, sex, and age.}
    \label{fig:isc_main_dist}
  \end{figure}
\end{landscape}
\restoregeometry

\subsubsection{Categorical variables}

Across all studies, 2,745 (59.7\%) of patients were male, ranging from 50.0\% to 75.0\% at the study level.

Relapse within 6 months was identified in 228 (5.0\%) patients, varying from 0\% to 10.4\% at the study level.

The majority of patients (52.4\%) had mild/normal malnutrition according to the definition provided in Section \ref{sec:malnutrition}. Moderate and severe malnutrition affected 28.7\% and 18.9\% of patients overall, although these proportions varied significantly across studies and were affected by considerable missing data (see Appendix Figure~\ref{fig:isc_comb_dist_cat}).

Approximately half of all patients (2,051 [49.8\%] patients) had severe anaemia at the time of recruitment, as defined by the 2024 WHO guidelines\cite{who_haem2024}.

At the patient level, the most common treatment regimen was 28 days of oral miltefosine (1,639 [35.6\%] patients), followed by 10~mg/kg single dose liposomal amphotericin B (1,331 [28.9\%] patients). The remaining treatment regimens (1,629 [35.4\%] patients), consisted of a broad range of experimental and non-experimental regimens, including amphotericin B deoxycholate, other (non-Gilead) lipid formulations of amphotericin B, paromomycin, SSG, pentamidine, and human placenta extract (Figure~\ref{fig:isc_treat}).

The most common baseline parasite grade on tissue aspirate was $1+$ (1,201 [40.7\%] patients), while the median grade was $2+$ (764 [25.9\%] patients at this grade; IQR: 1 to 3). The proportion of patients decreased with increasing parasite grade, with 54 (1.8\%) patients with a parasite grade of $5+$. Where reported, 92.0\% (2,717 patients) of tissue aspirates were obtained from the spleen. The remaining aspirates were from bone marrow and accounted for almost all aspirates performed in three studies from Nepal\cite{rijal2010a,rijal2003,koirala2003} (Appendix Figure~\ref{fig:isc_comb_dist_cat}).

% categorical variables table
\begin{table}[htbp]
    \centering
    \small
    \begin{threeparttable}
        \begin{tabular}{@{} l R{1.3cm} @{\hspace{4pt}} L{1.3cm} R{1.3cm} @{\hspace{4pt}} L{1.3cm} R{0.89cm} @{\hspace{4pt}} L{1.3cm} @{}}
            \toprule
            \textbf{Variable}                 & \multicolumn{2}{c}{\textbf{Overall (\%)}} & \multicolumn{2}{c}{\textbf{Final cure (\%)}} & \multicolumn{2}{c}{\textbf{Relapse (\%)}}                         \\
                                              & \multicolumn{2}{c}{n~=~4,599}             & \multicolumn{2}{c}{n~=~4,371}                & \multicolumn{2}{c}{n~=~228}                                       \\
            \midrule
            \textbf{Sex}                      &                                           &                                              &                                           &        &     &        \\
            \hspace{1em} Female               & 1,854                                     & (40.3)                                       & 1,771                                     & (40.5) & 83  & (36.4) \\
            \hspace{1em} Male                 & 2,745                                     & (59.7)                                       & 2,600                                     & (59.5) & 145 & (63.6) \\
            \textbf{Malnutrition}             &                                           &                                              &                                           &        &     &        \\
            \hspace{1em}Normal/mild           & 1,403                                     & (30.5)                                       & 1,333                                     & (30.5) & 70  & (30.7) \\
            \hspace{1em}Moderate              & 769                                       & (16.7)                                       & 732                                       & (16.7) & 37  & (16.2) \\
            \hspace{1em}Severe                & 506                                       & (11.0)                                       & 485                                       & (11.1) & 21  & (9.2)  \\
            \hspace{1em}(Missing)             & 1,921                                     & (41.8)                                       & 1,821                                     & (41.7) & 100 & (43.9) \\
            \textbf{Anaemia}                  &                                           &                                              &                                           &        &     &        \\
            \hspace{1em}Non-severe            & 2,089                                     & (45.4)                                       & 1,973                                     & (45.1) & 116 & (50.9) \\
            \hspace{1em}Severe                & 2,071                                     & (45.0)                                       & 1,985                                     & (45.4) & 86  & (37.7) \\
            \hspace{1em}(Missing)             & 439                                       & (9.5)                                        & 413                                       & (9.4)  & 26  & (11.4) \\
            \textbf{Treatment}                &                                           &                                              &                                           &        &     &        \\
            \hspace{1em}Miltefosine\tnote{1}  & 1,639                                     & (35.6)                                       & 1,526                                     & (34.9) & 113 & (49.6) \\
            \hspace{1em}Other                 & 1,629                                     & (35.4)                                       & 1,562                                     & (35.7) & 67  & (29.4) \\
            \hspace{1em}LAMB\tnote{2}         & 1,331                                     & (28.9)                                       & 1,283                                     & (29.4) & 48  & (21.1) \\
            \textbf{Parasite grade}           &                                           &                                              &                                           &        &     &        \\
            \hspace{1em}1+                    & 1,201                                     & (26.1)                                       & 1,152                                     & (26.4) & 49  & (21.5) \\
            \hspace{1em}2+                    & 764                                       & (16.6)                                       & 733                                       & (16.8) & 31  & (13.6) \\
            \hspace{1em}3+                    & 610                                       & (13.3)                                       & 567                                       & (13.0) & 43  & (18.9) \\
            \hspace{1em}4+                    & 323                                       & (7.0)                                        & 301                                       & (6.9)  & 22  & (9.6)  \\
            \hspace{1em}5+                    & 54                                        & (1.2)                                        & 53                                        & (1.2)  & 1   & (0.4)  \\
            \hspace{1em}(Missing)             & 1,647                                     & (35.8)                                       & 1,565                                     & (35.8) & 82  & (36.0) \\
            \textbf{Aspirate source}\tnote{3} &                                           &                                              &                                           &        &     &        \\
            \hspace{1em}Bone                  & 235                                       & (8.0)                                        & 221                                       & (7.9)  & 14  & (9.5)  \\
            \hspace{1em}Spleen                & 2,717                                     & (92.0)                                       & 2,585                                     & (92.1) & 132 & (90.4) \\
            \bottomrule
        \end{tabular}
        \begin{tablenotes}
            \footnotesize
            \item[1] 28 days of linear-dosed miltefosine at standard dosing.
            \item[2] Single dose liposomal amphotericin (Gilead) B 10mg/kg.
            \item[3] Denominator for \% in aspirate source: number of patients with documented parasite grade (overall: 2,952; final cure: 2,806; relapse: 146; no missing data).
        \end{tablenotes}
    \end{threeparttable}
    \caption{Summary of categorical candidate predictors and parasite source across contributed studies from the Indian subcontinent. Missing data are presented where present. SDA: Single dose liposomal amphotericin B 10mg/kg.}
    \label{tab:isc_categorical}
\end{table}

\subsubsection{Continuous variables}

The median age was 18 years, ranging from 1 to 80 years (IQR: 10 to 32 years). Overall the distribution was right-skewed, with study-specific distributions reflecting age-specific inclusion criteria. 138 (3.0\%) of patients were under 5 years, and 47 (1.0\%) were 65 years or over.

In adults ($\geq$ 19 years) the median BMI was 18.2~kg/m$^2$ (IQR: 16.4 to 20.8) (1,313 patients). In children ($\geq$ 5 and $<$ 19 years), the median BMI-for-age z-score was -1.68 (IQR: -2.67 to -0.72) (1,338 patients), and in younger children ($<$ 5 years) the median weight-for-height z-score was -1.80 (IQR: -2.97 to -0.99) (27 patients).

The median spleen size was 4 cm (IQR: 2 to 7 cm), and ranged from 0 to 22 cm. Eight studies reported patients without splenomegaly at baseline\cite{sundar2019, sundar2012, sundar2010, sundar2009,sundar2008,sundar2007a, pandey2017,bhattacharya2007}, corresponding to 218 (3.7\%) patients with recorded spleen sizes.

The median duration of fever prior to patient recruitment was 30 days (IQR: 20 to 60 days). The distribution was markedly right-skewed, ranging from 1 to 730 days.

For distributions of laboratory results (haemoglobin, white blood cells, platelets, creatinine, and alanine aminotransferase), please refer to Table~\ref{tab:isc_continuous} and Figure~\ref{fig:isc_pooled_dist_cont_lab}.

% continuous variables table
\begin{landscape}
    \begin{table}[htbp]
        \centering
        \small
        \begin{threeparttable}
            \begin{tabular}{@{} l @{} r @{\hspace{4pt}} l @{} r @{\hspace{4pt}} l @{} r @{\hspace{4pt}} l @{} r @{\hspace{4pt}} l @{} r @{\hspace{4pt}} l @{} r @{\hspace{4pt}} l @{}}
                \toprule
                \textbf{Variable}           & \multicolumn{4}{@{}c@{}}{\textbf{Overall} (n = 4,599)} & \multicolumn{4}{@{}c@{}}{\textbf{No relapse} (n = 4,371)} & \multicolumn{4}{@{}c@{}}{\textbf{Relapse} (n = 228)}                                                                                                                                        \\
                \cmidrule(r){2-5}\cmidrule(lr){6-9}\cmidrule(l){10-13}
                                            & Median                                                 & (IQR)                                                     & Missing\tnote{1}                                     & (\%)   & Median & (IQR)            & Missing\footnotemark[1] & (\%)   & Median & (IQR)            & Missing\footnotemark[1] & (\%)   \\
                \midrule
                Age (years)                 & 18                                                     & (10 -- 32)                                                & 6                                                    & (0.1)  & 18     & (10 -- 33)       & 6                       & (0.1)  & 14     & (8 -- 32)        & 0                       & (0.0)  \\
                Height (cm)                 & 150.0                                                  & (125.0 -- 161.5)                                          & 1,914                                                & (41.6) & 150.0  & (126.0 -- 161.0) & 1,815                   & (41.5) & 149.4  & (124.0 -- 164.6) & 99                      & (43.4) \\
                Weight (kg)                 & 36                                                     & (21 -- 46)                                                & 110                                                  & (2.4)  & 36     & (21 -- 46)       & 108                     & (2.5)  & 35     & (19 -- 48)       & 2                       & (0.9)  \\
                BMI (kg/m$^2$)\tnote{2}     & 18.22                                                  & (16.37 -- 20.81)                                          & 3,286                                                & (39.0) & 18.17  & (16.33 -- 20.70) & 3,118                   & (39.1) & 19.77  & (17.24 -- 24.07) & 168                     & (38.1) \\
                BMI-FA z-score\tnote{3}     & -1.68                                                  & (-2.67 -- -0.72)                                          & 3,261                                                & (41.9) & -1.67  & (-2.65 -- -0.71) & 3,098                   & (41.5) & -1.99  & (-2.81 -- -1.11) & 163                     & (48.0) \\
                WFH z-score\tnote{4}        & -1.80                                                  & (-2.97 -- -0.99)                                          & 4,572                                                & (80.4) & -1.53  & (-2.51 -- -0.97) & 4,347                   & (81.8) & -2.73  & (-3.13 -- -2.27) & 225                     & (50.0) \\
                Spleen size (cm)            & 4                                                      & (2 -- 7)                                                  & 642                                                  & (14.0) & 4      & (2 -- 7)         & 595                     & (13.6) & 3      & (2 -- 6)         & 47                      & (20.6) \\
                Fever duration (days)       & 30                                                     & (20 -- 60)                                                & 1,779                                                & (38.7) & 30     & (20 -- 60)       & 1,667                   & (38.1) & 20     & (15 -- 30)       & 112                     & (49.1) \\
                Parasite grade              & 2                                                      & (1 -- 3)                                                  & 1,647                                                & (35.8) & 2      & (1 -- 3)         & 1,565                   & (35.8) & 2      & (1 -- 3)         & 82                      & (36.0) \\
                WBC ($\times 10^9$/L)       & 3.4                                                    & (2.5 -- 4.5)                                              & 435                                                  & (9.5)  & 3.4    & (2.4 -- 4.5)     & 409                     & (9.4)  & 3.4    & (2.7 -- 4.5)     & 26                      & (11.4) \\
                Platelets ($\times 10^9$/L) & 112                                                    & (77 -- 155)                                               & 434                                                  & (9.4)  & 112    & (77 -- 155)      & 408                     & (9.3)  & 119.5  & (83 -- 160)      & 26                      & (11.4) \\
                Haemoglobin (g/L)           & 79                                                     & (67 -- 93)                                                & 433                                                  & (9.4)  & 79     & (67 -- 92)       & 407                     & (9.3)  & 82.5   & (69 -- 96)       & 26                      & (11.4) \\
                ALT (IU/L)                  & 31                                                     & (20 -- 52)                                                & 449                                                  & (9.8)  & 31.2   & (20 -- 52)       & 421                     & (9.6)  & 30     & (19 -- 52)       & 28                      & (12.3) \\
                Creatinine ($\mu$mol/L)     & 63.7                                                   & (51.3 -- 79.6)                                            & 646                                                  & (14.0) & 63.7   & (51.3 -- 79.6)   & 598                     & (13.7) & 63.7   & (51.3 -- 76.0)   & 48                      & (21.1) \\
                \bottomrule
            \end{tabular}
            \begin{tablenotes}
                \footnotesize
                \item[1] Denominator for missing \%: total number of patients in respective group (overall, relapse or no relapse). For measures of malnutrition (BMI, BMI-for-age z-score, and weight-for-height z-score), see further table footnotes.
                \item[2] Denominator for missing \%: number of patients aged $\geq$ 19 years, n = 2,154 (relapse: 97, no relapse: 2,057).
                \item[3] Denominator for missing \%: number of patients aged 5--18 year inclusive, n = 2,301 (relapse: 125, no relapse: 2,176).
                \item[4] Denominator for missing \%: number of patients aged $<$ 5 years, n = 138 (relapse: 6, no relapse: 132).
            \end{tablenotes}
        \end{threeparttable}
        \caption{Summary of continuous candidate predictors and additional variables used for the derivation of malnutrition status (height, weight, BMI, BMI-for-age z-score, weight-for-height z-score). Abbreviations: ALT: alanine aminotransferase; BMI(-FA): body mass index(-for age); cm: centimetres; IQR: inter-quartile range, IU: international units; kg: kilograms; L: litres; m: metres; WBC: white blood cells; WFH: weight-for-height; g: grams; $\mu$mol: micromoles.}
        \label{tab:isc_continuous}
    \end{table}
\end{landscape}

%  distributions of categorical variables
% cp /Users/jameswilson/proj/vl_model_isc/figures/dist/catOut/comb_cat.pdf figures/ch5/isc_cat_comb.pdf
\newgeometry{left=1cm, bottom=2.5cm, right=2cm, top=3cm}
\begin{landscape}
  \begin{figure}[tb]
    \centering
    \includegraphics[width=1.35\textwidth]{figures/ch5/isc_cat_comb.pdf}
    \caption{Marginal distributions and predictor-outcomes relationships for categorical candidate predictors. Excluding missing data. 95\% binomial confidence intervals calculated using the Wilson method. MF: miltefosine; Norm: normal; SDA: single dose liposomal amphotericin B.}\label{fig:isc_cat_comb}
  \end{figure}
\end{landscape}
\restoregeometry

\subsection{Univariable Associations}

Unadjusted relationships between variables (including all candidate predictors, excluding missing data) and relapse risk are presented in tabular form (Tables~\ref{tab:isc_categorical},~\ref{tab:isc_continuous}) and visually alongside their distributions (Figure~\ref{fig:isc_cat_comb} for categorical variables, and Figures~\ref{fig:isc_pooled_dist_cont_nolab} and~\ref{fig:isc_pooled_dist_cont_lab} for continuous non-laboratory and laboratory variables, respectively). The relationships are presented on the log-odds (logit) scale for continuous candidate predictors in Appendix Figure~\ref{fig:isc_logodds}.

Across continuous candidate predictors, GAM smooths suggested often non-linear relationships with relapse risk, with wider uncertainty at the extremes of the predictor distributions where data were sparse. Apparent trends were most evident for age, duration of fever and spleen size. Age showed a shallow U-shaped pattern, reaching a minimum relapse risk at approximately 20 years, while duration of fever showed a marked downward trend, with longer fever durations associated with lower relapse risk. With spleen size, a notable downward trend in risk was seen for spleen sizes over 2 cm, perhaps better appreciated on the logit scale in Appendix Figure~\ref{fig:isc_logodds}. For the laboratory predictors, weak upward trending monotonic patterns were observed for white blood cells and platelets, with higher values associated with increased relapse risk.

Trends in non-candidate predictors, including weight, height, and haemoglobin, were also apparent. A marked upward trend was seen with haemoglobin and relapse risk, which was also evidence in the categorical associations, with severe anaemia associated with a lower relapse risk compared to non-severe anaemia.

Treatment with miltefosine was associated with a higher unadjusted relapse risk when compared with 10~mg/kg single dose liposomal amphotericin B or `Other' treatment.

Patients with parasite grades of $3+$ and $4+$ were associated with higher relapse risk compared to patients with grades $1+$ and $2+$. Extrapolation of any trend to patients with $5+$ parasite grade was limited by small numbers.

A correlation matrix showing associations between continuous variables is presented in Appendix Figure~\ref{fig:isc_cont_cont}, and between continuous and categorical variables in Appendix Figure~\ref{fig:isc_cont_cat}. Inter-predictor correlations are considered further in the Discussion section.

% these are the pooled continuous distributions and relationships with relapse - PART 1
% cp /Users/jameswilson/proj/vl_model_isc/figures/dist/contOut/age_comb.pdf figures/ch5/isc_pool_age_comb.pdf
% cp /Users/jameswilson/proj/vl_model_isc/figures/dist/contOut/ss_comb.pdf figures/ch5/isc_pool_ss_comb.pdf
% cp /Users/jameswilson/proj/vl_model_isc/figures/dist/contOut/fd_comb.pdf figures/ch5/isc_pool_fd_comb.pdf
% cp /Users/jameswilson/proj/vl_model_isc/figures/dist/contOut/height_comb.pdf figures/ch5/isc_pool_height_comb.pdf
% cp /Users/jameswilson/proj/vl_model_isc/figures/dist/contOut/weight_comb.pdf figures/ch5/isc_pool_weight_comb.pdf

\clearpage
\begin{figure}[H]
  \centering
  \begin{subfigure}{\textwidth}
    \centering
    \begin{overpic}[width=\textwidth]{figures/ch5/isc_pool_age_comb.pdf}
      \put(2,19){\small Age}
    \end{overpic}
  \end{subfigure}
  \begin{subfigure}{\textwidth}
    \centering
    \begin{overpic}[width=\textwidth]{figures/ch5/isc_pool_weight_comb.pdf}
      \put(2,19){\small Weight}
    \end{overpic}
  \end{subfigure}
  \begin{subfigure}{\textwidth}
    \centering
    \begin{overpic}[width=\textwidth]{figures/ch5/isc_pool_height_comb.pdf}
      \put(2,19){\small Height}
    \end{overpic}
  \end{subfigure}
  \begin{subfigure}{\textwidth}
    \centering
    \begin{overpic}[width=\textwidth]{figures/ch5/isc_pool_fd_comb.pdf}
      \put(2,19){\small FevDur}
    \end{overpic}
  \end{subfigure}
  \begin{subfigure}{\textwidth}
    \centering
    \begin{overpic}[width=\textwidth]{figures/ch5/isc_pool_ss_comb.pdf}
      \put(2,19){\small SpnSize}
    \end{overpic}
  \end{subfigure}
  \caption{Distributions and predictor-outcome relationships for continuous non-laboratory candidate predictors. FevDur:~duration of fever; SpnSize:~spleen size. For each candidate predictor, left upper panel shows the overall density across studies and the left lower panel shows overlapping densities normalised by relapse status. The right panel shows a univariable generalised additive model spline fit, with 95\% confidence interval, of relapse.}
  \label{fig:isc_pooled_dist_cont_lab}
\end{figure}

% these are the pooled continuous distributions and relationships with relapse - PART 2
% cp /Users/jameswilson/proj/vl_model_isc/figures/dist/contOut/wbc_comb.pdf figures/ch5/isc_pool_wbc_comb.pdf
% cp /Users/jameswilson/proj/vl_model_isc/figures/dist/contOut/plt_comb.pdf figures/ch5/isc_pool_plt_comb.pdf
% cp /Users/jameswilson/proj/vl_model_isc/figures/dist/contOut/hb_comb.pdf figures/ch5/isc_pool_hb_comb.pdf
% cp /Users/jameswilson/proj/vl_model_isc/figures/dist/contOut/alt_comb.pdf figures/ch5/isc_pool_alt_comb.pdf
% cp /Users/jameswilson/proj/vl_model_isc/figures/dist/contOut/cr_comb.pdf figures/ch5/isc_pool_cr_comb.pdf
\begin{figure}[H]
  \centering
  \begin{subfigure}{\textwidth}
    \centering
    \begin{overpic}[width=\textwidth]{figures/ch5/isc_pool_hb_comb.pdf}
      \put(2,19){\small Hb}
    \end{overpic}
  \end{subfigure}
  \begin{subfigure}{\textwidth}
    \centering
    \begin{overpic}[width=\textwidth]{figures/ch5/isc_pool_plt_comb.pdf}
      \put(2,19){\small Plt}
    \end{overpic}
  \end{subfigure}
  \begin{subfigure}{\textwidth}
    \centering
    \begin{overpic}[width=\textwidth]{figures/ch5/isc_pool_wbc_comb.pdf}
      \put(2,19){\small WBC}
    \end{overpic}
  \end{subfigure}
  \begin{subfigure}{\textwidth}
    \centering
    \begin{overpic}[width=\textwidth]{figures/ch5/isc_pool_alt_comb.pdf}
      \put(2,19){\small ALT}
    \end{overpic}
  \end{subfigure}
  \begin{subfigure}{\textwidth}
    \centering
    \begin{overpic}[width=\textwidth]{figures/ch5/isc_pool_cr_comb.pdf}
      \put(2,19){\small Crt}
    \end{overpic}
  \end{subfigure}
  \caption{Marginal distributions and predictor-outcome relationships for continuous laboratory candidate predictors. All predictors presented on log scale. Hb:~haemoglobin; Plt:~platelet; WBC:~white blood cells; ALT:~alanine aminotransferase; Crt:~creatinine. For each candidate predictor, left upper panel shows the overall density across studies and the left lower panel shows overlapping densities normalised by relapse status. The right panel shows a univariable generalised additive model spline fit, with 95\% confidence interval, of relapse.}
  \label{fig:isc_pooled_dist_cont_nolab}
\end{figure}

\subsection{\label{sec:isc_missing_data}Missing Data}

Missing data were frequent among the candidate predictors, with 3,697 (80.4\%) of patients missing at least one candidate predictor data point. Just under half of all patients (2,177 [47.3\%] patients) were missing exactly one data point, and 1,520 (33.1\%) patients were missing two or more data points. No missing data were present in the outcome (relapse) or random-effect (study) variables.

The candidate predictor with the most missingness was malnutrition, which affected 1,921 (41.8\%) patients due to a lack of height information. After malnutrition, fever duration was missing in 1,779 (38.7\%) of patients, and parasite grade was missing in 1,647 (35.8\%) patients. Remaining predictors had under 15\% missingness. Missingness patterns, ordered by missingness in the candidate predictors, are presented in Figure~\ref{fig:isc_missing_summary} at the study level and overall.

\subsubsection{Multiple Imputation}

On review of the multiple imputation diagnostic plots (as described in Section \ref{sec:meth-missing} and available for review in the \href{https://github.com/jpwil/dphil}{Supplementary Material}), no clear or consistent patterns indicating violations of the missing-at-random assumption were apparent. Convergence was obtained well before the 20 iterations for the majority of imputed predictors, and the distributional assumptions of the imputation model appeared robust on inspection of the diagnostic density and scatter plots.

% missing data figure
% cp /Users/jameswilson/proj/vl_model_isc/figures/missing/summary.pdf figures/ch5/isc_missing_summary.pdf
\begin{figure}[tb]
  \centering
  \includegraphics[scale = 0.8, trim = 0 25 0 0]{figures/ch5/isc_missing_summary.pdf}
  \caption{Density plot illustrating the amount of missing data overall and across contributing studies from the Indian subcontinent. Study ordered by lead author and year of publication (or protocol). Variables ordered by amount of missingness. ALT: alanine aminotransferase; WBC: white blood cells.}\label{fig:isc_missing_summary}
\end{figure}

\section{Model Results}

Two models were fitted to the ISC IPD \textemdash\ one model including parasite grade as a candidate predictor, and one model excluding parasite grade.

\subsection{Model Specification and Coefficient Estimates}

Forest plots of the final model predictors are presented in Figure~\ref{fig:isc_var_forest_combined}. Full specification of the final models, including intercept terms, p-values, and predictor transformations, are presented in Appendix Tables~\ref{tab:isc_model_coeff_with_pg} and~\ref{tab:isc_model_coeff_without_pg}.

For both models, final predictors included: age (both linear and squared terms), duration of fever, treatment regimen, and presence of severe anaemia. Parasite grade was also included when considered as a candidate predictor. As can be appreciated from Figure~\ref{fig:isc_var_forest_combined}, the adjusted odds ratios were similar across the two models. The similarity between models can also be appreciated in Figure~\ref{fig:isc_adjusted_assoc}, which shows relapse probability predictions with the model intercept recalibrated to the Sundar 2019 dataset\cite{sundar2019}.

In both models, U-shaped relationships between age and relapse risk were shown, reflecting a significant positive age squared coefficient. Inverse relationships between fever duration and relapse risk were also demonstrated. In the model including parasite grade, the relapse odds decreased by 36.0\% (95\% CI\@: 23.1--46.7\%) for each doubling of fever duration, with similar findings with the model excluding parasite grade. When included, a unit increase in parasite grade (e.g.\ from $1+$ to $2+$) was associated with a 30.6\% increase in relapse odds (95\% CI\@: 12.1--52.2\%). Relapse odds in the 10~mg/kg single dose liposomal amphotericin B, or `Other' treatment groups were significantly lower (approximately half) than the odds in the standard dose miltefosine regimen, although with marked uncertainty. The presence of severe anaemia was also found to be associated with decreased odds of relapse by approximately one third (32.0\%, 95\% CI\@: 7.0--50.3\%) in the model including parasite grade. Appendix Tables~\ref{tab:isc_model_coeff_with_pg} and~\ref{tab:isc_model_coeff_without_pg} provide the complete predictor coefficients and standard errors for both models.

Pooled estimates of intraclass correlation coefficients (ICC)\footnote{ICC quantifies the proportion of total variance in the latent propensity for the outcome attributable to between-study differences.} were 3.5\% for the model including parasite grade, and 4.4\% for the model excluding parasite grade.

% cp /Users/jameswilson/proj/vl_model_isc/results/var_forest_combined.pdf figures/ch5/var_forest_combined.pdf
\begin{figure}[tb]
  \centering
  \includegraphics[width=\textwidth]{figures/ch5/var_forest_combined.pdf}
  \caption{Forest plots of adjusted odds ratios with 95\% confidence intervals for final model predictors. For age, the odds ratio represents a combination of linear and squared effects, with age centred by the mean and scaled by the standard deviation. Please refer to Figure~\ref{fig:isc_adjusted_assoc} for the adjusted relapse probabilities after model intercept recalibration to the observed relapse rate in the Sundar 2019 dataset\cite{sundar2019}. SDA: single dose liposomal amphotericin B 10~mg/kg.}
  \label{fig:isc_var_forest_combined}
\end{figure}

% cp /Users/jameswilson/proj/vl_model_isc/figures/multiAssocM1.pdf figures/ch5/isc_multiassoc_with_pg.pdf
% cp /Users/jameswilson/proj/vl_model_isc/figures/multiAssocM2.pdf figures/ch5/isc_multiassoc_without_pg.pdf
\newgeometry{left=2.5cm, bottom=2.5cm, right=2cm, top=3cm}
\begin{landscape}
  \begin{figure}[tb]
    \centering
    \begin{subfigure}{1.4\textwidth}
      \centering
      \begin{overpic}[width=\textwidth]{figures/ch5/isc_multiassoc_without_pg.pdf}
        \put(2,31){\small\textbf{ISC model: without parasite grade}}
      \end{overpic}
    \end{subfigure}
    \begin{subfigure}{1.4\textwidth}
      \centering
      \begin{overpic}[width=\textwidth]{figures/ch5/isc_multiassoc_with_pg.pdf}
        \put(2,26){\small\textbf{ISC model: with parasite grade}}
      \end{overpic}
    \end{subfigure}
    \caption{Adjusted associations between final predictors and predicted relapse probability, as estimated from the final ISC prognostic models. Probabilities were calculated from optimism-adjusted models and following logistic recalibration (intercept-term only) to data contributed from Sundar 2019\cite{sundar2019}. Where not varying in the plot, predictions are standardised to a representative reference participant: median age (18 years), median fever duration (30 days), treated with 10~mg/kg single dose liposomal amphotericin B, with severe anaemia, and \textemdash\ for the model including parasite grade \textemdash\ a median parasite count of 2+.}\label{fig:isc_adjusted_assoc}
  \end{figure}
\end{landscape}
\restoregeometry

\subsection{Model Performance and Internal Validation}

Apparent and optimism-adjusted c-statistics and calibration slopes are presented in Table~\ref{tab:isc_performance}. When assessed in studies with $>$5 relapse events, model discrimination was estimated at 0.70 (95\% CI\@: 0.66--0.74) in the model including parasite grade, and 0.69 (95\% CI\@: 0.64--0.74) in the model excluding parasite grade. These estimates reduced to 0.68 and 0.67, respectively, after adjusting for optimism. No evidence of significant between-study heterogeneity was identified in either model, with forest plots presented in Figure~\ref{fig:isc_forestCIM1} (model including parasite grade) and Appendix Figure~\ref{fig:isc_forestCIM2} (excluding parasite grade).

\begin{table}[htbp]
  \begin{tabular}{@{}llll@{}}
    \toprule
                                   & Estimate (95\% CI) & Average  & Optimism-adjusted \\
                                   &                    & optimism & performance       \\
    \midrule
    \textbf{Model: with PG}        &                    &          &                   \\
    \hspace{1em} C-statistic       & 0.70 (0.66--0.74)  & 0.014    & 0.68              \\
    \hspace{1em} Calibration slope & 1.01 (0.69--1.32)  & 0.093    & 0.91              \\        \midrule
    \textbf{Model: without PG}     &                    &          &                   \\
    \hspace{1em} C-statistic       & 0.69 (0.64--0.74)  & 0.016    & 0.67              \\
    \hspace{1em} Calibration slope & 1.02 (0.68--1.35)  & 0.098    & 0.92              \\        \bottomrule
  \end{tabular}
  \caption{Apparent and optimism-adjusted performance measures. Abbreviations: CI: confidence interval; PG: parasite grade}
  \label{tab:isc_performance}
\end{table}

% cp /Users/jameswilson/proj/vl_model_isc/graphs/forestCIM1.pdf figures/ch5/forestCIM1.pdf
% forest plots for C-statistic
\begin{figure}[tb]
  \centering
  \begin{overpic}[width=\textwidth, trim={3.5cm 0cm 1cm 0.3cm}, clip]{figures/ch5/forestCIM1.pdf}
    \put(0,5.5){\fcolorbox{black}{white}{\scriptsize $\tau^2$ = 0; $I^2$ = 0\%; p = 0.73}}
  \end{overpic}
  \caption{Forest plot showing individual and pooled study c-statistics, for the model \textbf{including} parasite grade. For the model excluding parasite grade refer to Appendix \ref{fig:isc_forestCIM2}. Pooled c-statistics are presented from both fixed-effects and random-effects meta-analysis models, after excluding studies with $\leq$ 5 relapse events. Blue diamonds: pooled summary estimates with 95\% confidence intervals. For Chakraborty 2008, no relapse events occurred and c-statistic is therefore undefined.}
  \label{fig:isc_forestCIM1}
\end{figure}

Optimism in calibration slopes were estimated at 0.093 and 0.098 in the models including and excluding parasite grade, respectively. Resulting uniform shrinkage factors were 0.91 and 0.92, respectively. Multiplying together these shrinkage factors with the estimated model coefficients will result in the optimism-adjusted coefficients (Appendix Tables \ref{tab:isc_model_coeff_with_pg} and \ref{tab:isc_model_coeff_without_pg}). No significant evidence of between-study heterogeneity in calibration slope was identified, with forest plots presented in Figure~\ref{fig:isc_forestCalM1} (model including parasite grade) and Appendix Figure~\ref{fig:isc_forestCalM2} (model excluding parasite grade).

% cp /Users/jameswilson/proj/vl_model_isc/graphs/forestCalM1.pdf figures/ch5/forestCalM1.pdf
% forest plots for calibration
\begin{figure}[tb]
  \centering
  \begin{overpic}[width=\textwidth, trim={3.3cm 0cm 0.3cm 0.3cm}, clip]{figures/ch5/forestCalM1.pdf}
    \put(0,13){\fcolorbox{black}{white}{\scriptsize CITL: $\tau^2$ = 0.094; $I^2$ = 46.1\%; p = .01}}
    \put(0,9.52){\fcolorbox{black}{white}{\scriptsize CS: $\tau^2$ = 0; $I^2$ = 0\%; p = .74}}
  \end{overpic}
  \caption{Forest plots showing individual and pooled study calibration measures for the model \textbf{including} parasite grade. Left: calibration intercept (calibration-in-the-large, CITL); Right: calibration slope (CS). Blue diamonds: summary estimates with 95\% confidence intervals. Calibration measures not presented for Chakraborty 2008 due to no relapse events. Calibration slope not presented for Rijal 2010(A) due to only one relapse event and few total participants leading to failure of model convergence.}\label{fig:isc_forestCalM1}
\end{figure}


% calibration plots (overall)
% cp /Users/jameswilson/proj/vl_model_isc/graphs/calPlot.pdf figures/ch5/isc_calPlot.pdf
\begin{figure}[tb]
  \centering
  \begin{overpic}[width=\textwidth, trim = 0 0 0 -25]{figures/ch5/isc_calPlot.pdf}
    \put(2,52){\small \textbf{ISC model: with parasite grade}}
    \put(50,52){\small \textbf{ISC model: without parasite grade}}
  \end{overpic}
  \caption{Calibration plots showing observed versus predicted probabilities for deciles of predicted probability. Red dashed line represents perfect calibration.  Observed probabilities are presented with 95\% confidence intervals (black error bars). A generalised additive model is fitted to show the smoothed mean observed probability (blue dotted line) with 95\% confidence intervals (blue ribbon). Histograms, normalised by outcome, are overlaid to illustrate the distribution of relapses and cures across the expected probabilities.}
  \label{fig:isc_calPlot}
\end{figure}

% calibration plots (parasite grade and fever duration)
% cp /Users/jameswilson/proj/vl_model_isc/graphs/calPlotFD1.pdf figures/ch5/isc_calPlotFD1.pdf
\begin{figure}[tb]
  \centering
  \includegraphics[width=\textwidth]{figures/ch5/isc_calPlotFD1.pdf}
  \caption{Calibration plots for different fever durations (model including parasite grade).}
  \label{fig:isc_calPlotFD1}
\end{figure}

Visual inspection of the calibration plots comparing predicted and observed relapse probabilities showed minimal deviation from perfect calibration (Figure~\ref{fig:isc_calPlot}). Calibration plots showing predicted and observed relapse probabilities for fever duration (Figure~\ref{fig:isc_calPlotFD1} for model including parasite grade), and other predictors (Appendix Figures~\ref{fig:isc_calPlotPD_ap} to \ref{fig:isc_calPlotRx1_ap} for model including parasite grade and~\ref{fig:isc_calPlotFD2_ap} to~\ref{fig:isc_calPlotRx2_ap} excluding parasite grade) and also showed good overall agreement, with predicted relapse probabilities lying within the 95\% confidence intervals of the observed probabilities.

Calibration intercepts (calibration-in-the-large) were found to vary significantly between studies for both models, as can be appreciated from the forest plots in Figure~\ref{fig:isc_forestCalM1} for the model including parasite grade, and Appendix Figure~\ref{fig:isc_forestCalM2} for the model excluding parasite grade (test for heterogeneity; p = 0.01 and p = 0.002 for models including and excluding parasite grade).

The distribution of the final predictors selected across the 2~$\times$~500 bootstrap models are presented in Appendix Table~\ref{tab:isc_model_stability}. Reassuringly, the most frequently selected predictors correspond to the final predictors selected in both final models. However, a degree of instability is apparent, with predictors such as spleen size, sex, alanine aminotransferase, and the cubic age term being selected in 100--250 bootstraps.

\section{Summary}

In summary, two prognostic models predicting relapse were fitted to the ISC IPD \textemdash\ one including parasite grade as a candidate predictor and one excluding it. Both final models retained the same core predictors: age (modelled using linear and squared terms, yielding a U-shaped association), duration of fever (inverse association), treatment regimen (lower relapse risk with LAMB and `Other' regimens compared with miltefosine), and severe anaemia (lower relapse risk with increased severity). When parasite grade was included, it was retained and associated with a higher risk of relapse.

Discriminative performance was modest. The pooled apparent c-statistics were 0.70 (95\% CI 0.66--0.74) for the model including parasite grade and 0.69 (95\% CI 0.64--0.74) for the model excluding it, reducing after optimism correction to 0.68 and 0.67, respectively. Overall agreement between observed and predicted relapse risk was good across predictors, although significant between-study variation in calibration intercepts (calibration-in-the-large) was observed for both models.

The following section presents an analogous summary of the East Africa models using the same structure, enabling direct comparison with the ISC model results.
