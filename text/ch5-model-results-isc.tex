\begin{savequote}[8cm]
  Quote goes here.
  \qauthor{--- James Wilson}
\end{savequote}

\chapter{\label{ch:5-isc-model-results}Indian subcontinent model results}
\minitoc

\section{Descriptive analysis}

% flow diagram
\begin{figure}[tb]
  \centering
  \includegraphics[width = 0.8\textwidth, trim={4cm 0cm 2cm 2cm}, clip]{figures/ch5/isc_flow_chart.pdf}
  \caption[Indian subcontinent model: flow diagram]{Flow diagram showing the studies and participants excluded from Indian subcontinent model development, following application of the eligibility criteria. HIV: human immunodeficiency virus; IDDO: Infectious Diseases Data Observatory; ISC: Indian subcontinent; PKDL: post kala-azar dermal leishmaniasis; VL: visceral leishmaniasis.}
  \label{fig:isc_flow_diagram}
\end{figure}

\subsection{Study and patient characteristics}
% studies tables
% study characteristics
% need to have `\\` after caption entries, otherwise compiler hangs
% reduced font size, expanded margins, remove page numbers, footer & headers - this reduces to 3 pages instead of 4-5
% sum of n (model) = 4599; sum of relapses = 228. 

{
\newgeometry{left=1.5cm, right=1.5cm, top=2cm, bottom=1.5cm}
\captionsetup{width=1\textheight, font=scriptsize, skip=7pt}

\begin{landscape}
  % \singlespacing
  \pagestyle{empty}
  \scriptsize
  \begin{ThreePartTable}
    \begin{TableNotes}
      \item[1] Study name is composed of the lead author and year of publication or most recent protocol.
      \item[2] Contributed number of participants from KAMRC site only.
      \item[3] Amphotericin B deoxycholate arm not reported in publication.
      \item[4] Protocol not in public domain.
    \end{TableNotes}

    \begin{longtable}[c]{L{1.5cm} L{3.5cm} L{1.5cm} L{2.5cm} L{2.5cm} L{2.5cm} L{2.5cm} L{1.5cm} L{1.2cm} L{1.3cm} L{1.3cm}}
      \caption{Key characteristics of included studies, ordered by lead author and year of publication/protocol. Where information not presented in the publication, information is extracted from the study protocol. -: not reported; ABLE: Amphotericin B lipid emulsion (Bharat Serum and Vaccines Ltd.); alt: alternative; ABD: Amphotericin B deoxycholate; BMGF: Bill and Melinda Gates Foundation; BPKIHS: B.P. Koirala Institute of Health Sciences; D: Day(s); EC: European Commission; Govt.: Government; ICMR: Indian Council of Medical Research; IM: Intramuscular; KAMRC: Kala-azar Medical Research Center; LAMB: Liposomal amphotericin B (Gilead formulation unless otherwise specified); MF: Miltefosine; mg/kg: milligrams per kilogram; NCVBDC: National Center for Vector Borne Diseases Control; OD: Once daily; P: Publication; PATH: Program for Appropriate Technology in Health; PM: Paromomycin; Pr: Protocol; Ref: Reference; RMRIMS: Rajendra Memorial Research Institute of Medical Sciences; SDA: Single-dose administration; SSG: Sodium stibogluconate; TDR: UNICEF/UNDP/World Bank/WHO Special Programme for Research and Training in Tropical Diseases; UN: United Nations; WHO: World Health Organization\label{tab:isc_study}} \\
      \toprule
      Study\tnote{1}                                  & Title                                                                                                                                                                                                                       & Journal                  & Sponsor/funding                                                                      & Location(s)                                       & Study design                                                                       & Study arm(s)                                                                                                                   & Age (years) & Study period & n (model)    & Relapses (\%)                                                                                                                                                                                                                                                                                                                                                                                                                                                                                                                          \\ \midrule
      \endfirsthead

      \caption[]{continued}                                                                                                                                                                                                                                                                                                                                                                                                                                                                                                                                                                                                                                                                                                                                                                                                                                                                                                                                                                                                                                                                                                                                                                                                                                          \\
      \toprule
      Study\tnote{1}                                  & Title                                                                                                                                                                                                                       & Journal                  & Sponsor/funding                                                                      & Location(s)                                       & Study design                                                                       & Study arm(s)                                                                                                                   & Age (years) & Study period & n (model)    & Relapses (\%)                                                                                                                                                                                                                                                                                                                                                                                                                                                                                                                          \\ \midrule
      \endhead

      \multicolumn{11}{r}{\textit{continued on next page}}                                                                                                                                                                                                                                                                                                                                                                                                                                                                                                                                                                                                                                                                                                                                                                                                                                                                                                                                                                                                                                                                                                                                                                                                           \\
      \endfoot

      \insertTableNotes
      \endlastfoot

      Bhattacharya 2007\cite{bhattacharya2007}        & Phase 4 trial of miltefosine for the treatment of Indian visceral leishmaniasis                                                                                                                                             & J Infect Dis             & ICMR                                                                                 & 13 locations in Bihar, India (outpatient setting) & Open label; phase 4; safety/efficacy                                               & 1: MF 28D                                                                                                                      & 2--65       & 2002--2004   & 352\tnote{2} & 22 (6.3)                                                                                                                                                                                                                                                                                                                                                                                                                                                                                                                               \\ \midrule
      Chakraborty 2008\cite{chakraborty2008}          & Human placental extract offers protection against experimental visceral leishmaniasis: a pilot study for a phase-I clinical trial                                                                                           & Am J Trop Med Hyg        & Indian Council of Scientific and Industrial Research; Albert David Ltd.              & KAMRC, Muzaffarpur, India                         & ``Pre-phase 1''; pilot/preliminary                                                 & 2\tnote{3}: 2.06mg human placental extract IM, single dose (or ABD 1mg/kg alt. days for 30D)                                   & $\geq$5     & 2003--2005   & 6            & 0 (0)                                                                                                                                                                                                                                                                                                                                                                                                                                                                                                                                  \\ \midrule
      Das 2009\cite{das2009}                          & A controlled, randomized nonblinded clinical trial to assess the efficacy of amphotericin B deoxycholate as compared to pentamidine for the treatment of antimony unresponsive visceral leishmaniasis cases in Bihar, India & Ther Clin Risk Manag     & -                                                                                    & RMRIMS, Patna, India                              & Randomised; open label; efficacy                                                   & 2: ABD 1mg/kg alt. days for 30D vs pentamidine IM 4mg/kg alt days, 30D                                                         & 6--60       & 2002         & 73           & 5 (6.8)                                                                                                                                                                                                                                                                                                                                                                                                                                                                                                                                \\ \midrule
      Koirala 2003                                    & Phase IV trial of miltefosine in the treatment of visceral leishmaniasis                                                                                                                                                    & Protocol only\tnote{4}   & -                                                                                    & BPKIHS, Dharan, Nepal                             & Open label; phase 4; safety/efficacy                                               & 1: MF 28D                                                                                                                      & 2--65       & 2003--2004   & 116          & 12 (10.3)                                                                                                                                                                                                                                                                                                                                                                                                                                                                                                                              \\ \midrule
      Pandey 2016\cite{pandey2016}                    & Pharmacovigilance of miltefosine in treatment of visceral leishmaniasis in endemic areas of Bihar                                                                                                                           & Am J Trop Med Hyg        & NCVBDC, Govt. of India; World Bank                                                   & 4 locations in Bihar, India                       & Open label; safety/efficacy                                                        & 1: MF 28D                                                                                                                      & 6--70       & 2012--2015   & 600          & 45 (7.5)                                                                                                                                                                                                                                                                                                                                                                                                                                                                                                                               \\ \midrule
      Pandey 2017\cite{pandey2017}                    & Efficacy and safety of liposomal amphotericin B for visceral leishmaniasis in children and adolescents at a tertiary care center in Bihar, India                                                                            & Am J Trop  Med Hyg       & -                                                                                    & RMRIMS, Patna, India                              & Open label; safety/efficacy                                                        & 1: LAMB 10mg/kg SDA                                                                                                            & $<$15       & 2014--2016   & 100          & 2 (2.0)                                                                                                                                                                                                                                                                                                                                                                                                                                                                                                                                \\ \midrule
      Rijal 2003\cite{rijal2003}                      & Treatment of visceral leishmaniasis in south-eastern Nepal: decreasing efficacy of sodium stibogluconate and need for a policy to limit further decline                                                                     & Trans R Soc Trop Med Hyg & WHO, Geneva University Hospital; Novartis Foundation                                 & BPKIHS, Dharan, Nepal                             & Non-randomised; efficacy                                                           & 2: SSG 20mg/kg OD, 30D either (i) in hospital or (ii) first 5--7 days in hospital; extended to 40D if positive aspirate at 30D & All         & 2000--2001   & 102          & 1 (1.0)                                                                                                                                                                                                                                                                                                                                                                                                                                                                                                                                \\ \midrule
      Rijal 2010(A)\cite{rijal2010a}                  & Efficacy and safety of liposomal amphotericin B in Nepalese patients with visceral leishmaniasis                                                                                                                            & Protocol only            & TDR                                                                                  & BPKIHS, Dharan, Nepal                             & Phase 2/3; safety/efficacy                                                         & 1: LAMB, 3mg/kg OD, 5D                                                                                                         & 12--65      & 2010--2011   & 32           & 1 (3.1)                                                                                                                                                                                                                                                                                                                                                                                                                                                                                                                                \\ \midrule
      Rijal 2010(B)\cite{rijal2010b}                  & Clinical risk factors for therapeutic failure in kala-azar patients treated with pentavalent antimonials in Nepal                                                                                                           & Trans R Soc Trop Med Hyg & EC (5th Framework Programme)                                                         & BPKIHS, Dharan, Nepal                             & Prospective cohort                                                                 & 1: SSG 20mg/kg OD, 30D                                                                                                         & -           & 2001--2003   & 178          & 1 (0.6)                                                                                                                                                                                                                                                                                                                                                                                                                                                                                                                                \\ \midrule
      Sundar 2007\cite{sundar2007a,sundar2005prot}    & Injectable paromomycin for visceral leishmaniasis in India                                                                                                                                                                  & N Engl J Med             & PATH (including UN, BMGF, TDR)                                                       & 4 locations in Bihar, India                       & Randomised; open label; phase 3; non-inferiority; safety/efficacy                  & 2: PM 15mg/kg IM, 21D vs. ABD 1mg/kg alt. days, 30D                                                                            & 5--55       & 2003--2004   & 250\tnote{2} & 4 (1.6)                                                                                                                                                                                                                                                                                                                                                                                                                                                                                                                                \\ \midrule
      Sundar 2008(A)\cite{sundar2008protC,sundar2008} & New treatment approach in Indian visceral leishmaniasis: single-dose liposomal amphotericin B followed by short-course oral miltefosine                                                                                     & Clin Infect Dis          & Banaras Hindu University                                                             & KAMRC, Muzaffarpur, India                         & Partially randomised; open label; phase 2; non-comparative; sequential; triangular & 5: LAMB 5mg/kg SD alone, or comb. LAMB 3.75--5mg/kg SD + MF 7--14D                                                             & $\geq$12    & 2006--2007   & 225          & 7 (3.1)                                                                                                                                                                                                                                                                                                                                                                                                                                                                                                                                \\ \midrule
      Sundar 2008(B)\cite{sundar2008B}                & Safety of a pre-formulated amphotericin B lipid emulsion for the treatment of Indian kala-azar                                                                                                                              & Trop Med Int Health      & Bharat Serum and Vaccines Ltd.                                                       & KAMRC, Muzaffarpur, India                         & Non-randomised; non-comparative; open label; phase 2; safety/efficacy              & 3: ABLE OD for 3D at (i) 5mg/kg (ii) 4mg/kg (iii) 3mg/kg                                                                       & 12--65      & 2004--2005   & 45           & 4 (8.9)                                                                                                                                                                                                                                                                                                                                                                                                                                                                                                                                \\ \midrule
      Sundar 2009\cite{sundar2008protB, sundar2009}   & Short-course paromomycin treatment of visceral leishmaniasis in India: 14-day vs 21-day treatment                                                                                                                           & Clin Infect Dis          & Banaras Hindu University                                                             & KAMRC, Muzaffarpur, India                         & Randomised; open label; phase 3; safety/efficacy                                   & 2: PM OD 15mg/kg/day for either (i) 21D or (ii) 14D                                                                            & 5--55       & 2007--2008   & 307          & 26 (8.5)                                                                                                                                                                                                                                                                                                                                                                                                                                                                                                                               \\ \midrule
      Sundar 2010\cite{sundar2008protA, sundar2010}   & Single-dose liposomal amphotericin B for visceral leishmaniasis in India                                                                                                                                                    & N Engl J Med             & Banaras Hindu University                                                             & KAMRC, Muzaffarpur, India                         & Randomised; open label; phase 3; safety/efficacy                                   & 2: ABD 1mg/kg alt. days for 30D, vs. LAMB 10mg/kg SDA                                                                          & 2--65       & 2008--2009   & 412          & 14 (3.4)                                                                                                                                                                                                                                                                                                                                                                                                                                                                                                                               \\ \midrule
      Sundar 2011\cite{sundar2006prot,sundar2011}     & Ambisome plus miltefosine for Indian patients with kala-azar                                                                                                                                                                & Trans R Soc Trop Med Hyg & Banaras Hindu University                                                             & KAMRC, Muzaffarpur; RMRIMS, Patna, India          & Open label; phase 2; safety/efficacy                                               & 1: LAMB 5mg/kg SDA followed by MF D2-D15                                                                                       & 2--65       & 2007--2009   & 128          & 5 (3.9)                                                                                                                                                                                                                                                                                                                                                                                                                                                                                                                                \\ \midrule
      Sundar 2012\cite{sundar2012}                    & Efficacy of miltefosine in the treatment of visceral leishmaniasis in India after a decade of use                                                                                                                           & Clin Infect Dis          & EC (Kaladrug-R); Sitaram Memorial Trust                                              & KAMRC, Muzaffarpur, India                         & Open label; safety/efficacy                                                        & 1: MF 28D                                                                                                                      & 6--70       & 2009--2010   & 571          & 34 (6.0)                                                                                                                                                                                                                                                                                                                                                                                                                                                                                                                               \\ \midrule
      Sundar 2014\cite{sundar2009prot, sundar2014}    & Efficacy and safety of amphotericin B emulsion versus liposomal formulation in Indian patients with visceral leishmaniasis: a randomised, open-label study                                                                  & PLoS Negl Trop Dis       & Bharat Serums and Vaccines Ltd; Department of Science and Technology, Govt. of India & 4 locations in Bihar, India.                      & Randomised; open label; phase 3; safety/efficacy                                   & 2: LAMB 15mg/kg SDA vs. ABLE 15mg/kg SDA                                                                                       & 5--65       & 2009--2011   & 144\tnote{2} & 9 (6.3)                                                                                                                                                                                                                                                                                                                                                                                                                                                                                                                                \\ \midrule
      Sundar 2015\cite{sundar2011prot,sundar2015}     & Single-dose indigenous liposomal amphotericin B in the treatment of Indian visceral leishmaniasis: A phase 2 study                                                                                                          & Am J Trop Med Hyg        & Lifecare Innovations; Department of Science and Technology, Govt. of India           & KAMRC, Muzaffarpur, India                         & Non-randomised; non-comparative; open label; phase 2; safety/efficacy              & 2: LAMB (Lifecare Innovations) 10mg/kg SDA vs. 15mg/kg SDA                                                                     & 12--60      & 2012--2013   & 30           & 3 (10.0)                                                                                                                                                                                                                                                                                                                                                                                                                                                                                                                               \\ \midrule
      Sundar 2019\cite{sundar2012prot,sundar2019}     & Effectiveness of single-dose liposomal amphotericin B in visceral leishmaniasis in Bihar                                                                                                                                    & Am J Trop Med Hyg        & Banaras Hindu University                                                             & KAMRC, Muzaffarpur, India                         & Observational; efficacy                                                            & 1: LAMB 10mg/kg SDA                                                                                                            & All         & 2013--2017   & 928          & 33 (3.6)                                                                                                                                                                                                                                                                                                                                                                                                                                                                                                                               \\ \bottomrule
    \end{longtable}
  \end{ThreePartTable}
\end{landscape}
}
\pagestyle{fancy}
\restoregeometry

% categorical variables table
\begin{table}[htbp]
    \centering
    \small
    \begin{threeparttable}
        \begin{tabular}{@{} l R{1.3cm} @{\hspace{4pt}} L{1.3cm} R{1.3cm} @{\hspace{4pt}} L{1.3cm} R{0.89cm} @{\hspace{4pt}} L{1.3cm} @{}}
            \toprule
            \textbf{Variable}                 & \multicolumn{2}{c}{\textbf{Overall (\%)}} & \multicolumn{2}{c}{\textbf{Final cure (\%)}} & \multicolumn{2}{c}{\textbf{Relapse (\%)}}                         \\
                                              & \multicolumn{2}{c}{n~=~4,599}             & \multicolumn{2}{c}{n~=~4,371}                & \multicolumn{2}{c}{n~=~228}                                       \\
            \midrule
            \textbf{Sex}                      &                                           &                                              &                                           &        &     &        \\
            \hspace{1em} Female               & 1,854                                     & (40.3)                                       & 1,771                                     & (40.5) & 83  & (36.4) \\
            \hspace{1em} Male                 & 2,745                                     & (59.7)                                       & 2,600                                     & (59.5) & 145 & (63.6) \\
            \textbf{Malnutrition}             &                                           &                                              &                                           &        &     &        \\
            \hspace{1em}Normal/mild           & 1,403                                     & (30.5)                                       & 1,333                                     & (30.5) & 70  & (30.7) \\
            \hspace{1em}Moderate              & 769                                       & (16.7)                                       & 732                                       & (16.7) & 37  & (16.2) \\
            \hspace{1em}Severe                & 506                                       & (11.0)                                       & 485                                       & (11.1) & 21  & (9.2)  \\
            \hspace{1em}(Missing)             & 1,921                                     & (41.8)                                       & 1,821                                     & (41.7) & 100 & (43.9) \\
            \textbf{Anaemia}                  &                                           &                                              &                                           &        &     &        \\
            \hspace{1em}Non-severe            & 2,089                                     & (45.4)                                       & 1,973                                     & (45.1) & 116 & (50.9) \\
            \hspace{1em}Severe                & 2,071                                     & (45.0)                                       & 1,985                                     & (45.4) & 86  & (37.7) \\
            \hspace{1em}(Missing)             & 439                                       & (9.5)                                        & 413                                       & (9.4)  & 26  & (11.4) \\
            \textbf{Treatment}                &                                           &                                              &                                           &        &     &        \\
            \hspace{1em}Miltefosine\tnote{1}  & 1,639                                     & (35.6)                                       & 1,526                                     & (34.9) & 113 & (49.6) \\
            \hspace{1em}Other                 & 1,629                                     & (35.4)                                       & 1,562                                     & (35.7) & 67  & (29.4) \\
            \hspace{1em}LAMB\tnote{2}         & 1,331                                     & (28.9)                                       & 1,283                                     & (29.4) & 48  & (21.1) \\
            \textbf{Parasite grade}           &                                           &                                              &                                           &        &     &        \\
            \hspace{1em}1+                    & 1,201                                     & (26.1)                                       & 1,152                                     & (26.4) & 49  & (21.5) \\
            \hspace{1em}2+                    & 764                                       & (16.6)                                       & 733                                       & (16.8) & 31  & (13.6) \\
            \hspace{1em}3+                    & 610                                       & (13.3)                                       & 567                                       & (13.0) & 43  & (18.9) \\
            \hspace{1em}4+                    & 323                                       & (7.0)                                        & 301                                       & (6.9)  & 22  & (9.6)  \\
            \hspace{1em}5+                    & 54                                        & (1.2)                                        & 53                                        & (1.2)  & 1   & (0.4)  \\
            \hspace{1em}(Missing)             & 1,647                                     & (35.8)                                       & 1,565                                     & (35.8) & 82  & (36.0) \\
            \textbf{Aspirate source}\tnote{3} &                                           &                                              &                                           &        &     &        \\
            \hspace{1em}Bone                  & 235                                       & (8.0)                                        & 221                                       & (7.9)  & 14  & (9.5)  \\
            \hspace{1em}Spleen                & 2,717                                     & (92.0)                                       & 2,585                                     & (92.1) & 132 & (90.4) \\
            \bottomrule
        \end{tabular}
        \begin{tablenotes}
            \footnotesize
            \item[1] 28 days of linear-dosed miltefosine at standard dosing.
            \item[2] Single dose liposomal amphotericin (Gilead) B 10mg/kg.
            \item[3] Denominator for \% in aspirate source: number of patients with documented parasite grade (overall: 2,952; final cure: 2,806; relapse: 146; no missing data).
        \end{tablenotes}
    \end{threeparttable}
    \caption{Summary of categorical candidate predictors and parasite source across contributed studies from the Indian subcontinent. Missing data are presented where present. SDA: Single dose liposomal amphotericin B 10mg/kg.}
    \label{tab:isc_categorical}
\end{table}

% continuous variables table
\begin{landscape}
    \begin{table}[htbp]
        \centering
        \small
        \begin{threeparttable}
            \begin{tabular}{@{} l @{} r @{\hspace{4pt}} l @{} r @{\hspace{4pt}} l @{} r @{\hspace{4pt}} l @{} r @{\hspace{4pt}} l @{} r @{\hspace{4pt}} l @{} r @{\hspace{4pt}} l @{}}
                \toprule
                \textbf{Variable}           & \multicolumn{4}{@{}c@{}}{\textbf{Overall} (n = 4,599)} & \multicolumn{4}{@{}c@{}}{\textbf{No relapse} (n = 4,371)} & \multicolumn{4}{@{}c@{}}{\textbf{Relapse} (n = 228)}                                                                                                                                        \\
                \cmidrule(r){2-5}\cmidrule(lr){6-9}\cmidrule(l){10-13}
                                            & Median                                                 & (IQR)                                                     & Missing\tnote{1}                                     & (\%)   & Median & (IQR)            & Missing\footnotemark[1] & (\%)   & Median & (IQR)            & Missing\footnotemark[1] & (\%)   \\
                \midrule
                Age (years)                 & 18                                                     & (10 -- 32)                                                & 6                                                    & (0.1)  & 18     & (10 -- 33)       & 6                       & (0.1)  & 14     & (8 -- 32)        & 0                       & (0.0)  \\
                Height (cm)                 & 150.0                                                  & (125.0 -- 161.5)                                          & 1,914                                                & (41.6) & 150.0  & (126.0 -- 161.0) & 1,815                   & (41.5) & 149.4  & (124.0 -- 164.6) & 99                      & (43.4) \\
                Weight (kg)                 & 36                                                     & (21 -- 46)                                                & 110                                                  & (2.4)  & 36     & (21 -- 46)       & 108                     & (2.5)  & 35     & (19 -- 48)       & 2                       & (0.9)  \\
                BMI (kg/m$^2$)\tnote{2}     & 18.22                                                  & (16.37 -- 20.81)                                          & 3,286                                                & (39.0) & 18.17  & (16.33 -- 20.70) & 3,118                   & (39.1) & 19.77  & (17.24 -- 24.07) & 168                     & (38.1) \\
                BMI-FA z-score\tnote{3}     & -1.68                                                  & (-2.67 -- -0.72)                                          & 3,261                                                & (41.9) & -1.67  & (-2.65 -- -0.71) & 3,098                   & (41.5) & -1.99  & (-2.81 -- -1.11) & 163                     & (48.0) \\
                WFH z-score\tnote{4}        & -1.80                                                  & (-2.97 -- -0.99)                                          & 4,572                                                & (80.4) & -1.53  & (-2.51 -- -0.97) & 4,347                   & (81.8) & -2.73  & (-3.13 -- -2.27) & 225                     & (50.0) \\
                Spleen size (cm)            & 4                                                      & (2 -- 7)                                                  & 642                                                  & (14.0) & 4      & (2 -- 7)         & 595                     & (13.6) & 3      & (2 -- 6)         & 47                      & (20.6) \\
                Fever duration (days)       & 30                                                     & (20 -- 60)                                                & 1,779                                                & (38.7) & 30     & (20 -- 60)       & 1,667                   & (38.1) & 20     & (15 -- 30)       & 112                     & (49.1) \\
                Parasite grade              & 2                                                      & (1 -- 3)                                                  & 1,647                                                & (35.8) & 2      & (1 -- 3)         & 1,565                   & (35.8) & 2      & (1 -- 3)         & 82                      & (36.0) \\
                WBC ($\times 10^9$/L)       & 3.4                                                    & (2.5 -- 4.5)                                              & 435                                                  & (9.5)  & 3.4    & (2.4 -- 4.5)     & 409                     & (9.4)  & 3.4    & (2.7 -- 4.5)     & 26                      & (11.4) \\
                Platelets ($\times 10^9$/L) & 112                                                    & (77 -- 155)                                               & 434                                                  & (9.4)  & 112    & (77 -- 155)      & 408                     & (9.3)  & 119.5  & (83 -- 160)      & 26                      & (11.4) \\
                Haemoglobin (g/L)           & 79                                                     & (67 -- 93)                                                & 433                                                  & (9.4)  & 79     & (67 -- 92)       & 407                     & (9.3)  & 82.5   & (69 -- 96)       & 26                      & (11.4) \\
                ALT (IU/L)                  & 31                                                     & (20 -- 52)                                                & 449                                                  & (9.8)  & 31.2   & (20 -- 52)       & 421                     & (9.6)  & 30     & (19 -- 52)       & 28                      & (12.3) \\
                Creatinine ($\mu$mol/L)     & 63.7                                                   & (51.3 -- 79.6)                                            & 646                                                  & (14.0) & 63.7   & (51.3 -- 79.6)   & 598                     & (13.7) & 63.7   & (51.3 -- 76.0)   & 48                      & (21.1) \\
                \bottomrule
            \end{tabular}
            \begin{tablenotes}
                \footnotesize
                \item[1] Denominator for missing \%: total number of patients in respective group (overall, relapse or no relapse). For measures of malnutrition (BMI, BMI-for-age z-score, and weight-for-height z-score), see further table footnotes.
                \item[2] Denominator for missing \%: number of patients aged $\geq$ 19 years, n = 2,154 (relapse: 97, no relapse: 2,057).
                \item[3] Denominator for missing \%: number of patients aged 5--18 year inclusive, n = 2,301 (relapse: 125, no relapse: 2,176).
                \item[4] Denominator for missing \%: number of patients aged $<$ 5 years, n = 138 (relapse: 6, no relapse: 132).
            \end{tablenotes}
        \end{threeparttable}
        \caption{Summary of continuous candidate predictors and additional variables used for the derivation of malnutrition status (height, weight, BMI, BMI-for-age z-score, weight-for-height z-score). Abbreviations: ALT: alanine aminotransferase; BMI(-FA): body mass index(-for age); cm: centimetres; IQR: inter-quartile range, IU: international units; kg: kilograms; L: litres; m: metres; WBC: white blood cells; WFH: weight-for-height; g: grams; $\mu$mol: micromoles.}
        \label{tab:isc_continuous}
    \end{table}
\end{landscape}




% these are the study specific distributions of outcome, sex and age
% cp /Users/jameswilson/proj/vl_model_isc/figures/dist/main_dist.pdf figures/ch5/isc_main_dist.pdf
\newgeometry{left=1cm, bottom=2.5cm, right=2cm, top=3cm}
\begin{landscape}
  \begin{figure}[tb]
    \centering
    \includegraphics[width=1.35\textwidth]{figures/ch5/isc_main_dist.pdf}
    \caption{Graphical summary of the Indian subcontinent study-specific sample sizes and distributions of relapse status, sex, and age.}
    \label{fig:isc_main_dist}
  \end{figure}

  \begin{figure}[tb]
    \centering
    \includegraphics[width=1.35\textwidth]{figures/ch5/treat.pdf}
    \caption{Bar chart showing the distribution of treatment regimens across contributing studies from the Indian subcontinent. Drugs are colour-coded (see legend). Important distinguishing dosing information provided in the overlaying labels, as space allows. Full treatment details presented in Table \ref{tab:isc_study}. 6 patients are included from Chakraborty 2008: 5 receiving alternate day amphotericin B deoxycholate and 1 receiving human placenta extract. ABD: amphotericin B deoxycholate; ALT: alternate days; CONS: consecutive days; D: days; HPE: human placenta extract; MF: miltefosine; PENT: pentamidine; mg: miligrams/kilogram; SSG: sodium stibogluconate.}
    \label{fig:isc_treat}
  \end{figure}

\end{landscape}
\restoregeometry

\subsection{Pooled univariable distributions}


% pooled distributions of categorical variables
% cp /Users/jameswilson/proj/vl_model_isc/figures/dist/catOut/comb_cat.pdf figures/ch5/isc_cat_comb.pdf
\newgeometry{left=1cm, bottom=2.5cm, right=2cm, top=3cm}
\begin{landscape}
  \begin{figure}[tb]
    \centering
    \includegraphics[width=1.35\textwidth]{figures/ch5/isc_cat_comb.pdf}
    \caption{Pooled distributions and predictor--outcomes relationships for categorical candidate predictors. Excluding missing data. 95\% binomial confidence intervals calculated using the Wilson method. MF: miltefosine; Norm: normal; SDA: single-dose liposomal amphotericin B.}
    \label{fig:isc_cat_comb}
  \end{figure}
\end{landscape}
\restoregeometry


% these are the pooled continuous distributions and relationships with relapse - PART 1
% cp /Users/jameswilson/proj/vl_model_isc/figures/dist/contOut/age_comb.pdf figures/ch5/isc_pool_age_comb.pdf
% cp /Users/jameswilson/proj/vl_model_isc/figures/dist/contOut/ss_comb.pdf figures/ch5/isc_pool_ss_comb.pdf
% cp /Users/jameswilson/proj/vl_model_isc/figures/dist/contOut/fd_comb.pdf figures/ch5/isc_pool_fd_comb.pdf
% cp /Users/jameswilson/proj/vl_model_isc/figures/dist/contOut/height_comb.pdf figures/ch5/isc_pool_height_comb.pdf
% cp /Users/jameswilson/proj/vl_model_isc/figures/dist/contOut/weight_comb.pdf figures/ch5/isc_pool_weight_comb.pdf

\clearpage
\begin{figure}[H]
  \centering
  \begin{subfigure}{\textwidth}
    \centering
    \begin{overpic}[width=\textwidth]{figures/ch5/isc_pool_age_comb.pdf}
      \put(2,19){\small Age}
    \end{overpic}
  \end{subfigure}
  \begin{subfigure}{\textwidth}
    \centering
    \begin{overpic}[width=\textwidth]{figures/ch5/isc_pool_weight_comb.pdf}
      \put(2,19){\small Weight}
    \end{overpic}
  \end{subfigure}
  \begin{subfigure}{\textwidth}
    \centering
    \begin{overpic}[width=\textwidth]{figures/ch5/isc_pool_height_comb.pdf}
      \put(2,19){\small Height}
    \end{overpic}
  \end{subfigure}
  \begin{subfigure}{\textwidth}
    \centering
    \begin{overpic}[width=\textwidth]{figures/ch5/isc_pool_fd_comb.pdf}
      \put(2,19){\small FevDur}
    \end{overpic}
  \end{subfigure}
  \begin{subfigure}{\textwidth}
    \centering
    \begin{overpic}[width=\textwidth]{figures/ch5/isc_pool_ss_comb.pdf}
      \put(2,19){\small SpnSize}
    \end{overpic}
  \end{subfigure}
  \caption{Distributions and predictor--outcome relationships for continuous non-laboratory candidate predictors. FevDur:~duration of fever; SpnSize:~spleen size. For each candidate predictor, left upper panel shows the overall density pooled across studies and the left lower panel shows overlapping densities normalised by relapse status. The right panel shows a univariable generalised additive model spline fit, with 95\% confidence interval, of relapse.}
  \label{fig:isc_pooled_dist_cont_lab}
\end{figure}

% these are the pooled continuous distributions and relationships with relapse - PART 2
% cp /Users/jameswilson/proj/vl_model_isc/figures/dist/contOut/wbc_comb.pdf figures/ch5/isc_pool_wbc_comb.pdf
% cp /Users/jameswilson/proj/vl_model_isc/figures/dist/contOut/plt_comb.pdf figures/ch5/isc_pool_plt_comb.pdf
% cp /Users/jameswilson/proj/vl_model_isc/figures/dist/contOut/hb_comb.pdf figures/ch5/isc_pool_hb_comb.pdf
% cp /Users/jameswilson/proj/vl_model_isc/figures/dist/contOut/alt_comb.pdf figures/ch5/isc_pool_alt_comb.pdf
% cp /Users/jameswilson/proj/vl_model_isc/figures/dist/contOut/cr_comb.pdf figures/ch5/isc_pool_cr_comb.pdf
\begin{figure}[H]
  \centering
  \begin{subfigure}{\textwidth}
    \centering
    \begin{overpic}[width=\textwidth]{figures/ch5/isc_pool_hb_comb.pdf}
      \put(2,19){\small Hb}
    \end{overpic}
  \end{subfigure}
  \begin{subfigure}{\textwidth}
    \centering
    \begin{overpic}[width=\textwidth]{figures/ch5/isc_pool_plt_comb.pdf}
      \put(2,19){\small Plt}
    \end{overpic}
  \end{subfigure}
  \begin{subfigure}{\textwidth}
    \centering
    \begin{overpic}[width=\textwidth]{figures/ch5/isc_pool_wbc_comb.pdf}
      \put(2,19){\small WBC}
    \end{overpic}
  \end{subfigure}
  \begin{subfigure}{\textwidth}
    \centering
    \begin{overpic}[width=\textwidth]{figures/ch5/isc_pool_alt_comb.pdf}
      \put(2,19){\small ALT}
    \end{overpic}
  \end{subfigure}
  \begin{subfigure}{\textwidth}
    \centering
    \begin{overpic}[width=\textwidth]{figures/ch5/isc_pool_cr_comb.pdf}
      \put(2,19){\small Crt}
    \end{overpic}
  \end{subfigure}
  \caption{Pooled distributions and predictor--outcome relationships for continuous laboratory candidate predictors. All predictors presented on log scale. Hb:~haemoglobin; Plt:~platelet; WBC:~white blood cells; ALT:~alanine aminotransferase; Crt:~creatinine. For each candidate predictor, left upper panel shows the overall density pooled across studies and the left lower panel shows overlapping densities normalised by relapse status. The right panel shows a univariable generalised additive model spline fit, with 95\% confidence interval, of relapse.}
  \label{fig:isc_pooled_dist_cont_nolab}
\end{figure}

\subsection{Missing data patterns}
% missing data figure
% cp /Users/jameswilson/proj/vl_model_isc/figures/missing/summary.pdf figures/ch5/isc_missing_summary.pdf
\begin{figure}[tb]
  \centering
  \includegraphics[scale = 0.8]{figures/ch5/isc_missing_summary.pdf}
  \caption{Density plot illustrating the amount of missing data overall and across contributing studies from the Indian subcontinent. Study ordered by lead author and year of publication (or protocol). Variable ordered by amount of missingness. ALT: alanine aminotransferase; WBC: white blood cells.}
  \label{fig:isc_missing_summary}
\end{figure}

\section{Model results}

% cp /Users/jameswilson/proj/vl_model_isc/results/var_forest_combined.pdf figures/ch5/var_forest_combined.pdf
\begin{figure}[tb]
  \centering
  \includegraphics[width=\textwidth]{figures/ch5/var_forest_combined.pdf}
  \caption{Forest plot of adjusted odds ratios with 95\% confidence intervals for final model predictors. Odds ratios are displayed on a logarithmic scale. For age, the odds ratio represents a combination of linear and quadratic effects for standardised age (centred by the mean and scaled by the standard deviation). Please refer to Figure \ref{fig:isc_adjusted_assoc} for a visualisation of the adjusted relapse probabilities after recalibration of the model intercepts to the observed relapse rate observed in Sundar 2019\cite{sundar2019}.}
  \label{fig:isc_var_forest_combined}
\end{figure}

% cp /Users/jameswilson/proj/vl_model_isc/figures/multiAssocM1.pdf figures/ch5/isc_multiassoc_with_pg.pdf
% cp /Users/jameswilson/proj/vl_model_isc/figures/multiAssocM2.pdf figures/ch5/isc_multiassoc_without_pg.pdf
\newgeometry{left=2.5cm, bottom=2.5cm, right=2cm, top=3cm}
\begin{landscape}
  \begin{figure}[tb]
    \centering
    \begin{subfigure}{1.4\textwidth}
      \centering
      \begin{overpic}[width=\textwidth]{figures/ch5/isc_multiassoc_without_pg.pdf}
        \put(2,31){\small ISC model: without parasite grade}
      \end{overpic}
    \end{subfigure}
    \begin{subfigure}{1.4\textwidth}
      \centering
      \begin{overpic}[width=\textwidth]{figures/ch5/isc_multiassoc_with_pg.pdf}
        \put(2,26){\small ISC model: with parasite grade}
      \end{overpic}
    \end{subfigure}
    \caption{Adjusted associations between final predictors and predicted relapse probability, as estimated from the final ISC prognostic models. Probabilities were calculated from optimism-adjusted models and following logistic recalibration (intercept--term only) to data contributed from Sundar 2019\cite{sundar2019}. Where not varying in the plot, predictions are standardised to a representative reference participant: median age (18 years), median fever duration (30 days), treated with single-dose liposomal amphotericin B (10 mg/kg), with severe anaemia, and \textemdash\ for the model including parasite grade \textemdash\ a median parasite count of 2+.}
    \label{fig:isc_adjusted_assoc}
  \end{figure}
\end{landscape}
\restoregeometry

% calibration plots (overall)
% cp /Users/jameswilson/proj/vl_model_isc/graphs/calPlot.pdf figures/ch5/isc_calPlot.pdf
\begin{figure}[tb]
  \centering
  \includegraphics[width=\textwidth]{figures/ch5/isc_calPlot.pdf}
  \caption{Calibration plots showing observed versus predicted probabilities for deciles of predicted probability. Red dashed line represents perfect calibration.  Observed probabilities are presented with 95\% confidence intervals (black error bars). A generalised additive model is fitted to show the smoothed mean observed probability (blue dotted line) with 95\% confidence intervals (blue ribbon). Histograms, normalised by outcome, are overlaid to illustrate the distribution of relapses and cures across the expected probabilities.}
  \label{fig:isc_calPlot}
\end{figure}

% calibration plots (parasite grade and fever duration)
% \begin{figure}[tb]
%   \centering
%   \includegraphics[width=\textwidth]{figures/ch5/isc_calPlotPD.pdf}
%   \caption{Calibration plots for different parasite grades (model including parasite grade). Left: Observed vs. predicted probabilities, grouped by parasite grade. Right: Observed and predicted probabilities on the $y$--axis against parasite grade on the $x$--axis.}
%   \label{fig:isc_calPlotPD}
% \end{figure}

\begin{figure}[tb]
  \centering
  \includegraphics[width=\textwidth]{figures/ch5/isc_calPlotFD1.pdf}
  \caption{Calibration plots for different fever durations (model including parasite grade). Left: Observed vs. predicted probabilities, grouped by fever duration group. Right: Observed and predicted probabilities on the $y$--axis against fever duration group on the $x$--axis.}
  \label{fig:isc_calPlotFD1}
\end{figure}

% cp /Users/jameswilson/proj/vl_model_isc/graphs/forestCIM1.pdf figures/ch5/forestCIM1.pdf
% forest plots for C-statistic
\begin{figure}[tb]
  \centering
  \begin{overpic}[width=0.9\textwidth, trim={3.5cm 0cm 1cm 0.3cm}, clip]{figures/ch5/forestCIM1.pdf}
    \put(0,5.5){\scriptsize $\tau^2$: 0.43; $I^2$: 77.4\%}
    \put(0,2.3){\scriptsize Q(df = 17): 122.48; p < .0001}
  \end{overpic}
  \caption{Forest plot showing individual and pooled study c--statistics, for the model \textbf{including} parasite grade. Pooled c--statistics are presented from both fixed--effects and random--effects meta-analysis models. Pooled random--effects c--statistics and variances are estimated using restricted maximum likelihood (REML) and the Hartung--Knapp--Sidik--Jonkman method. Blue diamonds: pooled summary estimates with 95\% confidence intervals. Red line: 95\% prediction interval. Study ordered by c--statistic. For Chakraborty 2008, no relapse events occurred and c--statistic is therefore undefined. Study--specific confidence intervals should be interpreted with caution due to small sample sizes and relapse events in some studies (see Methodology Section \ref{meth:discrimination}).}
  \label{fig:isc_forestCIM1}
\end{figure}

% cp /Users/jameswilson/proj/vl_model_isc/graphs/forestCalM1.pdf figures/ch5/forestCalM1.pdf
% forest plots for calibration
\begin{figure}[tb]
  \centering
  \begin{overpic}[width=0.9\textwidth, trim={3.4cm 0cm 0.3cm 0.3cm}, clip]{figures/ch5/forestCalM1.pdf}
    \put(27,13){\fcolorbox{black}{white}{\tiny $\tau^2$: 0.094; $I^2$: 46.1\%; Q(17): 33.6; p = .01}}
    \put(67,13){\fcolorbox{black}{white}{\tiny $\tau^2$: 0; $I^2$: 0\%; Q(16): 12.9; p = .74}}
  \end{overpic}
  \caption{Forest plots showing individual and pooled study calibration measures, for the model \textbf{including} parasite grade. Left: calibration intercept (calibration--in--the--large); Right: calibration slope. Pooled summary estimates and variances are estimated from random--effects meta-analysis models using restricted maximum likelihood (REML) and the Hartung--Knapp--Sidik--Jonkman method. Blue diamonds: summary estimates with 95\% confidence intervals. Red lines: 95\% prediction interval. Calibration measures not presented for Chakraborty 2008 due to no relapse events. Calibration slope not presented for Rijal 2010(A) due to only one relapse event and few total participants leading to failure of model convergence.}
  \label{fig:isc_forestCalM1}
\end{figure}
