% \begin{savequote}[8cm]
%   \textlatin{Many large baries [homesteads], in which there were formerly thirty or forty residents, have now been left with perhaps one solitary occupant... whole mohullas [neighbourhoods] and streets have been deserted, and large villages which formerly told their residents by thousands, can now almost number them by hundreds.
%   \qauthor{ --- [description of an early VL outbreak] The Annual Reports of the Sanitary Commissioners for Bengal, 1865\cite{gibson1983}}}
% \end{savequote}

\chapter{\label{ch:2-background}Background}
\minitoc

\section{Introduction}

% Currently, treatment response is predominantly assessed clinically, with tissue microscopy to determine cure restricted to patients with HIV co-infection or other complex cases. There is therefore an urgent need for non-invasive tests to determine cure and monitor treatment response. \cite{pareyn2025}

This chapter provides important context for the development and validation of prognostic models predicting VL relapse in the Indian subcontinent (ISC) and East Africa. It begins with an overview of VL, outlining its epidemiology, pathophysiology, clinical features, and management. Emphasis is placed on the epidemiology of VL and the ongoing WHO--supported elimination programmes in the ISC and East Africa, defining the public health landscape in which a VL prognostic model would be implemented.

The section on relapse is presented as a narrative synthesis informed by a systematic review of the literature (full methods in Appendix \ref{app:background}). Using pre-specified inclusion criteria, the review identifies and evaluates studies describing the burden, mechanisms, and determinants of relapse. The discussion highlights how relapse constitutes a high-risk infection reservoir that, together with PKDL and VL/HIV co-infection, poses a major threat to elimination efforts \textemdash\  a threat that could be mitigated through early identification of patients at high risk of relapse, enabling timely diagnosis and treatment.

\section{Visceral Leishmaniasis}
% speak more broadly about leishmaniases before narrowing in on VL specifically
The leishmaniases are a diverse group of neglected tropical diseases (NTDs) caused by protozoan parasites of the \textit{Leishmania} genus and transmitted between susceptible mammalian hosts via the bite of infected female sandflies\cite{pareyn2025}. At least 20 \textit{Leishmania} species infect humans, causing four principal disease forms: cutaneous leishmaniasis (CL), mucocutaneous leishmaniasis (MCL), visceral leishmaniasis (VL), and post kala--azar dermal leishmaniasis (PKDL) (see Box \ref{box:main-four})\cite{farrar2023manson, burza2018, pareyn2025, who_leish}. Disease form is largely determined by parasite species and strain, resulting in a spectrum of clinical presentations ranging in severity from the relatively common and usually self-healing skin lesions seen in CL, to the often disfiguring mucosal destruction of MCL, and life-threatening systemic illness of VL. PKDL is a disseminated dermal eruption that may occur following VL recovery, and while patients are systemically well, they have been shown to be infective to sandflies\cite{burza2018}.

VL represents the most severe manifestation of leishmaniasis, accounting for the vast majority of its morbidity and mortality. The disease overwhelmingly affects impoverished rural populations with poor access to healthcare. VL is also recognised as an opportunistic infection in patients living with human immunodeficiency virus (HIV), in whom co-infection leads to particularly high rates of treatment failure and death.

\begin{mybox}[label=box:main-four]{Principal disease forms of Leishmaniasis\cite{who_leish,farrar2023manson}}
  \begin{description}[nosep]
    \item[Cutaneous leishmaniasis (CL)] Results in lesions on exposed skin that can lead to ulceration and life-long scarring. Often self-healing within a year, but can manifest atypical and disseminated forms, especially in the immunocompromised. Up to 1 million new cases per year with most cases occurring in the Americas, Mediterranean basin, Middle East and Central Asia.
    \item[Mucocutaneous leishmaniasis (MCL)] Rare complication of CL seen especially in the Americas with most cases reported in Bolivia, Perú, and Brazil. Results in destructive ulceration of the oral and nasal mucosa. Highly stigmatising and challenging to treat.
    \item[Visceral leishmaniasis (VL)] Also known as kala--azar, the most severe form of leishmaniasis caused by \textit{L. donovani} in the Old World and \textit{L. infantum} in the New World. With an estimated 50,000--90,000 cases/year, VL presents with progressive weight loss, splenomegaly and fever. Considered fatal without treatment.
    \item[Post kala--azar dermal leishmaniasis (PKDL)] Benign macular and/or papular rash often including the face, arms, and trunk.  Affects 5--20\% of patients months to years after successful initial treatment for VL in the Indian subcontinent and East Africa. Often self-limiting, although known to be infective to sandflies and therefore acts as a disease reservoir.
  \end{description}
\end{mybox}

\subsection{Epidemiology}
% 2024 figures only became available (5th Nov 2025) - not updated yet.

% where, how much, how transmitted, who’s at risk 
This section provides an overview of the current global distribution, transmission dynamics, and risk factors for developing visceral leishmaniasis, highlighting major regional differences in disease burden and risk.

VL is endemic\footnote{Defined by the WHO as the occurrence of at least one autochthonous case with demonstrated local transmission within a country\cite{who_wer2023}.} in at least 80 countries across tropical, semi-tropical and temperate climes. The disease is caused by two closely related \textit{Leishmania} species whose distribution defines the four principal global regions of high endemicity (see Figure \ref{fig:gho_map}): \textit{L. donovani}, responsible for anthroponotic transmission (human$\rightarrow$sandfly$\rightarrow$ human) in the ISC and East Africa, and \textit{L. infantum}\footnote{Previously referred to as \textit{L. chagasi} in the Americas.}, responsible for zoonotic transmission (mammal$\rightarrow$sandfly$\rightarrow$human) in the Americas (principally Brazil) and Mediterranean basin, extending into the Middle East and Central Asia\cite{who_wer2023,who_gho}. Together, these two species comprise the \textit{L. donovani complex}.

% cp /Users/jameswilson/proj/vl_visualisation/results/gho_map.pdf figures/ch2/gho_map.pdf
\begin{landscape}
  \begin{figure}[tb]
    \centering
    \includegraphics[clip, trim={0cm 2.3cm 0cm 2.2cm}, width=1.55\textwidth]{figures/ch2/gho_map.pdf}
    \caption{Visceral leishmaniasis endemicity (reported 2023) and average annual cases (reported 2018--2023). `Considered endemic' \textemdash\  at least one autochthonous case has been reported, with or without the entire cycle of transmission being demonstrated. Data sourced from the World Health Organization Global Health Observatory, accessed October 2025\cite{who_gho}. }
    \label{fig:gho_map}
  \end{figure}
\end{landscape}
\restoregeometry

\subsubsection{Disease Burden}

Estimating the global burden of VL is problematic. True case numbers are obscured by significant underreporting due to limited access to healthcare, inadequate diagnostic facilities, misdiagnosis, and poor surveillance systems in many endemic countries\cite{chappuis2007,singh2006, mubayi2010, maia-elkhoury2007}. In 2012, Alvar et al published the results of a WHO-led update to the global incidence of leishmaniasis using country-level reporting from the mid-late 2000s\cite{alvar2012}. Underreporting rates were estimated through consultation with country representatives and disease experts. The global incidence was estimated at 200,000--400,000 cases/year, with approximately 80\% of the burden originating from the ISC and 15\% from East Africa. Compared to official reporting over the same period, this reflected a global underreporting rate of 3.5--7-fold.

Since the publication of Alvar et al's estimates, the number of cases has undisputedly fallen, driven largely by decreases in the ISC following the launch of the Kala-Azar Elimination Programme (KAEP) in 2005. Reported cases decreased from > 50,000 cases/year prior to 2012, to approximately 22,500 cases/year in 2017 and < 12,000 cases/year in 2023\cite{who_gho} (high incidence country breakdowns presented in Figure \ref{fig:gho_plot_top8}). Notably, this downward trend has persisted despite improvements in surveillance and reporting systems in many endemic countries\cite{who_wer2023}. Reflecting these changes, the WHO revised its estimated annual incidence in 2017 to 50,000--90,000 cases/year\footnote{According to the online WHO Leishmaniasis Fact Sheet: \url{https://www.who.int/news-room/fact-sheets/detail/leishmaniasis}. Alternate WHO online content estimates 30,000 cases/year since 2020: \url{https://www.who.int/health-topics/leishmaniasis} (online material last accessed October 2025).}\cite{WHO_2016_leishmaniasis,who_leish}.

\begin{figure}[h]
  \centering
  \includegraphics[width=1\textwidth]{figures/ch2/gho_plot_top8.pdf} % path to your figure file
  \caption{Temporal trends of visceral leishmaniasis incidence by country. Comparison between the top 8 countries with the highest average case numbers reported between 2013 and 2023. Data sourced from the World Health Organization Global Health Observatory\cite{who_gho}.}
  \label{fig:gho_plot_top8}
\end{figure}

Based on the most recent reporting data from 2023\cite{who_gho}, the five countries with the highest case numbers are now, in descending order, Sudan, Ethiopia, Brazil, Kenya and South Sudan \textemdash\  collectively comprising 72.4\% of the global total. In stark contrast to the situation 20 years ago, countries in the ISC now account for only 6.3\% of the reported total: India with 538 cases (4.6\%), Nepal with 168 cases (1.4\%), and Bangladesh with just 34 cases (0.3\%).

\subsubsection{Mortality}

The disease is widely described as `fatal without treatment'\cite{burza2018,farrar2023manson}. Supporting this statement are the high mortality figures recorded during conflict-related epidemics in East Africa over the last 40 years, and prior to effective therapy, 19\textsuperscript{th} century accounts of outbreaks devastating communities across the Gangetic plains\cite{collin2004, gibson1983,steverding2017}. Despite this, subclinical disease forms have been reported with spontaneous resolution\cite{mouri2015,badaro1986}.

With treatment, \textasciitilde5--15\% of cases result in death, although accurate estimates are challenged by a lack of reporting. Where deaths are reported, they frequently only reflect hospital deaths and omit those where a definite diagnosis was missed. In the 2000s, it was often cited that VL was responsible for up to 50,000 deaths/year \textemdash\  the second highest mortality from a parasitic disease after malaria\cite{chappuis2007} \textemdash\  although more recent estimates tentatively place the figure at a more modest 4,627 deaths/year with a wide uncertainty range of 1,853--8,725 deaths/year\cite{gbd2023}.

\subsubsection{Vector}
Measuring 2--4 mm and covered in dense hairs, phlebotomine sandflies (Diptera: Psychodidae) appear distinctly fuzzy under magnification. Females from an estimated 31 species across two genera are known to transmit the parasite between human hosts: \textit{Lutzomyia} in the New World and \textit{Phlebotomine} in the Old World\cite{akhoundi2016}. Sandflies occupy a wide range of ecological niches, found on every continent except Antarctica. Biting occurs from dusk to dawn, with females requiring a blood meal for larval development. During the day they are found in cool and sheltered locations, such as in cracks and crevices in walls as seen with \textit{Ph. argentipes}, responsible for transmission in the ISC. In East Africa, three sandfly vectors have been implicated in \textit{L. donovani} transmission, defining two distinct and non-overlapping ecological settings: (i) the \textit{Acacia-Balanites} and black cotton soil savannah regions in northern focus, incorporating Sudan, northern South Sudan, and northern Ethiopia, where \textit{Ph. orientalis} thrives, and (ii) the savannah and forested areas in the southern focus, incorporating southern Ethiopia, Kenya, and Uganda, where \textit{Ph. martini} and \textit{Ph. celiae} are seen in association with \textit{Macrotermes} termite mounds.

In addition to sandflies, needle sharing among people who inject drugs was considered an important route of transmission in the southern Mediterranean region during the 1990s and 2000s, particularly among people living with HIV \cite{alvar1997}. Exceptionally, transmission can also result from blood transfusion, organ transplantation, laboratory accidents\cite{farrar2023manson}, mother-to-child transmission\cite{pagliano2005}, and possibly even sexual contact\cite{guedes2020, symmers1960}.
% UK resident with no travel hx developed VL (symmers1960)

\subsubsection{Reservoirs}

Similar to the majority of \textit{Leishmania} spp. causing CL and MCL, \textit{L. infantum} demonstrates zoonotic transmission, with domestic dogs being the main reservoir host in both the Americas and the Old World. This being said, an ever-increasing list of wild and domestic animals are known to harbour the parasite, including cats, foxes, horses, rodents, bats and opossums, although their relevance to human infection is unclear\cite{alcover2020, ratzlaff2023}. An outbreak near Madrid (2009--2012) was attributed to hares\cite{molina2012}.

In contrast to \textit{L. infantum}, and importantly from an elimination perspective, \textit{L. donovani} transmission in the ISC and East Africa is predominantly anthroponotic. Xenodiagnosis studies confirm that patients with active VL and those with PKDL are competent human reservoirs, in contrast to individuals with asymptomatic infection\cite{singh2021, mondal2019}. Although \textit{L. donovani} infections have been detected in several animal species in both regions—including cattle, dogs, and rats—their relevance as sources of human transmission remains unproven\cite{kushwaha2024, jones2021}.

\subsubsection{Risk Factors}

From population prevalence studies we know that only a minority of people with detectable parasites develop symptoms\cite{pederiva2022, burza2018}. Risk factors for acquiring an initial asymptomatic infection and subsequent progression to symptomatic disease reflect a tangled ecology of determinants linking host factors (sandfly exposure, immunity, genetics) and parasite factors (strain, virulence, inoculum). A common theme, woven into many of these determinants, is poverty.

In both the ISC and East Africa the median age of infection during stable transmission is similar at 15--20 years. More men than women are infected and develop disease, likely reflecting their increased occupational exposure to sandflies (for example, cattle herding and other outdoor activities)\cite{who_wer2023}.
% other reasons for men: Cloots, K. et al. Male predominance in reported visceral leishmaniasis cases: nature or nurture? A comparison of population-based with health facility-reported data. PLoS Negl. Trop. Dis. 14, e0007995 (2020). Lockard, R. D., Wilson, M. E. & Rodríguez, N. E. Sex-related differences in immune response and symptomatic manifestations to infection with Leishmania species. J. Immunol. Res. 2019, 4103819 (2019).

% genetics: Fakiola, M. et al. Common variants in the HLA-DRB1-HLA-DQA1 HLA class II region are associated with susceptibility to visceral leishmaniasis. Nat. Genet. 45, 208–213 (2013). Blackwell, J. M., Fakiola, M. & Castellucci, L. C. Human genetics of Leishmania infections.Hum. Genet. 139, 813–819 (2020) Essential reading for an in-depth critique of host genetic factors affecting the outcome of leishmaniasis.

% sex: https://linkinghub.elsevier.com/retrieve/pii/S1471492224002496

In the ISC\footnote{Despite sporadic VL cases reported from Pakistan, Bhutan and Sri Lanka, in this thesis ISC refers to India, Nepal and Bangladesh.}, VL endemicity is centred on the fertile and low-lying alluvial plains of the Ganges river, where high humidity, heavy monsoon rains, and abundant vegetation provide ideal conditions for sustained transmission between sandflies and humans\cite{bhunia2010}. Significant clustering of cases is seen across the rural farming communities of Bihar, Jharkhand, Uttar Pradesh, and West Bengal in northeastern India, central and western Bangladesh, and southeastern Nepal. In a systematic review by Bern et al, determinants of VL transmission in the ISC included living in mud houses, proximity to prior cases (in the same or nearby household), presence of vegetation and standing water surrounding the house, sleeping on the floor or outside, malnutrition, and a lack of bed net use\cite{bern2010}.

The greatest concentration of cases in East Africa is reported in the northern focus, specifically between the eastern Sudanese state of Gederaf and the bordering northern states of Ethiopia. Many of the epidemiological determinants of VL in East Africa are shared with the ISC, with poverty remaining the central overarching factor. Notable determinants include living in rural settings near sandfly breeding and resting sites (living near termite mounds in the southern focus, sleeping under \textit{Acacia} trees in the northern focus), living in proximity to other VL infected (or recently infected) people, and malnutrition\cite{geto2024}.

VL/HIV co-infection has reshaped VL epidemiology in many endemic regions, and remains the most important risk factor for asymptomatic infection, disease progression, and poor treatment outcomes. In the mid-1980s in Spain and other southern European countries, VL shifted from a rare childhood disease to one predominantly affecting HIV-positive adults\cite{monge2014,alvar2008}. VL/HIV co-infection rates are currently increasing in Brazil (now reported in 20\% of new cases\cite{PAHO_Leishmaniasis_2024}) and India (> 10\% of VL episodes in 2023 and 2024\cite{NCVBDC2025}). In northern Ethiopia, VL/HIV co-infection presents a significant challenge to elimination, affecting between 20 and 50\% of cases predominantly in male migrant workers\cite{diro2014}.

% host immunity is key some of this thesis' conclusions, and explaining relapse predictors
Host immunity \textemdash\  particularly cell-mediated immunity \textemdash\  is a major determinant of disease onset and treatment outcomes, and thought to be a key driver of several devastating epidemics associated with war and natural disasters. A striking example is Sudan in the 1980s-2000s, where population displacement driven by conflict and famine brought immune-naïve populations into endemic areas, and spread infection into previously unaffected areas. Combined with severe malnutrition and the collapse of health infrastructure, the resulting mortality was catastrophic\cite{zijlstra1991, collin2004, al-salem2016}.

The geographic and demographic heterogeneity of VL reflects fundamental differences in host-parasite interactions and immune responses. To understand how these epidemiological patterns arise and why certain groups experience more severe disease, it is necessary to examine the pathophysiology of human \textit{Leishmania} infection.

\subsection{Pathophysiology}
% I'm including this chapter before clinical/diagnostic details because it provides important background  

Parasites of the genus \textit{Leishmania} belong to the family \textit{Trypanosomatidae} (class \textit{Kinetoplastea}, phylum \textit{Euglenozoa}) and share many features with two other human pathogens of the same family; \textit{Trypanosoma brucei} \textemdash\  the causative agent of African sleeping sickness (human African Trypanosomiasis), and \textit{Trypanosoma~cruzi} \textemdash\  the agent of Chagas disease in the Americas. These vector-borne NTDs are all single-celled protozoa characterized by a flagellum and a kinetoplast \textemdash\  a dense network of mitochondrial DNA giving them a distinctive microscopic appearance.

\begin{figure}[h]
  \centering
  \includegraphics[width=1\textwidth]{figures/ch2/cdc_lifecycle.png} % path to your figure file
  \caption{Life cycle of the \textit{Leishmania} parasite consists of a vector (sandfly) stage and reservoir (animal) stage. While vector transmission is by far the most common route, it is not essential \textemdash\  direct human-to-human transmission has been reported via needle sharing, organ transplantation, blood transfusion, and mother-to-child transmission. Illustration credit: Centers for Disease Control and Prevention/Alexander J. da Silva/Melanie Moser.}
  \label{fig:cdc_lifecycle}
\end{figure}

\subsubsection{Life Cycle}
In the sandfly gut, \textit{Leishmania} parasites exist in their extracellular flagellated promastigote forms. During a blood meal, the parasite is regurgitated into the bite wound and rapidly phagocytosed by macrophages. Intracellularly, in the human host, the protozoa lose their flagellum and transform into amastigotes: oval bodies measuring \textasciitilde2--6$\mu$m in length with their characteristic nucleus and kinetoplast clearly visible on nucleic acid staining. Intracellular amastigotes multiply through binary fission, eventually leading to cell lysis and subsequent uptake by neighboring macrophages to perpetuate infection. In viscerotropic disease caused by \textit{L. donovani complex}, this cycle drives widespread dissemination with macrophage proliferation and granuloma formation across the mononuclear phagocyte system, including in the spleen, liver, bone marrow, and lymph nodes. The cycle completes when sandflies ingest parasitised macrophages during a subsequent blood meal, initiating parasite development within the vector (Figure \ref{fig:cdc_lifecycle}).

\subsubsection{Immune Evasion}

Remarkably, intracellular amastigotes have evolved to survive and replicate within the hostile phagolysosomal environment of the macrophage. Although the full immunological mechanisms remain incompletely understood and beyond the scope of this chapter, key strategies include the presence of a lipophosphoglycan surface coat that protects promastigotes from complement-mediated lysis, and suppression of macrophage activation by inducing functional impairment of cytotoxic T-cell responses. Despite the presence of a pronounced humoral response, with a polyclonal hypergammaglobulinaemia typically seen, the resulting antibodies are predominantly non-neutralising and may even play a role in maintaining the immunosuppressive milieu required for amastigote survival\cite{stager2010,kaye2020}. As the disease progresses, so too does the host's immunosuppressive state, driven by rising levels of immunoregulatory cytokines (especially IL-10 and TGF$\beta$) and the destructive remodelling of lymphoid architecture\cite{kaye2020,nylen2010}.

% quiescent amastigote state perhaps important in relapse
% Dirkx, L. et al. Long-term hematopoietic stem cells trigger quiescence in Leishmania parasites. PLoS Pathog. 20, e1012181 (2024). although macrophages and other classical phagocytes are generally consideredthe major host cells supporting Leishmania replication, this study demonstrates that haematopoietic stem cells also harbour large numbers of parasites and may facilitate the development of a quiescent state.

\subsection{Clinical Features}

% could include a lot more information here, for example, WHO case classification, elaboration on frequency of symptoms
The clinical spectrum of \textit{Leishmania} infection is remarkably broad, with asymptomatic carriers outnumbering symptomatic cases by a factor of 5--200 in endemic settings\cite{pederiva2022,burza2018,werneck2002}. If symptomatic, disease onset is typically insidious, following an incubation period ranging from several weeks, to months, and occasionally years.

Classic symptoms result from a persistent systemic inflammatory response syndrome with infiltration of the mononuclear phagocyte system, resulting in intermittent fever, weight loss, splenomegaly, often with accompanying hepatomegaly and, in parts of East Africa, lymphadenopathy.  The spleen is usually non-tender on palpation, but massive splenomegaly $\pm$ hepatomegaly can result in severe abdominal pain. The greyish skin hyperpigmentation (`kala--azar' \textemdash\  literally `black disease' or `black fever' in Hindi), described in historical accounts of chronic disease in the ISC before effective therapy became available, is now uncommon\cite{chappuis2007, farrar2023manson}.

Characteristic laboratory abnormalities include raised inflammatory markers (CRP, ESR), pancytopenia, a polyclonal hypergammaglobulinaemia, and hypoalbuminaemia. Anaemia is often severe, normocytic, and normochromic, and likely multifactorial in origin, with splenic sequestration and anaemia of chronic disease thought to play important roles\cite{varma2010,goto2017}. Haemoglobin levels at presentation are frequently between 6--10 g/dL\cite{varma2010}, resulting in pallor and contributing to weakness.

Without treatment, disease severity correlates with increasing parasite burden\cite{zacarias2017,silva2014}. Patients may develop vomiting, diarrhoea, cough, and dyspnoea, reflecting a combination of direct mucosal invasion, and opportunistic infections arising from the host's immunosuppressed state. The presence of oedema, jaundice, severe co-infection, and bleeding with disseminated intravascular coagulation indicates a poor prognosis\cite{costa2023}.

\subsection{Diagnosis}

% could include WHO case definition - omitted given length concerns
The gold standard for diagnosing VL combines compatible clinical features with direct visualisation of \textit{Leishmania} amastigotes in tissue aspirates obtained from the spleen, bone marrow, or lymph node. Splenic aspiration offers the highest sensitivity (93--99\%) but carries a small risk of fatal haemorrhage. Bone marrow and lymph node aspirates are safer but considered less sensitive (\textasciitilde50--80\%)\cite{farrar2023manson,burza2018}. Parasite density is usually expressed using a logarithmic semiquantitative grading system (0--6+) originally described by Chulay and Bryceson (Table \ref{tab:grading})\cite{chulay1983}.

\begin{table}[ht]
  \centering
  \caption{Logarithmic grading system for parasite load on microscopy of tissue aspirates described by Chulay and Bryceson in the early 1980s\cite{chulay1983}. Smears are stained with Giemsa and examined using a $10\times$ eyepiece and $100\times$ objective lens.}
  \begin{tabular}{ll}
    \toprule
    \textbf{Grade} & \textbf{Average parasite density} \\
    \midrule
    $0$            & 0 amastigotes/1,000 fields        \\
    $1+$           & 1--10 amastigotes/1,000 fields    \\
    $2+$           & 1--10 amastigotes/100 fields      \\
    $3+$           & 1--10 amastigotes/10 fields       \\
    $4+$           & 1--10 amastigotes/field           \\
    $5+$           & 10--100 amastigotes/field         \\
    $6+$           & > 100 amastigotes/field           \\
    \bottomrule
  \end{tabular}
  \label{tab:grading}
\end{table}

Culture of tissue aspirates can improve diagnostic sensitivity but is limited by cost, technical complexity, and a slow turnaround of up to four weeks. Molecular assays have been developed but remain infrequently available outside research settings and regions of high endemicity. Notably, peripheral blood quantitative polymerase chain reaction (qPCR) assays amplifying kinetoplast DNA (kDNA) have shown high sensitivities in patients with both VL and VL/HIV co-infection\cite{galluzzi2018, rihs2025, verrest2024}, obviating the need for invasive sampling where available. Urine antigen detection has demonstrated specificity but suffers from low diagnostic sensitivity\cite{vangriensven2018, riera2004}.

A wide gamut of serological (antibody-detecting) tests is available for the diagnosis of VL, including enzyme-linked immunosorbent assays (ELISA), indirect fluorescent antibody (IFA) assays, immunoblots, and rapid diagnostic tests (RDTs)\cite{farrar2023manson}. The rK39 immunochromatographic test (ICT) is the most widely used RDT, providing a binary result within 10--20 minutes from a finger-prick blood sample. In the ISC, rK39 RDTs demonstrate excellent sensitivity (97.0\%, 95\% CI 90.0--99.5\%)\cite{boelaert2014} and have served as the first-line diagnostic test since the mid-2000s. In East Africa, however, sensitivity is lower (85.3\%, 95\% CI 74.5--93.2\%), and a second serological test \textemdash\  the direct agglutination test (DAT) \textemdash\  is recommended to confirm negative rK39 results\cite{boelaert2014}. Key limitations of serological tests include their inability to distinguish between active and past infection, with antibodies persisting for months to years following successful treatment and their reduced sensitivity in patients with severe immunosuppression, particularly those with VL/HIV co-infection\cite{burza2018}.

% more RDTs https://journals.plos.org/plosntds/article?id=10.1371/journal.pntd.0012905

% DAT systematic review & meta-analysis: https://link.springer.com/article/10.1186/s12879-023-08772-1


\subsection{Treatment}

Injectable pentavalent antimonials have been the workhorse of VL treatment since the 1940s\footnote{Previously, since the 1910s, tartar emetic was used \textemdash\  a toxic \textit{trivalent} antimonial compound also used as an emetic.}, and despite their relatively toxic adverse effects (pancreatitis, cardiac arrhythmias, hepatitis), they remain first-line therapy in both East Africa (sodium stibogluconate [SSG], as part of combination therapy) and in Brazil (as meglumine antimoniate [MA])\cite{WHO_TRS_949_2010,Brasil_2024_GuiaVigilanciaSaude}. From the early 1980s in Bihar, India, treatment failure rates exceeding 50\% were observed with SSG despite dose escalation (10$\rightarrow$20 mg/kg/day) and duration (6$\rightarrow$40 days). Blame was attributed to poor stewardship, with subtherapeutic dosing practices driving resistance\cite{sundar2001}. As efficacy declined, second-line agents were used \textemdash\  initially pentamidine (limited by severe toxicity), and later amphotericin B deoxycholate (ABD), effective but constrained by infusion reactions, nephrotoxicity, and the need for prolonged hospitalisation.

Over the past two decades, the introduction of three new agents \textemdash\  miltefosine, liposomal amphotericin B (AmBisome\textsuperscript{\textregistered}; Gilead Sciences), and paromomycin \textemdash\  has transformed the previously limited and often toxic treatment arsenal\cite{alves2018}.

Miltefosine (MF), a repurposed anticancer drug, is the only effective oral agent for VL. Introduced as the first-line treatment in the ISC in 2005, MF played a pivotal role in the early KAEP\cite{sundar2002,bhattacharya2007,who_sea_elim2005}. Its use has since declined. Concerns about teratogenicity, gastrointestinal toxicity, and decreasing efficacy\cite{rijal2013,sundar2012} led to its replacement in the KAEP by single dose liposomal amphotericin B (LAMB) (AmBisome\textsuperscript{\textregistered}; Gilead Sciences) in 2014--15\cite{chatterjee2025}. Outside the ISC, MF performs poorly as monotherapy in East Africa and Brazil\cite{alves2018, carnielli2019}.
% Eye side effects\cite{WHO2023MiltefosineAdvisory} - omit for now
% Need to write something briefly on supportive care: Supportive care, for example, rehydration, nutritional supplements and blood transfusion, may be considered

Supported by a WHO--Gilead donation programme, a single intravenous (IV) dose of 10 mg/kg LAMB remains the first-line regimen in the ISC owing to its efficacy and safety\cite{WHO2023GileadAgreement}. At higher cumulative doses of 21--30~mg/kg given over 5--10~days, LAMB is also used as first-line therapy in Europe and as a second-line option in Brazil and East Africa, particularly in patients intolerant of pentavalent antimonials (severe disease, extremes of age, pregnancy, renal or hepatic dysfunction, relapse, VL/HIV co-infection).

Paromomycin (PM), an injectable aminoglycoside, is effective both as monotherapy and in combination regimens in the ISC\cite{sundar2007a}. Despite its low cost and favourable safety profile, uptake in the ISC has been limited. In East Africa, by contrast, SSG--PM combination therapy has been first-line since 2010\cite{WHO_TRS_949_2010,musa2012}, with evidence also supporting its use alongside MF in both East Africa and the ISC\cite{musa2023,sundar2011comb}.

\subsection{Elimination Efforts}

% Where transmission is anthroponotic, xenodiagnosis studies confirm that the most important infectious reservoir are patients with active VL and PKDL\cite{singh2021, mondal2019}. 
% Concerns re: CL as potential reservoir \cite{bhattarai2025}

In areas of anthroponotic transmission, control of VL centres on two interdependent strategies: (i) reduction of the human infection reservoir through early diagnosis and treatment, and (ii) interruption of transmission events through vector control measures. Effective implementation of these approaches requires evidence-based decision-making and sustained political will.

% ISC elimination
Guided by these principles, in 2005 the governments of Nepal, Bangladesh, and India, with WHO support, signed a Memorandum of Understanding to reduce VL incidence to fewer than 1 case per 10,000 population at the district or sub-district level by 2015. This deadline was later extended and is now embedded within the 2021--2030 WHO NTD roadmap\cite{who_sea_elim2005, WHO_KalaAzarElimination_2015, WHO_SEA_CD_329_2021, WHO_NTDs_Roadmap_2021_2030, WHO_Global_Report_NTDs_2025}. The initial success of the KAEP was attributed to the introduction of highly sensitive rK39 RDTs, oral MF, and subsequently single dose LAMB, alongside active case detection and vector control measures such as indoor residual spraying (IRS) and insecticide treated nets (ITN)\cite{sundar2018}. In October 2023, Bangladesh became the first country to eliminate VL as a public health problem\cite{nagi2024}. Nepal and India are currently working to sustain their targets for 3 consecutive years to achieve elimination status\cite{pandey2025}.

As the ISC enters the consolidation and maintenance phases of the KAEP, concerns regarding sustainability are increasingly voiced\cite{chatterjee2025}. Waning political momentum and financial support threaten the viability of costly IRS and active case detection. Moreover, as the proportion of immunologically naïve individuals increases, so too does the risk of future outbreaks\cite{lerutte2017}.

% EA elimination
Buoyed by the KAEP experience, calls for VL elimination in East Africa\cite{alvar2021} culminated in the 2023 \textit{Nairobi Declaration}, endorsed by the health ministries of nine endemic East African countries\footnote{Chad, Djibouti, Eritrea, Ethiopia, Kenya, Somalia, South Sudan, Sudan and Uganda} and aiming to reduce VL incidence by 90\% by 2030. Similar to the KAEP, a phased approach has been proposed \textemdash\  consisting of planning, attack, consolidation, and maintenance phases \textemdash\  with emphasis placed on equitable access to diagnosis and treatment, integrated vector management adapted to diverse ecological settings, and strong cross-border collaboration\cite{WHO_nairobi_declaration, WHO_2024_VL_easternAfrica}.

However, in contrast to the KAEP, the East Africa initiative faces greater operational complexity. Challenges include heterogeneous transmission ecologies, weaker health systems, reliance on multi-dose treatments, and limited access to sensitive diagnostic tools\cite{WHO_2024_VL_easternAfrica}.

Early diagnosis and treatment of all human infection reservoirs remains a cornerstone of any successful elimination campaign where transmission is largely anthroponotic. The next section examines the burden, mechanisms, and determinants of VL relapse \textemdash\  a reservoir of particular concern and of critical relevance to elimination.

\section{\label{sec:relapse}Relapse}

This section is guided by a systematic review of the literature and presented as a narrative review (see Box \ref{box:lit-search1}). Search terms were constructed to identify the burden, timing, mechanisms, and determinants of VL relapse, with a focus on immunocompetent patients in the ISC and East Africa. Further details are available in Appendix \ref{app:background}.

VL relapse is defined as the reappearance of VL signs and symptoms following an initial treatment response\cite{WHO2010kalaazarIndicators, WHO2024_Leishmaniasis}. In both research and routine clinical settings, relapse is typically confirmed by direct visualisation of the parasite on a tissue aspirate smear\cite{chhajed2024}.

Relapse can only occur once an initial treatment response is achieved \textemdash\  termed initial cure in clinical efficacy studies. In research settings, this requires both clinical improvement (e.g., defervescence, reduction in spleen size, weight gain, improvement of haemoglobin) and confirmation of parasite clearance by microscopy. This is referred to as \textit{test--of--cure}, and usually occurs within a month of the end of treatment (EoT), although considerable heterogeneity exists in precisely how and when test--of--cure is assessed\cite{dahal2024}.

\begin{mybox}[label=box:lit-search1]{Literature search}
  Literature search performed from database inception to 11\textsuperscript{th} August 2025 in PubMed (below), Embase and Web of Science. Articles in English were reviewed. Full search details available in Appendix \ref{app:background}.
  \tcbline
  \texttt{(``Leishmaniasis, Visceral''[Mesh] OR ``visceral leishmaniasis'' OR ``leishmaniasis, visceral'' OR ``kala azar'' OR ``kala--azar'') AND (``Recurrence''[Mesh] OR relapse* OR recurrent OR recurrence OR recrudescence OR ``treatment failure'')}
\end{mybox}

% WHO TDR definition of relapse: \cite{WHO2010kalaazarIndicators}

\subsection{Burden of Relapse}

% general comment \textemdash\  high relapse rates in certain populations
Reported relapse rates among immunocompetent patients vary widely across studies, reflecting differences in host, treatment, and parasite factors\cite{chhajed2024}. As a `rule of thumb', relapse rates with current first-line regimens among patients without significant immunosuppression range between 2.5\% to 10\% (1 in 40 to 1 in 10 patients)\cite{chhajed2024}. According to a meta-analysis of VL clinical efficacy studies by Chhajed et al, the overall proportion of HIV-negative patients relapsing in the ISC within 6 months of treatment was estimated at 3.5\% (95\% confidence interval [CI]: 2.8--4.5\%) following first-line treatment with single dose LAMB\cite{chhajed2024}. Under pragmatic conditions in East Africa, slightly higher 6-month relapse rates of approximately 5\% are seen with the first-line combination therapy of PM and SSG\cite{kimutai2017,atia2015,melaku2007}.

These estimates, however, understate the true relapse burden: (i) relapses occurring beyond 6-months of follow-up are missed, and (ii) clinical trial populations do not reflect the broader patient population seen in routine care. Chhajed et al also showed that, across 21 studies reporting both 6 and 12 month relapse rates, one third of all relapses by 12 months occurred during the second half of the 12 month period\cite{chhajed2024}. Similarly, several large studies from the ISC have shown comparable or even higher relapses counts occurring between 6--12 months than in the initial 6 month period, prompting calls to extend routine follow-up from 6 to 12 months\cite{sundar2019,rijal2013,burza2013,burza2014,goyal2019}. Furthermore, trials often exclude patients at increased risk of relapse, such as those at the extremes of age, with more severe disease and comorbidities. Trial patients are also managed under controlled conditions that incentivise adherence. These factors collectively bias relapse estimates downward compared with outcomes observed under routine conditions.

% Recovery of a host's cell-mediated immunity can be demonstrated by the conversion of a Leishmanin skin test (LST) from negative to positive\footnote{Also known as the Montenegro test, initially developed in the 1920s to support the diagnosis of CL. Analogous to the Mantoux test for latent tuberculosis, a delayed-type hypersensitivity response from the LST manifests as visible skin induration occurring 48--72 hours after an intradermal injection of \textit{Leishmania} antigen\cite{carstens-kass2021}.}. During active infection, the LST is expected to be negative due to cellular anergy. Subsequent conversion to a positive test following treatment strongly indicates successful immune reconstitution, and conferring lasting protection from relapse and reinfection.

\subsection{Relapse vs. Reinfection}

Distinguishing relapse resulting from parasite recrudescence from re-infection with a new strain is important when assessing drug efficacy and relapse determinants. However, unlike malaria\cite{WHO_2009_surveillance_antimalarial_drug_efficacy}, VL lacks validated molecular targets for PCR-based confirmation. Nevertheless, accumulating evidence indicates that most relapses result from recrudescence rather than reinfection. For example, Rijal et al. performed kDNA fingerprinting in 8 pairs of bone marrow samples in HIV-negative patients prior to primary infection treatment and at the time of relapse. No evidence of reinfection was found, with 8 distinct kDNA fingerprints identified across the 16 samples that matched at the individual patient level\cite{rijal2013}. Similar studies have been performed in VL/HIV co-infection patients in East Africa and Europe employing a variety of molecular techniques, and showing that while reinfection does occur, it is considerably less common than recrudescence\cite{franssen2021, morales2002, gelanew2010,lachaud2009}.

\subsection{Relapse Determinants}

When selecting candidate predictors for inclusion in a prognostic model, it is important to draw on previously described predictors\cite{collins2024A}. To provide important context for the subsequent chapters on relapse model development, this section summarises the determinants of relapse identified through the literature review.

For immunocompetent patients, the review aims to be exhaustive. Studies were excluded if they (i) did not adopt a longitudinal design in which index VL cases were linked to subsequent relapse episodes in the same patients, (ii) included more than 5\% VL/HIV co-infection without reporting relapse predictors separately by HIV status; or (iii) focused solely on treatment regimens and/or biomarkers (including molecular targets or cytokines) that are not routinely measured in clinical practice. Although not intended to be comprehensive, key studies conducted in patients with VL/HIV co-infection, as well as those reviewing biomarkers or treatments predictive of relapse, are also discussed.

% systematic reviews
Four systematic reviews describing predictors of relapse were identified: two including patients with and without VL/HIV co-infection\cite{hirve2016, santos2023} and two focussing specifically on patients with VL/HIV co-infection \cite{cota2011,alemayehu2016}. Where appropriate, references from these reviews inform the narrative synthesis.

% comment on study heterogeneity
It is important to recognise that whether a study identifies a significant association between a predictor and relapse does not necessarily correspond to whether or not the predictor is important. For example, (i) not all predictors are considered by all studies, (ii) many studies have small sample sizes and are therefore underpowered to detect certain associations, and (iii) a wide range of study designs, statistical tests, predictor transformations, and modelling strategies are employed, affecting what is considered `significant'. Furthermore, the significance of an association will also depend on the other variables adjusted for in the prediction model.

\subsubsection{Immunocompetent Patients}

A total of 11 studies reporting relapse determinants in immunocompetent patients were identified, and are presented in Table \ref{tab:relapse-rf}. Eight studies were conducted in the ISC, including India\cite{burza2014, goyal2019, goyal2020, ostyn2014, sundar2019}, Bangladesh\cite{lucero2015, mondal2019}, and Nepal\cite{rijal2013}, two in East Africa, including South Sudan\cite{gorski2010} and Kenya\cite{kennedy2024}, and one in Georgia\cite{kajaia2011}. The median study size was 1,143 patients (range 115\cite{rijal2013} to 8,537\cite{burza2014}). Two studies presented outcomes from the same trial\cite{goyal2019, goyal2020}, although they are reported separately here due to differences in sample size and methodology. Follow-up ranged from 6 months to 4 years, and relapse proportions from 2.6\%\cite{lucero2015} to 20.8\%\cite{rijal2013}. Further details, including study design and methodologies, are provided in the \href{https://github.com/jpwil/dphil}{Supplementary Material}.

% age
Age, treated categorically ($\geq$ 2 levels), was the most frequently reported relapse determinant. All 11 studies considered age as a candidate predictor, with eight demonstrating statistically significant (p < 0.05) or borderline significant (0.05$\leq$p < 0.1) associations in unadjusted models\cite{burza2014, goyal2019, goyal2020, lucero2015, ostyn2014, rijal2013, sundar2019, kajaia2011}. When adjusted models were considered, age remained significant in all but one study\cite{sundar2019}. Two ISC studies that modelled $\geq$ 3 age categories showed a U-shaped relationship, with increased relapse risk in young children (< 5 years) and older adults ($\geq$ 40 or $\geq$ 45 years)\cite{burza2014, lucero2015}. Other studies identified increased risk only in younger groups; < 12 or < 15 years in the ISC\cite{goyal2019, goyal2020, rijal2013, sundar2019} and < 1 year in Georgia\cite{kajaia2011}. Notably, the two East African studies did not identify age as a significant predictor\cite{gorski2010, kennedy2024}, although this may reflect limited power (17 relapses in Kennedy et al.\cite{kennedy2024}), or selection bias due to incomplete linking of relapse and index cases\cite{gorski2010}. No study explicitly modelled age as a continuous variable.

% spleen size
Spleen size was the next most frequently identified determinant, considered in seven studies. Five reported significant associations in unadjusted analyses, and four in adjusted models\cite{burza2014, lucero2015, sundar2019, gorski2010, kajaia2011}. Definitions and timing varied. Lucero et al. reported that larger spleens at discharge predicted relapse (OR 1.27, 95\% CI 1.10--1.47 per cm below the costal margin). Sundar et al. found approximately twice the odds of relapse in patients with admission spleen size > 4 cm\cite{sundar2019}. Gorski et al. identified increased relapse odds with larger spleens at discharge \textit{and} admission, when measured with Hackett grade\cite{gorski2010}. Kajaia et al. reported increased relapse risk with larger spleens at admission when measured by the Kandelaki splenometric method\cite{kajaia2011, meliia2006}. Instead of absolute spleen size, Burza et al. modelled change in spleen size during admission: patients with a reduction $\leq$ 0.5 cm/day had 1.7-fold higher odds of relapse (95\% CI 1.1--2.5) compared with those with greater reductions\cite{burza2014}.

% symptom duration
Symptom duration prior to treatment was considered in six studies, with four identifying significant associations in both unadjusted and adjusted models\cite{burza2014, goyal2019, goyal2020, kajaia2011}. In the ISC, Burza et al. reported that patients with $\leq$ 4 weeks of symptoms had higher relapse odds (1.6-fold vs 4--8 weeks; 2.3-fold vs > 8 weeks), with similar effects after adjusting for age, sex, and change in spleen size\cite{burza2014}. Goyal et al. found that $\leq$ 8 weeks of symptoms was associated with 3.3-fold (95\% CI 1.3--8.4) higher relapse odds\cite{goyal2019} and a 3.6-fold (95\% CI 1.4--9.1) higher relapse hazard in time-to-event analysis\cite{goyal2020}. In contrast, Kajaia et al. reported that $\geq$ 90 days of symptoms was associated with higher relapse risk in Georgia (OR 3.9; 95\% CI 1.8--8.5) after adjusting for haemoglobin and age\cite{kajaia2011}. Kennedy et al. included symptom duration but did not identify a significant association, perhaps due to low statistical power\cite{kennedy2024}.

% male
Although sex was considered in all studies, only three ISC studies found significant associations with relapse\cite{burza2014, ostyn2014, sundar2019}. Each reported an approximate doubling in relapse odds\cite{burza2014, sundar2019} or rates\cite{ostyn2014}, with similar effect sizes seen in adjusted analyses.

% Hb
Admission haemoglobin (Hb) was included in eight studies\cite{burza2014, goyal2019, kajaia2011, kennedy2024, lucero2015, rijal2013, sundar2019} and was significant in three\cite{kajaia2011, kennedy2024, lucero2015}. Lucero et al. observed strong associations between lower Hb and relapse in partially adjusted models, although these were not retained in the fully adjusted model (OR 1.40, 95\% CI 1.09--1.72 and 1.45, 95\% CI 1.15--1.85, per 10 g/L decrease, for admission and discharge Hb, respectively). Kajaia et al. reported a strong association, with relapse 12-fold (95\%CI 4.1--34.8) more likely in patients with Hb < 60 g/L vs $\geq$ 80 g/L, persisting after adjustment for age and symptom duration.

% malnutrition
Markers of malnutrition were inconsistently defined across studies, and reporting was often unclear. Gorski et al. and Burza et al. used age-specific definitions combining BMI, BMI-for-age z--scores, and weight-for-height z--scores\cite{gorski2010, burza2014}. Lucero et al. used a similar approach, combining weight-for-height z--scores and BMI\cite{lucero2015}. Goyal et al. and Kennedy et al. described `severe wasting' and `malnutrition', respectively, without further definitions\cite{goyal2019, kennedy2024}. Rijal et al. and Sundar et al. relied on single indicators (BMI and weight, respectively)\cite{rijal2013, sundar2019}. Only Sundar et al. observed a significant association, with patients $\leq$ 30 kg at higher risk, although the causal interpretation is likely confounded by age, given that the study recruited participants of all ages\cite{sundar2019}.

% study characteristics

{% caption group
\newgeometry{left=2.5cm, right=2.6cm, top=2cm, bottom=2cm}
% \captionsetup{width=1\textheight, font=footnotesize, skip=7pt}
\begin{landscape}
    \footnotesize
    \begin{longtable}{@{} L{3.1cm} L{2.5cm} L{2cm} L{1.7cm} L{6cm} L{6cm} @{}}
        \toprule
        Study                                        & Location                                                          & Treatment                 & n (relapse) \%              & Significant predictors -- univariable\footnotemark[5]                                                                                                                                                                                                                  & Significant predictors -- multivariable                                                                                                                                                          \\
        \midrule
        \textbf{ISC}                                 &                                                                   &                           &                             &                                                                                                                                                                                                                                                                        &                                                                                                                                                                                                  \\
        \hspace{0.5em}Burza 2014\cite{burza2014}     & MSF clinics, Bihar, India                                         & 20mg/kg LAMB (4 x 5mg/kg) & 8,537~(119) 13.9\%          & age <5****, age $\geq$45** vs. 15--30~yrs; male sex***; other backwards caste* vs. general category; symptom duration\footnotemark[1] > 8 wks***, > 4 to $\leq$8wks** vs. $\leq$4 wks; spleen size change $\leq$0.5cm/day vs. >0.5cm/day                               & age~<5****, age~$\geq$45** vs. 15--30 yrs; male sex***; symptom duration\footnotemark[1] >8~wks**, >4 to~$\leq$8wks** vs. $\leq$4 wks; spleen size change $\leq$0.5cm/day vs.~>0.5cm/day         \\
        \hspace{0.5em}Goyal 2019\cite{goyal2019}     & Public health facilities, Bihar, India                            & SDA; LAMB+MF; MF+PM       & 1,353 (75) 5.5\%            & treatment****; age 2--12~yrs** vs. >12~yrs; symptom duration $\leq$8~wks*** vs. >8 wks                                                                                                                                                                                 & treatment MF+PM$^{\dagger}$ vs. SDA; age 2--12~yrs$^{\dagger}$ vs. >12~yrs;  symptom duration $\leq$8~wks$^{\dagger}$ vs. >8 wks                                                                 \\
        \hspace{0.5em}Goyal 2020\cite{goyal2020}     & [as above]                                                        & [as above]                & 1,750 (79) 4.5\%            & age 2--12~yrs$^{\dagger\dagger}$ vs. >12~yrs; treatment with MF+PM$^{\dagger\dagger}$                                                                                                                                                                                  & treatment MF+PM**** vs. SDA; age 2--12~yrs*** vs >12~yrs; symptom duration $\leq$8~wks*** vs. >8 wks                                                                                             \\
        \hspace{0.5em}Lucero 2015\cite{lucero2015}   & MSF clinic, Fulbaria, Bangladesh                                  & 15mg/kg LAMB (3 x 4mg/kg) & 1,521 (39) 2.6\%            & age <5~yrs***, age~$\geq$40 yrs** vs. 18--39~yrs; lower admission Hb***; lower discharge Hb***; larger discharge spleen size***                                                                                                                                        & age <5~yrs***, age~$\geq$40 yrs** vs. 18--39~yrs; larger discharge spleen size**                                                                                                                 \\
        \hspace{0.5em}Mondal 2019\cite{mondal2019}   & Hospital in Mymensingh, Bangladesh.                               & 8 different regimens      & 984 (69) 7.0\%.             & treatments compared only: MD-LAMB*, MF**, LAMB+PM**, LAMB+MF***, SDA****, PM****, MF+PM**** vs. SSG                                                                                                                                                                    & treatment\footnotemark[2]: MD-LAMB*, MF**, LAMB+PM**, SDA***, LAMB+MF****, PM****, MF+PM**** vs. SSG                                                                                             \\
        \hspace{0.5em}Ostyn 2014\cite{ostyn2014}     & Hospitals and clinics in Bihar, India                             & MF                        & 853 (78) 9.1\%              & age 10--14 yrs***, 2--9 yrs**** vs. $\geq$25; male sex**                                                                                                                                                                                                               & age 10--14 yrs***, 2--9 yrs**** vs. $\geq$25; male sex***                                                                                                                                        \\
        \hspace{0.5em}Rijal 2013\cite{rijal2013}     & BPKIHS, Eastern Nepal                                             & MF                        & 115 (24) 20.9\%             & age $\leq$12 yrs** vs. >12 yrs                                                                                                                                                                                                                                         & [not performed]                                                                                                                                                                                  \\
        \hspace{0.5em}Sundar 2019\cite{sundar2019}   & KAMRC, Bihar, India                                               & SDA                       & 1,143 (66) 5.8\%            & age $\leq$15 yrs* vs. >15 yrs; male sex**; weight $\leq$30kg** vs. >30kg; spleen size >4cm*** vs. $\leq$4cm                                                                                                                                                            & male sex***; weight $\leq$30kg** vs. >30kg; spleen size >4cm*** vs. $\leq$4cm                                                                                                                    \\
        \midrule
        continued\dots                               &                                                                   &                           &                             &                                                                                                                                                                                                                                                                        &                                                                                                                                                                                                  \\
        \pagebreak[4]
        \midrule
        \textbf{East Africa}                         &                                                                   &                           &                             &                                                                                                                                                                                                                                                                        &                                                                                                                                                                                                  \\
        \hspace{0.5em}Gorski 2010\cite{gorski2010}   & MSF clinics, South Sudan                                          & SSG, SSG/PM, LAMB         & 8,090 (166\footnotemark[3]) & treatment centre: Lankien**** vs other; admission spleen size\footnotemark[4]***; discharge spleen size***; treatment****; lower \% gain in body weight*                                                                                                               & admission spleen size\footnotemark[4] 1$^{\dagger}$, 2$^{\dagger}$, $\geq$3$^{\dagger}$ vs 0; discharge spleen size 2$^{\dagger}$, $\geq$3$^{\dagger}$ vs 0; treatment SSG/PM$^{\dagger}$ vs SSG \\
        \hspace{0.5em}Kennedy 2024\cite{kennedy2024} & Chemolingot Sub-county Hospital, Baringo County, Kenya            & SSG/PM, LAMB              & 248 (17) 6.9\%              & anaemia*; shorter hospital stay*                                                                                                                                                                                                                                       & [not performed]                                                                                                                                                                                  \\
        \textbf{Mediterranean}                       &                                                                   &                           &                             &                                                                                                                                                                                                                                                                        &                                                                                                                                                                                                  \\
        \hspace{0.5em}Kajaia 2011\cite{kajaia2011}   & Institute of Parasitology and Tropical Medicine, Tbilisi, Georgia & MA                        & 300 (21) 7.0\%              & age**: higher risk in <1 yrs; symptom duration***: higher risk with longer duration; Hb****: higher risk with Hb <60g/l; RBC*: higher risk <2.0$\times10^{12}/l$; lymphocytes**: higher risk with $\geq$80\%; admission spleen size**: higher risk with larger spleens & age <1 yr** vs. >1 yr; symptom duration $\geq$90 days*** vs <90 days; Hb <60g/l**** vs. Hb <60g/l                                                                                                \\
        \bottomrule
        \caption{
            Summary of studies predictors of VL relapse in immunocompetent patients. Excluding: studies only looking at treatment regimen or biomarkers.
        \\ BPKIHS: B.P. Koirala Institute of Health Sciences; ISC: Indian subcontinent; KAMRC: Kala-azar Medical Research Centre; MA: meglumine antimoniate; MSF: Médecins Sans Frontières; Hb: haemoglobin; LAMB: liposomal amphotericin B; MF: miltefosine; PM: paromomycin; RBC: red blood cell; SDA: single dose liposomal amphotericin B 10mg/kg; SSG: sodium stibogluconate; VL: visceral leishmaniasis; wks: weeks; yrs: years.
        \\ ****p <0.001; ***0.001$\leq$ p <0.01; **0.01$\leq$ p <0.05; *0.05$\leq$ p <0.1.
        \\ $^{\dagger}$p <0.05 inferred as confidence interval does not cover null; ; $^{\dagger\dagger}$Significance level not stated.
        \\ \textsuperscript{1}Shorter symptom duration associated with higher relapse risk in Burza 2014.
        \\ \textsuperscript{2}Mondal et al describe adjusting the multivariable model for 10 other predictors, including age and sex, but these were reported as not significant.
        \\ \textsuperscript{3}166 relapses cases could be matched with primary case records, and does not reflect total relapse number.
        \\ \textsuperscript{4}Spleen size recorded as Hackett grade in Gorski 2010.
        \\ \textsuperscript{5}For the significant univariable variables, Kajaia et al present the overall p-value across all predictor categories and 95\% CIs for each category. Overall p-values are presented, and the trend commented on.
        }
        \label{tab:relapse-rf}
    \end{longtable}
\end{landscape}
\restoregeometry
} % end caption group


\subsubsection{VL/HIV Co-infection}

Relapse rates in patients with VL/HIV co-infection often exceed 20--50\%\cite{cota2011, diro2019, burza2014a}, highlighting the importance of host immunity in achieving lasting cure.

Two systematic reviews were identified that summarised relapse determinants in this immunosuppressed population, drawing on 18 studies (Cota et al., 2011)\cite{cota2011} and 15 studies (Alemayehu et al., 2016)\cite{alemayehu2016}, respectively. Significant predictors of relapse included low baseline CD4\textsuperscript{+} T-cell counts (particularly < 100 cells/$\mu$L), previous history of VL relapse, lack of improvement in CD4\textsuperscript{+} counts during follow-up, absence of secondary prophylaxis, and the timing of highly active antiretroviral therapy (HAART) initiation.

Since the publication of these reviews, several important studies have shed further light on relapse determinants in VL/HIV co-infected patients\cite{abongomera2017a, costa2023-hiv, diro2019}. In 2017, Abongomera et al. reported outcomes from a retrospective study of 146 co-infected patients treated at an MSF-supported clinic in Northwest Ethiopia between 2008 and 2013, in which 44 (30.1\%) relapsed during follow-up. In addition to the timing of HAART initiation, further predictors included high baseline tissue parasite load (adjusted hazard ratio [aHR] 6.63, 95\% CI 2.64--16.63, 6+ vs $\leq$ 6+) and the presence of splenomegaly on admission (borderline significant in unadjusted analysis)\cite{abongomera2017a}.

More recently, Costa et al. (2023) published a prospective study of 169 co-infected patients from Maranhão, Brazil (2013--2020), in which 70 (41.1\%) relapsed during 12 months of follow-up. Baseline splenomegaly, lymphadenopathy, previous VL relapse, HAART regimen, HAART duration, elevated creatinine, and elevated urea were all statistically significant predictors of relapse in unadjusted analyses\cite{costa2023-hiv}.

In contrast to findings in immunocompetent patients, only one study identified age as a statistically significant predictor of relapse. Burza et al. demonstrated an approximate doubling of relapse hazard among patients $\geq$ 40 years, using routinely collected data from MSF treatment centres in Bihar, India (2007--2012)\cite{burza2014a}.

\subsubsection{Treatments}
\label{sec:treatments}

After VL/HIV co-infection, treatment is arguably the second most important predictor of relapse. This is evident across numerous studies, where the most common reason for failing to achieve the primary outcome of definitive cure \textemdash\  defined as initial cure and relapse-free survival for 6 months \textemdash\  is relapse\cite{alves2018, sundar2019, goyal2020, musa2012, musa2023}. Summarising outcomes from all VL efficacy studies is beyond the scope of this thesis, although several observations merit discussion.

Firstly, as summarised by Chhajed et al., although relapse risk varies considerably across regimens, combination therapies are generally associated with lower relapse risk than monotherapies\cite{chhajed2024}. After adjusting for age and symptom duration, Goyal et al. showed that in India, treatment with paromomycin and miltefosine combination therapy was associated with significantly reduced odds of relapse compared with single dose LAMB 10 mg/kg (aOR 0.21, 95\% CI 0.08--0.55). Similarly, in a large cohort study from Bangladesh, Mondal et al. found that, with the exception of SSG and multiple-dose LAMB, combination regimens were associated with lower relapse rates compared with paromomycin and miltefosine monotherapies\cite{mondal2019}. A possible exception to this pattern is SSG/PM combination therapy. Using programmatic MSF data from South Sudan, 1999--2007, Gorski et al. demonstrated an approximate doubling of relapse odds in the SSG/PM group compared to SSG monotherapy (OR 2.08, 95\% CI 1.21--3.58)\cite{gorski2010}.

Secondly, drug dose and treatment duration are also important predictors of treatment failure and relapse. Chhajed et al. showed this at the aggregate level by performing a meta-regression of relapse risk in the ISC against the dose of single dose LAMB, demonstrating that higher LAMB doses were associated with lower relapse risk (OR 0.81 per 1 mg/kg increase, 95\% CI 0.72--0.91)\cite{chhajed2024}. Similar relationships between dose and overall treatment failure (lack of initial cure or subsequent relapse) have been demonstrated under trial conditions for both LAMB\cite{khalil2014} and paromomycin\cite{musa2010, hailu2010}. In East Africa, Dorlo et al. used a paediatric pharmacokinetic--pharmacodynamic (PK--PD) model to show an inverse relationship between miltefosine exposure and relapse hazard, with similar findings reported in the ISC\cite{dorlo2012}. These results supported the development of allometric dosing strategies for paediatric miltefosine\cite{dorlo2017, mbui2019}.

Lastly, both the intrinsic infectivity of the parasite strain and the development of drug resistance contribute to increasing rates of initial treatment failure and, when initial cure is achieved, subsequent relapse. A well-known example is the emergence of antimony resistance in Bihar, India, where treatment failure and relapse rates following treatment with SSG increased from < 5\% to > 50\% during the 1980s and 1990s, despite incremental escalation in SSG dose and duration\cite{olliaro2005, sundar2001}. These failures have been attributed to both antimony resistance\cite{jeddi2011}, and also to the selective survival and transmission of more infectious parasite strains\cite{vanaerschot2011}. Similar concerns arose following the widespread introduction of miltefosine in the ISC in the late 2000s, when increasing relapse rates were observed\cite{rijal2013, sundar2012, singh2016}. Interestingly, comparisons of relapsing versus non-relapsing strains revealed no evidence of drug resistance, although relapsing strains were found to be more infectious\cite{rai2013}. More recently, Naylor-Leyland et al. reported increasing relapse rates between 2001 and 2018 that could not be explained by patient-level factors, prompting concerns about declining SSG/PM efficacy potentially due to emerging resistance\cite{naylor-leyland2022}.

% Given these findings, increasing relapse rates have been proposed as a potential bellwether of emerging drug resistance more broadly\cite{vangriensven2024}.

% Regardless of the treatment received, it is widely accepted that an effective host immunity is crucial to achieving lasting cure\cite{murray2005, khalil2005, franssen2021, alves2018}. In this context, the goal of treatment is not complete parasite eradication, but rather sufficient suppression of the parasite burden to facilitate reconstitution of the host's cell-mediated immune response and  subsequent parasite control.

% dose response - LAMB
% Sundar S, Jaya J. Liposomal amphotericin B and leishmaniasis: Dose and response. J Glob Infect Dis. 2010;2(2):159. 

\subsubsection{Biomarkers}

% intro
Since tissue aspirate sampling is invasive, substantial effort has been directed toward developing biomarkers that can serve both as a test--of--cure following treatment and as a diagnostic test of relapse\cite{WHO2024_Leishmaniasis}. In fact, a 2015 systematic review identified 53 biomarkers from 170 studies with potential to be used for monitoring post-treatment outcomes in Leishmaniasis\cite{kip2015}. Biomarkers can be broadly categorised into two categories: (1) direct markers of parasite burden, including DNA and antigen detection; and (2) indirect markers of host immunity, such as antibodies, cytokines and acute-phase proteins.

% molecular markers
A large portion of the VL biomarker literature focuses on the direct detection of \textit{Leishmania} DNA/RNA in both immunosuppressed\cite{antinori2007, bourgeois2008, cota2017, molina2013, nicodemo2013, verma2017} and immunocompetent patients\cite{hossain2017, mary2006, sudarshan2011, verrest2021, verrest2024}. In particular, real-time quantitative PCR (qPCR) targeting kDNA, performed on the buffy coat of peripheral blood, has been confirmed as a strong predictor of relapse\cite{hossain2017, sudarshan2011, verrest2021, verrest2024}, although routine use is limited by the need for specialised laboratory expertise and high costs. In a landmark 2021 study, Verrest et al. used qPCR measurements from 177 immunocompetent patients enrolled in DNDi trials in East Africa\footnote{LEAP 0714 (30 children treated with allometric miltefosine), LEAP 0208 (151 patients receiving miltefosine plus LAMB combination regimens), FEXI VL 001 (14 patients receiving fexinidazole)} and demonstrated that qPCR levels measured between treatment initiation and day 56 significantly predicted relapse, with areas under the receiver operating curve (AUC) of 0.71, 0.74, and 0.92 on days 14, 28, and 56, respectively\cite{verrest2021}. Using the same qPCR data, Verrest et al. (2024) developed a semi-mechanistic population PK--PD model describing parasite replication, drug action, and post-treatment parasite clearance. The model successfully predicted relapse based on modelled day 28 and 56 parasite loads and provided the first direct evidence that relapse risk depends not only on initial parasite clearance but also on subsequent host immune responses\cite{verrest2024}. Interestingly, inclusion of haematological and biochemical parameters,\footnote{including Hb, white blood cells (WBC), platelets, and creatinine} however, did not account for variation in post-treatment qPCR values.

% antigen testing
Beyond molecular testing, several studies have shown an association between relapse and the degree of \textit{Leishmania} antigenuria in patients with VL/HIV co-infection\cite{vangriensven2018, riera2004}. In Ethiopian patients, higher levels of urinary antigen at test--of--cure were strongly associated with relapse over the following 12 months (HR 9.8, 95\% CI 1.8--82.1, comparing 1--10 parasites/10 fields with no parasites detected using the KATex assay).

% serology (antibody)
Post-treatment IgG subclass titres have also been linked to relapse in both immunocompetent and immunosuppressed individuals\cite{bhattacharyya2014, mondal2019-serology, mollett2019, kuschnir2021}. However, because of the long half-life of immunoglobulins, their ability to predict relapse at the time of initial cure assessment remains uncertain. In asymptomatic patients, however, serological studies have shown that individuals with higher antibody titres are more likely to progress to symptomatic disease\cite{chapman2015, vangriensven2024_preleish}.

% other biomarkers - cytokines
A variety of non-serological and indirect biomarker panels have been evaluated for assessing disease activity during asymptomatic infection, active disease, and post-treatment recovery\cite{samant2021, takele2022, tadesse2021, guedes2022, kip2018, torres2025}. In East African patients, Kip et al. reported that the change in neopterin, a marker of macrophage activation, between EoT and day 60 of follow-up was predictive of relapse, with an AUC of 0.84. In a subsequent study of 34 Ethiopian VL/HIV co-infected patients, half of whom relapsed, Takele et al. showed that relapse was associated with (i) failure to restore antigen-specific IFN$\gamma$ production, (ii) persistently low CD4\textsuperscript{+} T-cell counts, and (iii) high T-cell expression of PD1. These results underscore the importance of effective cell-mediated immunity in achieving sustained cure.

\subsubsection{Summary of Relapse Determinants}

A substantial and rapidly growing evidence base has been identified describing the determinants of VL relapse, although methodological heterogeneity complicates synthesis.

Among immunocompetent patients \textemdash\  the focus of this thesis \textemdash\  the most consistently identified predictors are clear. Extremes of age (young children and older adults), larger spleen size, whether measured at admission or at EoT, male sex, and the severity of anaemia (at admission and EoT) are repeatedly associated with higher relapse risk. In contrast, evidence for malnutrition is weak, perhaps in part due to the varying definitions, and nuances in how measures are adjusted for age. Symptom duration shows conflicting associations. Shorter pre-treatment symptom intervals were strongly linked to higher relapse risk in two ISC studies, yet an opposite association was reported in the study from Georgia \textemdash\  highlighting potential contextual or methodological modifiers.

In patients with VL/HIV co-infection, predictors align more with immunological plausibility: markers of profound immunosuppression (notably low baseline CD4\textsuperscript{+} counts) and indicators of inadequate immune reconstitution (including timing and duration of HAART, absence of secondary prophylaxis) were the dominant determinants of relapse identified.

As anticipated, treatment-related factors are also highly influential. Treatment regimen, dose, and duration modify both initial cure and subsequent relapse risk, and therefore any prognostic model should be accounting for treatment variables where possible.

Finally, biomarker research (particularly peripheral blood qPCR) shows promising predictive performance, but current approaches are constrained by cost, laboratory requirements and limited routine availability, restricting their immediate applicability in many endemic settings.

\section{Summary}

VL remains a highly neglected tropical disease, disproportionately affecting the poorest and most marginalised communities. Despite substantial gains over the last two decades \textemdash\  most notably in the ISC, where coordinated elimination efforts have driven steep reductions in incidence \textemdash\  VL continues to impose a considerable clinical and economic burden. Sustaining these achievements is challenging, and translating success to high-burden regions such as East Africa remains an urgent global health priority\cite{WHO_NTDs_Roadmap_2021_2030}.

Relapse is a critical, yet often understated, component of this challenge, with consequences both at the individual patient level and for public health more broadly.

For the patient, relapse is associated with heightened morbidity and mortality, exposure to prolonged and often toxic second-line therapies, and considerable direct and indirect financial costs for already vulnerable households.

From a public health and elimination perspective, relapse carries further implications. In anthroponotic transmission settings, early identification and effective treatment of relapse cases are essential to interrupt transmission. Moreover, parasite strains isolated from relapse cases can harbour resistance mutations and demonstrate greater fitness, underscoring their potential role as a consequential parasite reservoir.

Given the clinical, economic, and epidemiological significance of relapse, there is a clear and globally recognised need to improve the identification of individuals at greatest risk\cite{WHO2024_Leishmaniasis,WHO_2024_VL_easternAfrica}. Motivated by this, the overarching aim of this thesis is to determine whether clinical trial data contained within the IDDO VL data platform can be leveraged to support the risk stratification of patients who are likely to progress to relapse following initial cure. In particular, this thesis explores whether such data can inform the development of a robust prognostic model for relapse.

% link to next chapter
If a prognostic model for VL relapse already exists, data from the IDDO VL data platform could be used to validate and, where necessary, update the model\cite{collins2024A}. Therefore, to establish the current evidence base, the next chapter presents a systematic review of prognostic models developed for VL, encompassing both relapse-specific models and those predicting broader clinical outcomes.

%Rising relapse rates have been proposed as a bellwether of emerging drug resistance\cite{vangriensven2024}. 
