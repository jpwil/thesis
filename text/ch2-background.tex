\begin{savequote}[8cm]
    \textlatin{Many large baries [homesteads], in which there were formerly thirty or forty residents, have now been left with perhaps one solitary occupant... whole mohullas [neighbourhoods] and streets have been deserted, and large villages which formerly told their residents by thousands, can now almost number them by hundreds.
        \qauthor{--- exerpt from The Annual Reports of the Sanitary Commissioners for Bengal, 1865}}
\end{savequote}

\chapter{\label{ch:2-background}Visceral leishmaniasis}
\minitoc

\section{Introduction}

% speak more broadly about leishmaniases before narrowing in on VL specifically
The leishmaniases are a diverse group of neglected tropical diseases (NTDs) caused by protozoan parasites from the genus \textit{Leishmania} and transmitted between susceptible mammalian hosts by the bite of infected female sandflies. Humans are vulnerable to at least 20 \textit{Leishmania} species, manifesting as four principal disease forms: cutaneous leishmaniasis (CL), mucocutaneous leishmaniasis (MCL), visceral leishmaniasis (VL) and post kala-azar dermal leishmaniasis (PKDL)\cite{farrar2023manson, burza2018, who_leish}. These distinct forms are determined largely by parasite species and strain, and associated with a spectrum of clinical presentations ranging in severity from the relatively common and typically self-healing skin lesions seen in CL, to the often disfiguring mucosal destruction characteristic of MCL, and life-threatening systemic illness of VL\cite{burza2018}.

VL is by far the most severe manifestation of leishmaniasis and responsible for the lion's share of the leishmaniases' mortality and morbidity. The disease overwhelmingly affects impoverished populations, often living in rural areas with poor access to healthcare.

The aim of this chapter is to lay the groundwork for subsequent sections on the development and validation of prognostic models predicting VL relapse in the Indian subcontinent (ISC)\footnote{Despite sporadic VL cases reported from Pakistan, Bhutan and Sri Lanka, in this thesis ISC refers to India, Nepal and Bangladesh.} and East Africa. The chapter begins with an overview of VL, covering its epidemiology, clinical features, life cycle, pathophysiology, immunology and management. Particular emphasis is placed on the ongoing WHO-supported elimination programmes in the ISC and East Africa, which shape the public health landscape in which a VL prognostic model would be implemented. This chapter concludes with a narrative review of current knowledge on the determinants of VL relapse, informed by a systematic review of the literature.

\section{Epidemiology}

VL is endemic\footnote{Defined by the WHO as the occurrence of at least one autochthonous case with demonstrated local transmission within a country\cite{who_wer2023}.} in at least 80 countries across tropical, semi-tropical and temperate regions. The disease is caused by two closely related \textit{Leishmania} species whose distribution defines the four principal global regions of high endemicity (see figure \ref{fig:gho_map}): \textit{L. donovani}, responsible for anthroponotic transmission in the ISC and East Africa, and \textit{L. infantum}\footnote{Previously referred to as \textit{L. chagasi} in the Americas.}, responsible for zoonotic transmission in the Americas (principally Brazil) and Mediterranean basin, extending into the Middle East and Central Asia\cite{who_wer2023,who_gho}. Together, these two species comprise the \textit{L. donovani complex}.

\subsection{Disease Burden and Geographical Distribution}

% cp /Users/jameswilson/proj/vl_visualisation/results/gho_map.pdf figures/ch2/gho_map.pdf
\begin{landscape}
    \begin{figure}[htbp]
        \centering
        \includegraphics[clip, trim={0cm 2.3cm 0cm 2.2cm}, width=1.55\textwidth]{figures/ch2/gho_map.pdf}
        \caption{Visceral leishmaniasis endemicity (reported 2023) and average annual cases (reported 2018 - 2023). `Considered endemic' - at least one autochthonous case has been reported, with or without the entire cycle of transmission being demonstrated. Data sourced from the World Health Organization Global Health Observatory, accessed October 2025\cite{who_gho}. }
        \label{fig:gho_map}
    \end{figure}
\end{landscape}

Estimating the global burden of VL is problematic, with true case numbers obscured by significant underreporting due limited access to healthcare, inadequate diagnostic facilities, misdiagnosis, and poor surveillance systems in many endemic countries\cite{chappuis2007,singh2006, mubayi2010, maia-elkhoury2007}. In 2012, Alvar et al. published the results of an important WHO-led update to the global incidence of leishmaniasis using country-level reporting from the mid-late 2000s\cite{alvar2012};  underreporting rates were estimated systematically through consultation with country representatives and disease experts. The global incidence was estimated at 200,000--400,000/year, with approximately 80\% of the burden originating from the ISC and 15\% from East Africa. Compared to official reporting during the same time period of 58,000/year, this reflects a global underreporting rate of 3.5--7-fold.

Since the publication of Alvar et al.'s estimates, the number of cases has undisputedly fallen, driven largely by decreases in the ISC following the launch of the Kala-Azar Elimination Programme (KEAP) in 2005. Officially reported cases decreased from >50,000/year prior to 2012, to approximately 22,500/year in 2017 and <12,000/year in 2023\cite{who_gho} (selected country breakdowns presented in Figure \ref{fig:gho_plot_top8}). Notably, this downward trend has persisted despite improvements in surveillance and reporting systems in many endemic countries\cite{who_wer2023}. Reflecting these changes, the WHO revised its estimated annual incidence in 2017 to 50,000--90,000/year\footnote{According to the online WHO Leishmaniasis Fact Sheet: \url{https://www.who.int/news-room/fact-sheets/detail/leishmaniasis}. Alternate WHO online material reports an estimated 30,000/year since 2020: \url{https://www.who.int/health-topics/leishmaniasis} (online material last accessed October 2025).}\cite{who_leish}.

\begin{figure}[h]
    \centering
    \includegraphics[width=1\textwidth]{figures/ch2/gho_plot_top8.pdf} % path to your figure file
    \caption{Temporal trends of reported new visceral leishmaniasis cases by country. Comparison between the top 8 countries with the highest average case numbers reported between 2013 and 2023. Data sourced from the World Health Organization Global Health Observatory\cite{who_gho}.}
    \label{fig:gho_plot_top8}
\end{figure}

Based on the most recent reporting data from 2023\cite{who_gho}; the five countries with the highest case numbers are now  Sudan, Ethiopia, Brazil, Kenya and South Sudan -- collectively comprising 72.4\% of the global total. In stark contrast to the situation 20 years ago, countries in the ISC now account for only 6.3\% of the reported total: India with 538 cases (4.6\%), Nepal with 168 cases (1.4\%), and Bangladesh with just 34 cases (0.3\%).

Approximately 5--15\% of cases are thought to result in death, although accurate mortality estimates are challenged by a paucity of reporting and, where deaths are reported, they frequently only reflect hospital deaths and omit those where a definite diagnosis was missed. Whilst in the 2000s, VL was blamed for causing the second highest number of deaths from a parasitic disease after malaria\cite{chappuis2007}, more recent estimates tentatively place the figure at a more modest 4,627/year with a wide uncertainty range of 1,853--8,725 deaths/year\cite{gbd2023}.

\subsection{Transmission}

\subsubsection{Vector}
Covered in dense hairs and measuring 2-4mm in length, Phlebotomine sandflies (Diptera, Psychodidae) have a distinctly fuzzy appearance under magnification. Females from an estimated 31 species across two genera are known to transmit the parasite between human hosts: \textit{Lutzomyia} in the New World and \textit{Phlebotomine} in the Old World\cite{akhoundi2016}. Sandflies occupy a wide range of ecological niches, found on every continent except Antarctica. Biting occurs from dusk, with females requiring a blood meal for larval development. During the day they are found in cool and sheltered locations, such as in cracks and crevices in walls as seen with \textit{Ph. argentipes}, responsible for transmission in the ISC. In East Africa, three sandfly vectors have been incriminated for \textit{L. donovani} transmission, defining two distinct and non-overlapping ecological settings: (i) the \textit{Acacia-Balanites} and black cotton soil savannah regions in northern Ethiopia, Sudan and northern South Sudan, where \textit{Ph. orientalis} thrives, and (ii) the savannah and forest areas in the southern focus of southern Ethiopia, Kenya and Uganda, where \textit{Ph. martini} and \textit{Ph. celiae} are seen in association with \textit{Macrotermes} termite mounds.

In addition to sandflies, needle sharing among people who inject drugs was considered an important route of transmission in the southern Mediterranean region during the 1990s and 2000s, particularly among people living with HIV \cite{alvar1997}. Exceptionally, transmission can  result from blood transfusion, organ transplantation, congenital infection, laboratory accidents\cite{farrar2023manson}, and possibly even sexual contact\cite{guedes2020, symmers1960}.

\subsubsection{Reservoirs}

Similar to the majority of \textit{Leishmania} spp. causing CL and MCL, \textit{L. infantum} demonstrates zoonotic transmission (animal $\rightarrow$ sandfly $\rightarrow$ human), with domestic dogs being the main reservoir host in both the Americas and the Old World. This being said, an ever-increasing list of wild and domestic animals are known to harbour the parasite, including cats, foxes, horses, rodents, bats and opossums, although their relevance to human infection is unclear\cite{alcover2020, ratzlaff2023}. An outbreak near Madrid (2009--2012) was attributed to hares\cite{molina2012}.

In contrast to \textit{L. infantum}, and crucially from an elimination perspective, \textit{L. donovani} transmission in the ISC and East Africa is predominantly anthroponotic (human $\rightarrow$ sandfly $\rightarrow$ human). Although \textit{L. donovani} infections have been reported in several animal species in both regions, including cattle and dogs, the significance of these infections as potential sources of human transmission has yet to be established \cite{kushwaha2024, jones2021}.

\subsection{Risk Factors}

From population prevalence studies we know that across all endemic areas, only a minority of people with detectable parasites develop symptoms\cite{pederiva2022, burza2018}. Risk factors for acquiring an initial asymptomatic infection and subsequent progression to symptomatic disease reflect a tangled ecology of determinants linking host factors (sandfly exposure, immunity, genetics) and parasite factors (strain, virulence, inoculum). A common theme, woven into many of these determinants, is poverty.

In both the ISC and East Africa the median age of infection during stable transmission is similar at 15--20 years. More men than women are infected and develop disease, reflecting their increased occupational exposure to sandflies (for example, cattle herding and other farming activities).

In the ISC, VL endemicity is centred in the fertile and low-lying alluvial plains of the Ganges river, where high humidity, heavy monsoon rains, and abundant vegetation provide ideal conditions for sustained transmission between sandflies and humans\cite{bhunia2010}. Significant clustering of cases is seen across the rural farming communities of Bihar, Jharkhand, Uttar Pradesh, and West Bengal in northeastern India, central and western Bangladesh, and southeastern Nepal. In a systematic review by Bern et al., risk factors for VL transmission in the ISC were identified, and included living in mud houses, proximity to prior cases (in the same or nearby household), presence of vegetation and standing water surrounding the house, sleeping on the floor or outside, malnutrition, and a lack of bed net use\cite{bern2010}.

The greatest concentration of cases in East Africa is reported in the northern focus, specifically the border area between the eastern Sudanese states and the northern states of Ethiopia. For the purpose of elimination, the WHO is targeting nine endemic East African countries (Chad, Djibouti, Eritrea, Ethiopia, Kenya, Somalia, South Sudan Sudan and Uganda), representing a wide diversity of ecological settings, populations, and region-specific risk factors\cite{WHO_2024_VL_easternAfrica,geto2024}. Many of the epidemiological determinants of VL in East Africa are shared with the ISC, with poverty remaining the central overarching factor. Notable determinants include living in rural settings near sandfly breeding and resting sites (living near termite mounds in the southern focus, sleeping under \textit{Acacia} trees in the northern focus), living in proximity to other VL infected (or recently infected) people, and malnutrition\cite{geto2024}.

Human immunodeficiency virus (HIV) infection has reshaped VL epidemiology in many endemic regions, and remains the most important risk factor for asymptomatic infection, disease progression, and poor treatment outcomes. In the mid-1980s in Spain and other southern European countries, VL shifted from a rare childhood disease to one predominantly affecting HIV-positive adults\cite{monge2014, alvar2012}. HIV-VL co-infection rates are currently increasing in Brazil (now reported in 20\% of new cases\cite{PAHO_Leishmaniasis_2024}) and India (>10\% of VL episodes in 2023 and 2024\cite{NCVBDC2025}). In northern Ethiopia, HIV-VL coinfection presents a significant challenge to elimination, affecting between 20 and 50\% of cases predominantly in male migrant workers\cite{diro2014}.

% host immunity is key some of this thesis' conclusions, and explaining relapse predictors
Host immunity as a risk factor for disease onset and poor treatment outcomes cannot be understated, and is considered an important driving factor for a number of devastating epidemics related to war and natural disasters. A striking example is Sudan in the 1980s-2000s, where population displacement driven by conflict famine brought immune-naïve populations into endemic areas and spread infection into previously unaffected areas. Combined with severe malnutrition and the collapse of health infrastructure, the resulting mortality was catastrophic\cite{zijlstra1991, collin2004, al-salem2016}.

\section{Pathophysiology}
% I'm including this chapter before clinical/diagnostic details because it provides important background  

Parasites of the genus \textit{Leishmania} belong to the family \textit{Trypanosomatidae} (class \textit{Kinetoplastea}, phylum \textit{Euglenozoa}) and share many features with two other human pathogens of the same family; \textit{Trypanosoma brucei} -- the causative agent of African sleeping sickness (human African Trypanosomiasis), and \textit{Trypanosoma~cruzi} -- the agent of Chagas disease in the Americas. These vector-borne NTD siblings are all single-celled protozoa characterized by the presence of a flagellum and a kinetoplast -- a dense network of mitochondrial DNA that conveys a distinctive microscopic appearance.

\begin{figure}[h]
    \centering
    \includegraphics[width=1\textwidth]{figures/ch2/cdc_lifecycle.png} % path to your figure file
    \caption{Life cycle of the \textit{Leishmania} parasite consists of a vector (sandfly) stage and reservoir (animal) stage. Whilst the vector stage is by the far the most common mode of transmission, it is not essential, with direct transmission between humans reported from needle sharing, organ and blood donation. Illustration credit: Centers for Disease Control and Prevention/Alexander J. da Silva/Melanie Moser}
    \label{fig:cdc_lifecycle}
\end{figure}

\subsection{Life Cycle}
In the sandfly gut, \textit{Leishmania} parasites exist in their extracellular flagellated promastigote form which are regurgitated into the bite wound during a blood meal and rapidly phagocytosed by macrophages. Intracellularly, the protozoa lose their flagellum and transform into amastigotes: oval bodies measuring approximately 2--6$\mu$m in length with their characteristic nucleus and kinetoplast clearly visible with nucleic acid staining. Intracellular amastigotes multiply through binary fission, eventually leading to cell lysis and subsequent uptake by neighboring macrophages to perpetuate infection. In viscerotropic disease caused by \textit{L. donovani complex} spp., this cycle drives widespread dissemination with macrophage proliferation and granuloma formation across the mononuclear phagocyte system, including the spleen, liver, bone marrow and lymph nodes. The cycle continues when sandflies ingest infected macrophages during a subsequent blood meal, initiating parasite development within the vector (figure \ref{fig:cdc_lifecycle}).

\subsection{Immune Evasion}

Remarkably, intracellular amastigotes have evolved to survive and replicate within the hostile phagolysosomal environment of the macrophage. Whilst full immunological details are poorly understood, and beyond the scope of this chapter, key strategies include the suppression of macrophage activation with elevated IL-10 production and the functional impairment of cytotoxic CD8\textsuperscript{+} T-cell responses. Despite the presence of a pronounced humoral response, with marked polyclonal hypergammaglobulinaemia typically seen, the resulting antibodies are predominantly non-neutralising, and may even play a role in maintaining the immunosuppressive milieu required for parasite survival\cite{stager2010,kaye2020}.

\section{Clinical Features}

The spectrum of illness is perhaps surprising, with prevalence studies revealing that asymptomatic carriers often outnumber symptomatic cases 5--20-fold in endemic settings\cite{pederiva2022,burza2018}. Where manifested, symptoms typically present insidiously following an incubation period of weeks--months, and sometimes years. Classical symptoms are a consequence of a persistent systemic inflammatory response syndrome with infiltration of the mononuclear phagocyte system, resulting in intermittent fever, weight loss, splenomegaly, often with accompanying hepatomegaly and, in parts of East Africa, lymphadenopathy. Often-described laboratory findings include pancytopenia and (where measured), a polyclonal hypergammaglobulinaemia.

Anaemia is often severe, normocytic and normochromic in nature, and likely multifactorial in origin, with splenic sequestration and anaemia of chronic disease thought to be important\cite{varma2010,goto2017}.Haemoglobin levels are frequently seen between 6--10 g/dL\cite{varma2010}, contributing to weakness. Whilst the spleen is usually non-tender on palpation, massive splenomegaly and hepatomegaly can result in severe abdominal pain. The term `kala-azar' translates to `black disease' in Hindi, coined in the 19\textsuperscript{th} century due to the observation of a grayish skin hyperpigmentation, although this is an infrequent finding in modern case series\cite{farrar2023manson}.



The disease is widely described as `fatal without treatment'\cite{burza2018,farrar2023manson}. Supporting this statement are the high mortality figures estimated during conflict-related epidemics in East Africa over the last 40 years, and prior to effective therapy, 19\textsuperscript{th} century accounts of outbreaks devastating communities across the Ganges delta\cite{collin2004, gibson1983,steverding2017}. Despite this, subclinical forms of the disease have been reported with spontaneous resolution\cite{mouri2015,badaro1986}.

\section{Management}

\section{Control and Elimination}
