\begin{savequote}[8cm]
    \textlatin{Many large baries [homesteads], in which there were formerly thirty or forty residents, have now been left with perhaps one solitary occupant... whole mohullas [neighbourhoods] and streets have been deserted, and large villages which formerly told their residents by thousands, can now almost number them by hundreds.
        \qauthor{--- exerpt from The Annual Reports of the Sanitary Commissioners for Bengal, 1865\cite{gibson1983}}}
\end{savequote}

\chapter{\label{ch:2-background}Background}
\minitoc

\section{Visceral leishmaniasis}

% speak more broadly about leishmaniases before narrowing in on VL specifically
The leishmaniases are a diverse group of neglected tropical diseases (NTDs) caused by protozoan parasites from the genus \textit{Leishmania} and transmitted between susceptible mammalian hosts by the bite of infected female sandflies. Humans are vulnerable to at least 20 \textit{Leishmania} species, manifesting as four principal disease forms: cutaneous leishmaniasis (CL), mucocutaneous leishmaniasis (MCL), visceral leishmaniasis (VL) and post kala-azar dermal leishmaniasis (PKDL) (see Box \ref{box:main-four})\cite{farrar2023manson, burza2018, who_leish}. These distinct forms are determined largely by parasite species and strain, and associated with a spectrum of clinical presentations ranging in severity from the relatively common and typically self-healing skin lesions seen in CL, to the often disfiguring mucosal destruction characteristic of MCL, and life-threatening systemic illness of VL\cite{burza2018}.

VL is by far the most severe manifestation of leishmaniasis and responsible for the lion's share of the leishmaniases' mortality and morbidity. The disease overwhelmingly affects impoverished populations, often living in rural areas with poor access to healthcare. co-infection with both VL and human immunodeficiency virus (HIV)...

The aim of this chapter is to lay the groundwork for subsequent sections on the development and validation of prognostic models predicting VL relapse in the Indian subcontinent (ISC)\footnote{Despite sporadic VL cases reported from Pakistan, Bhutan and Sri Lanka, in this thesis ISC refers to India, Nepal and Bangladesh.} and East Africa. The chapter begins with an overview of VL, covering its epidemiology, clinical features, life cycle, pathophysiology, immunology and management. Particular emphasis is placed on the ongoing WHO-supported elimination programmes in the ISC and East Africa, which shape the public health landscape in which a VL prognostic model would be implemented. The chapter concludes with a narrative review of current knowledge on the determinants of VL relapse, informed by a systematic review of the literature.

\begin{mybox}{Principal disease forms of Leishmaniasis\cite{who_leish,farrar2023manson}}
    \begin{description}
        \item[Cutaneous leishmaniasis (CL)] Results in lesions on exposed skin that can lead to ulceration and life-long scarring. Often self-healing within a year, but can manifest atypical and disseminated forms, especially in the immunocompromised. Up to 1 million new cases per year with most cases occurring in the Americas, Mediterranean basin, Middle East and Central Asia.
        \item[Mucocutaneous leishmaniasis (MCL)] Rare complication of CL seen especially in the Americas with most cases reported in Bolivia, Peru and Brazil. Results in destructive ulceration of the oral and nasal mucosa. Highly stigmatising and challenging to treat.
        \item[Visceral leishmaniasis (VL)] Also known as kala-azar, the most severe form of leishmaniasis caused by \textit{L. donovani} in the Old World and \textit{L. infantum} in the New World. With an estimated 50,000--90,000 cases/year, VL presents with progressive weight loss, splenomegaly and fever. Typically fatal without treatment.
        \item[Post kala-azar dermal leishmaniasis (PKDL)] Benign macular and/or papular rash often including the face, arms and trunk.  Affects 5--20\% of patients months to years after successful initial treatment for VL in the Indian subcontinent and East Africa. Often self-limiting, although known to be infective to sandflies and therefore act as a disease reservoir.
    \end{description}
    \label{box:main-four}
\end{mybox}

\subsection{Epidemiology}

VL is endemic\footnote{Defined by the WHO as the occurrence of at least one autochthonous case with demonstrated local transmission within a country\cite{who_wer2023}.} in at least 80 countries across tropical, semi-tropical and temperate regions. The disease is caused by two closely related \textit{Leishmania} species whose distribution defines the four principal global regions of high endemicity (see figure \ref{fig:gho_map}): \textit{L. donovani}, responsible for anthroponotic transmission in the ISC and East Africa, and \textit{L. infantum}\footnote{Previously referred to as \textit{L. chagasi} in the Americas.}, responsible for zoonotic transmission in the Americas (principally Brazil) and Mediterranean basin, extending into the Middle East and Central Asia\cite{who_wer2023,who_gho}. Together, these two species comprise the \textit{L. donovani complex}.

% cp /Users/jameswilson/proj/vl_visualisation/results/gho_map.pdf figures/ch2/gho_map.pdf
\begin{landscape}
    \begin{figure}[htbp]
        \centering
        \includegraphics[clip, trim={0cm 2.3cm 0cm 2.2cm}, width=1.55\textwidth]{figures/ch2/gho_map.pdf}
        \caption{Visceral leishmaniasis endemicity (reported 2023) and average annual cases (reported 2018 - 2023). `Considered endemic' - at least one autochthonous case has been reported, with or without the entire cycle of transmission being demonstrated. Data sourced from the World Health Organization Global Health Observatory, accessed October 2025\cite{who_gho}. }
        \label{fig:gho_map}
    \end{figure}
\end{landscape}


\subsubsection{Disease burden}

Estimating the global burden of VL is problematic, with true case numbers obscured by significant underreporting due limited access to healthcare, inadequate diagnostic facilities, misdiagnosis, and poor surveillance systems in many endemic countries\cite{chappuis2007,singh2006, mubayi2010, maia-elkhoury2007}. In 2012, Alvar et al. published the results of an important WHO-led update to the global incidence of leishmaniasis using country-level reporting from the mid-late 2000s\cite{alvar2012};  underreporting rates were estimated systematically through consultation with country representatives and disease experts. The global incidence was estimated at 200,000--400,000 cases/year, with approximately 80\% of the burden originating from the ISC and 15\% from East Africa. Compared to official reporting during the same time period of 58,000 cases/year, this reflects a global underreporting rate of 3.5--7-fold.

Since the publication of Alvar et al.'s estimates, the number of cases has undisputedly fallen, driven largely by decreases in the ISC following the launch of the Kala-Azar Elimination Programme (KEAP) in 2005. Officially reported cases decreased from >50,000 cases/year prior to 2012, to approximately 22,500 cases/year in 2017 and <12,000 cases/year in 2023\cite{who_gho} (selected country breakdowns presented in Figure \ref{fig:gho_plot_top8}). Notably, this downward trend has persisted despite improvements in surveillance and reporting systems in many endemic countries\cite{who_wer2023}. Reflecting these changes, the WHO revised its estimated annual incidence in 2017 to 50,000--90,000 cases/year\footnote{According to the online WHO Leishmaniasis Fact Sheet: \url{https://www.who.int/news-room/fact-sheets/detail/leishmaniasis}. Alternate WHO online material reports an estimated 30,000 cases/year since 2020: \url{https://www.who.int/health-topics/leishmaniasis} (online material last accessed October 2025).}\cite{WHO_2016_leishmaniasis,who_leish}.

\begin{figure}[h]
    \centering
    \includegraphics[width=1\textwidth]{figures/ch2/gho_plot_top8.pdf} % path to your figure file
    \caption{Temporal trends of reported new visceral leishmaniasis cases by country. Comparison between the top 8 countries with the highest average case numbers reported between 2013 and 2023. Data sourced from the World Health Organization Global Health Observatory\cite{who_gho}.}
    \label{fig:gho_plot_top8}
\end{figure}

Based on the most recent reporting data from 2023\cite{who_gho}; the five countries with the highest case numbers are now  Sudan, Ethiopia, Brazil, Kenya and South Sudan -- collectively comprising 72.4\% of the global total. In stark contrast to the situation 20 years ago, countries in the ISC now account for only 6.3\% of the reported total: India with 538 cases (4.6\%), Nepal with 168 cases (1.4\%), and Bangladesh with just 34 cases (0.3\%).

\subsubsection{Mortality}

The disease is widely described as `fatal without treatment'\cite{burza2018,farrar2023manson}. Supporting this statement are the high mortality figures recorded during conflict-related epidemics in East Africa over the last 40 years, and prior to effective therapy, 19\textsuperscript{th} century accounts of outbreaks devastating communities across the Ganges delta\cite{collin2004, gibson1983,steverding2017}. Despite this, subclinical forms of the disease have been reported with spontaneous resolution\cite{mouri2015,badaro1986}.

With treatment, \textasciitilde5--15\% of cases result in death, although accurate estimates are challenged by a lack of reporting. Where deaths are reported, they frequently only reflect hospital deaths and omit those where a definite diagnosis was missed. Whilst in the 2000s, VL was blamed for causing the second highest number of deaths from a parasitic disease after malaria\cite{chappuis2007}, more recent estimates tentatively place the figure at a more modest 4,627 deaths/year with a wide uncertainty range of 1,853--8,725 deaths/year\cite{gbd2023}.

\subsubsection{Vector}
Covered in dense hairs and measuring 2-4mm in length, Phlebotomine sandflies (Diptera, Psychodidae) have a distinctly fuzzy appearance under magnification. Females from an estimated 31 species across two genera are known to transmit the parasite between human hosts: \textit{Lutzomyia} in the New World and \textit{Phlebotomine} in the Old World\cite{akhoundi2016}. Sandflies occupy a wide range of ecological niches, found on every continent except Antarctica. Biting occurs from dusk, with females requiring a blood meal for larval development. During the day they are found in cool and sheltered locations, such as in cracks and crevices in walls as seen with \textit{Ph. argentipes}, responsible for transmission in the ISC. In East Africa, three sandfly vectors have been incriminated for \textit{L. donovani} transmission, defining two distinct and non-overlapping ecological settings: (i) the \textit{Acacia-Balanites} and black cotton soil savannah regions in northern focus of northern Ethiopia, Sudan and northern South Sudan, where \textit{Ph. orientalis} thrives, and (ii) the savannah and forest areas in the southern focus of southern Ethiopia, Kenya and Uganda, where \textit{Ph. martini} and \textit{Ph. celiae} are seen in association with \textit{Macrotermes} termite mounds.

In addition to sandflies, needle sharing among people who inject drugs was considered an important route of transmission in the southern Mediterranean region during the 1990s and 2000s, particularly among people living with HIV \cite{alvar1997}. Exceptionally, transmission can  result from blood transfusion, organ transplantation, congenital infection, laboratory accidents\cite{farrar2023manson}, and possibly even sexual contact\cite{guedes2020, symmers1960}.

\subsubsection{Reservoirs}

Similar to the majority of \textit{Leishmania} spp. causing CL and MCL, \textit{L. infantum} demonstrates zoonotic transmission (animal $\rightarrow$ sandfly $\rightarrow$ human), with domestic dogs being the main reservoir host in both the Americas and the Old World. This being said, an ever-increasing list of wild and domestic animals are known to harbour the parasite, including cats, foxes, horses, rodents, bats and opossums, although their relevance to human infection is unclear\cite{alcover2020, ratzlaff2023}. An outbreak near Madrid (2009--2012) was attributed to hares\cite{molina2012}.

In contrast to \textit{L. infantum}, and crucially from an elimination perspective, \textit{L. donovani} transmission in the ISC and East Africa is predominantly anthroponotic (human $\rightarrow$ sandfly $\rightarrow$ human). Although \textit{L. donovani} infections have been reported in several animal species in both regions including cattle, dogs and rats, the significance of these infections as potential sources of human transmission has yet to be established \cite{kushwaha2024, jones2021}.

\subsubsection{Risk factors}

From population prevalence studies we know that across all endemic areas, only a minority of people with detectable parasites develop symptoms\cite{pederiva2022, burza2018}. Risk factors for acquiring an initial asymptomatic infection and subsequent progression to symptomatic disease reflect a tangled ecology of determinants linking host factors (sandfly exposure, immunity, genetics) and parasite factors (strain, virulence, inoculum). A common theme, woven into many of these determinants, is poverty.

In both the ISC and East Africa the median age of infection during stable transmission is similar at 15--20 years. More men than women are infected and develop disease, reflecting their increased occupational exposure to sandflies (for example, cattle herding and other farming activities)\cite{who_wer2023}.

In the ISC, VL endemicity is centred in the fertile and low-lying alluvial plains of the Ganges river, where high humidity, heavy monsoon rains, and abundant vegetation provide ideal conditions for sustained transmission between sandflies and humans\cite{bhunia2010}. Significant clustering of cases is seen across the rural farming communities of Bihar, Jharkhand, Uttar Pradesh, and West Bengal in northeastern India, central and western Bangladesh, and southeastern Nepal. In a systematic review by Bern et al., determinants of VL transmission in the ISC were identified, and included living in mud houses, proximity to prior cases (in the same or nearby household), presence of vegetation and standing water surrounding the house, sleeping on the floor or outside, malnutrition, and a lack of bed net use\cite{bern2010}.

The greatest concentration of cases in East Africa is reported in the northern focus, specifically between the eastern Sudanese states and the bordering northern states of Ethiopia. Many of the epidemiological determinants of VL in East Africa are shared with the ISC, with poverty remaining the central overarching factor. Notable determinants include living in rural settings near sandfly breeding and resting sites (living near termite mounds in the southern focus, sleeping under \textit{Acacia} trees in the northern focus), living in proximity to other VL infected (or recently infected) people, and malnutrition\cite{geto2024}.

Human immunodeficiency virus (HIV) infection has reshaped VL epidemiology in many endemic regions, and remains the most important risk factor for asymptomatic infection, disease progression, and poor treatment outcomes. In the mid-1980s in Spain and other southern European countries, VL shifted from a rare childhood disease to one predominantly affecting HIV-positive adults\cite{monge2014, alvar2012}. HIV-VL co-infection rates are currently increasing in Brazil (now reported in 20\% of new cases\cite{PAHO_Leishmaniasis_2024}) and India (>10\% of VL episodes in 2023 and 2024\cite{NCVBDC2025}). In northern Ethiopia, HIV-VL co-infection presents a significant challenge to elimination, affecting between 20 and 50\% of cases predominantly in male migrant workers\cite{diro2014}.

% host immunity is key some of this thesis' conclusions, and explaining relapse predictors
Host immunity as a risk factor for disease onset and poor treatment outcomes cannot be understated, and is considered an important driving factor for a number of devastating epidemics related to war and natural disasters. A striking example is Sudan in the 1980s-2000s, where population displacement driven by conflict famine brought immune-naïve populations into endemic areas and spread infection into previously unaffected areas. Combined with severe malnutrition and the collapse of health infrastructure, the resulting mortality was catastrophic\cite{zijlstra1991, collin2004, al-salem2016}.

\subsection{Pathophysiology}
% I'm including this chapter before clinical/diagnostic details because it provides important background  

Parasites of the genus \textit{Leishmania} belong to the family \textit{Trypanosomatidae} (class \textit{Kinetoplastea}, phylum \textit{Euglenozoa}) and share many features with two other human pathogens of the same family; \textit{Trypanosoma brucei} -- the causative agent of African sleeping sickness (human African Trypanosomiasis), and \textit{Trypanosoma~cruzi} -- the agent of Chagas disease in the Americas. These vector-borne NTD siblings are all single-celled protozoa characterized by the presence of a flagellum and a kinetoplast -- a dense network of mitochondrial DNA that conveys a distinctive microscopic appearance.

\begin{figure}[h]
    \centering
    \includegraphics[width=1\textwidth]{figures/ch2/cdc_lifecycle.png} % path to your figure file
    \caption{Life cycle of the \textit{Leishmania} parasite consists of a vector (sandfly) stage and reservoir (animal) stage. Whilst the vector stage is by the far the most common mode of transmission, it is not essential, with direct transmission between humans reported from needle sharing, organ and blood donation. Illustration credit: Centers for Disease Control and Prevention/Alexander J. da Silva/Melanie Moser}
    \label{fig:cdc_lifecycle}
\end{figure}

\subsubsection{Life cycle}
In the sandfly gut, \textit{Leishmania} parasites exist in their extracellular flagellated promastigote form which are regurgitated into the bite wound during a blood meal and rapidly phagocytosed by macrophages. Intracellularly, the protozoa lose their flagellum and transform into amastigotes: oval bodies measuring \textasciitilde2--6$\mu$m in length with their characteristic nucleus and kinetoplast clearly visible with nucleic acid staining. Intracellular amastigotes multiply through binary fission, eventually leading to cell lysis and subsequent uptake by neighboring macrophages to perpetuate infection. In viscerotropic disease caused by \textit{L. donovani complex} spp., this cycle drives widespread dissemination with macrophage proliferation and granuloma formation across the mononuclear phagocyte system, including the spleen, liver, bone marrow and lymph nodes. The cycle continues when sandflies ingest infected macrophages during a subsequent blood meal, initiating parasite development within the vector (figure \ref{fig:cdc_lifecycle}).

\subsubsection{Immune evasion}

Remarkably, intracellular amastigotes have evolved to survive and replicate within the hostile phagolysosomal environment of the macrophage. While full immunological details are poorly understood and beyond the scope of this chapter, key strategies include the presence of a lipophosphoglycan coat that enables the promastigotes to evade complement mediated lysis, and suppression of macrophage activation by inducing functional impairment in cytotoxic CD8\textsuperscript{+} T-cell responses. Despite the presence of a pronounced humoral response, with a polyclonal hypergammaglobulinaemia typically seen, the resulting antibodies are predominantly non-neutralising and may even play a role in maintaining the immunosuppressive milieu required for survival of the intracellular amastigotes\cite{stager2010,kaye2020}. As the disease progressives, so does the host's immunosuppressive state, attributed to increasing levels of immunoregulatory cytokines (especially IL-10 and TGF$\beta$), and destructive remodelling of lymphoid tissue\cite{kaye2020,nylen2010}.

\subsection{Clinical features}

% could include a lot more information here, for example, WHO case classification, elaboration on frequency of symptoms

The spectrum of illness is perhaps surprising, with asymptomatic carriers often outnumbering symptomatic cases by 5--200-fold in endemic settings\cite{pederiva2022,burza2018,werneck2002}. When manifested, symptoms typically present insidiously following an incubation period of weeks--months, and up to years. Classic symptoms are a consequence of a persistent systemic inflammatory response syndrome with infiltration of the mononuclear phagocyte system, resulting in intermittent fever, weight loss, splenomegaly, often with accompanying hepatomegaly and, in parts of East Africa, lymphadenopathy.  The spleen is usually non-tender on palpation, but massive splenomegaly $\pm$ hepatomegaly can result in severe abdominal pain. The grayish skin hyperpigmentation, described in historical accounts of chronic disease in the ISC prior to effective treatment, and where the name `kala-azar' originates (`black disease' or `black fever' in Hindi), is now uncommon\cite{chappuis2007, farrar2023manson}.

Often-described laboratory findings, where measured, include raised inflammatory markers (CRP, ESR), pancytopenia, a polyclonal hypergammaglobulinaemia, and hypoalbuminaemia. Anaemia is often severe, normocytic and normochromic in nature, and likely multifactorial in origin, with splenic sequestration and anaemia of chronic disease thought to play important roles\cite{varma2010,goto2017}. Haemoglobin levels are frequently between 6--10 g/dL\cite{varma2010} at presentation, resulting in pallor and contributing to weakness.

In the absence of treatment, disease severity increases in-line with an increasing parasite burden\cite{zacarias2017,silva2014}. Vomiting, diarrhoea, cough, and shortness of breath are often reported, likely reflecting a combination of direct parasite invasion to the mucosa, and opportunistic infections due to the host's immunosuppresed state. The presence of oedema, jaundice, severe co-infection, and bleeding with disseminated intravascular coagulation are poor prognostic factors\cite{costa2023}.

\subsection{Diagnosis}

The \textit{gold standard} diagnosis of VL classically combines compatible clinical features with direct visualisation of parasites in tissue aspirates from the spleen, bone marrow and lymph nodes. Splenic aspiration offer the highest sensitivity (93--99\%) but carries a small risk of fatal haemorrhage. Bone marrow and lymph node aspirates are safer but less sensitive (\textasciitilde50--80\%)\cite{farrar2023manson,burza2018}. Parasite density is commonly quantified using the logarithmic grading system (0--6+) described by Chulay and Bryceson (Table \ref{tab:grading})\cite{chulay1983}.

\begin{table}[ht]
    \centering
    \caption{Logarithmic grading system for parasite load on microscopy of tissue aspirates described by Chulay and Bryceson\cite{chulay1983}. Slides are stained with Giemsa and examined using a $10\times$ eyepiece and $100\times$ oil objective.}
    \begin{tabular}{ll}
        \toprule
        \textbf{Grade} & \textbf{Average parasite density} \\
        \midrule
        $0$            & 0 amastigotes/1,000 fields        \\
        $1+$           & 1--10 amastigotes/1,000 fields    \\
        $2+$           & 1--10 amastigotes/100 fields      \\
        $3+$           & 1--10 amastigotes/10 fields       \\
        $4+$           & 1--10 amastigotes/field           \\
        $5+$           & 10--100 amastigotes/field         \\
        $6+$           & >100 amastigotes/field            \\
        \bottomrule
    \end{tabular}
    \label{tab:grading}
\end{table}

Culture of tissue aspirates can improve sensitivity but is limited by cost, technical complexity, and a slow turnaround of up to 4 weeks. Molecular tests have been developed but are infrequently available outside of research settings and regions of high endemicity. Notably, peripheral blood quantitative polymerase chain reaction (qPCR) assays amplifying kDNA have shown high sensitivities in patients with both VL and VL/HIV co-infection\cite{galluzzi2018, rihs2025, verrest2024}, averting the need for invasive sampling where available. Urine antigen detection has demonstrated specificity, but suffers from low diagnostic sensitivity\cite{vangriensven2018, riera2004}.

A wide gamut of serological (antibody-detecting) tests are available for the diagnosis of VL, including enzyme-linked immunosorbent assays (ELISA), indirect fluorescent antibody (IFA) assays, immunoblots, and a range of rapid diagnostic tests (RDTs)\cite{farrar2023manson}. The rK39 immunochromatographic test (ICT) is by far the most widely used RDT, giving a binary result within 10--20 minutes of providing a finger-prick blood sample. In the ISC, rK39 RDTs demonstrate excellent sensitivity (97.0\%, 95\% CI: 90.0--99.5\%)\cite{boelaert2014} and have served as the first-line diagnostic test since the mid-2000s. In East Africa, however, sensitivity is lower (85.3\%, 95\% CI: 74.5--93.2\%), and a second serological test - the direct agglutination test (DAT) - is recommended to confirm negative rK39 results\cite{boelaert2014}. Key limitations of serological tests include their inability to distinguish between active and past infection, with antibodies persisting for months to years following successful treatment, and their poor sensitivities in patients with severe immunosuppression, particularly those with HIV co-infection\cite{burza2018}.

\subsection{Treatment}

Injectable antimonial compounds have been the workhorse of VL treatment since the 1910s\footnote{Initially as tartar emetic - a toxic trivalent antimonial compound also used to induce vomiting.}, and despite their relatively toxic side effects (including pancreatitis, cardiac arrhythmias and hepatitis), to this day remain first-line therapy in both East Africa (sodium stibogluconate (SSG), as part of a combination therapy) and in Brazil (as meglumine antimoniate (MA))\cite{WHO_TRS_949_2010,Brasil_2024_GuiaVigilanciaSaude}. From the early 1980s in Bihar, India, treatment failure rates exceeding 50\% were observed with SSG despite dose escalation (from 10 mg/kg to 20 mg/kg per day) and extending the duration (from 6--10 days to 20--40 days). Blame was attributed to poor stewardship, with subtherapeutic dosing practices driving selective pressure for resistance within the human reservoir\cite{sundar2001}. As efficacy declined, second-line options were introduced, initially with the more toxic pentamidine (associated with hypoglycaemia, diabetes mellitus, shock and nephrotoxicity), and later, amphotericin B deoxycholate (ABD), which, although effective, was limited by frequent infusion-related reactions, nephrotoxicity, and prolonged hospitalisation.

The last twenty years have seen significant advances in the management of VL, with three new drugs added to the previously limited and toxic armamentarium\cite{alves2018}.

Miltefosine, a repurposed anticancer drug, is the only effective oral agent for VL. Introduced as the first-line treatment option in the ISC in 2005, miltefosine played a central role in the launch of the KAEP\cite{sundar2002,bhattacharya2007,who_sea_elim2005}. Unfortunately, use is limited by teratogenicity and common gastrointestinal side effects. Concerns of increasing relapse rates\cite{rijal2013,sundar2012} led to the replacement of miltefosine by single-dose liposomal amphotericin B (LAMB) (AmBisome\textsuperscript{\textregistered}; Gilead Sciences) in 2014-15\cite{chatterjee2025}, although the link between increasing relapse rates to increased miltefosine resistance is not proven\cite{rijal2013,hendrickx2015}. Outside of the ISC, miltefosine has been shown to perform poorly as monotherapy in East Africa and Brazil\cite{alves2018, carnielli2019}.

% Eye side effects\cite{WHO2023MiltefosineAdvisory}

Supported by a donation agreement between the WHO and Gilead, a single IV dose of 10mg/kg LAMB continues as the first-line regimen in the ISC due to its proven efficacy and safety profile\cite{WHO2023GileadAgreement}. LAMB, at a higher cumulative dose of 21--30mg/kg over 5--10 days, is also the first-line treatment for VL in Southern Europe and an increasingly used second-line regimen in Brazil and East Africa where antimony treatment is poorly tolerated (e.g severe disease, extremes of age, pregnancy, renal and liver toxicity, relapse, HIV/VL co-infection).

Paromomycin, an injectable aminoglycoside, has demonstrated efficacy both as monotherapy and as part of combination therapy in the ISC, with registration achieved in India in 2006\cite{sundar2007a}. Despite its relatively low cost and reassuring safety profile, it has never seen widespread use in the ISC. In contrast, in East Africa it has shown efficacy in trials of combination therapy with SSG, and since 2010 has formed part of the current first-line treatment option\cite{WHO_TRS_949_2010, musa2012}. Evidence supports its use in combination with miltefosine in both East Africa and the ISC\cite{musa2023,sundar2011comb}.

\subsection{Control and elimination}

In areas of anthroponotic transmission, control and elimination of VL centres on two logically interdependent strategies: (i) reduction of the human infection reservoir through early diagnosis and treatment, and (ii) interruption of transmission events through vector control measures. Effective implementation of these approaches, however, requires evidence-based decision-making supported by a firm political commitment.

Guided by these principles, in 2005 the governments of Nepal, Bangladesh and India, with WHO support, signed a Memorandum of Understanding aiming to reduce VL incidence to fewer than 1 case per 10,000 population at the district or sub-district levels by 2015. The deadline was subsequently extended to 2020 and is now integrated into the 2021--2030 WHO NTD roadmap\cite{who_sea_elim2005, WHO_KalaAzarElimination_2015, WHO_SEA_CD_329_2021, WHO_NTDs_Roadmap_2021_2030, WHO_Global_Report_NTDs_2025}. Initial success of the kala-azar elimination programme (KAEP) was attributed to the introduction of the rK39 rapid diagnostic tests, oral miltefosine and subsequently single dose LAMB, alongside active case detection and vector control measures including indoor residual spraying (IRS) and insecticide treated nets (ITN)\cite{sundar2018}. In October 2023, Bangladesh was the first country to be declared as having eliminated VL as a public health problem\cite{nagi2024}. Nepal and India are now working to sustain their targets for 3 years to also achieve elimination status\cite{pandey2025}.

As the ISC enters the consolidation and maintenance phases of the KAEP, concerns over sustainability are now being voiced\cite{chatterjee2025}. Waning political momentum and financial support threatens the ongoing viability of costly IRS and active case detection. As the proportion of immunologically individuals increases, so does the risk of future outbreaks\cite{lerutte2017}.

Booned by the success of the KEAP, calls to action for a similar effort in the East Africa\cite{alvar2021} were answered in 2023 with the publication of the Nairobi declaration by the ministries of health of nine endemic East African countries\footnote{Chad, Djibouti, Eritrea, Ethiopia, Kenya, Somalia, South Sudan, Sudan and Uganda\cite{WHO_2024_VL_easternAfrica}}.

\section{Relapse}

VL relapse can be defined as the reappearance of signs and symptoms of VL following an initial treatment response\cite{WHO_SEA_CD_329_2021, chhajed2024}. Recurrence is typically confirmed by direct visualisation of the parasite from a tissue aspirate smear.

Importantly, a patient may only relapse once an initial treatment response is first achieved, also known as initial cure in clinical efficacy studies. Achievement of initial cure typically requires not only improvement of the patient's signs and symptoms (e.g., defervescence, reduction in spleen size, weight gain, improvement of haemoglobin), but also visualisation of amastigotes on microscopy of a tissue aspirate smear. Assessment of initial cure, (`test of cure') most often occurs within a month of treatment completion, although considerable heterogeneity exists in exactly how and when treatment outcomes are defined\cite{dahal2024}.

% WHO TDR definition of relapse: \cite{WHO2010kalaazarIndicators}

\subsection{Burden of relapse}

% general comment - high relapse rates in certain populations
The proportion of patients that relapse following an initial treatment response varies dramatically across studies, reflecting differences in host and parasite factors\cite{chhajed2024}.

Perhaps the most important host factor predictive of relapse is HIV co-infection, with studies frequently describing relapse rates upwards of 20\% in patients with low CD4 counts\cite{cota2011,diro2019,burza2014a}. Choice of treatment regimen and the presence of parasite resistance also play important roles, with relapse rates of over 50\% frequently reported during the 1990s in Bihar, India, due to the development of antimony resistance\cite{olliaro2005}.

% estimate of current burdens in East Africa & ISC
As a `rule of thumb', relapse rates with current first-line regimens in patients \textit{without} significant immunosuppression range between 1 in 40 to 1 in 10 patients (2.5--10\%)\cite{chhajed2024}. In a meta-analysis of efficacy studies recently published by Chhajed et al., the overall proportion of HIV-negative patients relapsing in the ISC at 6-months was estimated at 3.5\% (95\% confidence interval (CI):2.8--4.5\%) following first-line treatment with single-dose LAMB\cite{chhajed2024}. Under pragmatic conditions in East Africa, slightly higher 6-month relapse rates of approximately 5\% are seen with the first-line combination therapy of PM and SSG\cite{kimutai2017,atia2015,melaku2007}.

\subsubsection{Timing}

Efficacy studies with 6 months' follow-up are likely underestimating the true relapse rate by a considerable margin. Chhajed et al. compared 21 studies that reported both 6 month and 12 month relapse rates, and noted that only two-thirds of all relapses by 1 year were detected within the first 6 months\cite{chhajed2024}. Recent large studies from the ISC report similar findings, leading to calls to extend the routine follow-up period from 6 months to a year\cite{sundar2019,rijal2013,burza2013,burza2014,goyal2019}.

% studies where more relapses occur between 6-12 vs 0-6 months \cite{lucero2015} \cite{rijal2013} - equal number, 32 vs 34 in \cite(sundar2019)

\subsection{Relationship to immune response}

Treatment does not lead to sterile cure. As demonstrated by the poor treatment outcomes observed in patients with advanced HIV co-infection, effective host immunity is fundamental to achieving lasting cure\cite{murray2005, khalil2005, franssen2021,alves2018}. In this context, the goal treatment is not complete parasite eradication, but rather sufficient suppression of the parasite burden to facilitate reconstitution of the host's cell-mediated immune response, and allowing parasite control.

Recovery of the cell-mediated immunity can be demonstrated by a positive Leishmanin skin test (LST)\footnote{Also known as the Montenegro test, initially developed in the 1920s to support the diagnosis of CL. Analogous to the Mantoux test for latent tuberculosis, a delayed-type hypersensitivity response from the LST manifests as visible skin induration occurring 48--72 hours after an intradermal injection of \textit{Leishmania} antigen\cite{carstens-kass2021}.}. During active infection, the LST is expected to be negative from cellular anergy. Subsequent conversion to a positive test following treatment strongly indicates successful immune recovery, conferring lasting protection from relapse and reinfection.

\subsection{Relapse vs. reinfection}

Distinguishing `true relapse' due to parasite recrudescence from re-infection due to a new strain is important when assessing drug efficacy and evaluating relapse determinants. However, unlike in malaria\cite{WHO_2009_surveillance_antimalarial_drug_efficacy}, in VL there are no validated molecular targets for PCR confirmation. Despite this, mounting evidence exists that the majority of observed relapses due indeed occur due to recrudescence. Rijal et al. performed kDNA fingerprinting in 8 pairs of bone marrow samples in HIV-negative patients before treatment of their primary infection and at the time of relapse, and found no evidence of reinfection\cite{rijal2013}. Similar studies have been performed in VL-HIV co-infection patients in East Africa and southern Europe using a variety of molecular techniques, and showing that while reinfection does occur, it is considerably less common than recrudescence\cite{franssen2021, morales2002, gelanew2010,lachaud2009}.

\subsection{Why is relapse important?}

Relapse is harmful at the individual level not only from the morbidity and mortality linked to recurrent infection, but also due to the potentially toxic second-line regimens and prolonged hospital admissions. Both direct and indirect costs -— including loss of patient income -— can impose significant financial strain on households already affected by poverty.

Relapse is also important from a public health perspective. In areas of anthroponotic transmission, early diagnosis and treatment of all infections, including relapses, is a core pillar of both the East Africa and ISC elimination programmes, and supported by mathematical modelling\cite{lerutte2017}. Parasite strains isolated from relapse cases, however, are distinct from primary cases, presenting additional challenges. For example, there is mounting evidence that relapse strains exhibit increased infectiousness, likely resulting from selection during initial drug pressure\cite{rai2013,hendrickx2016, garcia-hernandez2015,vanaerschot2011}. Relapse cases are also more likely to both \textit{spread} drug resistant strains, as seen during the 1980s and 1990s with antimony resistance in India\cite{sundar2001}, and also be the source of de novo drug resistance mutations, as demonstrated \textit{in vitro}, and frequently described in clinical practice in when HIV-VL coinfected....

Where transmission is anthroponotic, xenodiagnosis studies confirm that the most important infectious reservoir are patients with active VL and PKDL\cite{singh2021, mondal2019}.

Concerns re: CL as potential reservoir \cite{bhattarai2025}

\subsection{Determinants of relapse}

Drug-resistant parasites also seem to display increased fitness.\cite{, hendrickx2016, garcia-hernandez2015}

Antimony resistant strains leading to higher parasite burdens in \cite{vanaerschot2011}
Increased fitness of miltefosine relapse strains (in vivo)\cite{rai2013}

\subsection{Relapse determinants}

Guided by the results of a literature search (box \ref{box:lit-search1}), a narrative review is performed exploring the determinants of VL relapse, with a focus on immunocompetent patients in the ISC and East Africa. Full details in Appendix \ref{app:background}.

\begin{mybox}{Literature search}
    Literature search performed from database inception to August 11\textsuperscript{th} 2025 in PubMed (below), Embase and Web of Science. Articles in English were reviewed. Full search details available in Appendix \
    \tcbline
    \texttt{(``Leishmaniasis, Visceral''[Mesh] OR ``visceral leishmaniasis'' OR ``leishmaniasis, visceral'' OR ``kala azar'' OR ``kala-azar'') AND (``Recurrence''[Mesh] OR relapse* OR recurrent OR recurrence OR recrudescence OR ``treatment failure'')}
    \label{box:lit-search1}
\end{mybox}

%% PREDICTIVE BIOMARKERS

Baseline parasite count associated with relapse \cite{vangriensven2018, riera2004}
KATex associated with treatment failure and relapse in HIV/VL co-infection.\cite{vangriensven2018, riera2004}

Summary \cite{vangriensven2012}

\cite{takele2022,takele2023}
% Immune markers predictive of relapse in HIV/VL patients (Takele/Kropf 2022)

% Cross sectional study: 25 healthy male, non-endemic controls (from LTRC, University of Gondar), and 99 male patients with HIV (49 with VL/HIV and 50 HIV uninfected). Excluded < 18 years. 

% VL patients recruited at: 1) ToD, EoT, 3/12 after EoT, and 6-12/12 after EoT. 3 year follow-up for relapse. If no relapse, VL/HIV patients were contacted by phone to confirm no relapse. Range of treatments used (esp in VL/HIV patients). SSG/PM 17D most common in VL patients, and LAMB/MF 28D most common in VL/HIV patients. 

% It appears 34 patients were recruited at EoT. Multinomial logistic regression - 3 outcomes of relapse at 3/12, 6-12/12 snd no relapse  (n = 10, 7 and 17 respectively).

% PD1 expression by CD4+ cells, IFN-gamma and CD4 T-cell counts, measured at EoT, were predictive of relapse. BUT - not clear how the model was constructed (multinomial logistic regression, but not clear if univariable or multivariable). 

% Takele 2023 shows the same markers are more common in VL/HIV with recurrent infections vs. primary infections. Also recurrent VL/HIV patients have higher parasite loads and less weight gain, worse recovery of blood counts during treatment. Recurrent VL/HIV patients are more likely to have further infections. 

% Baseline parasite load is significantly associated with relapse. The Verrest paper deserves a special mention\cite{verrest2021,verrest2024}.
% Verrest paper: aim to speed up treatment evaluation by developing a biomarker that serve as a surrogate endpoint.

% LEAP 0714: Mbui 2019, allometric MF in children, n = 30 children in Uganda+Kenya, all allometric dosing
% LEAP 0208: Wasunna 2016, combination therapy in Sudan+Kenya, MF vs LAMB+MF vs LAMB+SSG, n = 151
% FEXI VL 001: Fexinidazole efficacy in Sudan, n = 14, terminated early due to lack of efficacy

% Parasite aspirates performed at baseline and day 28 in all patients. If indicated, repeated at day 56, 210 or an unscheduled visit
% Tissue PCR also in LEAP 0714 and FEXI VL 001
% Quantitative PCR: for Leishmania kinetoplastid DNA
% Simulations also performed looking at power estimates.
% Excluded: initial treatment failures, samples collected after relapse, unreliable samples
% Absolute and relative PCRs. Days 14, 28, 56 (relative to baseline). ROC curves. AUC. Spearman's rank correlation.

% Tissue samples qPCR - 40 at baseline, 42 at 28 days, (5 at 56)
% Microscopy 'score' - 174 at baseline, 168 at 28 days, 9 at 56 days

% EDTA qPCR - 175 at D0, 183 at D14, 173 at D28, 178 at D56 (including excluded samples)
% Absolute parasite load AUC: 0.92 (D56), 0.74 (D28), 0.71 (D14)
% Relative parasite loads were comparable
% Figure 1
% "When no parasites were detected by microscopy on day 28, parasites were still detected by qPCR in 36\% of blood samples"


% Also, in East Africa, HIV/VL co-infection, baseline parasite load is important: \cite{abongomera2017a}

% % NEOPTERIN: \cite{kip2018}

% DAWIT - axis labels too small
% Mortality by age - 75% at 100 years - extrapolating?
% INFANTS < 1 year

% case controls: \cite{santos2002} and Naylor-Leyland 2024. 