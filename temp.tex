As outlined in Chapter~\ref{ch:2-background}, VL relapse is consequential not only for individual patients but also poses a threat to sustained elimination efforts. Accordingly, the development of a non-invasive tool to predict relapse \textemdash\ functioning as a `test of cure' following initial treatment \textemdash\ has been identified by the WHO as a research priority\cite{WHO2024_Leishmaniasis,WHO_2024_VL_easternAfrica,who_sea_elim}. A prognostic model represents a potential solution: by quantifying the relationship between patient characteristics and subsequent relapse events, relapse risk in future patients can be estimated and clinical decision-making informed. However, as demonstrated in Chapter~\ref{ch:3-sys-review}, no prognostic models for VL relapse have been published to date.

To address this evidence gap, four prognostic models are developed using IPD from the IDDO VL data platform: two for patients from the ISC and two for patients from East Africa. Within each region, one model includes parasite grade at initial cure assessment, and one model excludes parasite grade, reflecting differences in data availability and clinical practice. All models use routinely collected information available at the time of initial cure assessment to predict six-month relapse among VL patients without HIV co-infection.

The development and evaluation of clinical prediction models is supported by an extensive and growing methodological literature. Over the past decade, reporting guidelines for prediction model studies have been established\cite{collins2015, moons2015, debray2023, collins2024A}, alongside an increasing number of reviews and recommendations that define best practice\cite{efthimiou2024, collins2024B, riley2024A, riley2024B, vanSmeden2021}. In particular, the application of meta-analysis techniques to IPD from multiple studies presents exciting new opportunities for prediction model development and evaluation\cite{riley2021_book_ch17, debray2023}. Notable opportunities include increased sample sizes leading to greater statistical power, and the ability to explore heterogeneity in predictor effects and model performance across different settings. Additionally, IPD can be used to standardise inclusion criteria and outcome/predictor definitions across included studies. However, as we lay out in this chapter, the use of IPD in prediction model research also introduces challenges; specifically (i) the need for statistical models that account for clustering of participants within studies and (ii) the presence of missing data, which can be sporadically missing within studies, or entirely missing from one or more studies\cite{debray2023}.

The aim of this chapter is to describe and justify the methodology used for model development and evaluation \textemdash\ from data acquisition to final model presentation. Guidelines and methodological texts are cited accordingly, and checklists provided for current reporting guidelines on prediction model studies (TRIPOD-AI and TRIPOD-Cluster)\cite{collins2024A,debray2023}. Additional material is presented in Appendix \ref{app:methodology} and in the \href{https://github.com/jpwil/dphil}{Supplementary Material}.\footnote{Available at \url{https://github.com/jpwil/dphil}.} A protocol is available on the \href{https://osf.io/z4bdn}{Open Science Framework}.\footnote{Created Nov 8, 2024, available at \url{https://osf.io/z4bdn}.}

This chapter's structure closely mirrors the methodological workflow as outlined in Figure~\ref{fig:workflow}, with sections on data harmonisation, model development, and internal validation. In keeping with best practice, and similar to the approach adopted in Chapter \ref{ch:3-sys-review}, the research question is presented in Box \ref{box:picots} using the PICOTS (population, index model, comparator model, outcome, timing, and setting) framework\cite{riley2021_book_ch17, debray2017}. Further elaboration of the eligibility criteria, and standardised definitions of predictors and the outcome are considered in the following section.

All analyses were performed using R version 4.4.1\cite{r2025}, with R packages cited in the relevant sections below. R scripts used for model development and evaluation are provided in the \href{https://github.com/jpwil/dphil}{Supplementary Material}.

\begin{mybox}{Definition of the research question: a PICOTS approach}
    \begin{description}
        \item[Population] HIV-negative patients that are prospectively recruited into a clinical trial with a diagnosis of visceral leishmaniasis, confirmed either serologically or parasitologically. No restrictions are placed on age, sex or treatment regimen.
        \item[Index models] For each setting, two prognostic models are developed; one including baseline parasite grade from a tissue aspirate, and one without. The models predict the \textit{future} occurrence of relapse, using patient information collected at treatment baseline. The intended time of model use is following a successful assessment of treatment response.
        \item[Comparator model] As established in Chapter \ref{ch:3-sys-review}, no published relapse models are available for comparison or updating.
        \item[Outcome] Relapse is defined as the recurrence of signs and symptoms of VL requiring rescue treatment, and following demonstration of an initial treatment response.
        \item[Timing] Relapse occurring within 6--months of test of cure (typically occurring at the time of treatment completion, or within 30 days of starting treatment).
        \item[Setting] Participants from either the Indian subcontinent or East Africa.
    \end{description}
    \label{box:picots}
\end{mybox}

\section{Data harmonisation}

Here, data harmonisation refers to the process of data acquisition, curation, and any subsequent data manipulation required to produce a single analysis dataset ready for model development.

In the interest of full disclosure, data acquisition was completed by IDDO colleagues prior to the commencement of this DPhil project. The first stage of data curation \textemdash\ conversion of the contributed datasets to the Clinical Data Interchange Standards Consortium (CDISC) Study Data Tabulation Model (SDTM) standard \textemdash\ was performed by the IDDO data engineering team with support from the IDDO science team (myself included). Subsequent methodological steps were led by myself.

\subsection{Data acquisition}

A systematic review of the scientific literature was first performed in 2016, with the aim of comprehensively cataloguing all existing VL clinical trials (PROSPERO: \href{https://www.crd.york.ac.uk/PROSPERO/view/CRD42021284622}{CRD42021284622})\cite{bush2017}. 145 trials were initially identified (1980--2016, n = 26,986 patients), with further trials added during periodic updates according to an open protocol\cite{Singh-Phulgenda2022_IDDO_VL_protocol}. Between 2018--2022, corresponding authors of the identified VL clinical trials were invited to share their IPD with the IDDO VL data platform, in line with the General Data Protection Regulation (GDPR)--compliant IDDO data sharing policy\cite{IDDO_DataGovernance, dahal2025}.

\subsection{Data curation}

Conversion of the contributed datasets to an analysis--ready dataset of all eligible IPD occurred in two key stages.

\subsubsection{Stage 1: CDISC SDTM curation}

To facilitate reusability and interoperability, contributed datasets were standardised to a common storage format: the CDISC--compliant SDTM standard\cite{cdisc2024}, adapted by IDDO for VL\cite{iddo2020}. During this process, contributed datasets underwent \textit{psuedonymisation},\footnote{\ `\dots processing of personal data in such a manner that the personal data can no longer be attributed to a specific data subject without the use of additional information' \url{https://ico.org.uk/for-organisations/uk-gdpr-guidance-and-resources/data-sharing/anonymisation/pseudonymisation/} (accessed 15 Dec 2025).} prior to being available for data sharing requests. Briefly, SDTM format datasets comprise a number of standardised domains (tables) containing related information (e.g. patient demographics, laboratory results, treatment administration, clinical signs and symptoms). Each domain contains a set of standard variables (table columns, e.g. STUDYID, USUBJID, VISITDY) alongside VL--specific variables defined by IDDO (e.g. parasite grade, spleen size). Further details of the curation process are available in the \href{https://www.iddo.org/tools-and-resources/data-tools}{IDDO SDTM Implementation Guide}.\footnote{Available at \url{https://www.iddo.org/tools-and-resources/data-tools}, free registration required.}


Subsequently, in the second stage of curation, SDTM format datasets were converted to a single analysis--ready dataset, primed for model development. This stage consisted of multiple steps, refined iteratively over several months and in close consultation with the IDDO data engineering team:

\begin{itemize}[noitemsep]
    \item Creation of a standardised outcome variable according to a pre-defined definition
    \item Application of inclusion and exclusion criteria
    \item Identification and extraction of candidate predictors
    \item Identification and removal of spurious data points (e.g. outliers)
    \item Conversion of the datasets from a long to wide format, consisting of one row per participant
    \item Merging of all datasets into a single analysis dataset
\end{itemize}

Data wrangling during the second curation stage was performed with the \texttt{tidyverse} suite of R packages\cite{wickham2019}. Identification and removal of spurious data points was performed through subgroup tabulations and visual inspection of histograms and scatter plots. Where two incongruous data points were identified, for example, incompatible height and weight values, a third variable, such as BMI or age, would be used to identify the spurious value. Data points considered to be outliers were changed to missing values. A complete record of all data cleaning steps, including outlier identification and removal, was maintained and documented in commented R scripts.

\subsection{Outcome definition}

\subsection{Participant eligibility}

The following inclusion and exclusion criteria were applied:

\begin{itemize}[noitemsep]
    \item \textbf{Study--level \underline{inclusion} criteria}
          \begin{itemize}
              \item Studies conducted in either the ISC (India, Nepal, Bangladesh) or East Africa (Ethiopia, Sudan, South Sudan, Kenya, Uganda)
              \item Prospective design, defined as participants having provided informed consent
              \item Participants recruited with a diagnosis of VL as defined by a combination of clinical symptoms and either parasitological or serological confirmation
              \item Studies that reported, as a minimum, the treatment regimen including at least the drug name(s), dose and duration
              \item Included a minimum of 6 months of prospective follow-up from treatment initiation
              \item Reported VL relapse events during the 6-month follow-up period or later
          \end{itemize}
    \item \textbf{Participant--level \underline{exclusion} criteria}
          \begin{itemize}
              \item Participants with HIV co-infection or from a setting with high HIV co-infection prevalence and without a negative HIV test
              \item Participants who were confirmed pregnant at the time of treatment initiation
              \item Participants with symptomatic treatment failure requiring rescue treatment before or at initial cure assessment
          \end{itemize}
\end{itemize}

Individual participants with a positive human immunodeficiency virus (HIV) test or positive pregnancy test were excluded from the analysis. Where study inclusion criteria required a positive tissue aspirate, participants with a negative tissue aspirate were excluded. No individual participants were excluded based on age, sex or prior VL history.

Where reported, relapse is defined as new clinical symptoms compatible with VL and confirmed with a positive tissue aspirate\cite{dahal2024}. For the outcome (relapse) to occur, participants must first have demonstrated an initial response to treatment, termed `initial cure', typically assessed between 15-30 days with some variability around the timing . Participants without confirmed initial cure were therefore excluded. The assessment of initial cure can be based on clinical criteria or a combination of clinical and/or parasitological criteria. We did not exclude studies based on their criteria for defining initial cure.

% Study-specific definitions of initial cure and relapse are presented in \hyperref[sec:add-files]{Additional file 1}.
