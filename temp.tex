In the absence of treatment, disease severity increases in-line with an increasing parasite burden\cite{zacarias2017,silva2014}. Vomiting, diarrhoea, cough, and shortness of breath are often reported, likely reflecting a combination of direct parasite invasion to the mucosa, and opportunistic infections due to the host's immunosuppresed state. The presence of oedema, jaundice, severe co-infection, and bleeding with disseminated intravascular coagulation are poor prognostic factors\cite{costa2023}.

\section{Management}

\subsection{Diagnosis}

\subsection{Treatment}

Injectable antimonial compounds have been the workhorse of VL treatment since the early 1900s, and to this day remain first-line therapy in both East Africa (sodium stibogluconate (SSG), as part of a combination therapy) and in Brazil (as meglumine antimoniate (MA))\cite{WHO_TRS_949_2010,Brasil_2024_GuiaVigilanciaSaude}. From the early 1980s in Bihar, India, treatment failure rates exceeding 50\% were observed with SSG despite dose escalation (from 10 mg/kg to 20 mg/kg per day) and extending the duration (from 6--10 days to 20--40 days). Blame was attributed to poor treatment stewardship, with subtherapeutic dosing driving selective pressure for resistance within the human reservoir\cite{sundar2001}. As efficacy declined, second-line options were introduced, initially with the more toxic pentamidine (associated with hypoglycaemia, diabetes mellitus, shock, nephrotoxicity, and increasing failure rates), and later, amphotericin B deoxycholate (ABD), which, although highly effective, was constrained by frequent infusion-related reactions, prolonged hospitalisation, and nephrotoxicity.

The last twenty years have seen significant advances in the management of VL, with three new drugs added to the previously limited armamentorium, and exciting new candidates in development\cite{alves2018}.

Miltefosine, a repurposed anticancer drug, is currently the only effective oral agent for VL. Introduced as the first-line option in India, Nepal and Bangladesh in 2005, miltefosine played a central role in the initiation of the tripartite elimination campaign\cite{sundar2002,bhattacharya2007,who_sea_elim2005}. Unfortunately, its use is limited by teratogenicity and gastrointestinal side effects. Additional concerns of increasing resistance led to its replacement in the the ISC by single-dose liposomal amphotericin B (LAMB) in 2011\cite{rijal2013,sundar2012,chatterjee2025}. Outside of the ISC, miltefosine has be shown to perform poorly as monotherapy in East Africa and Brazil\cite{alves2018}

Paromomycin, an injectable aminoglycoside (IM/IV), has demonstrated good efficacy both as monotherapy and as part of combination therapy in the ISC, with registration achieved in India in 2006\cite{sundar2007a}. Despite its relatively low cost and reassuring safety profile, paromomycin has never seen widespread use in the ISC. In contrast, in East Africa it has shown efficacy in trials of combination therapy, including with ...