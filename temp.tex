
The spectrum of illness is perhaps surprising, with prevalence studies revealing that asymptomatic carriers often outnumber symptomatic cases by 5--20-fold in endemic settings\cite{pederiva2022,burza2018}. Where manifested, symptoms typically develop indolently following an incubation period of weeks - months, and sometimes years. Classical symptoms are a consequence of a persistent systemic inflammatory response with infiltration of the mononuclear phagocyte system, resulting in intermittent fever, weight loss, splenomegaly, often with accompanying hepatomegaly and, in parts of East Africa, lymphadenopathy.

At presentation patients are classically pancytopenic, with leukopenia, thrombocytopenia and often severe anaemia.\footnote{Typically normocytic and normochromic in nature and likely multifactorial in origin, with splenic sequestration and anaemia of chronic disease likely to be important\cite{varma2010,goto2017}.} Haemoglobin levels are frequently seen between 6--10 g/dL\cite{varma2010}, contributing to weakness. Whilst the spleen is normally non-tender on palpation, massive splenomegaly and hepatomegaly can result in abdominal pain that can be severe in nature. The term `kala-azar' means `black disease' in Hindi, coined in the 19\textsuperscript{th} century due to a grayish skin hyperpigmentation, although this is an infrequent finding in modern case series\cite{farrar2023manson}.

The disease is widely described as `fatal without treatment'\cite{burza2018,farrar2023manson}. Supporting this statement are the high mortality figures estimated during conflict-related epidemics in East Africa over the last 40 years, and prior to effective therapy, 19\textsuperscript{th} century accounts of outbreaks devastating communities across the Ganges delta\cite{collin2004, gibson1983,steverding2017}. Despite this, subclinical forms of the disease have been reported with spontaneous resolution\cite{mouri2015,badaro1986}.
