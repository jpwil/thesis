The \textit{gold standard} diagnosis of VL classically combines compatible clinical features with direct visualisation of parasites in tissue aspirates from the spleen, bone marrow and lymph nodes. Splenic aspiration offer the highest sensitivity (93--99\%) but carries a small risk of fatal haemorrhage. Bone marrow and lymph node aspirates are safer but less sensitive (\textasciitilde50--80\%)\cite{farrar2023manson,burza2018}. Parasite density is commonly quantified using the logarithmic grading system (0--6+) described by Chulay and Bryceson (Table \ref{tab:grading})\cite{chulay1983}.

\begin{table}[ht]
    \centering
    \caption{Logarithmic grading system for parasite load on microscopy of tissue aspirates described by Chulay and Bryceson\cite{chulay1983}. Slides are stained with Giemsa and examined using a $10\times$ eyepiece and $100\times$ oil objective.}
    \begin{tabular}{ll}
        \toprule
        \textbf{Grade} & \textbf{Average parasite density} \\
        \midrule
        $0$            & 0 amastigotes/1,000 fields        \\
        $1+$           & 1--10 amastigotes/1,000 fields    \\
        $2+$           & 1--10 amastigotes/100 fields      \\
        $3+$           & 1--10 amastigotes/10 fields       \\
        $4+$           & 1--10 amastigotes/field           \\
        $5+$           & 10--100 amastigotes/field         \\
        $6+$           & >100 amastigotes/field            \\
        \bottomrule
    \end{tabular}
    \label{tab:grading}
\end{table}

Culture of tissue aspirates can improve sensitivity but is limited by cost, technical complexity, and a slow turnaround of up to 4 weeks. Molecular tests have been developed but are infrequently available outside of research settings and regions of high endemicity. Notably, peripheral blood quantitative polymerase chain reaction (qPCR) assays amplifying kDNA have shown high sensitivities in patients with both VL and VL/HIV co-infection\cite{galluzzi2018, rihs2025, verrest2024}, averting the need for invasive sampling where available. Urine antigen detection has demonstrated specificity, but suffers from low sensitivity\cite{burza2018}.

A wide gamut of serological (antibody-detecting) tests are available for the diagnosis of VL, including enzyme-linked immunosorbent assays (ELISA), indirect fluorescent antibody (IFA) assays, immunoblots, and a range of rapid diagnostic tests (RDTs)\cite{farrar2023manson}. The rK39 immunochromatographic test (ICT) is by far the most widely used RDT, giving a binary result within 10--20 minutes of providing a finger-prick blood sample. In the ISC, rK39 RDTs demonstrate excellent sensitivity (97.0\%, 95\% CI: 90.0--99.5\%)\cite{boelaert2014} and have served as the first-line diagnostic test since the mid-2000s. In East Africa, however, sensitivity is lower (85.3\%, 95\% CI: 74.5--93.2\%), and a second serological test -— the direct agglutination test (DAT) -- is recommended to confirm negative rK39 results\cite{boelaert2014}. Key limitations of serological tests include their inability to distinguish between active and past infection, with antibodies persisting for months to years following successful treatment, and their poor sensitivities in patients with severe immunosuppression, particularly those with HIV co-infection\cite{burza2018}.