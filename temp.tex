
\chapter{\label{ch:5-isc-model-results}Results: Indian Subcontinent Models}


\section{Descriptive Analysis}

On application of the eligibility criteria to the IDDO VL data platform, a total of 19 studies and 4,599 patients were selected for inclusion. A flow diagram is presented in Figure \ref{fig:isc_flow_diagram}.

\subsection{Study Characteristics}

Key characteristics of the 19 included studies are presented in Table \ref{tab:isc_study}.

A median of 144 patients were recruited in each study, ranging from 6 patients\cite{chakraborty2008} to 928 patients\cite{sundar2019} (IQR: 86.5 to 329.5 patients). Patients were recruited between 2000 and 2017 from sites in India (15 studies, 4,171 [90.7\%] patients)\cite{bhattacharya2007,chakraborty2008, das2009,pandey2016,pandey2017,sundar2007a,sundar2008,sundar2008B,sundar2009,sundar2010,sundar2011,sundar2012,sundar2014,sundar2015,sundar2019} and Nepal (4 studies, 428 [9.3\%] patients)\cite{rijal2003, rijal2010a, rijal2010b}. All Indian studies recruited patients from the state of Bihar in northeastern India, with the majority of studies led by investigators from either a government referral hospital in Patna (Rajendra Memorial Research Institute of Medical Sciences: RMRIMS) (4 studies, 1,125 [24.5\%] patients)\cite{bhattacharya2007, das2009,pandey2016,pandey2017}, or a private research facility in Muzaffarpur (Kala--azar Medical Research Centre: KAMRC) (9 studies, 2,652 [57.7\%] patients)\cite{chakraborty2008,sundar2008,sundar2008B,sundar2009,sundar2010,sundar2011,sundar2012,sundar2015,sundar2019}. All Nepalese studies were led by investigators based at the B.P. Koirala Institute of Health Sciences, a health sciences university in the city of Dharan, Koshi Province, eastern Nepal.


\subsubsection{Study Design and Treatment Arms}

A range of study designs were included and are summarised in Table \ref{tab:isc_study} as described in the publication or protocol. Four studies (428 [9.3\%] patients) were described as Phase 2 trials\cite{sundar2008,sundar2008B,sundar2011,sundar2015}, four studies (1,113 [24.2\%] patients) as Phase 3 trials\cite{sundar2007a,sundar2009,sundar2010,sundar2014}, and one study (32 [0.7\%] patients) as a Phase 2/3 trial\cite{rijal2010a}.

Ten studies (1,625 [35.3\%] patients) allocated patients to more than one treatment arm\cite{das2009, sundar2007a,sundar2008, sundar2009, sundar2010, sundar2014, sundar2015, sundar2008B, chakraborty2008, rijal2010b}, of which six studies (1,411 [30.7\%] patients) were randomised\cite{das2009,sundar2007a,sundar2009,sundar2010,sundar2014} or partially randomised\cite{sundar2008}.

Treatment arms are summarised by study in Figure \ref{fig:isc_treat}. The three largest studies were all monotherapy trials with either single dose liposomal amphotericin B (10mg/kg) (1 study, 928 [20.2\%] patients)\cite{sundar2019}, or 28--day miltefosine (2 studies, 1,171 [25.5\%] patients).

Figure \ref{fig:isc_treat} presents the different treatment regimens across all studies.

\subsubsection{Study Eligibility Criteria}

All studies recruited both male and female patients, and all but three studies restricted inclusion by age. Four studies only recruited patients aged 12 years and above (314 [6.8\%] patients)\cite{sundar2015,sundar2008B,sundar2008,rijal2010a}. Where specified, the lower age limit was otherwise between 2 and 6 years. One study (100 [2.2\%] patients) recruited children only (<15 years). Otherwise, the upper age limit ranged between 60 and 70 years.

In the 16 studies that included baseline haemoglobin level in their eligibility criteria, the median threshold was 50g/dL (range 35 to 60g/dL).\footnote{with one study reporting boosting haemoglobin with blood transfusions prior to recruitment\cite{das2009}.} Almost all studies excluded patients with serious illness or the presence of significant co-existing diseases.

Twelve studies (2,021 [43.9\%] patients) explicitly reported performing HIV testing on all recruited patients\cite{bhattacharya2007,chakraborty2008,rijal2013,rijal2010a,sundar2007a,sundar2008,sundar2008B,sundar2009,sundar2010,sundar2014,sundar2015}, and patients with confirmed VL/HIV co-infection were excluded from all but two studies (280 [6.1\%] patients)\cite{rijal2003,rijal2010b}. Patients with HIV/VL co-infection from these two studies were excluded at the individual participant level, where reported.

Inclusion criteria based on prior VL treatment were reported in 12 studies (2,629 [57.2\%] patients)\cite{sundar2019,sundar2015,sundar2014,sundar2011,sundar2010,sundar2009,sundar2008B,sundar2007a,rijal2010b,rijal2010a,rijal2003,das2009}, with timing since last VL treatment ranging from 10 days\cite{sundar2008B} to lifelong\cite{sundar2019}. One study (73 [1.6\%] patients) only recruited patients who had failed treatment with SSG \cite{das2009}.

Specific wording of the inclusion and exclusion criteria, and additional limits for platelets, white blood cells, clotting, renal, and liver function, are available in the \href{https://github.com/jpwil/dphil}{Supplementary Material}.

\subsubsection{Diagnostic Criteria}

The majority of studies specified clinical criteria for inclusion, with six studies (1,984 [43.1\%] patients) requiring $\geq$2 weeks' fever and splenomegaly\cite{sundar2019,sundar2014,rijal2010b,rijal2010a,rijal2003,pandey2016}. The remaining studies either did not specify clinical criteria (two studies, 36 [0.8\%] patients)\cite{chakraborty2008,sundar2015}, or referred more generally to the presence of typical signs and symptoms (13 studies, 2,681 [58.3\%] patients).

VL was routinely confirmed by tissue aspirate in all but two studies (1,528 [33.2\%] patients)\cite{pandey2016,sundar2019}, which included patients with rK39 RDT confirmation only. Three studies (276 [6.0\%] patients)\cite{pandey2017,rijal2010a,sundar2014}, screened patients with the rK39 RDT prior to confirmation with a tissue aspirate.

\subsubsection{Relapse}

Nine studies directly defined relapse in their publications or protocols\cite{sundar2019,koirala2003,sundar2007a,rijal2010a, rijal2010b,rijal2003,pandey2017,das2009,bhattacharya2007}, with the remaining studies allowing relapse events to be inferred from definitions of initial cure and treatment failure/success at 6 months.  Confirmation of relapse with tissue aspirates was reported as performed in 13 studies, and reported (at least partially) in the IPD in 12 studies (see \href{https://github.com/jpwil/dphil}{Supplementary Material} for study--specific details).

In the three studies that reported relapse beyond 6 months\cite{sundar2019,sundar2008,rijal2010b}, IPD interrogation allowed identification of those events occurring within 6 months of initial cure assessment. Active follow-up strategies were used to identify relapse events in all but one study (928 [20.2\%] patients)\cite{sundar2019}.
\subsubsection{Initial Cure}

Initial cure was assessed 28--30 days after treatment initiation in 14 studies (3,740 [81.3\%] patients)\cite{bhattacharya2007,chakraborty2008,das2009,pandey2016,pandey2017,rijal2003,rijal2010b,sundar2010,sundar2011,sundar2012,sundar2014,sundar2015,sundar2019,koirala2003}. In the remaining five studies, initial cure was assessed either at two different time points depending on the treatment arm (two studies, 532 [11.6\%] patients)\cite{sundar2007a,sundar2009}, or between days 16 and 19 (three studies, 302 [6.6\%] patients)\cite{rijal2010a,sundar2008,sundar2008B}. Clinical criteria for initial cure was described in all but two studies\cite{bhattacharya2007,chakraborty2008}.

A tissue aspirate formed part of the initial cure assessment in all but one study (928 [20.2\%] of patients without tissue aspirate)\cite{sundar2019}. In nine studies (1,378 [30.0\%] patients)\cite{bhattacharya2007,koirala2003,rijal2003,rijal2010a,sundar2007a,sundar2008B,sundar2009,sundar2014,sundar2015}, patients with an initial $1+$ parasite grading underwent a subsequent aspirate 10 to 60 days later. If the subsequent aspirate were negative, the patient would often be referred to as a `slow--responder', and could progress to either definite cure or relapse.


\subsection{Patient Characteristics}

Overall (marginal) distributions of categorical and continuous variables are tabulated in Tables \ref{tab:isc_categorical} and \ref{tab:isc_continuous}, respectively. These distributions also displayed graphically in Figure \ref{fig:isc_cat_comb} for bar charts of categorical variables, and Figures \ref{fig:isc_pooled_dist_cont_nolab} and \ref{fig:isc_pooled_dist_cont_lab} for histograms of non--laboratory and laboratory variables, respectively. Study--specific distributions are presented for age, sex, and relapse in Figure \ref{fig:isc_main_dist} (ordered by study size). In the Appendix, study--specific distributions of all categorical and continuous variables are presented in Figures \ref{fig:isc_comb_dist_cat} and \ref{fig:isc_age_comb}--\ref{fig:isc_cr_log_comb}, respectively.

Patient numbers and proportions presented in this section exclude missing data. See Section \ref{sec:isc_missing_data} for further information on missing data.

\subsubsection{Categorical variables}

Across all studies, 2,745 (59.7\%) of patients were male, ranging from 50.0\% to 75.0\% at the study level.

Relapse within 6 months was identified in 228 (5.0\%) patients, varying from 0\% to 10.4\% at the study level.

The majority of patients (52.4\%) of patients had mild--normal malnutrition. Moderate and severe malnutrition affected 28.7\% and 18.9\% of patients overall, although these proportions varied significantly across studies and are affected by a high amount of missing data (see Appendix Figure \ref{fig:isc_comb_dist_cat}).

Approximately half of all patients (49.8\%) had severe anaemia at the time of recruitment.

At the patient level, the most common treatment regimen was 28 days of oral miltefosine (35.6\%), followed by 10mg/kg single dose liposomal amphotericin B (28.9\%). The remaining treatment regimens (35.4\% of patients), consisted of a broad range of experimental and non--experimental regimens, including amphotericin B deoxycholate, other (not Gilead) lipid formulations of amphotericin B, paromomycin, SSG, pentamidine, and human placenta extract (Figure \ref{fig:isc_treat}).

The most common baseline parasite grade on tissue aspirate was $1+$ (40.7\% of patients), while the median grade was $2+$ (25.9\% of patients at this grade; IQR: 1 to 3). The proportion of patients decreased sequentially with increasing parasite grade, with only 54 patients (1.8\%) exhibiting a grade of $5+$. When reported, 92.0\% of tissue aspirates were obtained from the spleen. The remaining aspirates were from bone marrow and accounted for almost all aspirates performed in three studies from Nepal\cite{rijal2010a,rijal2003,koirala2003} (Appendix Figure~\ref{fig:isc_comb_dist_cat}).

\subsubsection{Continuous variables}

The median age was 18 years, ranging from 1 to 80 years (IQR: 10 to 32 years). Overall the distribution is right--skewed, with study--specific distributions reflecting their age--specific inclusion criteria.

In adults ($\geq$19 years) the median BMI was 18.2 kg/m$^2$ (IQR: 16.4 to 20.8) (1,313 patients). In children ($\geq$5 and $<$19 years), the median BMI--for--age z--score was -1.68 (IQR: -2.67 to -0.72) (1,338 patients), and in infants ($<$5 years) the median weight--for--height z--score was -1.80 (IQR: -2.97 to -0.99) (27 patients).

The median spleen size was 4cm (IQR: 2 to 7cm), and ranged from 0 to 22cm. Eight studies reported patients without splenomegaly at baseline\cite{sundar2019, sundar2012, sundar2010, sundar2009,sundar2008,sundar2007a, pandey2017,bhattacharya2007}, corresponding to 5.5\% of patients with recorded spleen sizes.

The median duration of fever prior to patient recruitment was exactly 30 days (IQR: 20 to 60 days). The distribution was markedly right--skewed, ranging from 1 to 730 days.

For distributions of laboratory results (haemoglobin, white blood cells, platelets, creatinine, and alanine aminotransferase) the reader is referred to Table \ref{tab:isc_continuous} and Figure \ref{fig:isc_pooled_dist_cont_lab}.


\subsection{Pooled Univariable Associations}

Unadjusted relationships between variables (including all candidate predictors, excluding missing data) and relapse risk are presented in tabular form (Tables \ref{tab:isc_categorical}, \ref{tab:isc_continuous}) and visually alongside their distributions (Figure \ref{fig:isc_cat_comb} for categorical variables, and Figures \ref{fig:isc_pooled_dist_cont_nolab} and \ref{fig:isc_pooled_dist_cont_lab} for continuous non--laboratory and laboratory variables, respectively). The relationships are presented on the log--odds (logistic) scale for continuous candidate predictors in Appendix Figure \ref{fig:isc_logodds}.

Across continuous candidate predictors, GAM smooths suggested often non--linear relationships with relapse risk, with wider uncertainty at the extremes of the predictor distributions where data were sparse. Apparent trends were most evident for age, duration of fever and spleen size. Age showed a shallow U-shaped pattern, reaching a minimum relapse risk at approximately 20 years, while duration of fever showed a marked downward trend, with longer fever durations associated with lower relapse risk. With spleen size, a notable downward trend in risk was seen for spleen sizes over 2cm, perhaps better appreciated on the logit--log scale in Appendix Figure \ref{fig:isc_logodds}. For the laboratory predictors, weak upward trending monotonic patterns were observed for white blood cells and platelets, with higher values associated with increased relapse risk.

Trends in non--candidate predictors, including weight, height, and haemoglobin, were also apparent. A marked upward trend was seen with haemoglobin and relapse risk, which was also evidence in the categorical associations, with severe anaemia associated with a lower relapse risk compared to non--severe anaemia. Treatment with miltefosine was associated with a higher unadjusted relapse risk when compared with 10mg/kg single--dose liposomal amphotericin B or `Other' treatment.

Patients with parasite grades of $3+$ and $4+$ were associated with higher relapse risk compared to patients with grades $1+$ and $2+$. Extrapolation of any trend to patients with $5+$ parasite grade was limited by small numbers.

A correlation matrix showing associations between continuous variables is presented in Appendix Figure \ref{fig:isc_cont_cont}, and between continuous and categorical variables in Appendix \ref{fig:isc_cont_cat}.

\subsection{\label{sec:isc_missing_data}Missing Data}

Missing data were frequent among the candidate predictors, with 3,697 (80.4\%) of patients missing at least one candidate predictor data point. Just under half of all patients (2,177 [47.3\%] patients) were missing exactly one data point, and 1,520 (33.1\%) patients were missing two or more data points. No missing data were present in the outcome (relapse) or random--effect (study) variables.

The candidate predictor with the most missing data was malnutrition, which was unknown in 1,921 (41.8\%) patients due to a lack of height information. After malnutrition, fever duration was missing in 1,779 (38.7\%) of patients, and parasite grade was missing in 1,647 (35.8\%) patients. Remaining predictors had under 15\% missingness. Missingness patterns, ordered by missingness in the candidate predictors, are presented in Figure \ref{fig:isc_missing_summary} at the study level and overall.

On review of the missing data diagnostic plots (as described in Section \ref{sec:meth-missing} and available for review in the \href{https://github.com/jpwil/dphil}{Supplementary Material}), no clear or consistent patterns indicating violations of the missing--at--random assumption, were apparent. Convergence was obtained well before the 20 iterations for the majority of imputed predictors, and the distributional assumptions of the imputation model appeared robust on inspection of the diagnostic density and scatter plots.

\section{Model Results}

\subsubsection{Model Specification and Coefficient Estimates}

Forest plots of the final model predictors are presented in Figure \ref{fig:isc_var_forest_combined}. Full specification of the final models, including intercept terms, p--values, and predictor transformations, are presented in Appendix Tables \ref{tab:isc_model_coeff_with_pg} and \ref{tab:isc_model_coeff_without_pg}.

For both models, final predictors included: age (both linear and quadratic terms), duration of fever, treatment regimen, and presence of severe anaemia. Parasite grade was also included when considered as a candidate predictor. As can be appreciated from Figure \ref{fig:isc_var_forest_combined}, the magnitude of the adjusted associations (odds ratios) between the predictors and relapse risk were similar between the two models. The similarity between models can also be appreciated in Figure \ref{fig:isc_adjusted_assoc}, which shows the relapse probability predictions with the model intercept recalibrated to the Sundar 2019 dataset\cite{sundar2019}.

In both models, U--shaped relationships between age and relapse risk were shown, reflecting a significant positive age$^2$ coefficient. Inverse relationships between fever duration and relapse risk were also demonstrated. In the model including parasite grade, the relapse odds decreased by 34.8\% (95\% CI: 24.0--44.1\%) for each doubling of fever duration, with similar findings with the model excluding parasite grade. When included, a unit increase in parasite grade (e.g. from $1+$ to $2+$) was associated with a 30.6\% increase in relapse odds (95\% CI: 12.1--52.2\%). Relapse odds in the single dose liposomal amphotericin B, or `Other' treatment groups was found to be significantly less (approximately half) than that of the standard dose miltefosine regimen, although with marked uncertainty. The presence of severe anaemia was also found to be associated with decreased relapse odds by approximately 25\% (25.3\%, 95\% CI: 44.4--0.2\% in the model including parasite grade). Appendix Tables \ref{tab:isc_model_coeff_with_pg} and \ref{tab:isc_model_coeff_without_pg} provide the full (log) odds ratios for both models.

Pooled estimates of intraclass correlation coefficients (ICC)\footnote{ICC quantifies the proportion of total variance in the latent propensity for the outcome attributable to between--study differences.} were 3.54\% for the model including parasite grade, and 4.37\% for the model excluding parasite grade.

\subsubsection{Model Performance and Internal Validation}

Apparent and optimism--adjusted c--statistics and calibration slopes are presented in Table \ref{tab:isc_performance}. When assessed in studies with $>$5 relapse events, model discrimination was estimated at 0.70 (95\% CI: 0.66--0.74) in the model including parasite grade, and 0.69 (95\% CI: 0.64--0.74) in the model excluding parasite grade. These estimates reduced to 0.68 and 0.67, respectively, after adjusting for optimism. No evidence of significant between--study heterogeneity was identified in either model, with forest plots presented in Figure \ref{fig:isc_forestCIM1} (model including parasite grade) and Appendix Figure \ref{fig:isc_forestCIM2} (excluding parasite grade).

Optimism in calibration slopes were estimated at 0.093 and 0.098 in the models including and excluding parasite grade, respectively. Resulting uniform shrinkage factors were 0.91 and 0.92, respectively. Multiplying together these shrinkage factors with the estimated model coefficients will result in the optimism--adjusted coefficients (Appendix Tables \ref{tab:isc_model_coeff_with_pg} and \ref{tab:isc_model_coeff_without_pg}). No significant evidence of between--study heterogeneity in calibration slope was identified with forest plots presented in Figure \ref{fig:isc_forestCalM1} (model including parasite grade) and Appendix Figure \ref{fig:isc_forestCalM2} (model excluding parasite grade).

Visual inspection of the calibration plots comparing predicted and observed relapse probabilities showed minimal deviation from perfect calibration (Figure \ref{fig:isc_calPlot}). Calibration plots showing predicted and observed relapse probabilities for fever duration (Figure \ref{fig:isc_calPlotFD1} for model including parasite grade), and other predictors (Appendix Figures \ref{fig:isc_calPlotPD_ap} to \ref{fig:isc_calPlotRx1_ap} for model including parasite grade and \ref{fig:isc_calPlotFD2_ap} to \ref{fig:isc_calPlotRx2_ap} excluding parasite grade) and also showed good overall agreement, with predicted relapse probabilities lying within the 95\% confidence intervals of the observed probabilities.

Calibration intercepts (calibration--in--the--large) were found to vary significantly between studies for both models, as can be appreciated from the forest plots in Figure \ref{fig:isc_forestCalM1} for the model including parasite grade, and Appendix Figure \ref{fig:isc_forestCalM2} for the model excluding parasite grade (test for heterogeneity; p = 0.01 and p = 0.002 for models including and excluding parasite grade).

The distribution of the final predictors selected across the 2 x 500 bootstrap models are presented in Appendix Table \ref{tab:isc_model_stability}. Reassuringly, the most frequently selected predictors correspond to the final predictors selected in both final models. However, a degree of stability is apparent, with predictors such as spleen size, sex, alanine aminotransferase, and the cubic age term being selected in 100--250 bootstraps.



