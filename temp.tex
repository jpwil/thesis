Relapse is harmful at the individual level not only from the morbidity and mortality linked to recurrent infection, but also due to the potentially toxic second-line regimens and prolonged hospital admissions. Both direct and indirect costs -— including loss of patient income -— can impose significant financial strain on households already affected by poverty.

Relapse is also important from a public health perspective. In areas of anthroponotic transmission, early diagnosis and treatment of all infections, including relapses, is a core pillar of both the East Africa and ISC elimination programmes, and supported by mathematical modelling\cite{lerutte2017}. Parasite strains isolated from relapse cases, however, are distinct from primary cases, presenting additional challenges. For example, there is mounting evidence that relapse strains exhibit increased infectiousness, likely resulting from selection during initial drug pressure\cite{rai2013,hendrickx2016, garcia-hernandez2015,vanaerschot2011}. Relapse cases are also more likely to both \textit{spread} drug resistant strains, as seen during the 1980s and 1990s with antimony resistance in India\cite{sundar2001}, and also be the source of de novo drug resistance mutations, as demonstrated \textit{in vitro}, and frequently described in clinical practice in when HIV-VL coinfected....

Where transmission is anthroponotic, xenodiagnosis studies confirm that the most important infectious reservoir are patients with active VL and PKDL\cite{singh2021, mondal2019}.