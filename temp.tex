A total of 11 studies were identified reporting relapse determinants in immunocompetent patients, after excluding studies that were not longitudinal and those solely reporting treatment and/or biomarker associations. As presented in Table \ref{tab:relapse-rf}, eight studies were from the ISC, including India\cite{burza2014, goyal2019, goyal2020,ostyn2014, sundar2019}, Bangladesh\cite{lucero2015, mondal2019}, and Nepal\cite{rijal2013}, two from East Africa, including South Sudan\cite{gorski2010} and Kenya\cite{kennedy2024}, and one study from Georgia\cite{kajaia2011}. A wide range of study designs and methodologies are identified, with further details, where reported, presented in [Supplemental file]. The median study size was 1,143 patients, ranging from 115\cite{rijal2013} to 8,537 patients\cite{burza2014}. Two studies report outcomes from the same trial, although reported separately here given the variation in patient numbers and methodologies used\cite{goyal2019,goyal2020}. Follow-up periods ranged from 6 months to 4 years, with relapse proportions varying from 2.6\%\cite{lucero2015} to 20.9\%\cite{rijal2013}.

Age, as a categorical variable with $\geq$2 levels, is the most frequently reported relapse determinant. All 11 studies consider age as a candidate predictor, of which eight studies demonstrate statistically significant (p<0.05) or borderline statistically significant (0.05$\leq$p<0.1) associations in the unadjusted models\cite{burza2014, goyal2019, goyal2020, lucero2015, ostyn2014, rijal2013,sundar2019,kajaia2011}. Where considered, age remains significant in all but one study in the adjusted models\cite{sundar2019}. In two ISC studies that divide age into $\geq$3 categories, a `U'-shaped relationship is seen, with increased relapse risk in younger (<5 years) and older ($\geq$40 or $\geq$45 years) groups\cite{burza2014, lucero2015}. The remaining studies show increased risk in the younger age groups only, defined variously as <12 years or <15 years in the ISC\cite{goyal2019, goyal2020, rijal2013,sundar2019}, or <1 year in the the Mediterranean region study\cite{kajaia2011}. Interestingly, the two East Africa studies do not identify age as a significant predictor of relapse\cite{gorski2010,kennedy2024}, although this may simply reflect reduced statistical power, with only 17 relapses in Kennedy et al\cite{kennedy2024}, or selection bias from incomplete matching of relapse and index cases\cite{gorski2010}. No studies explicitly model age as a continuous variable.

After age, spleen size is the next most frequently identified relapse determinant. Of the seven studies that consider spleen size, five and four studies report significant associations in the unadjusted and adjusted models, respectively\cite{burza2014, lucero2015, sundar2019,gorski2010,kajaia2011}. Heterogeneity is seen in the how and when spleen size is measured and modelled. For example, Lucero et al report a significant association between larger spleens at discharge and relapse (OR 1.27, 95\% CI: 1.10--1.47, per cm below lower costal margin). Sundar et al report an approximate doubling of the relapse odds in patients with an admission spleen size of over 4cm\cite{sundar2019}. Gorski et al find increased relapse odds in patients with larger spleen sizes at discharge \textit{and} admission, as measured by Hackett grade\cite{gorski2010}, and Kajaia et al report increased relapse risk with larger spleens at admission as measured by the Kandelaki splenometric method\cite{kajaia2011, meliia2006}. Instead of modelling absolute spleen sizes, Burza et al instead model the change in spleen size during admission, and found that patients with a reduction in spleen size of $\leq$0.5cm/day had a 1.7 (95\% CI: 1.1--2.5) higher odds of relapse compared to >0.5cm/day, with similar findings in the adjusted model\cite{burza2014}.

Of the six studies that consider symptom duration prior to diagnosis or treatment, four studies identify significant associations with relapse in both their adjusted and unadjusted models\cite{burza2014,goyal2019,goyal2020,kajaia2011}. In the ISC, Burza et al report that patients with $\leq$4 weeks of symptoms had a 1.6 (95\% CI: 1.0--2.8) and 2.3 (95\% CI: 1.2--4.8) higher odds of relapse compared to patients with 4--8 weeks and >8 weeks of symptoms, respectively, with similar effect sizes seen after adjusting for age, sex and change in spleen size\cite{burza2014}. Also in the ISC, Goyal et al demonstrate that, after adjusting for treatment and age, patients with symptom duration $\leq$8 weeks had a 3.3 (95\% CI: 1.3--8.4) higher odds of relapse compared to those with >8weeks of symptoms\cite{goyal2019}. When performing a time-to-event analysis in patients from the same study, those with a symptom duration of $\leq$8 weeks have a 3.6 (95\% CI: 1.4--9.1) higher relapse hazard compared to those with >8 weeks of symptoms\cite{goyal2020}. In contrast to these findings, Kajaia et al report that in Georgia, patients with symptom durations of $\geq$90 days had a 3.9 (95\% CI: 1.8--8.5) \textit{higher} relapse risk compared to those with <90 of symptoms, after adjusting for haemoglobin and age\cite{kajaia2011}. Kennedy et al considered symptom duration in their analysis, however it was not found to be a significant predictor of relapse, perhaps due to low statistical power\cite{kennedy2024}

Despite sex being considered in all studies, only three identify significant associations with relapse\cite{burza2014,ostyn2014,sundar2019}. All from the ISC, these studies report an approximate doubling of the relapse odds\cite{burza2014,sundar2019} or rate\cite{ostyn2014}, with similar effects sizes seen in both unadjusted and adjusted models.

Admission haemoglobin levels are considered in eight studies\cite{burza2014,goyal2019,kajaia2011,kennedy2024,lucero2015,rijal2013, sundar2019} and reported as significant in three\cite{kajaia2011,kennedy2024,lucero2015}. When adjusting for age and sex, Lucero et al found a strong statistical association between relapse risk and low admission and discharge haemoglobin levels, when considered on a linear scale (OR 1.40, 95\% CI: 1.09--1.72 and 1.45, 95\% CI: 1.15--1.85, per g/dL decrease, respectively). However, in the fully adjusted model these associations did not retain significance. In the unadjusted analysis performed by Kajaia et al, a strong relationship between haemoglobin and relapse was seen, with relapse found to be 12.0 (95\%CI: 4.1--34.8) times more likely in those with a haemoglobin <60g/l vs $\geq$80g/l. This relationship remained significant after adjusting for age and symptom duration.
