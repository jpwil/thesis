As evidenced by poor treatment outcomes in patients with advanced HIV co-infection, the presence of an effective host immunity is considered fundamental to achieving lasting cure. In this regard, the role of the initial treatment course is not to eradicate all parasites from the body, but instead to suppress the parasite burden sufficiently such that the host's cell-mediated immunity can recover enough to maintain parasite suppression.

In VL, development of an effective cell-mediated immunity can be demonstrated with a positive Leishmanin skin test (LST)\footnote{Also known as the Montenegro test, initially developed in the 1920s to support the diagnosis of CL. Akin to the Mantoux test for latent tuberculosis, a delayed-type hypersensitivity response from the LST manifests as visible skin induration occurring 48--72 hours after an intradermal injection of the \textit{Leishmania} antigen.\cite{carstens-kass2021}.}. During VL infection the test is expected to be negative, reflecting cellular anergy. Eventual conversion to a positive test following treatment is a strong indicator that treatment has been successful, with lasting protection from relapse and reinfection.