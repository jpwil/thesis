
\section{Introduction}

As outlined in Chapter~\ref{ch:2-background}, VL relapse is consequential not only for individual patients but also poses a threat to sustained elimination efforts. Accordingly, the development of a non-invasive tool to predict relapse \textemdash\ functioning as a `test--of--cure' following initial treatment \textemdash\ has been identified by the WHO as a research priority\cite{WHO2024_Leishmaniasis,WHO_2024_VL_easternAfrica,who_sea_elim}. A prognostic model represents a potential solution: by quantifying the relationship between patient characteristics and subsequent relapse events, relapse risk in future patients can be estimated and clinical decision-making informed. However, as demonstrated in Chapter~\ref{ch:3-sys-review}, no prognostic models for VL relapse have been published to date.

To address this evidence gap, four prognostic models are developed using IPD from the IDDO VL data platform: two for patients from the ISC and two for patients from East Africa. Within each region, one model includes parasite grade at initial cure assessment, and one model excludes parasite grade, reflecting differences in data availability and clinical practice. All models use routinely collected information available at the time of initial cure assessment to predict six-month relapse among VL patients without HIV co-infection.

The development and evaluation of clinical prediction models is supported by an extensive and growing methodological literature. Over the past decade, reporting guidelines for prediction model studies have been established\cite{collins2015, moons2015, debray2023, collins2024A}, alongside an increasing number of reviews and recommendations that define best practice\cite{efthimiou2024, collins2024B, riley2024A, riley2024B, vanSmeden2021}. In particular, the application of meta-analysis techniques to IPD from multiple studies presents exciting new opportunities for prediction model development and evaluation\cite{riley2021_book_ch17, debray2023}. Notable opportunities include increased sample sizes leading to greater statistical power, and the ability to explore heterogeneity in predictor effects and model performance across different settings. Additionally, IPD can be used to standardise inclusion criteria and outcome/predictor definitions across included studies. However, as we lay out in this chapter, the use of IPD in prediction model research also introduces challenges; specifically (i) the need for statistical models that account for clustering of participants within studies and (ii) the presence of missing data, which can be sporadically missing within studies, or entirely missing from one or more studies\cite{debray2023}.

The aim of this chapter is to describe and justify the methodology used for model development and evaluation \textemdash\ from data acquisition to final model presentation. Guidelines and methodological texts are cited accordingly, and checklists provided for current reporting guidelines on prediction model studies (TRIPOD-AI and TRIPOD-Cluster)\cite{collins2024A,debray2023}. Additional material is presented in Appendix \ref{app:methodology} and in the \href{https://github.com/jpwil/dphil}{Supplementary Material}.\footnote{Available at \url{https://github.com/jpwil/dphil}.} A protocol is available on the \href{https://osf.io/z4bdn}{Open Science Framework}.\footnote{Created Nov 8, 2024, available at \url{https://osf.io/z4bdn}.}

This chapter's structure closely mirrors the methodological workflow as outlined in Figure~\ref{fig:workflow}, with sections on data harmonisation, model development, and internal validation. In keeping with best practice, and similar to the approach adopted in Chapter \ref{ch:3-sys-review}, the research question is presented in Box \ref{box:picots} using the PICOTS (population, index model, comparator model, outcome, timing, and setting) framework\cite{riley2021_book_ch17, debray2017}. Further elaboration of the eligibility criteria, and standardised definitions of predictors and the outcome are considered in the following section.

All analyses were performed using R version 4.4.1\cite{r2025}, with R packages cited in the relevant sections below. R scripts used for model development and evaluation are provided in the \href{https://github.com/jpwil/dphil}{Supplementary Material}.

\begin{mybox}{Definition of the research question: a PICOTS approach}
    \begin{description}[nosep]
        \item[Population] HIV-negative patients that are prospectively recruited into a clinical trial with a diagnosis of visceral leishmaniasis, confirmed either serologically or parasitologically. No restrictions are placed on age, sex or treatment regimen.
        \item[Index models] For each setting, two prognostic models are developed; one including baseline parasite grade from a tissue aspirate, and one without. The models predict the \textit{future} occurrence of relapse, using patient information collected at treatment baseline. The intended time of model use is following a successful assessment of treatment response.
        \item[Comparator model] As established in Chapter \ref{ch:3-sys-review}, no published relapse models are available for comparison or updating.
        \item[Outcome] Relapse is defined as the recurrence of signs and symptoms of VL requiring rescue treatment, and following demonstration of an initial treatment response.
        \item[Timing] Relapse occurring within 6--months of test--of--cure (typically occurring at the time of treatment completion, or within 30 days of starting treatment).
        \item[Setting] Participants from either the Indian subcontinent or East Africa.
    \end{description}
    \label{box:picots}
\end{mybox}

\section{Data harmonisation}

Here, data harmonisation refers to the process of data acquisition, curation, and any subsequent data manipulation required to produce a single analysis dataset ready for model development.

In the interest of full disclosure, data acquisition was completed by IDDO colleagues prior to the commencement of this DPhil project. The first stage of data curation \textemdash\ conversion of the contributed datasets to the Clinical Data Interchange Standards Consortium (CDISC) Study Data Tabulation Model (SDTM) standard \textemdash\ was performed by the IDDO data engineering team with support from the IDDO science team (myself included). Subsequent methodological steps were led by myself.

\subsection{Data acquisition}

A systematic review of the scientific literature was first performed in 2016, with the aim of comprehensively cataloguing all existing VL clinical trials (PROSPERO: \href{https://www.crd.york.ac.uk/PROSPERO/view/CRD42021284622}{CRD42021284622})\cite{bush2017}. 145 trials were initially identified (1980--2016, n = 26,986 patients), with further trials added during periodic updates according to an open protocol\cite{Singh-Phulgenda2022_IDDO_VL_protocol}. Between 2018--2022, corresponding authors of the identified VL clinical trials were invited to share their IPD with the IDDO VL data platform, in line with the General Data Protection Regulation (GDPR)--compliant IDDO data sharing policy\cite{IDDO_DataGovernance, dahal2025}.

\subsection{Data curation}

Conversion of the contributed datasets to an analysis--ready dataset of all eligible IPD occurred in two key stages.

\subsubsection{Stage 1: CDISC SDTM curation}

To facilitate reusability and interoperability, contributed datasets were standardised to a common storage format: the CDISC--compliant SDTM standard\cite{cdisc2024}, adapted by IDDO for VL\cite{iddo2020}. During this process, contributed datasets underwent \textit{psuedonymisation},\footnote{\ `\dots processing of personal data in such a manner that the personal data can no longer be attributed to a specific data subject without the use of additional information' \url{https://ico.org.uk/for-organisations/uk-gdpr-guidance-and-resources/data-sharing/anonymisation/pseudonymisation/} (accessed 15 Dec 2025).} prior to being available for data sharing requests. Briefly, SDTM format datasets comprise a number of standardised domains (tables) containing related information (e.g. patient demographics, laboratory results, treatment administration, clinical signs and symptoms). Each domain contains a set of standard variables (table columns, e.g. STUDYID, USUBJID, VISITDY) alongside VL--specific variables defined by IDDO (e.g. parasite grade, spleen size). Further details of the curation process are available in the \href{https://www.iddo.org/tools-and-resources/data-tools}{IDDO SDTM Implementation Guide}.\footnote{Available at \url{https://www.iddo.org/tools-and-resources/data-tools}, free registration required.}

\newgeometry{left=1cm, bottom=2.5cm, right=2cm, top=3cm}
\begin{landscape}
    \begin{figure}[tb]
        \centering
        \includegraphics[width=1.2\textwidth, trim={1.6cm 1cm 1.6cm 1.3cm}, clip]{figures/ch4/workflow.pdf}
        \caption{Schema of methodological workflow. \ding{192} Data harmonisation is performed for each contributed dataset including: initial curation (by the IDDO data engineering team) to the CDISC SDTM format, application of inclusion and exclusion criteria, and removal of outliers. Curated and cleaned datasets are converted into an analysis (wide) format and merged prior to \ding{193} model development. Multiple imputation is used to create (m = 30) imputed (complete) datasets. Variable selection, model fitting, and apparent performance evaluation is performed on all imputed datasets. Estimates are pooled using Rubin's rules. Bootstrapping is used to perform \ding{194} internal validation, allowing (i) review of model stability and (ii) optimism adjustment of performance measures and original model coefficients. All model development steps, including multiple imputation, are repeated for each of the (b = 500) bootstrap datasets. The resulting bootstrap models (b = 500) are evaluated both in the corresponding bootstrap (imputed) datasets and the original (imputed) datasets, and pooled using Rubin's rules. The mean of the differences of the pooled performance measures (in bootstrap vs. original dataset) are used to shrink the original model coefficients and apparent performance measures, resulting in the final optimism-adjusted model. AP: apparent performance; b: number of bootstraps; BP: bootstrap model performance; CDISC: Clinical Data Interchange Standards Consortium; IDDO: infectious diseases data observatory; m: number of imputations; n: number of contributed studies. SDTM: Study Data Tabulation Model.}
        \label{fig:workflow}
    \end{figure}
\end{landscape}
\restoregeometry

\subsubsection{Stage 2: Analysis--ready dataset curation}

Subsequently, SDTM format datasets were converted to a single analysis--ready dataset, primed for model development. This stage consisted of multiple steps, refined iteratively over several months and in close consultation with the IDDO data engineering team:

\begin{itemize}[noitemsep]
    \item Identification and removal of spurious data points (e.g. outliers, discussed below)
    \item Application of study and participant eligibility criteria (Section \ref{sec:eligibility})
    \item Creation of a standardised outcome variable according to a pre-defined definition (Section \ref{sec:out-definition})
    \item Conversion of the datasets from a long to wide format, consisting of one row per participant
    \item Merging of all datasets into a single analysis dataset
\end{itemize}

Data wrangling during the second curation stage was performed with the \texttt{tidyverse} suite of R packages\cite{wickham2019}. Identification and removal of spurious data points was performed through subgroup tabulations and visual inspection of histograms and scatter plots. Where two incongruous data points were identified, for example, incompatible height and weight values, a third variable, such as BMI or age, would be used to identify the spurious value. Data points considered to be outliers were converted to missing values. A complete record of all data cleaning steps, including outlier identification and removal, was maintained and documented in commented R scripts.

In Section \ref{sec:relapse} of Chapter \ref{ch:2-background}, relapse was defined broadly as `the reappearance of VL signs and symptoms following an initial treatment response', and `typically confirmed by direct visualisation of the parasite on a tissue aspirate smear'. Despite appearing a clear definition, on closer inspection it can be appreciated that even subtle variations in eligibility criteria, study design, and the definition of efficacy endpoints can, at times unexpectedly, impact the proportion of patients experiencing relapse as a study outcome.

\subsection{\label{sec:eligibility}Population at risk}

Clear specification of the population at risk is fundamental to understanding the model's real-world applicability\cite{riley2021_book_ch17, collins2024A}. Particular attention is given to the definition of initial cure, since (i) relapse can only occur following an initial treatment response, and (ii) heterogeneity in study-level cure definitions can be partially addressed through IPD--based standardisation.

The following inclusion and exclusion criteria were applied at the study and participant levels, respectively:

\begin{itemize}[noitemsep]
    \item \textbf{Study--level \underline{inclusion} criteria}
          \begin{itemize}
              \item Studies conducted in either the ISC (India, Nepal, Bangladesh) or East Africa (Ethiopia, Sudan, South Sudan, Kenya, Uganda)
              \item Prospective design, defined as participants having provided informed consent
              \item Participants recruited with a diagnosis of VL as defined by a combination of clinical symptoms and either parasitological or serological confirmation
              \item Studies that report, as a minimum, the treatment regimen including at least the drug name(s), dose and duration
              \item Recruited a minimum of 6 patients
              \item Included a minimum of 6 months of prospective follow-up from treatment initiation
              \item Reported VL relapse events during the 6-month follow-up period
          \end{itemize}
    \item \textbf{Participant--level \underline{exclusion} criteria}
          \begin{itemize}
              \item Participants with HIV co-infection or from a setting with high HIV co-infection prevalence and without a negative HIV test
              \item Participants who were confirmed pregnant at the time of treatment initiation
              \item Participants with symptomatic treatment failure requiring rescue treatment, identified either before or at initial cure assessment
          \end{itemize}
\end{itemize}

Inclusion and exclusion criteria are chosen according to (i) the eligibility criteria applied in the original systematic review from which identified study authors were invited to contribute their IPD\cite{bush2017}, (ii) the range of studies available in the IDDO VL data platform, and (iii) the resulting impact and applicability of models developed.

\subsubsection{Study--level criteria}

Study-level inclusion criteria were applied to ensure that contributing datasets were sufficiently comparable in terms of epidemiological context, study design, and outcome ascertainment to permit meaningful harmonisation and pooled analysis.

Studies were limited to those conducted in East Africa and the ISC, reflecting both the public health relevance of relapse prediction in regions with ongoing VL elimination programmes, and the availability of IPD within the IDDO VL data platform. On review of the IDDO inventory \cite{iddo2025vlinventory}, only two studies were conducted outside these regions \textemdash\ one in Greece conducted in the 1990s \cite{syriopoulou2003}, and one in Brazil in the 2010s \cite{romero2017}. These were excluded to preserve geographical and epidemiological coherence.

Only prospectively conducted studies were included. Prospective designs allow for systematic and active follow-up, predefined outcome definitions, and contemporaneous outcome recording, all of which are important for the reliable identification of initial cure and subsequent relapse. However, reliance on IPD derived from clinical trials limits the direct applicability of the resulting model to real-world patients \textemdash\ those who are managed outside trial settings, and may not meet the often--stringent eligibility criteria. These limitations are discussed further in Chapter~\ref{ch:7-discussion}.

A minimum study size was imposed to exclude very small cohorts with unstable relapse estimates. Finally, studies were required to report relapse events during follow-up, either explicitly or in a form that allowed relapse to be inferred from the available IPD.

\subsubsection{Participant--level criteria}

With respect to clinical presentation, treatment response, and outcomes, patients with VL/HIV co-infection constitute an important but distinct clinical population. Accordingly, patients with and without VL/HIV co-infection were \textit{not} combined within a single prediction model, given the substantial uncertainty in extrapolating relapse associations derived from HIV-negative patients to those with VL/HIV co-infection. Since the majority of contributing studies excluded patients with VL/HIV co-infection, insufficient IPD were available to develop a separate model for this group.

As with VL/HIV co-infection, very few contributing studies included pregnant participants, reflecting their systematic exclusion from VL trials and precluding the development of a separate prediction model.

\subsubsection{Initial cure}

Understanding study--specific definitions of initial cure is important, as all studies require the patient to demonstrate a treatment response, measured with a test--of--cure, in order to be at subsequent risk of relapse. Consequently, patients \textit{not} achieving initial cure \textemdash\ described as initial treatment failure \textemdash\ should be excluded. A direct consequence of excluding these patients is that model--derived risk estimates are only applicable to patients demonstrating initial cure.

Initial cure definitions based solely on clinical improvement, as is common in routine practice, are likely to classify some patients as cured despite persistently positive tissue aspirates, were these assessed. These patients form a subgroup at increased risk of relapse and would instead be classified as initial treatment failures under more stringent, parasitology-based test--of--cure criteria, thereby being excluded from subsequent follow-up. Consequently, all else being equal, studies applying stricter definitions of initial cure will observe a lower subsequent relapse risk.

Of the 89 studies identified by Dahal et al, 71 (79.8\%) included parasitological assessment (with or without demonstration of clinical improvement) in the test--of--cure, while 13 (14.6\%) required clinical improvement only. The remaining studies did not report the criteria\cite{dahal2024}. Timing also varied considerably, with the 68 (76.4\%) of studies performing the test--of--cure from 15--30 days following treatment completion. Similar patterns are observed in the contributed studies, as reported in the \href{https://github.com/jpwil/dphil}{Supplemental Material}, and are discussed further in subsequent results chapters. Importantly, criteria for `clinical improvement' are often not specified. Further complicating interpretation, many studies describe a subgroup of `slow responders', in whom the test--of--cure tissue aspirate remains positive despite clinical improvement. These patients may undergo repeat assessment at variable time points (e.g. 2--4 weeks later), with or without treatment extension, and may or may not ultimately be classified as having achieved initial cure. Such variation in both the definition and timing of the initial cure assessment can challenge standardisation efforts, leading to heterogeneity in observed relapse rates that is independent of other relapse risk factors.


\item \textbf{An initial treatment response is achieved.} Where initial cure (or initial treatment failure) is \textit{not} directly recorded in the IPD as an efficacy outcome, or where it is recorded but the study definition considers `slow responders' as initial treatment failures, it can be inferred from (i) improvement of signs and symptoms between baseline and test--of--cure, and (ii) not requiring rescue treatment during initial treatment. Importantly, reflecting both routine clinical practice and a number of study definitions, detection of parasites at test--of--cure should not preclude the subsequent development of relapse, so long as points (i) and (ii) are met.


----

\subsection{Outcome}

Relapse itself, where described at the study--level, is also subject to substantial variation with respect to its (i) definition \textemdash\ including whether patients require initial cure, the severity of symptoms required to trigger a repeat aspirate, and the tissue type chosen for aspirate, and (ii) timing \textemdash\ e.g. whether patients were actively screened at set time points with clinical examination $\pm$ routine aspirates, or whether dependent on patients attending voluntarily (passively) based on recurrent symptoms and discharge advice. In line with findings by Dahal et al, a significant proportion of contributing studies do not directly define relapse as a study outcome\cite{dahal2024}. Instead, for most studies, a relapse event can be inferred from patients achieving initial cure who subsequently do not meet the definition of `definite cure', which itself is typically defined as patients requiring rescue treatment.

\subsubsection{Outcome standardisation}

Access to study--level IPD allows for identification of relapse both directly, where relapse events are presented as a pre-defined efficacy outcome, and indirectly, where relapse events are identified using a combination of related efficacy outcome data (initial cure status, definite cure status), and information recorded on tissue aspirates, timing of rescue treatment, and patient signs and symptoms recorded at pre-defined follow-up periods.

Where direct identification of relapse events is not provided in the contributed IPD, the following two--part pragmatic definition will be applied:

\begin{enumerate}[noitemsep]

    \item \textbf{Subsequent recurrence of symptoms considered to be a relapse event.} Where relapse is not directly recorded in the IPD as an efficacy outcome, it can be inferred from (i) absence of rescue treatment within 6 months of test--of--cure assessment, (ii) the presence of a positive tissue aspirate in addition to a recurrence of compatible signs and symptoms.
\end{enumerate}

Since the large majority of contributed studies prospectively follow-up patients for 6 months following test--of--cure, this will also define the prediction horizon \textemdash\ the maximum time between the moment of intended model application (test--of--cure) and outcome (relapse) event occurring.

Relapse is recorded and modelled as a binary outcome variable (occurred vs. not occurred, or 1/0). Unfortunately, modelling relapse as a time--to--event variable is not feasible, since \textit{timing} information is (i) inconsistently recorded across the contributed IPD, and (ii) where recorded, is frequently limited to fixed, predetermined study visits (e.g. 3 months, 6 months).


- loss of follow up
- cross-referencing with study publications (where available)

