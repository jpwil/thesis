\chapter{\label{ch:4-isc-model-methodology}Model methodology}
\minitoc

\section{Introduction}

As outlined in Chapter~\ref{ch:2-background}, VL relapse is consequential not only for individual patients but also poses a threat to sustained elimination efforts. Accordingly, the development of a non-invasive tool to predict relapse \textemdash\ functioning as a `test--of--cure' following initial treatment \textemdash\ has been identified by the WHO as a research priority\cite{WHO2024_Leishmaniasis,WHO_2024_VL_easternAfrica,who_sea_elim}. A prognostic model represents a potential solution: by quantifying the relationship between patient characteristics and subsequent relapse events, relapse risk in future patients can be estimated and clinical decision-making informed. However, as demonstrated in Chapter~\ref{ch:3-sys-review}, no prognostic models for VL relapse have been published to date.

To address this evidence gap, four prognostic models are developed using IPD from the IDDO VL data platform: two for patients from the ISC and two for patients from East Africa. Within each region, one model includes parasite grade at initial cure assessment, and one model excludes parasite grade, reflecting differences in data availability and clinical practice. All models use routinely collected information available at the time of initial cure assessment to predict six-month relapse among VL patients without HIV co-infection.

The development and evaluation of clinical prediction models is supported by an extensive and growing methodological literature. Over the past decade, reporting guidelines for prediction model studies have been established\cite{collins2015, moons2015, debray2023, collins2024A}, alongside an increasing number of reviews and recommendations that define best practice\cite{efthimiou2024, collins2024B, riley2024A, riley2024B, vanSmeden2021}. In particular, the application of meta-analysis techniques to IPD from multiple studies presents exciting new opportunities for prediction model development and evaluation\cite{riley2021_book_ch17, debray2023}. Notable opportunities include increased sample sizes leading to greater statistical power, and the ability to explore heterogeneity in predictor effects and model performance across different settings. Additionally, IPD can be used to standardise inclusion criteria and outcome/predictor definitions across included studies. However, as we lay out in this chapter, the use of IPD in prediction model research also introduces challenges; specifically (i) the need for statistical models that account for clustering of participants within studies and (ii) the presence of missing data, which can be sporadically missing within studies, or entirely missing from one or more studies\cite{debray2023}.

The aim of this chapter is to describe and justify the methodology used for model development and evaluation \textemdash\ from data acquisition to final model presentation. Guidelines and methodological texts are cited accordingly, and checklists provided for current reporting guidelines on prediction model studies (TRIPOD-AI and TRIPOD-Cluster)\cite{collins2024A,debray2023}. Additional material is presented in Appendix \ref{app:methodology} and in the \href{https://github.com/jpwil/dphil}{Supplementary Material}.\footnote{Available at \url{https://github.com/jpwil/dphil}.} A protocol is available on the \href{https://osf.io/z4bdn}{Open Science Framework}.\footnote{Created Nov 8, 2024, available at \url{https://osf.io/z4bdn}.}

This chapter's structure closely mirrors the methodological workflow as outlined in Figure~\ref{fig:workflow}, with sections on data harmonisation, model development, and internal validation. In keeping with best practice, and similar to the approach adopted in Chapter \ref{ch:3-sys-review}, the research question is presented in Box \ref{box:picots} using the PICOTS (population, index model, comparator model, outcome, timing, and setting) framework\cite{riley2021_book_ch17, debray2017}. Further elaboration of the eligibility criteria, and standardised definitions of predictors and the outcome are considered in the following section.

All analyses were performed using R version 4.4.1\cite{r2025}, with R packages cited in the relevant sections below. R scripts used for model development and evaluation are provided in the \href{https://github.com/jpwil/dphil}{Supplementary Material}.

\begin{mybox}[label=box:picots]{Definition of the research question: a PICOTS approach}
    \begin{description}[nosep]
        \item[Population] HIV-negative patients that are prospectively recruited into a clinical trial with a diagnosis of visceral leishmaniasis, confirmed either serologically or parasitologically. No restrictions are placed on age, sex or treatment regimen.
        \item[Index models] For each setting, two prognostic models are developed; one including baseline parasite grade from a tissue aspirate, and one without. The models predict the \textit{future} occurrence of relapse, using patient information collected at treatment baseline. The intended time of model use is following a successful assessment of treatment response.
        \item[Comparator model] As established in Chapter \ref{ch:3-sys-review}, no published relapse models are available for comparison or updating.
        \item[Outcome] Relapse is defined as the recurrence of signs and symptoms of VL requiring rescue treatment, and following demonstration of an initial treatment response.
        \item[Timing] Relapse occurring within 6--months of test--of--cure (typically occurring at the time of treatment completion, or within 30 days of starting treatment).
        \item[Setting] Participants from either the Indian subcontinent or East Africa.
    \end{description}
\end{mybox}

\section{Data harmonisation}

Here, data harmonisation refers to the process of data acquisition, curation, and any subsequent data manipulation required to produce a single analysis dataset ready for model development.

In the interest of full disclosure, data acquisition was completed by IDDO colleagues prior to the commencement of this DPhil project. The first stage of data curation \textemdash\ conversion of the contributed datasets to the Clinical Data Interchange Standards Consortium (CDISC) Study Data Tabulation Model (SDTM) standard \textemdash\ was performed by the IDDO data engineering team with support from the IDDO science team (myself included). Subsequent methodological steps were led by myself.

\subsection{Data acquisition}

A systematic review of the scientific literature was first performed in 2016, with the aim of comprehensively cataloguing all existing VL clinical trials (PROSPERO: \href{https://www.crd.york.ac.uk/PROSPERO/view/CRD42021284622}{CRD42021284622})\cite{bush2017}. 145 trials were initially identified (1980--2016, n = 26,986 patients), with further trials added during periodic updates according to an open protocol\cite{Singh-Phulgenda2022_IDDO_VL_protocol}. Between 2018--2022, corresponding authors of the identified VL clinical trials were invited to share their IPD with the IDDO VL data platform, in line with the General Data Protection Regulation (GDPR)--compliant IDDO data sharing policy\cite{IDDO_DataGovernance, dahal2025}.

\subsection{Data curation}

Conversion of the contributed datasets to an analysis--ready dataset of all eligible IPD occurred in two key stages.

\subsubsection{Stage 1: CDISC SDTM curation}

To facilitate reusability and interoperability, contributed datasets were standardised to a common storage format: the CDISC--compliant SDTM standard\cite{cdisc2024}, adapted by IDDO for VL\cite{iddo2020}. During this process, contributed datasets underwent \textit{psuedonymisation},\footnote{\ `\dots processing of personal data in such a manner that the personal data can no longer be attributed to a specific data subject without the use of additional information' \url{https://ico.org.uk/for-organisations/uk-gdpr-guidance-and-resources/data-sharing/anonymisation/pseudonymisation/} (accessed 15 Dec 2025).} prior to being available for data sharing requests. Briefly, SDTM format datasets comprise a number of standardised domains (tables) containing related information (e.g. patient demographics, laboratory results, treatment administration, clinical signs and symptoms). Each domain contains a set of standard variables (table columns, e.g. STUDYID, USUBJID, VISITDY) alongside VL--specific variables defined by IDDO (e.g. parasite grade, spleen size). Further details of the curation process are available in the \href{https://www.iddo.org/tools-and-resources/data-tools}{IDDO SDTM Implementation Guide}.\footnote{Available at \url{https://www.iddo.org/tools-and-resources/data-tools}, free registration required.}



\subsubsection{Stage 2: Analysis--ready dataset curation}

Subsequently, SDTM format datasets were converted to a single analysis--ready dataset, primed for model development. This stage consisted of multiple steps, refined iteratively over several months and in close consultation with the IDDO data engineering team:

\begin{itemize}[noitemsep]
    \item Identification and removal of spurious data points (e.g. outliers, discussed below)
    \item Application of study and participant eligibility criteria (Section \ref{sec:eligibility})
    \item Creation of a standardised outcome variable according to a pre-defined definition (Section \ref{sec:methods-outcome})
    \item Conversion of the datasets from a long to wide format, consisting of one row per participant
    \item Merging of all datasets into a single analysis dataset
\end{itemize}

Data wrangling during the second curation stage was performed with the \texttt{tidyverse} suite of R packages\cite{wickham2019}. Identification and removal of spurious data points was performed through subgroup tabulations and visual inspection of histograms and scatter plots. Where two incongruous data points were identified, for example, incompatible height and weight values, a third variable, such as BMI or age, would be used to identify the spurious value. Data points considered to be outliers were converted to missing values. A complete record of all data cleaning steps, including outlier identification and removal, was maintained and documented in commented R scripts.

In Section \ref{sec:relapse} of Chapter \ref{ch:2-background}, relapse was defined broadly as `the reappearance of VL signs and symptoms following an initial treatment response', and `typically confirmed by direct visualisation of the parasite on a tissue aspirate smear'. Despite appearing a clear definition, on closer inspection it can be appreciated that even subtle variations in eligibility criteria, study design, and the definition of efficacy endpoints can, at times unexpectedly, impact the proportion of patients experiencing relapse as a study outcome.

\subsection{\label{sec:eligibility}Population at risk}

Clear specification of the population at risk is fundamental to understanding the model's real-world applicability\cite{riley2021_book_ch17, collins2024A}. Particular attention is given to the definition of initial cure, since (i) relapse can only occur following an initial treatment response, and (ii) heterogeneity in study-level cure definitions can be partly addressed through IPD--based standardisation.

Inclusion and exclusion criteria were applied at the study and participant levels and are presented in Box \ref{box:eligibility}. Criteria are chosen according to (i) the eligibility criteria applied in the original systematic review from which identified study authors were invited to contribute their IPD\cite{bush2017}, (ii) the range of studies available in the IDDO VL data platform, and (iii) the resulting impact and applicability of models developed.

\begin{mybox}[floatplacement=tb, label=box:eligibility]{Eligibility criteria}
    \begin{itemize}[nosep]
        \item \textbf{Study--level \underline{inclusion} criteria}
              \begin{itemize}
                  \item Studies conducted in either the ISC (India, Nepal, Bangladesh) or East Africa (Ethiopia, Sudan, South Sudan, Kenya, Uganda)
                  \item Prospective design, defined as participants having provided informed consent
                  \item Participants recruited with a diagnosis of VL as defined by a combination of clinical symptoms and either parasitological or serological confirmation
                  \item Studies that report, as a minimum, the treatment regimen including at least the drug name(s), dose and duration
                  \item Recruited a minimum of 6 patients
                  \item Included a minimum of 6 months of prospective follow-up from treatment initiation
                  \item Reported VL relapse events during the 6-month follow-up period
              \end{itemize}
        \item \textbf{Participant--level \underline{exclusion} criteria}
              \begin{itemize}
                  \item Participants with HIV co-infection or from a setting with high HIV co-infection prevalence and without a negative HIV test
                  \item Participants who were confirmed pregnant at the time of treatment initiation
                  \item Participants with symptomatic treatment failure requiring rescue treatment, identified either before or at initial cure assessment
              \end{itemize}
    \end{itemize}
\end{mybox}

\subsubsection{Study--level criteria}

Study-level inclusion criteria were applied to ensure that contributed studies were sufficiently comparable in terms of epidemiological context, study design, and outcome ascertainment to permit meaningful harmonisation and pooled analysis.

Studies were limited to those conducted in East Africa and the ISC, reflecting both the public health relevance of relapse prediction in regions with ongoing VL elimination programmes, and the availability of IPD the IDDO VL data platform. On review of the IDDO inventory\cite{iddo2025vlinventory}, only two studies were conducted outside these regions \textemdash\ one in Greece conducted in the 1990s \cite{syriopoulou2003}, and one in Brazil in the 2010s \cite{romero2017}. These were excluded to preserve geographical and epidemiological coherence.

Only prospectively conducted studies were included. Prospective designs allow for systematic and active follow-up, predefined outcome definitions, and contemporaneous outcome recording, all of which are important for the reliable identification of initial cure and subsequent relapse. However, reliance on IPD from clinical trial settings limits model applicability to real-world patients \textemdash\ those who are managed outside trial settings, and may not meet the often--stringent eligibility criteria. These limitations are discussed further in Chapter~\ref{ch:7-discussion}.

A minimum study size was imposed to exclude very small cohorts with unstable relapse estimates. Finally, studies were required to report relapse events during follow-up, either explicitly or in a form that allowed relapse to be inferred from the available IPD.

\subsubsection{Participant--level criteria}

With respect to clinical presentation, treatment response, and outcomes, patients with VL/HIV co-infection constitute an important but distinct clinical population. Accordingly, patients with and without VL/HIV co-infection were \textit{not} combined within a single prediction model, given the substantial uncertainty in extrapolating relapse associations derived from HIV-negative patients to those with VL/HIV co-infection. Since the majority of contributing studies excluded patients with VL/HIV co-infection, insufficient IPD were available to develop a separate model for this group.

As with VL/HIV co-infection, very few contributing studies included pregnant participants, reflecting their systematic exclusion from VL trials and precluding the development of a separate prediction model.

\subsubsection{Initial cure}

Understanding study--specific definitions of initial cure is important, as all studies require the patient to demonstrate a treatment response, measured with a test--of--cure, in order to be at subsequent risk of relapse. Consequently, patients \textit{not} achieving initial cure \textemdash\ described as initial treatment failure \textemdash\ should be excluded. A direct consequence of excluding these patients is that model--derived risk estimates are only applicable to patients demonstrating initial cure.

Initial cure definitions based solely on clinical improvement, as is common in routine practice, are likely to classify some patients as cured despite persistently positive tissue aspirates, were these assessed. These patients form a subgroup at increased risk of relapse and would instead be classified as initial treatment failures under more stringent, parasitology-based test--of--cure criteria, thereby being excluded from subsequent follow-up. Consequently, all else being equal, studies applying stricter definitions of initial cure will observe a lower subsequent relapse risk.

Recently, Dahal and colleagues performed a systematic review of the design, conduct, analysis, and reporting of VL therapeutic efficacy studies, published between 2000--2021. Of the 89 studies identified, 71 (79.8\%) included parasitological assessment, with or without demonstration of clinical improvement, as part of the test--of--cure, while 13 (14.6\%) required clinical improvement only. The remaining studies did not provide a definition. Timing also varied considerably, with the 68 (76.4\%) of studies performing the test--of--cure between 15--30 days following treatment completion\cite{dahal2024}. Similar patterns are observed in the contributed studies, as reported in the \href{https://github.com/jpwil/dphil}{Supplemental Material}, and discussed further in subsequent results chapters. Importantly, criteria for `clinical improvement' are often not specified. Further complicating interpretation, many studies describe a subgroup of `slow responders', who remain in the study despite a positive tissue aspirate in the test--of--cure. These patients may undergo repeat assessment at subsequent time points (e.g. 2--4 weeks later), with or without treatment extension, and may or may not ultimately be classified as having achieved initial cure.

Such variation in the initial cure definition can challenge standardisation efforts, leading to differences in observed relapse rates  stemming from differences in the population at risk. These differences, however, can often be mitigated through interrogation of the IPD. Box \ref{box:ipd-events} provides a working definition for initial cure, which is applied during data harmonisation.

\begin{mybox}[floatplacement=tb, label=box:ipd-events]{IPD--based working definitions}
    \begin{description}[nosep]
        \item[\textbf{Initial cure}] Where initial cure (or initial treatment failure) is \textit{not} directly recorded in the IPD as an efficacy outcome, or where it is recorded but the study definition considers `slow responders' as initial treatment failures, it can be inferred from (i) improvement of signs and symptoms between baseline and test--of--cure, and (ii) not requiring rescue treatment during initial treatment. Importantly, reflecting both routine clinical practice and a number of study definitions, detection of parasites at test--of--cure should not preclude the subsequent development of relapse, so long as points (i) and (ii) are met.
    \end{description}
    \tcbline
    \begin{description}[nosep]
        \item[\textbf{Relapse}] Where relapse is not directly recorded in the IPD as an efficacy outcome, the event can be inferred from two or more of: (i) the need for rescue treatment within 6 months of initial cure assessment (test--of--cure), (ii) the presence of a positive tissue aspirate, and (iii) in addition to a recurrence of compatible signs and symptoms.
    \end{description}
\end{mybox}

\subsection{\label{sec:methods-outcome}Outcome}

Relapse itself, where described at the study--level, is also subject to substantial variation with respect to its (i) definition \textemdash\ including the severity of symptoms required to trigger a repeat aspirate and the tissue type chosen for aspirate, and (ii) timing \textemdash\ including whether patients were actively screened at set time points with clinical examination $\pm$ routine aspirates, or whether dependent on patients attending voluntarily based on recurrent symptoms and discharge advice. In line with findings by Dahal et al, a significant proportion of contributing studies do not directly define relapse as a study outcome\cite{dahal2024}. Instead, for most studies, a relapse event can be inferred from patients achieving initial cure who subsequently do not meet the definition of `definite cure', which itself is typically defined as patients requiring rescue treatment.

Similar to the approach described for identifying patients that achieve initial cure, access to IPD allows relapse events to be inferred from other variables, including: definite cure status, tissue aspirates, timing of rescue treatment initiation, and patient signs and symptoms. Box \ref{box:ipd-events} provides an IPD--based working definition of relapse.

Relapse is recorded and modelled as a binary outcome variable (occurred vs. not occurred, or 1/0). Unfortunately, modelling relapse as a time--to--event variable is not feasible, since \textit{timing} information is (i) inconsistently presented across the contributed IPD, and (ii) where presented, is often limited to fixed, predetermined study visits (e.g. 3 months, 6 months).

% Study-specific definitions of initial cure and relapse are presented in \hyperref[sec:add-files]{Additional file 1}.
\section{Model development}


\subsection{\label{sec:meth-samp}Sample size}

A common `rule of thumb' is that at least 10--20 outcome events per predictor parameter (EPP) are needed to prevent model overfitting\cite{peduzzi1996,harrell2015}. However, both the validity of this threshold, and the broader premise that a single rule applies universally, have been increasingly debated in the prediction modelling literature\cite{courvoisier2011}.

Responding to these concerns, in 2018 Riley et al.\cite{riley2019A} proposed a new sample size methodology. For prediction models with binary or time--to--event outcomes, the approach defines a minimum sample size that satisfies three criteria: (i) minimal optimism in predictor effect estimates, quantified by a global shrinkage factor of $\geq$0.9; (ii) a small absolute difference ($\leq$0.05) between the model's apparent and adjusted Nagelkerke's R\textsuperscript{2}; and (iii) precise estimation of the overall population risk (i.e.\ the model intercept). When the number of participants and outcome events is fixed, as in the present study, the maximum allowable EPP can be derived from an estimate of the model's overall performance. In the absence of previously published prediction models for relapse, we followed the authors' recommendation and assumed a Nagelkerke R\textsuperscript{2} of 0.15\cite{riley2019A}.

This methodology has been adopted in the model development presented in this thesis, as implemented in the \texttt{pmsampsize} R package\cite{ensor2023}.

\subsubsection{ISC model}

With a total of 228 relapses identified in 4,599 participants (5.0\% event rate), the maximum number of predictor parameters satisfying Riley et al.'s three criteria is 25, corresponding to 8.86 EPP. Relaxing the permitted overall shrinkage from 0.90 to 0.85 (modifying criteria (i)) allows for 40 predictor parameters, corresponding to 5.57 EPP.

\subsubsection{East Africa model}

With a total of 99 relapses identified in 2,051 participants (4.8\% event rate), the maximum number of predictor parameters satisfying Riley et al.'s three criteria is 11, corresponding to 8.81 EPP. Relaxing the permitted overall shrinkage from 0.90 to 0.85 allows for 18 predictor parameters, corresponding to 5.54 EPP.

\subsection{\label{sec:meth-candidate-pred}Candidate predictors}

All model variables, including candidate predictors, are listed in Table \ref{tab:meth-cp}.

For the ISC models, a total of 17 candidate predictor parameters are included (16 when excluding parasite grade), corresponding to a study--specific random intercept term and 12 (11) participant-level candidate predictors, EPP~=~13.41 (14.25).

For the East Africa models, a total of 14 candidate predictor parameters are included (13 when excluding parasite grade), corresponding to the same variables included in the ISC models minus treatment group, due to convergence issues discussed below, EPP~=~7.07 (7.62). Full specification of all candidate predictors are presented in Table \ref{tab:meth-cp}.

\definecolor{darkgreen}{RGB}{0,90,0}
\newcommand{\yes}{\textcolor{green!66!black}{\ding{51}}}
\newcommand{\no}{\textcolor{red}{\ding{55}}}
\newcommand{\gap}{1.1cm}

\begin{table}[htbp]
    \centering
    \small
    \begin{threeparttable}
        \begin{tabular}{@{} l l l l >{\centering\arraybackslash}p{\gap} >{\centering\arraybackslash}p{\gap} >{\centering\arraybackslash}p{\gap} >{\centering\arraybackslash}p{\gap} @{}}
            \toprule
            Variable                & Specification       & dof & log & \multicolumn{2}{c}{ISC} & \multicolumn{2}{c}{EA}                                              \\
            \cmidrule(lr){5-6} \cmidrule(lr){7-8}
                                    & (categories)        &     &     & $\mathrm{PG}$           & $\overline{\mathrm{PG}}$ & $\mathrm{PG}$ & $\overline{\mathrm{PG}}$ \\ \midrule
            Relapse (outcome)       & Categorical (2)     & -   & -   & \yes                    & \yes                     & \yes          & \yes                     \\ \midrule
            Study                   & Random intercept    & 1   & -   & \yes                    & \yes                     & \yes          & \yes                     \\ \midrule
            Treatment               & Categorical (3)     & 2   & -   & \yes                    & \yes                     & \no           & \no                      \\ \midrule
            Sex                     & Categorical (2)     & 1   & -   & \yes                    & \yes                     & \yes          & \yes                     \\ \midrule
            Malnutrition            & Categorical (3)     & 2   & -   & \yes                    & \yes                     & \yes          & \yes                     \\ \midrule
            Anaemia severity        & Categorical (2)     & 1   & -   & \yes                    & \yes                     & \yes          & \yes                     \\ \midrule
            Age                     & Continuous\tnote{1} & 3   & No  & \yes                    & \yes                     & \yes          & \yes                     \\ \midrule
            Fever duration          & Continuous          & 1   & Yes & \yes                    & \yes                     & \yes          & \yes                     \\ \midrule
            WBCs                    & Continuous          & 1   & Yes & \yes                    & \yes                     & \yes          & \yes                     \\ \midrule
            Spleen size             & Continuous\tnote{2} & 1   & No  & \yes                    & \yes                     & \yes          & \yes                     \\ \midrule
            Platelets               & Continuous          & 1   & Yes & \yes                    & \yes                     & \yes          & \yes                     \\ \midrule
            ALT                     & Continuous          & 1   & Yes & \yes                    & \yes                     & \yes          & \yes                     \\ \midrule
            Creatinine              & Continuous          & 1   & Yes & \yes                    & \yes                     & \yes          & \yes                     \\ \midrule
            Parasite grade\tnote{3} & Continuous          & 1   & No  & \yes                    & \no                      & \yes          & \no                      \\
            \bottomrule
        \end{tabular}
        \begin{tablenotes}
            \footnotesize
            \item[1] Age is modelled as a cubic polynomial.
            \item[2] A small additive constant (+1) is added to spleen size to allow inclusion of patients with non-palpable spleens.
            \item[3] Parasite grade is recorded on a semi-quantitative logarithmic scale, ranging from 1+ to 6+.
        \end{tablenotes}
    \end{threeparttable}
    \caption{Variables used in the development of the Indian subcontinent (ISC) and East Africa (EA) models. Models are fitted either with parasite grade ($\mathrm{PG}$) or without parasite grade ($\overline{\mathrm{PG}}$). -: not applicable, \yes:~included in model, \no:~not included in model, dof: degrees of freedom, log: whether modelled on a natural logarithmic scale, ALT: alanine aminotransferase, WBC: white blood cell.}
    \label{tab:meth-cp}
\end{table}

The following points were considered when selecting candidate predictors:

\begin{itemize}[noitemsep]
    \item To avoid excessive missing data, predictors must be available for at least 50\% of participants.
    \item Predictors should be routinely measured, or at least available for measurement, in the majority of treatment centres in endemic areas.
    \item Since the model is applied at the time of initial cure, predictors should be available at or \textit{prior} to this time point (typically measured within a month of starting treatment).
    \item Predictors should be \textit{preferentially} included if (i) they have previously been shown to predict relapse, or (ii) other compelling reasons exist to include \textemdash\ such as expert opinion and arguments supporting a causal association between the predictor and relapse. Discussions in Section \ref{sec:relapse} provide further insight.
    \item Sample size considerations explored in Section \ref{sec:meth-samp} should guide the maximum number of predictor parameters to prevent model overspecification and overfitting.
    \item To facilitate model convergence, excessive collinearity should be avoided.
\end{itemize}

When considering the final point, it is worth highlighting that model convergence issues can occur due to collinearity not only between predictors, but also between predictors and random--effects (clustering) structures. Given the significant heterogeneity in study design and outcome definitions, and further reasons addressed further in Section \ref{sec:meth-spec}, each contributing study is modelled as a random--intercept term. Consequently, caution must be exercised with categorical predictors that are uniquely, or near--uniquely, identified at the study level.

\subsubsection{Treatment}

Treatment regimen \textemdash\ already established as an important predictor relapse \textemdash\ represents a categorical predictor affected by collinearity at the study level. For example, including a treatment category of \textit{14 days of paromomycin} in the ISC model would lead to convergence failure when included in a model with study as a random--intercept term. As can be appreciated from Figure \ref{fig:isc_treat}, only one contributing study (Sundar 2009) includes patients receiving this treatment regimen. Consequently, a model including both would not be able to distinguish between relapse risk related to the study or the treatment regimen.\footnote{Or more technically, this induces near--non--identifiability between the fixed treatment effect and the study--level random intercept, resulting in an ill--conditioned information matrix and failure of numerical optimization.}

Considering the distribution of treatment regimens across studies within the ISC, treatment will be categorised into three categories that occur in at least three studies (see Table \ref{tab:isc_study} and Figure \ref{fig:isc_treat}):

\begin{itemize}[noitemsep]
    \item Single dose liposomal amphotericin B (10mg/kg)
    \item 28 days miltefosine (standard dose)
    \item Other
\end{itemize}

As a consequence of creating a `catch--all' treatment group: \emph{Other}, some of the relationship between treatment and relapse not already accounted for at the study level will be lost. This is an important limitation and addressed further in the discussion (Chapter \ref{ch:7-discussion}).

Unfortunately, convergence issues preclude the inclusion of treatment as a categorical predictor in the East Africa models. This is not surprising, given fewer participants, fewer studies, and higher treatment--study collinearity when compared to the ISC models (see Figure \ref{fig:ea_treat}). Convergence issues persisted despite exploring different treatment groupings, including separate groups for SSG and SSG/PM combination therapy. Therefore, in the East Africa models, the impact of treatment on relapse risk is incorporated into the study level random intercept.

\subsubsection{Malnutrition}

Malnutrition is a well-established determinant of progression from asymptomatic infection to clinical VL and of adverse outcomes following initial treatment, including treatment failure and mortality\cite{pareyn2025,abongomera2020}. Evidence linking malnutrition to relapse, however, remains sparse. This may partly reflect the absence of a unified framework for defining malnutrition across age groups, leading to frequent omission or inconsistent classification in prognostic factor and prediction model studies. Anthropometric assessment of malnutrition is inherently age-specific, and no formal guidance exists on how to combine measures across the life course.

We therefore adopt a pragmatic three-level severity scale using age-appropriate metrics (Table~\ref{tab:malnutrition}). In children under five years, malnutrition is classified using weight-for-height (WFH) z-scores, largely consistent with WHO guidelines on the definition of acute malnutrition\cite{WHO-nutrition-2023}. For individuals aged five to under nineteen years, body--mass--index--for--age (BMI-FA) z-scores are used with identical cut-points, supported by the WHO 2007 growth reference, which was explicitly constructed to ensure continuity with the under-five standards at age five\cite{deonis2007}. In adults aged 19 years and over, malnutrition severity is defined using established BMI thresholds. Cole et~al.\ showed that adult BMI cut-offs of 16 and 17 kg/m$^2$ at 18 years correspond approximately to BMI-for-age z-scores of $-3$ and $-2$ in children and adolescents, providing a statistical basis for approximate continuity of severity definitions across the adolescent--adult boundary\cite{cole2007}. While imperfect, this approach preserves broadly comparable degrees of nutritional deficit across age groups.

For modelling purposes, individuals with mild (18.5 $\leq$ BMI < 25 kg/m$^2$) or obesity (BMI $\geq$ 25 kg/m$^2$), or equivalent z-scores (z > -1), were combined into a single \emph{mild/normal} category due to small numbers.

The approach adopted here is consistent with methods used in previous studies of VL outcomes that also use age-specific anthropometric indicators to define pragmatic malnutrition severity groupings\cite{burza2014,dorlo2017,gorski2010,naylor-leyland2022}. Z-scores were calculated using the \texttt{anthro} and \texttt{anthroplus} R packages \cite{schumacher2023,schumacher2021}.

\subsubsection{Anaemia}

Anaemia was grouped into two categories: \emph{severe} and \emph{non-severe}, using haemoglobin cut-offs stratified by age and sex thresholds as defined by 2024 WHO guidelines\cite{who_haem2024}. Additional subdivision of the non-severe anaemia group was limited by the small number of participants in the mild and normal categories.

\subsubsection{Age}

To account for an anticipated non-linear relationship between age and relapse, age was modelled as a third-degree polynomial term (including linear, squared, and cubic components).

\subsubsection{Parasite grade}

Baseline parasite grade, when available, was assessed from splenic, bone marrow, or lymph node aspirates. When reported, the logarithmic counting method of Chulay and Bryceson (1983) was either described or directly cited {Additional file 1}\cite{chulay1983}.

\subsubsection{Logarithmic transformations}

A number of continuous predictors \textemdash\ including fever duration, spleen length, and all blood tests \textemdash\ were log-transformed to reduce skewness and better approximate normality. For spleen size, a value of 1 was added prior to transformation to accommodate zero values and avoid undefined logarithmic results.

\subsection{\label{sec:meth-desc}Descriptive analysis}

All candidate predictors (pooled and study--specific) are summarised in both tabular and graphical forms.

The correlation between continuous predictors is illustrated with facetted scatter plots using the \texttt{ggpairs()} function from the \texttt{GGally} R package\cite{GGally2025}. The correlation between continuous and categorical predictors is presented with facetted box--and--whisker plots.

Unadjusted relationships between the pooled candidate predictors and the outcome (relapse) are presented in graphical form. Confidence intervals for the relationships are presented in preference to p--values\cite{altman1986}.

For continuous predictors, relapse is modelled using a generalized additive model (GAM) with a binomial error distribution and logit link, fitted using the \texttt{gam()} function in the R package \texttt{mgcv}\cite{wood2011}. The effect of the continuous predictor is represented by a smooth function estimated using penalised thin-plate regression splines. Model fitting is performed by penalised maximum likelihood, with the degree of smoothness selected automatically via restricted maximum likelihood (REML).

For categorical predictors, relapse is presented with bar charts with 95\% confidence intervals calculated using the Wilson method\cite{wilson1927}.



