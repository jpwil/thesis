In its broadest sense, VL relapse refers to the recurrence of signs and symptoms of VL following an initial treatment response\cite{chhajed2024}.

Importantly, a patient may only relapse once an initial treatment response is first achieved, also known as initial cure in clinical efficacy studies. Achievement of initial cure typically requires not only improvement of the patient's signs and symptoms (e.g., defervescence, reduction in spleen size, weight gain, improvement of haemoglobin), but also visualisation of amastigotes on microscopy of a tissue aspirate smear. Assessment of initial cure, (`test of cure') most often occurs within a month of treatment completion, although considerable heterogeneity exists in exactly how and when treatment outcomes are defined\cite{dahal2024}.

\subsection{Burden of relapse}

% general comment - high relapse rates in certain populations
The proportion of patients that relapse following an initial treatment response varies dramatically across studies, reflecting differences in host and parasite factors\cite{chhajed2024}. The most important host-factor predictive of relapse is the presence of HIV co-infection, with studies frequently describing relapse rates upwards of 20\%, depending on the degree of immunosuppression\cite{cota2011,diro2019,burza2014a}. Treatment regimen and the presence of parasite resistance also play important roles, with relapse rates of over 50\% frequently reported during the 1980s and 1990s in Bihar, India, due to the development of antimony resistance\cite{olliaro2005}.

% estimate of current burdens in East Africa & ISC
As a rough `rule of thumb', relapse rates with current first-line regimens in patients \textit{without} significant immunosuppression range between 1 in 40 to 1 in 10 patients (2.5--10\%)\cite{chhajed2024}. In a meta-analysis of efficacy studies recently published by Chhajed et al., the overall proportion of HIV-negative patients relapsing in the ISC at 6-months was estimated at 3.5\% (95\% confidence interval (CI):2.8--4.5\%) following first-line treatment with single-dose LAMB\cite{chhajed2024}. Under pragmatic conditions in East Africa, slightly higher 6-month relapse rates of around 5\% are seen with the first-line combination therapy of PM and SSG\cite{kimutai2017,atia2015,melaku2007}.

% timing of relapse

Efficacy studies that limit follow-up to 6-months will underestimate the true relapse rate by a considerable margin. Chhajed et al. compared 21 studies that reported both 6-month and 12-month relapse rates, and noted that two-thirds of all relapses were detected by 6-months. Recent large studies from the ISC report similar findings, leading to calls to extend the routine follow-up period from 6-months to a year\cite{sundar2019,burza2013,burza2014,goyal2019}.
