Since obtaining tissue aspirate samples is invasive, considerable effort has focused on developing non-invasive biomarkers to support both VL diagnosis and relapse prognostication. These biomarkers can be broadly categorised into direct markers of parasite burden --- such as those obtained from molecular methods --- and markers assessing the host immune response.

A variety of molecular approaches have been evaluated for the diagnosis of VL\cite{sundar2018-mol, deruiter2014, reithinger2007, osman1998}. Among these, real-time quantitative PCR (qPCR) of kDNA, performed on the buffy coat of peripheral blood, has demonstrated excellent sensitivity and specificity compared with tissue aspirate microscopy, while also serving as a significant predictor of relapse\cite{hossain2017, mary2006, sudarshan2011, verrest2021, verrest2024}. Of particular note, using qPCR measurements from 177 immunocompetent patients recruited into DNDi trials in East Africa\footnote{LEAP 0714 (30 children with allometric miltefosine), LEAP 0208 (151 patients receiving miltefosine and LAMB combination regimens), FEXI VL 001 (14 patients receiving fexinidazole)}, Verrest et al showed that qPCR levels performed between treatment initiation and day 56 all significantly predicted relapse, with areas under the receiver operating curve (AUC) of 0.71, 0.74 and 0.92 when using qPCR on days 14, 28 and 56 respectively\cite{verrest2021}. With the same qPCR data, Verrest et al subsequently developed a semi-mechanistic population PK-PD model, characterising parasite replication, drug action, and subsequent post-treatment parasite clearance. The model successfully predicted relapse based on the modelled day 28 and 56 parasite loads, and provided the first direct evidence that the development of relapse depends not only on initial parasite clearance during treatment, but also on the subsequent host immune response\cite{verrest2024}. Interestingly, inclusion of haematological and biochemical parameters, including Hb, white blood cells (WBC), platelets and creatinine, did not help explain the variation in post-treatment qPCR measurements.