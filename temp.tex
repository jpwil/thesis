This chapter describes the development and internal validation of prognostic models predicting visceral leishmaniasis (VL) relapse in immunocompetent patients from the Indian subcontinent (ISC).

As outlined in Chapter~\ref{ch:2-background}, relapse is not only consequential for individual patients but also represents an infection reservoir that threatens the success of ongoing VL elimination efforts. Development of a non-invasive tool predicting relapse, as a test of cure following initial treatment, has been identified by the WHO as a current research priority\cite{WHO2024_Leishmaniasis,WHO_2024_VL_easternAfrica,who_sea_elim}. A prediction model, using routinely collected patient information, could fulfil this role by identifying treatment-responsive patients at increased risk of relapse, enabling targeted counselling and follow-up to support early detection and management.

The Infectious Diseases Data Observatory (IDDO) VL data platform hosts over 14,000 individual participant data (IPD) from almost 50 clinical studies across all endemic regions\cite{IDDO_VisceralLeishmaniasis2025}. Since no prognostic models for VL relapse have previously been published (Chapter~\ref{ch:3-sys-review}), model updating is not possible. The IDDO VL data platform therefore provides a unique opportunity to develop and validate the first relapse prediction models tailored to patients in the ISC.

Two complementary models are presented. Model~1 incorporates parasite grade from a pre-treatment tissue aspirate, whereas Model~2 excludes this variable to allow use in settings where diagnosis relies solely on serological or clinico-epidemiological criteria. The chapter is divided into a methodology section, detailing the