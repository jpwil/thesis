Understanding study--specific definitions of initial cure is important, as most studies require the patient to achieve initial cure in order to be at risk of relapse. Initial cure definitions based solely on clinical symptoms, reflecting routine practice, are likely to include patients with a persistently positive tissue aspirate, were this measured. Such patients represent a higher risk subgroup for subsequent relapse, and would otherwise have been classified as initial treatment failures and excluded from further follow-up if the study were to apply a more stringent, parasitology--based test--of--cure. It therefore follows that, all else being equal, stricter criteria for achieving initial cure are associated with a lower observed risk of subsequent relapse.

Of the 89 studies identified by Dahal et al, 71 (79.8\%) required confirmation with parasitological assessment (with or without demonstration of clinical improvement), while 13 (14.6\%) required clinical improvement only; the remaining studies did not report cure criteria\cite{dahal2024}. The timing of initial cure assessment also varied considerably, with the 68 (76.4\%) of studies performing the test--of--cure from 15--30 days following treatment completion. Similar patterns are observed in the contributed studies, as reported in the \href{https://github.com/jpwil/dphil}{Supplemental Material}, and discussed further in subsequent results chapters. Importantly, criteria for `clinical improvement' are often not specified. Further complicating interpretation, many studies describe a subgroup of `slow responders', in whom the test--of--cure tissue aspirate remains positive despite clinical improvement. These patients may undergo repeat assessment at variable time points (e.g. 2--4 weeks later), with or without treatment extension, and may or may not ultimately be classified as having achieved initial cure. Such variation in both the definition and timing of initial cure assessment can challenge standardisation efforts, leading to heterogeneity in observed relapse rates that is independent of other relapse risk factors.

\subsubsection{Relapse and `definite cure`'}

Relapse itself, where described at the study--level, is also subject to substantial variation with respect to its (i) definition \textemdash\ including whether patients require initial cure, the severity of symptoms required to trigger a repeat aspirate, and the tissue type chosen for aspirate, and (ii) timing \textemdash\ e.g. whether patients were actively screened at set time points with clinical examination $\pm$ routine aspirates, or whether dependent on patients attending voluntarily (passively) based on recurrent symptoms and discharge advice. In line with findings by Dahal et al, a significant proportion of contributing studies do not directly define relapse as a study outcome\cite{dahal2024}. Instead, for most studies, a relapse event can be inferred from patients achieving initial cure who subsequently do not meet the definition of `definite cure', which itself is typically defined as patients requiring rescue treatment.