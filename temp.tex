Host immunity as a risk factor for disease onset and poor treatment outcomes cannot be understated, and is considered an important driving factor for a number of devastating epidemics related to war and natural disasters. A striking example is Sudan in the 1980s-2000, where population displacement driven by conflict famine brought immune-naïve populations into endemic areas and spread infection into previously unaffected areas. Combined with severe malnutrition and the collapse of health infrastructure, the resulting mortality was catastrophic\cite{collin2004, al-salem2016}.

\section{Pathophysiology}

Parasites of the genus \textit{Leishmania} belong to the family \textit{Trypanosomatidae} (class \textit{Kinetoplastea}, phylum \textit{Euglenozoa}) and share many features with the two other human pathogens of the same family; \textit{Trypanosoma brucei} -- the causative agent of African sleeping sickness (human African Trypanosomiasis), and \textit{Trypanosoma~cruzi} -- the agent of Chagas disease in the Americas. These vector-borne NTD siblings are all single-celled protozoa characterized by the presence of a flagellum and a kinetoplast -- a dense network of mitochondrial DNA that conveys a distinctive microscopic appearance.

In the sandfly gut, \textit{Leishmania} parasites exist in their flagellated promastigote form (figure \ref{fig:cdc_lifecycle}), which are regurgitated into the bite wound during a blood meal and rapidly phagocytosed by macrophages. Intracellularly, the promastigotes lose their flagellum and transform into amastigotes: oval bodies measuring approximately 2-6$\mu$m in length with their nucleus and kinetoplast clearly visible with nucleic acid staining. Remarkably, amastigotes have evolved to survive and replicate within the hostile phagolysosomal environment. Immune evasion mediated by a dense surface coat that prevents complement-mediated lysis, alongside active suppression of cell-mediated immunity via elevated IL-10 production that promotes an anti-inflammatory M2 macrophage phenotype; the full immunological mechanisms are beyond the scope of this chapter. In viscerotropic disease caused by \textit{L. donovani complex} spp., infected macrophages disseminate throughout the mononuclear phagocyte system to the spleen, liver, bone marrow and lymph nodes.