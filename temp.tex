In areas of anthroponotic transmission, control and elimination of VL centres on two logically interdependent strategies: (i) reduction of the human infection reservoir through early diagnosis and treatment, and (ii) the interruption of transmission events through vector control measures. Effective implementation of these approaches, however, requires evidence-based decision-making supported by a firm political commitment.

Guided by these principles, in 2005 the governments of Nepal, Bangladesh, and India, supported by the WHO, signed a Memorandum of Understanding aiming to reduce VL incidence to fewer than 1 case per 10,000 population at the district or sub-district levels by 2015 --  subsequently extended to 2020 and now integrated into the 2021--2030 WHO NTD roadmap\cite{who_sea_elim2005, WHO_KalaAzarElimination_2015, WHO_SEA_CD_329_2021, WHO_NTDs_Roadmap_2021_2030}. Initial success of the kala-azar elimination programme (KAEP) was attributed to the introduction of the rK39 rapid diagnostic tests, oral miltefosine, and subsequently single dose LAMB, alongside active case detection and vector control measures including indoor residual spraying (IRS) and insecticide treated nets (ITN)\cite{sundar2018}. In October 2023, Bangladesh was the first country to be declared as having eliminated VL as a public health problem\cite{nagi2024}. Nepal and India are now working to sustain their targets for 3 years to also achieve elimination status\cite{pandey2025}.
