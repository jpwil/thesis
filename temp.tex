As outlined in Chapter~\ref{ch:2-background}, VL relapse is consequential not only for individual patients but also poses a threat to sustained elimination efforts. Accordingly, the development of a non-invasive tool to predict relapse \textemdash\ functioning as a `test-of-cure' following initial treatment \textemdash\ has been identified by the WHO as a research priority\cite{WHO2024_Leishmaniasis,WHO_2024_VL_easternAfrica,who_sea_elim}. A prognostic model represents a potential solution: by quantifying the relationship between patient characteristics and subsequent relapse events, relapse risk in future patients can be estimated and clinical decision-making informed. However, as demonstrated in Chapter~\ref{ch:3-sys-review}, no prognostic models for VL relapse have been published to date.

To address this evidence gap, four prognostic models are developed using IPD from the IDDO VL data platform: two for patients from the ISC and two for patients from East Africa. Within each region, one model includes parasite grade at initial cure assessment and one excludes parasite grade, reflecting differences in data availability and clinical practice. All models use routinely collected information available at the time of initial cure assessment to predict six-month relapse among VL patients without HIV co-infection.

The development and evaluation of clinical prediction models is supported by an extensive methodological literature. Over the past decade, reporting guidelines for prediction model studies have been established\cite{collins2015, moons2015, debray2023, collins2024A}, alongside a growing number of reviews and recommendations that define best practice\cite{efthimiou2024, collins2024B, riley2024A, riley2024B, vanSmeden2021}. The use of IPD from multiple studies introduces additional methodological challenges; specifically (i) clustering of participants within studies and (ii) the presence of missing data\cite{debray2023}.

The aim of this chapter is to transparently describe and justify the methodology used for model development and evaluation \textemdash\ from data acquisition to final model presentation. Guidelines and methodological texts are cited as required, and checklists are provided for current reporting guidelines on prediction model studies (TRIPOD-AI and TRIPOD-Cluster)\cite{collins2024A,debray2023}. Supplementary material is provided, including the R code.
