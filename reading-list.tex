Emerging trends and Innovative strategies for the diagnosis of Leishmaniasis: A quantum leap from classical to modern era
https://www.sciencedirect.com/science/article/pii/S0001706X25002906?via%3Dihub

Prevalence of human visceral leishmaniasis and its risk factors in Eastern Africa: a systematic review and meta-analysis
https://www.frontiersin.org/journals/public-health/articles/10.3389/fpubh.2024.1488741/full

Chronic High-Level Parasitemia in Human Immunodeficiency Virus-Infected Individuals With or Without Visceral Leishmaniasis in an Endemic Area in Northwest Ethiopia: Potential Superspreaders?
https://academic.oup.com/cid/article/79/1/240/7513292

Visceral leishmaniasis outbreak investigation and risk factors among communities in South Omo Zone, Southern Ethiopia, 2022–2023
https://www.sciencedirect.com/science/article/pii/S2405673125000595?via%3Dihub

Underreporting in Sudan.
Analyzing the Impact of Control Strategies on Visceral Leishmaniasis: A Mathematical Modeling Perspective
https://dx.doi.org/10.29020/nybg.ejpam.v17i2.5121

From Infection to Death: An Overview of the Pathogenesis of Visceral Leishmaniasis
Carlos Costa. MDPI.
10.3390/pathogens12070969

Predicting leishmaniasis outbreaks in Brazil using machine learning models based on disease surveillance and meteorological data
https://dx.doi.org/10.1016/j.orhc.2024.100453

Systematic Review of Treatment Failure and Clinical Relapses in Leishmaniasis from a Multifactorial Perspective: Clinical Aspects, Factors Associated with the Parasite and Host
https://www.mdpi.com/2414-6366/8/9/430

Portable smartphone-based molecular test for rapid detection of Leishmania spp
https://dx.doi.org/10.1007/s15010-024-02179-z

### BRAZIL MORTALITY REVIEW ###
Clinical-epidemiological aspects and prognostic factors associated with death from visceral leishmaniasis between the years 2010 to 2019 in the Central-West region of Brazil.
https://dx.doi.org/10.1016/j.parint.2023.102824

# HIV VL T CELL RESPONSES
Persistent T cell unresponsiveness associated with chronic visceral leishmaniasis in HIV-coinfected patients.
https://dx.doi.org/10.1038/s42003-024-06225-2

# Anaemia in dogs is associated with infectiousness
Clinical anemia predicts dermal parasitism and reservoir infectiousness during progressive visceral leishmaniosis.
https://dx.doi.org/10.1371/journal.pntd.0012363

# Predictive factor study
Time to recovery from visceral leishmaniasis and its predictors of mature visceral leishmaniasis patients admitted at Metema Hospital, Metema, Ethiopia.
https://dx.doi.org/10.1038/s41598-024-83716-6

